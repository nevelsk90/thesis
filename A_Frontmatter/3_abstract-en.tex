% ************************** Thesis Abstract *****************************
% Use `abstract' as an option in the document class to print only the titlepage and the abstract.
\begin{abstract}
% Background

Ageing individuals exhibit a pervasive decline in adaptive immune function, with important implications for health and lifespan. Systemic changes observed in the structure and diversity of antibody repertoires with age are thought to play an important role in this immunosenescent phenotype; however, the relatively long lifespan of most vertebrate model organisms makes thorough investigation of the ageing repertoire challenging. As a naturally short-lived vertebrate, the turquoise killifish (\nfu) offers an exciting new opportunity to study the ageing of the adaptive immune system in general and antibody repertoires in particular.

In this thesis, I used a combination of existing genomic assemblies and new sequencing data to assemble and characterise the immunoglobulin heavy chain (\igh{}) locus sequence in the turquoise killifish and compare it to those of closely related species, revealing a history of dynamic locus evolution and repeated duplication and loss of the specialised mucosal isotype \igh{Z}. The \Nfu locus itself lacks \igh{Z}, making it one of the few known teleost species not to possess this isotype. 

Having characterised the \igh{} locus sequence in \Nfu, I used it to establish targeted immunoglobulin sequencing in this species, enabling quantitative interrogation of the antibody repertoire. Applying this protocol to whole-body killifish samples revealed complex and individualised antibody repertoires which decline rapidly in within-individual diversity and increase in between-individual variability with age, demonstrating that turquoise killifish exhibit a rapid repertoire-ageing phenotype in line with their short lifespans. This loss of diversity with age was particularly strong in isolated gut samples, a phenomenon that may be related to the constant strong antigenic exposure experienced at mucosal surfaces. Taken together, these results establish the turquoise killifish as a novel model for vertebrate immunosenescence and lay the groundwork for future interrogation of -- and intervention in -- adaptive-immune ageing.

% Comparing the \igh{} loci of these two closely-related species (\Cref{sec:nfu-locus,sec:xma-locus}) reveals dramatic and unexpected differences in immune structure, which when combined with information from previously-published loci from related species suggest unexpected patterns of locus evolution within this group of teleost fishes. Most strikingly, the specialised mucosal antibody isoform \igh{Z} appears to have been convergently lost in multiple closely-related lineages. 
%
%To further investigate the history of \igh{Z} and other surprising features of \igh{} locus evolution in the Cyprinodontiformes, I performed a partial reconstruction and analysis of the \igh{} loci from ten further cyprinodontiform species (\Cref{fig:species-tree-large-taxa}), as well as from a new and improved genome assembly of medaka (\textit{Oryzias latipes}), with a focus on the constant-region exons present in each species (\Cref{sec:locus_comparative}). Phylogenetic analysis confirms the repeated independent loss of \igh{Z} in this lineage and provides evidence for multiple, independent \igh{Z} subclasses present ancestrally in the clade. Taken together, this analysis significantly extends our knowledge of constant-region diversity in teleost fish, and establishes the cyprinodontiforms, and especially the African killifishes, as a highly promising collection of model systems for comparative evolutionary immunology.
%
%
%In this thesis, therefore, I establish the turquoise killifish as a model for the study of comparative immunology and humoral adaptive immunosenescence. Using a combination of existing genomic assemblies and new sequence data, I assembled and characterised the immunoglobulin heavy chain (\igh{}) gene locus of the turquoise killifish and compared it to other newly-assembled loci from closely-related species, revealing a group of complex and rapidly-evolving loci with a number of surprisingly ideosyncratic features (\Cref{chap:locus}). Using the sequences from this newly-characterised locus, I established the first working immunoglobulin-sequencing protocol in this species, which I used to investigate the diversity and complexity of heavy-chain immune repertoires in adult killifish and how this diversity changes with age in the whole body and the gut (\Cref{chap:locus}). The results of these investigations demonstrate that the turquoise killifish possesses a complex, diverse and individualised antibody repertoire, which undergoes a rapid decline in within-individual diversity and increase in between-individual variability with age. This phenomenon is particularly strong in the gut mucosal repertoire, likely as a result of the much greater relative prelavence of large, antigen-experienced clones and the intense antigen exposure of mucosal B-cell populations. Taken together, these results demonstrate the value of the turquoise killifish as an emerging model system in vertebrate immunology. % TODO: Daniel: mention gut-microbiota transfer stuff
%
%In this 


\end{abstract}

%
%
%To establish the turquoise killifish as a model for adaptive immune ageing, we sequenced, assembled and characterised the IgH locus of this species, revealing a complex and repetitive structure comprising two complete tandem subloci in antisense on chromosome 6. Unusually for a teleost IgH locus, the turquoise killifish does not possess the teleost-specific mucosal antibody class IgZ, raising the question of how it carries out mucosal immunity. This ideosyncratic lack of IgZ is shared with the IgH locus in medaka, the most closely-related species whose locus had previously been characterised, suggesting it may be the result of a shared ancestral deletion event. To test this hypothesis, we characterised the locus of a second important model species in the Cyprinodontiformes, the southern platyfish Xiphophorus maculatus. To our surprise, the platyfish locus contained multiple intact IgZ constant regions, strongly suggesting that the absence of IgZ in turquoise killifish and medaka is the result of multiple independent deletion events. Phylogenetic analysis of IgH regions in the genomes of a further ten cyprinodontiform species confirmed the presence of multiple independent losses of IgZ within the lineage, suggesting a surprising level of evolutionary volatility in the constant-region classes available to cyprinodontiform fishes. 
%Further experiments will exploit this variability to further investigate comparative evolutionary immunology in teleost fishes, for example by comparing mucosal adaptive immunity in closely-related IgZ+ and IgZ- cyprinodontiform species.
%
%Using the newly-assembled IgH locus in the turquoise killifish, we established and validated a protocol for immunoglobulin sequencing in killifish tissues, and performed the first repertoire-sequencing experiments in this species. The results of these experiments indicate that adult male turquoise killifish express a whole-body repertoires with diversity broadly comparable to that of other studied teleosts, which undergo a rapid decline in clonal diversity with age even over the very short lifespan of this species. In addition to a loss of within-repertoire (alpha) diversity, aged turquoise killifish show a marked increase in between-repertoire (beta) diversity compared to younger killifsh, a pattern which mirrors the loss of alpha-diversity and increase in beta-diversity seen in the gut microbiota of turquoise killifish with age. These results demonstrate the utility of the turquoise killifish as a model for quantitative, high-resolution investigation of vertebrate adaptive immunosenescence. Further experiments will investigate the connection between antibody-repertoire diversity and the gut microbiota with age, with particular focus on the effects of gut-microbial transfer  from young to old fish on the diversity and health of the mucosal repertoire. The results of these investigations could have important implications for our understanding of adaptive immunosenescence in vertebrate mucosal organs, both in normal ageing and in ageing-related disease. 
%
%Ageing individuals exhibit a pervasive decline in B-cell immune function, with important implications for health and lifespan. Systemic changes observed in the structure and diversity of the antibody repertoire with age are thought to play an important role in this immunosenescent phenotype; however, the relatively long lifespan of most vertebrate model organisms has made thorough investigation of the ageing repertoire challenging.
%
%The African turquoise killifish (\nfu) is the shortest-lived vertebrate to be bred in captivity, and has undergone rapid development as an emerging model organism for ageing research. The advent of the killifish as an ageing model presents an exciting new opportunity to investigate immune-repertoire ageing in a detailed and controlled manner. In this thesis, I set out to establish the turquoise killifish as a model system for vertebrate adaptive immunity and immunosenescence.
%
%In this project, I aim to establish the turquoise killifish as a model organism for vertebrate adaptive immunity and immunosenescence. Using high-throughput-sequencing approaches, I have assembled and characterised the killifish IGH locus and compared it to characterised loci of closely related species. Using  this locus sequence, I have established and validated a protocol for immunoglobulin sequencing in killifish tissues, and performed the first repertoire-sequencing experiments in this species. Preliminary results from adult males indicate whole-body repertoire diversity broadly comparable to that of other studied teleosts. 
%
%Further repertoire sequencing experiments currently underway in the turquoise killifish will examine the changes in repertoire sequencing that occur during development and ageing in this model, with a focus on those changes that occur following reproductive maturation. In addition, the effect of anti-ageing interventions (such as gut microbial transfer) on repertoire ageing in the killifish will be investigated, with the long-term goal of identifying interventions that maintain adaptive immune health in old age. The results of these experiments could have important implications for our understanding of adaptive immunosenescence in clinical and other applied contexts.
