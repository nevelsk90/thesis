% ************************** Thesis Abstract *****************************
% Use `abstract' as an option in the document class to print only the titlepage and the abstract.
\begin{zusammenfassung}
\textbf{Motivation:} %Zirkul\"are Ribonukleins\"auren (circRNA) sind eine spezifische Gruppe von RNA, welche einen kovalent geschlossenen Kreis bilden. Dieser Prozess wird als r\"uckw\"arts-splei\ss en bezeichnet. Bisher ist nicht viel \"uber die funktionsweise von circRNAs bekannt. F\"ur einige wenige circRNAs wurden m\"ogliche Funktionen als Schwamm f\"ur microRNAs (miRNA) oder RNA-bindende Proteine (RBP) gezeigt. Au\ss erdem k\"onnen sie die Transkription ihres Wirts-Gens regulieren. Zirkul\"are RNA k\"onnen anhand von chim\"aren Sequenzen, welche die r\"uckw\"arts gesplei\ss te Verbindung \"uberbr\"ucken, in rRNA-verdauten RNA-Seq Bibliotheken detektiert werden. Zur Zeit gibt es eine Vielfalt an unterschiedlichen Programmen, die circRNAs in RNA-Seq Daten identifizieren. Allerdings ist keines der Programme in der Lage die circRNAs weiter zu characterisieren oder zusammenzufassen. Um weiterf\"uhrende Analysen an den entdeckten circRNAs durch f\"uhren zu k\"onnen ist es unabl\"asslich die genau Exon-Intron Struktur von circRNAs zu kennen. Vor kurzem wurden zwei neue Programme ver\"offentlicht, die alternatives Splei\ss en in circRNAs beschreiben. In meiner Arbeit stelle ich \texttt{FUCHS} und \texttt{FUCHS\textit{denovo}} vor um entdeckte circRNAs zusammenzufassen und die Exon-Intron Struktur anhand von linearen Splei\ss  Signalen von chim\"aren Sequenzen zu rekonstruieren.

\textbf{Methoden:} %Zuerst habe ich drei der aktuellsten circRNA Identifikations Progamme miteinander vergleichen, um basierend auf den Ergebnissen des besten Programms eine Pipeline in Python zu entwicklen. Diese Pipeline hei\ss t \texttt{FUCHS}, kurz f\"ur "\textbf{FU}ll \textbf{CH}aracterization of circular RNA using RNA-\textbf{S}equencing". Sie fasst circRNAs nach ihren Wirts-Genen zusammen, findet \"ubersprungene Exons, findet Doppeltchim\"are Sequenzen, generiert Abdeckungsprofile und fasst circRNAs basierend auf ihrem Abdeckungsprofil zusammen. Das Anwenden von \texttt{FUCHS} auf einem Beispieldatensatz hat gezeigt, dass annotierte Strukturen oft nicht ausreichen um die zirkul\"aren Strukturen zu beschreiben. Deswegen habe ich \texttt{FUCHS} erweitert. Das neue Programm hei\ss t \texttt{FUCHS\textit{denovo}}, da es die Exon-Intron Strukturen von circRNAs \textit{de novo} rekonstruieren kann. Um die Funktionsweise beider Programme vorzustellen, habe ich sie auf einem Beispieldatensatz bestehend aus Leber und Herz Proben von jungen und alten Mäusen angewendet.

\textbf{Ergebnisse:} %Im Vergleich von drei circRNA Identifikations Progamme (\texttt{DCC}, \texttt{CIRI}, and \texttt{KNIFE}) hob sich \texttt{DCC} als schnellstes und präzisestes Program ab. Die Anwendung von \texttt{FUCHS} auf vier Maus Proben zeigte, dass es weniger unterschiedliche circRNAs im Herzen als in der Leber gibt, diese daf\"ur aber in h\"oherer Anzahl. Betrachtet man nur annotierte Exons, zeigt sich, dass die circRNAs im Herzen l\"anger sind als die in der Leber. Die durchschnittliche L\"ange der circRNAs betr\"agt 500 BP. Aus den Abdeckungsprofilen habe ich geschlossen, dass die annotierten Exon-Intron Strukturen nicht immer mit den Exon-Intron Strukturen der circRNAs \"ubereinstimmen. Ein Vergleich zwischen den annotierten und den mit \texttt{FUCHS\textit{denovo}} rekonstruierten Strukturen zeigte einen Gewinn von 15 \% an zusätzlicher Information. Weiterhin hat \texttt{FUCHS\textit{denovo}} alternatives Splei\ss en in 8 - 10 \% der circRNAs finden k\"onnen. Eine Analyse von differenziel angereicherten Motifen in den Introns um circRNAs zeigte, dass die Introns um circRNAs mit alternativen Splei\ss  Isoformen gegen\"uber circRNAs ohne alternativen Splei\ss  Isoformen mit FOXO Bindemotifen angereichert sind. Bindemotife f\"ur CPEB1 und HOX waren in Introns um circRNAs von multi-circRNA Genen gegen\"uber Introns um circRNAs von single-circRNA Genen angereichert. Somit k\"onnten sowohl FOXO als auch CPEB1 und HOX eine Rolle in der Biogenese von circRNAs spielen. Eine miRNA und RBP Bindemotif Suche hat gezeigt, dass Exons von circRNAs dichter mit Bindemotifen best\"uckt sind als Exons von linearen mRNAs. Daraus schlie\ss e ich, dass circRNAs eine weitere Ebene im Genregulationsnetzwerk darstellen k\"onnen, indem sie mit linearen mRNAs f\"ur die Bindung von miRNAs und RBPs konkurrieren.

\textbf{Verf\"ugbarkeit:} %\url{https://github.com/dieterich-lab/FUCHS.git}

\end{zusammenfassung}