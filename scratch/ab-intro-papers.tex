\citep{mix2006immunoglobulins}:

The essential feature  of vertebrate immunity is specificity of recognition: the ability to specifically and selectively identify and respond to molecular signatures of invasion and disease. % The essential feature of adaptive immunity is memory

A complete antibody is a heterotetrameric protein complex comprising identical pairs of heavy and light peptide chains % More detail on structure, length etc of heavy/light chains

This four-chain structure is known as an antibody monomer, and constitutes the fundamental unit of vertebrate humoral immunity. Some secreted antibodies (such as IgM in most vertebrates and IgA in mammals) oligomerise further to form complexes comprising multiple antibody monomers; these complexes have increased overall avidity for antigen and so can respond more strongly to low levels of low-specificity antigen. % Learn more about J-chains; how do antibody multimers form in teleosts?

The terminology surrounding the structure and function of an antibody monomer is complex and overlapping, and is summarised in % figure describing constant vs variable region, Fab vs Fac, isotype vs idiotype, etc

The N-terminal regions of a heavy-chain/light-chain pair (corresponding to the 5-prime regions of the corresponding heavy- and light-chain genes) constitutes the antigen-binding region (paratope) of the antibody, with the antigen-specificity of this paratope determined in concert by the structures of the heavy- and light-chain antigen-binding sites. Since antibody monomers each comprise two heavy chains and two light chains, they are bivalent, with each capable of binding antigen at two independent (but identical) paratopes. % Are the paratopes always identical? I don't think so?

Meanwhile, the C-terminal regions of the heavy chains constitute the effector domain of the antibody, responsible for its interactions with other parts of the immune system. The Fc region also determines whether the antibody is secreted or expressed on the cell-surface membrane of the B-lymphocyte that produced it and, if secreted, whether it is capable of crossing mucosal or placental barriers via recepter-mediated transport. 

Effector functions of antibodies in mammals include opsonisation, recruitment of cytotoxic or pro-inflammatory immune cells, and activation of cytotoxic complement cascades%; multimeric antibodies can also aggregate antigen sources (such as pathogenic bacteria) together for immobilisation and removal from the body. 

Antibodies are (principally?/exclusively?) produced by specialised B-lymphocytes%, which in teleosts develop... (More detail on development part as necessary once Davina's report is ready)

The variable region of an antibody heavy chain is composed of V (``variable"), D (``diversity") and J (``joining") sequences... %describe basic locus structure, CDR vs FR, etc

during the development of a B-cell, individual V, D and J regions a

This recombination process is mediated by the action of recombination-activating genes 1 and 2 (RAG1 \& RAG2)% which are highly conserved among jawed vertebrates

which recognise recombination signal sequences (RSSs) flanking the gene segments to be recombined %sequence composition of RSSs, downstream processes in VDJR

Since the V and J regions in an IgH locus are typically variable in length, this recombination frequently results in shifts in frame leading to non-functional protein sequences%; this propensity to frame-shift mutations is greatly exacerbated by junctional diversity

If both alleles in a diploid B-cell undergo non-productive rearrangements, the cell dies by apoptosis. 


When stimulated by antigen in a suitable signalling context, the naive B-cell will...

% Focus on teleosts, not mammals and teleosts; only mention mammals where they differ

% Focus on antibodies, not B-cells, explain antibodies first, then mention B-cells where necessary to explain antibodies

\citep{patel2018aid}:

Class-switch recombination between isotypes, as seen in mammals and other tetrapods, is absent in fish; however, teleost AID is able to catalyse CSR in AID-deficient mammalian cells%, suggesting that this absence may have more to do with differences in locus structure than with the properties of the underlying enzyme.

VDJ recombination is initiated by a complex of recombination activating genes (RAG) 1 and 2, which recognises recombination signal sequences flanking V, D and J segments on the IgH locus (see figure); induce double-strand breaks (DSBs) in the DNA between the gene segment and the RSS, and co-ordinate recruitment of the cell's DNA-repair machinery (in particular, the non-homologous end-joining, or NHEJ, pathway of DSB repair) to resolve the DSBs and join the gene segments together. % Cite something for NHEJ

% these RSSs have a highly conserved structure, with (heptamer/spacer/nonamer). [cleavage by RAG1/2]

These DSBs are then repaired by the cell's DNA damage repair machinery, % ...esp. NHEJ

% Gene conversion in teleost Ig loci

As in mammals, somatic hypermutation in teleosts depends on the activity of activation-induced cytidine deaminase (AID), an APOBEC-family cytidine deaminase enzyme capable of inducing various nucleotide substitution mutations 

Like mammals, teleost fish possess both B- and T-lymphocytes and are capable of mounting significant memory responses to previously encountered pathogens. Also like mammals, they are capable of undergoing affinity maturation, whereby the affinity of antibody for an antigen increases post-exposure. However, the size of this increase in affinity is smaller in teleosts than in mammals, and they lack the specialised germinal-centre organs that act as loci of affinity maturation in mammals.

Specialised germinal centre organs for affinity maturation are absent in teleosts % however...

Loss of AID has been shown to abrogate SHM and CSR (as well as antibody-associated gene conversion) in diverse organisms; however, it seems ho have no effect on the efficiency of VDJ recombination. 

In both teleosts and tetrapods, AID has four active domains that play an important role in its activities in antibody diversification...% Elaborate if you include this: a APOBEC-type cytosine-conversion domain, ...

A typical Ig gene cluster in cartilaginous fish comprises one V segment, two or three D segments, one J segment and a set of constant-region exons.

While some teleost IgL loci exhibit cluster arrangements like those seen in cartilaginous fish, teleost IgH loci have a broadly translocon arrangement%, albeit with more internal repetition than is seen in most mammalian loci

The choice between IgZ and IgM/D is made during VDJ recombination %based on the configuration of the V/D recombination event
: if the selected V-segment is recombined with a D$\zeta$ segment, IgZ is expressed, whereas selection of a D$\mu$ segment necessitates the irreversible excision of the IgZ constant region. 

Conversely, the choice between IgM and IgD is determined by alternative splicing of a single V/D/J/IgM/IgD mRNA, % and the same B-cell is capable of expressing both IgM and IgD

N and P nucleotide insertions (and deletions) in teleosts occur in a manner very similar to that in mammals, and relies on similar enzymes (i.e. RAG and NHEJ)

Whereas VDJ recombination of the teleost IgH locus appears to take place in a manner very similar to that of mammals, VJ recombination in teleost IgL may be somewhat more different, % inversions in fish Ig diversification?

The locations of the main enhancers involved in IgH gene expression differ somewhat from those in mammals; % include stuff about Emu3 enhancer from this paper if you decide to include this (and understand it)

AID preferentially targets cytidine residues on single-stranded DNA, and prefers a sequence context of WR\textbf{C}H % in mammals or teleosts?
; it preferentially induces cytidine-to-uridine transitions leading to C:G to T:A mutations during somatic hypermutation; %however, it has also been observed to be capable of inducing other types of mismatch mutations - Wikipedia

The action of AID in somatic hypermutation appears to rely on active transcription of the region to be mutated%, possibly to produce an open DNA structure accessible to the enzyme?
; transcription and hypermutation rates along the V sequence appear to correlate linearly % what exactly does this mean?

The repetitive switch regions upstream of constant regions, which are targeted by AID as part of CSR in mammals, are absent in teleosts %?

While AID hotspot motifs are shared between mammals and teleosts, their number in fish is substantially less than in mammals, %resulting in markedly lower rates of somatic hypermutation 

\citep{hikima2011ig}:

To date, three heavy chain isotypes (M, D and Z/T) and four light-chain isotypes (...) have been found in teleosts...

The transcription of IgH genes in teleosts is regulated by a VH promoter sequence (upstream of each V-gene) and an E$\mu$3' enhancer region (located ...). This enhancer region differs importantly from other vertebrates in its location, structure, and mode of function.

The teleost lineage has undergone repeated rounds of whole-genome duplication, followed by widespread gene loss.

Teleost IgH loci typically adopt a translocon configuration similar to those found in tetrapods%, with large tandem blocks of V, D and J segments from which single segments are selected during VDJ recombination. However, this translocon configuration is complicated in teleosts by two factors, the location of the IgZ/T constant region... and the frequent presence of multiple tandem repetitions of the translocon structure...

Different teleosts differ in the exact location of the DZ/JZ/IgZ/T, with some species (e.g. zebrafish and fugu) having it located between the V-segment block and the DM/JM/IGM block and others (e.g. rainbow trout) having it located within the V region.

Most teleosts possess multiple tandem copies of the IgH locus structure, resulting in extremely large and complex IgH loci%; however, of the species examined so far, only the salmonids possess two complete loci on different chromosomes, suggesting ...

% Figure 1 of Hikima 2011 might be useful (depicts tree of locus structures)

In most fish, the most prevalent serum antibody is tetrameric secreted IgM, though monomeric serum IgM has also been found in some species.

The IgM gene in teleosts encodes four constant-domain (C$\mu$1-4) and two transmembrane ($\mu$TM1-2) exons. Secretory IgM in all examined teleosts comprises all four C$\mu$ exons, while transmembrane IgM typically comprises C$\mu$1-3 and TM1-2, with C$\mu$4 excluded. %One important exception to the latter is medaka, in which C$\mu$2 in transmembrane IgM is spliced directly 

Unlike tetrapods, in which $\mu$TM1 is spliced into a cryptic splice site within C$\mu$4, teleost transmembrane IgM excludes the entirety of C$\mu$4, resulting in a five-exon CDS comprising ... . The loss of C$\mu$4 does not appear to impair the functionality of transmembrane IgM in teleosts.

The C-terminal transmembrane domain of IgM-TM comprises a hydrophobic membrane-spanning helical region %(encoded by the TM1 exon?) 
and a short cytoplasmic tail %(encoded by TM2?)
#; this very short cytoplasmic region has no direct signalling function and relies on accessory molecules to mediate the antigen-signalling cascade. % How does the transmembrane helix and tail correspond to the different TM exons? Helix = TM1, tail = TM2?

The J-chain peptide, which in mammals forms an important part of polymeric IgM and IgA complexes, has not been found in teleosts to date.

As in mammals, the IgD constant region is found immediately 3' of IgM in teleost IgH loci. However, the structure and organisation of IgD differs markedly between mammals and teleosts. Whereas most mammalian IgD contains two or three C-domain exons, in teleost the number is much larger, % ranging from 7 (in ...) to at least 17 (in zebrafish). 
This very long constant-region structure is most often the result of tandem duplications of blocks of C$\delta$ exons; in many species the C$\delta$2-C$\delta$3-C$\delta$4 exon block is repeated multiple times, and other repetitive configurations have also been observed. 

In most teleost species, IgD has only been observed in transmembrane form, in which all C-domain exons plus two transmembrane exons are included; secreted IgD has been identified in two teleost species to date, % but with different modes of expression...

In addition to this expanded number of C$\delta$ exons, all teleost IgD transcripts observed to date also contain a chimeric C$\mu$1 exon between the variable region and the C$\delta$1 exon; in several species, the cysteine normally expected to form a disulfide bond with the antibody light chain is missing in C$\delta$1 % replaced by a serine in catfish
, with this function instead served by the chimeric C$\mu$1 exon. % see if this is true in killifish

In addition to the much larger number of C-domain exons, teleost IgD differs from mammals in lacking the flexible hinge domain, which in mammals is encoded by one or two hinge exons located between C$\delta$1 and C$\delta$2. % Which mammalian classes have hinges?

Teleost loci also lack any class-switch sequence between IgM and IgD (or elsewhere), with this inter-constant region instead containing the important E$\mu$3' enhancer region (see below).
