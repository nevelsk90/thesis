% ******************************************************************************
% ****************************** Custom Margin *********************************

% Add `custommargin' in the document class options to use this section
% Set {innerside margin / outerside margin / topmargin / bottom margin}  and
% other page dimensions
\ifsetCustomMargin
  \RequirePackage[left=37mm,right=30mm,top=35mm,bottom=30mm]{geometry}
  \setFancyHdr % To apply fancy header after geometry package is loaded
\fi

% Add spaces between paragraphs
\setlength{\parskip}{0.5em}
% Ragged bottom avoids extra whitespaces between paragraphs
\raggedbottom
% To remove the excess top spacing for enumeration, list and description
%\usepackage{enumitem}
%\setlist[enumerate,itemize,description]{topsep=0em}

% *****************************************************************************
% ******************* Fonts (like different typewriter fonts etc.)*************

% Add `customfont' in the document class option to use this section

\ifsetCustomFont
  % Set your custom font here and use `customfont' in options. Leave empty to
  % load computer modern font (default LaTeX font).
  %\RequirePackage{helvet}

  % For use with XeLaTeX
  %  \setmainfont[
  %    Path              = ./libertine/opentype/,
  %    Extension         = .otf,
  %    UprightFont = LinLibertine_R,
  %    BoldFont = LinLibertine_RZ, % Linux Libertine O Regular Semibold
  %    ItalicFont = LinLibertine_RI,
  %    BoldItalicFont = LinLibertine_RZI, % Linux Libertine O Regular Semibold Italic
  %  ]
  %  {libertine}
  %  % load font from system font
  %  \newfontfamily\libertinesystemfont{Linux Libertine O}
\fi

% *****************************************************************************
% **************************** Custom Packages ********************************
% *****************************************************************************
% NB: Packages (without options) loaded in thesis class:
%  lscape, setspace, calc, ifthen, ifpdf, ifxetex, epstopdf, datetime,
%  everypage, textcomp, mathptmx, fourier, lmodern, amsmath, fontspec,
%  unicode-math, amsfonts, amssymb, url, breakurl, fancyhdr, makeidx,
% NB: Packages (with [options]) loaded in thesis class:
%  tocbibind[nottoc], appendix[title,titletoc], color[usenames, dvipsnames],
%  lineno[switch,pagewise,mathlines], textpos[absolute], graphicx[pdftex,demo],
%  biblatex[various options depending on other settings],
%  natbib[various options depending on other settings],
%  geometry[various options depending on other settings],
%  inputenc[utf8], fontenc[T1], microtype[final],
%  hyperref[unicode=true, hidelinks], nomencl[intoc],
%\usepackage{fancyvrb}
%\usepackage{framed}
\usepackage{subcaption}
\usepackage{wrapfig}
\usepackage{setspace}
\usepackage{todonotes}
\usepackage{tcolorbox}
\usepackage{threeparttable}
\usepackage{booktabs}
\usepackage{xstring}
\usepackage{mfirstuc}
\usepackage{xspace}
\usepackage{afterpage}
\usepackage{placeins}
\usepackage{array}
%\usepackage{subfiles}
% ************************* Algorithms and Pseudocode **************************

%\usepackage{algpseudocode}


% ********************Captions and Hyperreferencing / URL **********************

% Captions: This makes captions of figures use a boldfaced small font.
%\RequirePackage[small,bf]{caption}

\RequirePackage[labelsep=colon,tableposition=top,labelfont=bf,font=small]{caption}
\renewcommand{\figurename}{Figure} %to support older versions of captions.sty


% *************************** Graphics and figures *****************************
\hyphenation{circRNA}
\hyphenation{circRNAs}

%\usepackage{rotating}

% Uncomment the following two lines to force Latex to place the figure.
% Use [H] when including graphics. Note 'H' instead of 'h'
%\usepackage{float}
%\restylefloat{figure}

% Subcaption package is also available in the sty folder you can use that by
% uncommenting the following line
% This is for people stuck with older versions of texlive
%\usepackage{sty/caption/subcaption}
\usepackage{subcaption}

% ********************************** Tables ************************************
\usepackage{booktabs} % For professional looking tables
\usepackage{multirow}

%\usepackage{multicol}
%\usepackage{longtable}
%\usepackage{tabularx}


% *********************************** SI Units *********************************
\usepackage{siunitx} % use this package module for SI units


% ******************************* Line Spacing *********************************

% Choose linespacing as appropriate. Default is one-half line spacing as per the
% University guidelines

% \doublespacing
% \onehalfspacing
% \singlespacing


% ************************ Formatting / Footnote *******************************
\usepackage{listings}

% Don't break enumeration (etc.) across pages in an ugly manner (default 10000)
%\clubpenalty=500
%\widowpenalty=500

%\usepackage[perpage]{footmisc} %Range of footnote options


% *****************************************************************************
% *************************** Bibliography  and References ********************

\usepackage{cleveref} %Referencing without need to explicitly state fig /table

% Add `custombib' in the document class option to use this section
\ifuseCustomBib
   %\RequirePackage[square, sort, numbers, authoryear]{natbib} % CustomBib

% If you would like to use biblatex for your reference management, as opposed to the default `natbibpackage` pass the option `custombib` in the document class. Comment out the previous line to make sure you don't load the natbib package. Uncomment the following lines and specify the location of references.bib file

\RequirePackage[backend=biber, citestyle=numeric-comp, style=numeric, sorting=none, 
	natbib=true, maxnames=4, minnames=4, url=false, giveninits=true]{biblatex}
\renewbibmacro{in:}{\ifentrytype{article}{}{\printtext{\bibstring{in}\intitlepunct}}}
%\bibliography{References/references} %Location of references.bib only for biblatex
\bibliography{D_References/Locus,D_References/IgSeq}

\fi

% changes the default name `Bibliography` -> `References'
\renewcommand{\bibname}{References}


% ******************************************************************************
% ************************* User Defined Commands ******************************
% ******************************************************************************

% *********** To change the name of Table of Contents / LOF and LOT ************

%\renewcommand{\contentsname}{My Table of Contents}
%\renewcommand{\listfigurename}{My List of Figures}
%\renewcommand{\listtablename}{My List of Tables}

\usepackage{calc}
\makeatletter
\newcommand{\tocfill}{\cleaders\hbox{$\m@th \mkern\@dotsep mu . \mkern\@dotsep mu$}\hfill}
\makeatother
\newcommand{\abbrlabel}[1]{\makebox[8cm][l]{#1\ \tocfill}}
\newenvironment{abbreviations}{\begin{list}{}{\renewcommand{\makelabel}{\abbrlabel}%
        \setlength{\labelwidth}{8cm}\setlength{\leftmargin}{\labelwidth+\labelsep}%
                                              \setlength{\itemsep}{0pt}}}{\end{list}}

% ********************** TOC depth and numbering depth *************************

\setcounter{secnumdepth}{3}
\setcounter{tocdepth}{3}


% ******************************* Nomenclature *********************************

% To change the name of the Nomenclature section, uncomment the following line

%\renewcommand{\nomname}{Symbols}


% ********************************* Appendix ***********************************

% The default value of both \appendixtocname and \appendixpagename is `Appendices'. These names can all be changed via:

%\renewcommand{\appendixtocname}{List of appendices}
%\renewcommand{\appendixname}{Appndx}

% *********************** Configure Draft Mode **********************************

% Uncomment to disable figures in `draft'
%\setkeys{Gin}{draft=true}  % set draft to false to enable figures in `draft'

% These options are active only during the draft mode
% Default text is "Draft"
%\SetDraftText{DRAFT}

% Default Watermark location is top. Location (top/bottom)
%\SetDraftWMPosition{bottom}

% Draft Version - default is v1.0
%\SetDraftVersion{v1.1}

% Draft Text grayscale value (should be between 0-black and 1-white)
% Default value is 0.75
%\SetDraftGrayScale{0.8}


% ******************************** Todo Notes **********************************
%% Uncomment the following lines to have todonotes.

%\ifsetDraft
%	\usepackage[colorinlistoftodos]{todonotes}
%	\newcommand{\mynote}[1]{\todo[author=kks32,size=\small,inline,color=green!40]{#1}}
%\else
%	\newcommand{\mynote}[1]{}
%	\newcommand{\listoftodos}{}
%\fi

% Example todo: \mynote{Hey! I have a note}

%%%%%%%%%%%%%%%%%%%%%%%%%%%%%%%%%%%%%%%%%
%% UNIT COMMANDS FOR PROTOCOL TEMPLATE %%
%%      Will Bradshaw, March 2017      %%
%%%%%%%%%%%%%%%%%%%%%%%%%%%%%%%%%%%%%%%%%

% ******************************** Units **********************************

%% CUSTOM UNIT COMMAND %%
\newcommand{\cu}[2]{\SI[detect-weight]{#2}{#1}} % custom unit

%% TIME %%
\newcommand{\secs}{\cu{\second}} % Seconds
\newcommand{\mins}{\cu{\minute}} % Minutes
\newcommand{\hr}{\cu{\hour}} % Hours

%% VOLUME %%
\DeclareSIUnit\vols{vol} % volumes (relative to some reference)
\newcommand{\vol}{\cu{\vols}} % volumes
\newcommand{\ml}{\cu{\milli\litre}} % millilitres
\newcommand{\ul}{\cu{\micro\litre}} % microlitres
\newcommand{\lt}{\cu{\liter}} % litres

%% MASS %%
\let\ng\undefined
\newcommand{\ng}{\cu{\nano\gram}} % nanograms
\newcommand{\ug}{\cu{\micro\gram}} % micrograms
\newcommand{\mg}{\cu{\milli\gram}} % milligrams
\newcommand{\gr}{\cu{\gram}} % grams
\DeclareSIUnit\Unit{U} % enzyme units
\newcommand{\units}{\cu{\Unit}} % units of enzyme

%% CONCENTRATION %%
% Molar
\DeclareSIUnit\Molar{M} % Molar, = 1 mole per litre
\newcommand{\mol}{\cu{\Molar}} % molar
\newcommand{\mmol}{\cu{\milli\Molar}} % millimolar
\newcommand{\umol}{\cu{\micro\Molar}} % micromolar
\newcommand{\nmol}{\cu{\nano\Molar}} % nanomolar
% Mass per volume (e.g. for nucleic acids)
\newcommand{\ngul}{\cu{\nano\gram\per\micro\litre}} % ng/ul
\newcommand{\ngml}{\cu{\nano\gram\per\milli\litre}} % ng/ml
\newcommand{\ugml}{\cu{\micro\gram\per\milli\litre}} % ug/ml
\newcommand{\mgml}{\cu{\milli\gram\per\milli\litre}} % mg/ml
\newcommand{\gl}{\cu{\gram\per\liter}} % g/L
% Units per volume (for enzymes)
\newcommand{\unitsul}{\cu{\Unit\per\micro\litre}} % units/ul, for enzymes
% Percentages
\newcommand{\pc}{\cu{\percent}} % percent concentration (e.g. ethanol)
% Multiples
\DeclareSIUnit\X{\times} % x concentration, e.g. for buffers
\let\x\undefined
\newcommand{\x}{\cu{\X}}

%% TEMPERATURE %%
\newcommand{\degC}[1]{\SI[detect-weight]{#1}{\degreeCelsius}} % degrees celcius
\newcommand{\degrees}[1]{\SI[detect-weight]{#1}{\degree}}

%% PH %%
\DeclareSIUnit{\pH}{pH~}
\newcommand{\ph}[1]{\SI[detect-weight]{#1}[\pH]{}} % pH (as pre-unit)

%% CENTRIFUGE SPEED %%
\DeclareSIUnit{\G}{\mathit{g}} % x standard gravity
\DeclareSIUnit{\RPM}{rpm} % revolutions per minute
\newcommand{\g}[1]{\IfStrEq{#1}{top speed}{top speed}{\SI[detect-weight]{#1}{\G}}}
\newcommand{\rpm}{\cu{\RPM}}

%% BIOLOGICAL SEQUENCE LENGTH %%

\DeclareSIUnit{\B}{b} % Base (for kb etc)
\DeclareSIUnit{\BP}{bp} % Base pairs
\DeclareSIUnit{\NT}{nt} % Nucleotides
\newcommand{\bp}{\cu{\BP}}
\newcommand{\kb}{\cu{\kilo\B}}
\newcommand{\mb}{\cu{\mega\B}}
\newcommand{\nt}{\cu{\NT}}

%% LENGTH %%
\newcommand{\nm}{\cu{\nano\metre}}
\newcommand{\um}{\cu{\micro\metre}}
\newcommand{\mm}{\cu{\milli\metre}}


% ******************************** Autoref settings **********************************
%\renewcommand*{\sectionautorefname}{Section}
%\renewcommand*{\subsectionautorefname}{Section}
%\renewcommand*{\chapterautorefname}{Chapter}
%\newcommand*{\appendixautorefname}{Appendix}
%\newcommand*{\Appendixautorefname}{Appendix}

% ******************************** Misc. commands **********************************

\newcommand{\cm}[1]{C$_\mu$#1}
\newcommand{\cd}[1]{C$_\delta$#1}
\newcommand{\cz}[1]{C$_\zeta$#1}

\newcommand{\question}[1]{\subsection{#1}}
%\newcommand{\question}[1]{\textit{\Large #1}\\}
\newcommand{\q}[1]{\question{#1}}

\renewcommand\thesubfigure{\Alph{subfigure}}
\renewcommand\thesubtable{\Alph{subtable}}

\definecolor{codegreen}{rgb}{0,0.6,0}
\definecolor{codegray}{rgb}{0.5,0.5,0.5}
\definecolor{codepurple}{rgb}{0.58,0,0.82}
\definecolor{backcolour}{rgb}{0.95,0.95,0.92}

\lstset{
    backgroundcolor=\color{backcolour},   
    commentstyle=\color{codegreen},
	basicstyle=\ttfamily,
	columns=fixed,
	%frame=single,
    breaklines=true,
    breakatwhitespace=true,
    postbreak=\mbox{\textcolor{red}{$\hookrightarrow$}\space}, % TODO: Change linebreak character?
    language=bash,
    keepspaces=true,
    aboveskip=0.6em,
    belowskip=0.3em,
    }

% Quick formatting commands
\newcommand{\gene}[1]{\textit{#1}}
\newcommand{\igh}[1]{\gene{IGH{#1}}}

% Species names (with and without genus abbreviation)
\newcommand{\species}[2]{\textit{\capitalisewords{#1} #2}\xspace}
\newcommand{\nfu}{\species{Nothobranchius}{furzeri}}
\newcommand{\Nfu}{\species{N.}{furzeri}}
\newcommand{\xma}{\species{Xiphophorus}{maculatus}}
\newcommand{\Xma}{\species{X.}{maculatus}}


% Program/package names and code snippets
\newcommand{\program}[2][bash]{\lstinline[language=#1]{#2}} %[]
\newcommand{\snippet}[2][bash]{\lstinline[language=#1]{#2}} %[]

% IGH locus components
\let\dh\undefined
\newcommand{\vh}{VH\xspace}
\newcommand{\dh}{DH\xspace}
\newcommand{\jh}{JH\xspace}
\newcommand{\ch}{CH\xspace}

% File formats
\newcommand{\fmt}[1]{\texttt{\MakeUppercase{#1}}}

% Sequence
\newcommand{\sequence}[1]{\texttt{\MakeUppercase{#1}}}

% Ig-Seq
\newcommand{\igseq}{IgSeq\xspace}
\newcommand{\Igseq}{immunoglobulin sequencing\xspace}
\newcommand{\IGSEQ}{Immunoglobulin sequencing\xspace}
\newcommand{\naive}{na\"{i}ve\xspace}

%% Start of subappendices environment
%\AtBeginEnvironment{subappendices}{%
%\chapter*{Appendices}
%\addcontentsline{toc}{chapter}{Appendices}
%\counterwithin{figure}{section}
%\counterwithin{table}{section}
%}
%
%% End of subappendices environment
%\AtEndEnvironment{subappendices}{%
%\counterwithout{figure}{section}
%\counterwithout{table}{section}
%}

\makeatletter
\newcommand\notsotiny{\@setfontsize\notsotiny\@viipt\@viiipt}
\newcommand\evenlesstiny{\@setfontsize\notsotiny\@viiipt\@ixpt}
\makeatother

\newcommand{\subsubsubsection}[1]{\noindent\textit{#1}\noindent}
