% ************************** Thesis Abstract *****************************
{
\cleardoublepage
\setsinglecolumn
\chapter*{\centering \LARGE Abstract}
\thispagestyle{empty}

Ageing individuals exhibit a pervasive decline in adaptive immune function, with important implications for health and lifespan. Systemic changes observed in the structure and diversity of antibody repertoires with age are thought to play an important role in this immunosenescent phenotype; however, the relatively long lifespan of most vertebrate model organisms makes thorough investigation of the ageing repertoire challenging. As a naturally short-lived vertebrate, the turquoise killifish (\nfu) offers an exciting new opportunity to study the ageing of the adaptive immune system in general and antibody repertoires in particular.

In this thesis, I used a combination of existing genomic assemblies and new sequencing data to assemble and characterise the immunoglobulin heavy chain (\igh{}) locus sequence in the turquoise killifish and compare it to those of closely related species, revealing a history of dynamic locus evolution and repeated duplication and loss of the specialised mucosal isotype \igh{Z}. The \Nfu locus itself lacks \igh{Z}, making it one of the few known teleost species not to possess this isotype. These results support a high rate of evolution in teleost \igh{} loci and set a strong foundation for the study of comparative evolutionary immunology in cyprionodontiform fishes.

Having characterised the \igh{} locus sequence in \Nfu, I used it to establish targeted immunoglobulin sequencing in this species, enabling quantitative interrogation of the antibody repertoire. Applying this protocol to whole-body killifish samples revealed complex and individualised antibody repertoires which decline rapidly in within-individual diversity and increase in between-individual variability with age, demonstrating that turquoise killifish exhibit a rapid repertoire-ageing phenotype in line with their short lifespans. This loss of diversity with age was particularly strong in isolated gut samples, a phenomenon that may be related to the constant strong antigenic exposure experienced at mucosal surfaces and has not been previously investigated in a vertebrate model. Taken together, these results establish the turquoise killifish as a novel model for vertebrate immunosenescence and lay the groundwork for future interrogation of -- and intervention in -- adaptive-immune ageing.
}
