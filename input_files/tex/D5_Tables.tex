\chapter{Supplementary tables}
\label{app:tables}

\begin{table}
\caption{Software versions used in computational analyses}
\label{tab:software-versions}
\centering
\begin{tabular}{ll}
  \toprule Program & Version \\ 
  \midrule Basemount & 0.15.96.2154 \\ 
  BLAST & 2.7.1 \\ 
  Bowtie 2 & 2.2.6 \\ 
  CD-HIT-EST & 4.6.8 \\ 
  Change-O & 0.4.5 \\ 
  EMBOSS (FUZZNUC) & 6.6.0 \\ 
  FigTree & 1.4.2 \\ 
  HMMER & 3.2 \\ 
  IgBLAST & 1.7.0 \\ 
  IGoR & 1.3.0 \\ 
  IGV & 2.3.68 \\ 
  IMGT/DomainGapAlign & 4.9.2 \\ 
  PRANK & v.170427 \\ 
  pRESTO & 0.5.10 \\ 
  Primer3 & 2.3.6 \\ 
  Python 2 & 2.7.14 \\ 
  Python 3 & 3.6.4 \\ 
  QuorUM & 1.0.0 \\ 
  R & 3.4.1/3.5.2 \\ 
  RAxML & 8.2.12 \\ 
  RepeatMasker & 4.0.6 \\ 
  SAMtools & 1.9 \\ 
  sed & 4.2.2 \\ 
  seqtk & 1.3 \\ 
  Snakemake & 5.3.0 \\ 
  SPAdes & 3.6.1 \\ 
  SSPACE & 3.0 \\ 
  STAR & 2.5.2b \\ 
  Trimmomatic & 0.32 \\ 
  VSEARCH & 2.8.0 \\ 
   \bottomrule 
\end{tabular}
\end{table}

\begin{table}
\caption{RNA-sequencing datasets used for \textit{IGH} locus characterisation}
\centering
\begin{threeparttable}
\begin{tabular}{>{\bfseries}c|c|c}\toprule
Species & \Nfu & \Xma \\\midrule
Tissues & Gut & Various\tnote{a}\\\midrule
BioProject Accession & PRJNA379208 & PRJNA420092\\\midrule
\multirow{26}{*}{SRA Run Accessions} & SRR5344350 & SRR6327069\\
& SRR5344343 & SRR6327070\\
& SRR5344344 & SRR6327071\\
& SRR5344345 & SRR6327072\\
& SRR5344346 & SRR6327073\\
& SRR5344347 & SRR6327074\\
& SRR5344348 & SRR6327075\\
& SRR5344349 & SRR6327076\\
& SRR5344350 & SRR6327077\\
&&SRR6327078\\
&&SRR6327079\\
&&SRR6327080\\
&&SRR6327081\\
&&SRR6327082\\
&&SRR6327083\\
&&SRR6327084\\
&&SRR6327085\\
&&SRR6327086\\
&&SRR6327087\\
&&SRR6327088\\
&&SRR6327089\\
&&SRR6327090\\
&&SRR6327091\\
&&SRR6327092\\
&&SRR6327093\\
&&SRR6327094\\\midrule
Source & \parencite{smith2017microbiota} & Citation not given\\
\bottomrule\end{tabular}
	\begin{tablenotes}
	\item[a] Tissues used for \Xma RNA-sequencing included brain, heart, liver, gut, skin or whole fish; see BioProject entry for details.
	\end{tablenotes}
\end{threeparttable}
\label{tab:rnaseq-sources}
\end{table}

% 3 - LOCUS CHAPTER
\centering

\begin{table}\centering
    \caption{Co-ordinate table of constant-region exons in the \Nfu \igh{} locus}
    	\input{output_files/tables/locus-nfu-ch-coords}
    \label{tab:nfu-ch-coords}
\end{table}

    \begin{landscape}
        \centering
        \vspace*{\fill}
        \scriptsize
		\input{output_files/tables/locus-nfu-vh-coords}
		\normalsize\vspace{1em}
        \captionof{table}{Co-ordinate table of \vh segments in the \Nfu \igh{} locus}
        \label{tab:nfu-vh-coords}
        \vspace*{\fill}
    \end{landscape}

        {\centering
        \captionof{table}{Co-ordinate table of \dh segments in the \Nfu \igh{} locus}\vspace{-0.3em}
        \label{tab:nfu-dh-coords-seg}
        \scriptsize
		\input{output_files/tables/locus-nfu-dh-coords-seg}
		\normalsize\vspace{1em}
        \captionof{table}{Co-ordinate table of \dh 5'-RSSs in the \Nfu \igh{} locus}\vspace{-0.3em}
        \label{tab:nfu-dh-coords-rss5}
        \scriptsize
        	\input{output_files/tables/locus-nfu-dh-coords-rss5}
        	\normalsize\vspace{1em}
        \captionof{table}{Co-ordinate table of \dh 3'-RSSs in the \Nfu \igh{} locus}\vspace{-0.3em}
        \label{tab:nfu-dh-coords-rss3}
        \scriptsize
		\input{output_files/tables/locus-nfu-dh-coords-rss3}
		\normalsize
		}

    \begin{landscape}
        \centering
        \notsotiny
		\input{output_files/tables/locus-nfu-jh-coords-seg}
		\normalsize\vspace{0.6em}
        \captionof{table}{Co-ordinate table of \jh segments in the \Nfu \igh{} locus}
        \label{tab:nfu-jh-coords-seg}
        \notsotiny
		\input{output_files/tables/locus-nfu-jh-coords-rss}
		\normalsize\vspace{0.6em}
        \captionof{table}{Co-ordinate table of \jh RSSs in the \Nfu \igh{} locus}
        \label{tab:nfu-jh-coords-rss}
    \end{landscape}

\begin{table}\centering
    \caption{Co-ordinate table of constant-region exons in the \Xma \igh{} locus}
    	\input{output_files/tables/locus-xma-ch-coords}
    \label{tab:xma-ch-coords}
\end{table}

    \begin{landscape}
        \centering
        \vspace*{\fill}
        \scriptsize
		\input{output_files/tables/locus-xma-vh-coords-1}
		\normalsize\vspace{1em}
        \captionof{table}{Co-ordinate table of \vh segments in the \Xma \igh{} locus, part 1}
        \label{tab:xma-vh-coords-1}
        \vspace*{\fill}
    \end{landscape}

    \begin{landscape}
        \centering
        \vspace*{\fill}
        \scriptsize
		\input{output_files/tables/locus-xma-vh-coords-2}
		\normalsize\vspace{1em}
        \captionof{table}{Co-ordinate table of \vh segments in the \Xma \igh{} locus, part 2}
        \label{tab:xma-vh-coords-2}
        \vspace*{\fill}
    \end{landscape}

    \begin{landscape}
        \centering
        \vspace*{\fill}
        \scriptsize
		\input{output_files/tables/locus-xma-vh-coords-3}
		\normalsize\vspace{1em}
        \captionof{table}{Co-ordinate table of \vh segments in the \Xma \igh{} locus, part 3}
        \label{tab:xma-vh-coords-3}
        \vspace*{\fill}
    \end{landscape}

    \begin{landscape}
        \centering
        \vspace*{\fill}
        \scriptsize
		\input{output_files/tables/locus-xma-vh-coords-4}
		\normalsize\vspace{1em}
        \captionof{table}{Co-ordinate table of \vh segments in the \Xma \igh{} locus, part 4}
        \label{tab:xma-vh-coords-4}
        \vspace*{\fill}
    \end{landscape}

    \begin{landscape}
        \centering
        \vspace*{\fill}
        \scriptsize
		\input{output_files/tables/locus-xma-vh-coords-5_edited}
		\normalsize\vspace{1em}
        \captionof{table}{Co-ordinate table of \vh segments in the \Xma \igh{} locus, part 5}
        \label{tab:xma-vh-coords-5}
        \vspace*{\fill}
    \end{landscape}

        {\centering
        \captionof{table}{Co-ordinate table of \dh segments in the \Xma \igh{} locus}\vspace{-0.3em}
        \label{tab:xma-dh-coords-rss3}
        \scriptsize
		\input{output_files/tables/locus-xma-dh-coords-seg}
		\normalsize\vspace{1em}
        \captionof{table}{Co-ordinate table of \dh 5'-RSSs in the \Xma \igh{} locus}\vspace{-0.3em}
        \label{tab:xma-dh-coords-seg}
        \scriptsize
        	\input{output_files/tables/locus-xma-dh-coords-rss5}
        	\normalsize\vspace{1em}
        \captionof{table}{Co-ordinate table of \dh 3'-RSSs in the \Xma \igh{} locus}\vspace{-0.3em}
        \label{tab:xma-dh-coords-rss5}
        \scriptsize
		\input{output_files/tables/locus-xma-dh-coords-rss3}
		\normalsize
}

    \begin{landscape}
        \centering
        \notsotiny
		\input{output_files/tables/locus-xma-jh-coords-seg}
		\normalsize\vspace{0.6em}
        \captionof{table}{Co-ordinate table of \jh segments in the \Xma \igh{} locus}
        \label{tab:xma-jh-coords-seg}
        \notsotiny
		\input{output_files/tables/locus-xma-jh-coords-rss}
		\normalsize\vspace{0.6em}
        \captionof{table}{Co-ordinate table of \jh RSSs in the \Xma \igh{} locus}
        \label{tab:xma-dh-coords-rss}
    \end{landscape}

	\begin{landscape}
	\centering
	\vspace*{\fill}
    \scriptsize
    \begin{threeparttable}
\begin{tabular}{>{\itshape}lllllllp{4cm}}
  \toprule \textnormal{\textbf{Species}} & \textbf{Scaffold(s)} & \textbf{Region} & \textbf{Isotype} & \textbf{Known Exons} \tnote{1} & \textbf{Complete?} & \textbf{Pseudo-exons} & \textbf{Comments} \\ 
  \midrule Nothobranchius orthonotus & scf33878 & IGHM1 & M & 1,2,3,TM1 & \textbf{No} & -- & CM4 missing (missing sequence) \\ 
  Nothobranchius orthonotus & scf33878 & IGHD1 & D & 1,2,3,4,2,3,4,5,6,7,TM1 & Yes & -- &  \\ 
  Nothobranchius orthonotus & scf34438 & IGHM2 & M & 1,2,3,4,TM1 & Yes & -- &  \\ 
  Nothobranchius orthonotus & scf34438, scf33917 & IGHD2 & D & 1,2,3,4,2,3,4,5,6,7,TM1 & Yes & -- &  \\ 
  Nothobranchius orthonotus & scf33917 & IGHD3 & D & 1,2,3,4,2,3,4,5,6,7,TM1 & Yes & -- &  \\ 
  Nothobranchius orthonotus & scf33917 & IGHD4 & D & 1,2,3,4,2,3,4,5,6,7,TM1 & Yes & -- &  \\ 
  Nothobranchius orthonotus & scf9255, scf26119, scf33917 & IGHD5 & D & 3,4,2,3,4,5,6,7,TM1 & \textbf{No} & -- & CD1 \& CD2A missing (missing sequence) \\ 
  Nothobranchius orthonotus & scf27951, scf33789 & IGHM3 & M & 1,2,3,4,TM1 & Yes & -- &  \\ 
  Nothobranchius orthonotus & scf27951, 32033 & IGHD6 & D & 1,2,3,4,2,3,4,5,6,7,TM1 & Yes & -- &  \\ 
  Nothobranchius orthonotus & scf32137, scf21286 & IGHM4 & M & 1,2,3,4,TM1 & Yes & -- &  \\ 
  Nothobranchius furzeri & chr6 \tnote{2} & IGH1M & M & 1,2,3,4,TM1 & Yes & -- &  \\ 
  Nothobranchius furzeri & chr6 \tnote{2} & IGH1D & D & 1,2,3,4,2,3,4,5,6,7,TM1 & Yes & -- &  \\ 
  Nothobranchius furzeri & chr6 \tnote{2} & IGH2M & M & 1,2,3,4,TM1 & Yes & -- &  \\ 
  Nothobranchius furzeri & chr6 \tnote{2} & IGH2D & D & 1,2,3,4,2,3,4,5,6,7,TM1 & Yes & -- &  \\ 
  Aphyosemion australe & scf373 & IGHM & M & 1,2,3,4,TM1 & Yes & -- &  \\ 
  Aphyosemion australe & scf373 & IGHD & D & 1,2,3,4,5,6,7,TM1 & Yes & -- &  \\ 
  Callopanchax toddi & scf107 & IGHZ1 & Z & 1,2,3,4,TM1 & Yes & -- &  \\ 
  Callopanchax toddi & scf107 & IGHZ2 & Z & 1,2,3,4,TM1 & Yes & -- &  \\ 
  Callopanchax toddi & scf1209 & IGHZ3 & Z & 1,2,3,4,TM1 & Yes & -- &  \\ 
  Callopanchax toddi & scf1209 & IGHM1 & M & 1 & \textbf{No} & -- & Isolated CM1 exon \\ 
  Callopanchax toddi & scf945 & IGHZ4 & Z & 1,2,3,4,TM1 & Yes & -- &  \\ 
  Callopanchax toddi & scf945 & IGHM2 & M & 1,2,3,4,TM1 & Yes & -- &  \\ 
  Callopanchax toddi & scf945 & IGHD1 & D & 1,2,3,4,5,6,7,TM1 & Yes & 1,4,5 & Frameshift mutations in CD1, CD4 \& CD5 \\ 
  Callopanchax toddi & scf265 & IGHM3 & M & 1,2,3,4,TM1 & Yes & -- &  \\ 
  Callopanchax toddi & scf265 & IGHD2 & D & 1,5,7,TM1 & \textbf{No} & -- & CD2-4 \& CD5-6 missing (not in sequence) \\ 
   \bottomrule \end{tabular}
	\begin{tablenotes}
	\item[1] Excluding TM2 and secretory exons.
	\item[2] Expanded \igh{} locus sequence from \Cref{sec:nfu-locus}.
	\end{tablenotes}
	\end{threeparttable}
	\normalsize\vspace{1em}
    \captionof{table}{\igh{} constant regions in cyprinidontiform fish, part 1}
	\label{tab:multispecies-ch-regions-1}
    \vspace*{\fill}
    \end{landscape}

	\begin{landscape}
	\centering
	\vspace*{\fill}
    \scriptsize
    \begin{threeparttable}
\begin{tabular}{>{\itshape}lllllllp{4cm}}
  \toprule \textnormal{\textbf{Species}} & \textbf{Scaffold(s)} & \textbf{Region} & \textbf{Isotype} & \textbf{Known Exons} \tnote{1} & \textbf{Complete?} & \textbf{Pseudo-exons} & \textbf{Comments} \\ 
  \midrule Pachypanchax playfairii & scf547 & IGHZ & Z & 1,2,3,4,TM1 & Yes & -- &  \\ 
  Pachypanchax playfairii & scf125 & IGHM1 & M & 1,2,3,4,TM1 & Yes & -- &  \\ 
  Pachypanchax playfairii & scf125 & IGHD & D & 1,2,3,4,5,6,7,TM1 & Yes & -- &  \\ 
  Pachypanchax playfairii & scf547 & IGHM2 & M & 1 & \textbf{No} & -- & Isolated CM1 exon \\ 
  Austrofundulus limnaeus & NW\_013954375.1 & IGHZ & Z & TM1 & \textbf{No} & TM1 & Isolated TM1 exon with frameshift mutation \\ 
  Austrofundulus limnaeus & NW\_013952673.1 & IGHM & M & 1,2,3,4,TM1 & Yes & -- &  \\ 
  Austrofundulus limnaeus & NW\_013952673.1, NW\_013956335.1 & IGHD & D & 1,2,3,4,5,6,7,TM1 & Yes & -- &  \\ 
  Kryptolebias marmoratus & NW\_016094348.1 & IGHZ1 & Z & 1,2,3,4,TM1 & Yes & -- &  \\ 
  Kryptolebias marmoratus & NW\_016094348.1 & IGHZ2 & Z & 1,4,TM1 & \textbf{No} & -- & CZ2 \& CZ3 missing (not in sequence) \\ 
  Kryptolebias marmoratus & NW\_016094301.1 & IGHM1 & M & 1,2,3,4,TM1 & Yes & -- &  \\ 
  Kryptolebias marmoratus & NW\_016094301.1 & IGHD1 & D & 1,2,3,4,5,6,7,TM1 & Yes & -- &  \\ 
  Kryptolebias marmoratus & NW\_016094277.1 & IGHM2 & M & 1,2,3,4,TM1 & Yes & -- &  \\ 
  Kryptolebias marmoratus & NW\_016094277.1 & IGHD2 & D & 1,2,3,4,5,6,TM1 & \textbf{No} & -- & CD7 missing (not in sequence) \\ 
  Poecilia reticulata & NC\_024338.1 & IGHZ1 & Z & 1,2,3,4 & \textbf{No} & -- & TM1 missing (missing sequence) \\ 
  Poecilia reticulata & NC\_024338.1 & IGHZ2 & Z & 1,2,3,4,TM1 & Yes & -- &  \\ 
  Poecilia reticulata & NC\_024338.1 & IGHM & M & 1,2,3,4,TM1 & Yes & -- &  \\ 
  Poecilia reticulata & NC\_024338.1 & IGHD & D & 1,2,3,4,2,3,4,5,6,7,TM1 & Yes & -- &  \\ 
  Poecilia formosa & NW\_006800081.1 & IGHZ1 & Z & 1,2,3,4,TM1 & Yes & -- &  \\ 
  Poecilia formosa & NW\_006800081.1 & IGHZ2 & Z & 1,2,3,4,TM1 & Yes & -- &  \\ 
  Poecilia formosa & NW\_006800081.1 & IGHZ3 & Z & 1,2,3,4,TM1 & Yes & -- &  \\ 
  Poecilia formosa & NW\_006800081.1 & IGHM & M & 1,2,3,4,TM1 & Yes & -- &  \\ 
  Poecilia formosa & NW\_006800081.1 & IGHD & D & 1,2,3,4,5,6,7,TM1 & Yes & -- &  \\ 
  Xiphophorus maculatus & NC\_036458 & IGHZ1 & Z & 1,2,3,4,TM1 & Yes & -- &  \\ 
  Xiphophorus maculatus & NC\_036458 & IGHZ2 & Z & 1,2,3,4,TM1 & Yes & -- &  \\ 
  Xiphophorus maculatus & NC\_036458 & IGHM & M & 1,2,3,4,TM1 & Yes & -- &  \\ 
   \bottomrule \end{tabular}
	\begin{tablenotes}
	\item[1] Excluding TM2 and secretory exons.
	\end{tablenotes}
	\end{threeparttable}
	\normalsize\vspace{1em}
    \captionof{table}{\igh{} constant regions in cyprinidontiform fish, part 2}
	\label{tab:multispecies-ch-regions-2}
    \vspace*{\fill}
    \end{landscape}

	\begin{landscape}
	\centering
	\vspace*{\fill}
    \scriptsize
    \begin{threeparttable}
\begin{tabular}{>{\itshape}lllllllp{4cm}}
  \toprule \textnormal{\textbf{Species}} & \textbf{Scaffold(s)} & \textbf{Region} & \textbf{Isotype} & \textbf{Known Exons} \tnote{1} & \textbf{Complete?} & \textbf{Pseudo-exons} & \textbf{Comments} \\ 
  \midrule Xiphophorus maculatus & NC\_036458 & IGHD & D & 1,2,3,4,2,3,4,5,6,7,TM1 & Yes & -- &  \\ 
  Fundulus heteroclitus & NW\_012234561.1 & IGHZ1 & Z & 1,2,3,4,TM1 & Yes & -- &  \\ 
  Fundulus heteroclitus & NW\_012230737.1 & IGHZ2 & Z & 4,TM1 & \textbf{No} & -- & CZ1 to CZ3 missing (missing sequence) \\ 
  Fundulus heteroclitus & NW\_012234542.1 & IGHM & M & 1,2,3,4,TM1 & Yes & -- &  \\ 
  Fundulus heteroclitus & NW\_012234542.1 & IGHD & D & 1,2,3,4,2,3,4,5,6,7,TM1 & Yes & -- &  \\ 
  Cyprinodon variegatus & NW\_015154250.1, NW\_015151047.1 & IGHZ & Z & 1,2,3,4,TM1 & Yes & -- &  \\ 
  Cyprinodon variegatus & NW\_015151047.1 & IGHM & M & 1,2,3,4,TM1 & Yes & -- &  \\ 
  Cyprinodon variegatus & NW\_015151047.1 & IGHD & D & 1,2,3,4,2,3,4,5,6,7,TM1 & Yes & -- &  \\ 
  Oryzias latipes & NC\_019866.2 & IGHM1 & M & 1,2,3,4,TM1 & Yes & -- &  \\ 
  Oryzias latipes & NC\_019866.2 & IGHD1 & D & 1,2,3,4,6,7,TM1 & Yes & 7 & Nonsense mutation in CD7 \\ 
  Oryzias latipes & NC\_019866.2 & IGHM2 & M & 1,2,3,4,TM1 & Yes & -- &  \\ 
  Oryzias latipes & NC\_019866.2 & IGHD2 & D & 1,2,3,4,6,7,TM1 & Yes & -- &  \\ 
  Oryzias latipes & NC\_019866.2 & IGHM3 & M & 1,2,3,4,TM1 & Yes & -- &  \\ 
  Oryzias latipes & NC\_019866.2 & IGHD3 & D & 1,2,3,4,6,7,TM1 & Yes & -- &  \\ 
  Oryzias latipes & NC\_019866.2 & IGHM4 & M & 1,2,3,4,TM1 & Yes & -- &  \\ 
  Oryzias latipes & NC\_019866.2 & IGHD4 & D & 2,7,TM1 & \textbf{No} & -- & CD1 \& CD3-6 missing (not in sequence) \\ 
  Oryzias latipes & NC\_019866.2 & IGHM5 & M & 1,2,3,4,TM1 & Yes & -- &  \\ 
  Oryzias latipes & NC\_019866.2 & IGHD5 & D & 1,2,3,4,6,7,TM1 & Yes & -- &  \\ 
  Oryzias latipes & NC\_019866.2 & IGHM6 & M & 1,2,3,4,TM1 & Yes & -- &  \\ 
  Oryzias latipes & NC\_019866.2 & IGHD6 & D & 1,2,3,4,6,7,TM1 & Yes & -- &  \\ 
  Oryzias latipes & NC\_019866.2 & IGHD7 & D & 1,2,3,6 & \textbf{No} & -- & CD4, CD5, CD7 and TM1 missing (not in sequence) \\ 
   \bottomrule \end{tabular}
	\begin{tablenotes}
	\item[1] Excluding TM2 and secretory exons.
	\end{tablenotes}
	\end{threeparttable}
	\normalsize\vspace{1em}
    \captionof{table}{\igh{} constant regions in cyprinidontiform fish, part 3}
	\label{tab:multispecies-ch-regions-3}
    \vspace*{\fill}
    \end{landscape}

% 4 - IGSEQ CHAPTER

\begin{table}
\caption[Turquoise killifish individuals used in \igseq pilot and ageing experiments]{Turquoise killifish used in \igseq pilot and ageing experiments. All fish are GRZ-AD strain and male.}
\label{tab:igseq-cohorts-fish}
\begin{threeparttable}
\input{output_files/tables/igseq-cohorts-fish_edited}
\begin{tablenotes}
\item[1] grz-AD\_...\_E
\end{tablenotes}
\end{threeparttable}
\end{table}

\begin{landscape}
\centering
\vspace*{\fill}
\scriptsize
\input{output_files/tables/igseq-gut-cohorts-fish}
\normalsize\vspace{1em}
\captionof{table}[Turquoise killifish individuals used in \igseq gut experiment]{Turquoise killifish used in \igseq gut experiment. All fish are GRZ-Bellemans strain and male.}
\label{tab:gut-cohorts-fish}
\vspace*{\fill}
\end{landscape}
