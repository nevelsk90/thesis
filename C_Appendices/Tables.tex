\chapter{Supplementary tables}
\label{app:tables}

% 2 - METHODS CHAPTER

\begin{table}
\caption{RNA-sequencing datasets used for \textit{IGH} constant-region exon refinement and isoform identification}
\centering
\begin{threeparttable}
\begin{tabular}{>{\bfseries}c|c|c}\toprule
Species & \Nfu & \Xma \\\midrule
Tissues & Gut & Various\tnote{2}\\\midrule
BioProject Accession & PRJNA379208 & PRJNA420092\\\midrule
\multirow{26}{*}{SRA Run Accessions} & SRR5344350 & SRR6327069\\
& SRR5344343 & SRR6327070\\
& SRR5344344 & SRR6327071\\
& SRR5344345 & SRR6327072\\
& SRR5344346 & SRR6327073\\
& SRR5344347 & SRR6327074\\
& SRR5344348 & SRR6327075\\
& SRR5344349 & SRR6327076\\
& SRR5344350 & SRR6327077\\
&&SRR6327078\\
&&SRR6327079\\
&&SRR6327080\\
&&SRR6327081\\
&&SRR6327082\\
&&SRR6327083\\
&&SRR6327084\\
&&SRR6327085\\
&&SRR6327086\\
&&SRR6327087\\
&&SRR6327088\\
&&SRR6327089\\
&&SRR6327090\\
&&SRR6327091\\
&&SRR6327092\\
&&SRR6327093\\
&&SRR6327094\\\midrule
Source & \parencite{smith2017microbiota} & Citation not given\\
\bottomrule\end{tabular} % TODO: Table notes, sources
	\begin{tablenotes}
	\item[1] Tissues used for \Xma RNA-sequencing included brain, heart, liver, gut, skin or whole fish; see BioProject entry for details.
	\end{tablenotes}
\end{threeparttable}
\label{tab:rnaseq-sources}
\end{table}

% 3 - LOCUS CHAPTER
\centering

\begin{table}\centering
    \caption{Co-ordinate table of constant-region exons in the \nfu \igh{} locus.}
    	% latex table generated in R 3.5.2 by xtable 1.8-3 package
% Fri Jan  4 11:18:29 2019
\begin{tabular}{llrrrl}
  \toprule Name & Isotype & Start & End & Length & Strand \\ 
  \midrule IGH1M-1 & M & 130848 & 131144 & 297 & + \\ 
  IGH1M-2 & M & 131971 & 132312 & 342 & + \\ 
  IGH1M-3 & M & 132394 & 132705 & 312 & + \\ 
  IGH1M-4 & M & 132816 & 133288 & 473 & + \\ 
  IGH1M-TM1 & M & 134262 & 134413 & 152 & + \\ 
  IGH1M-TM2 & M & 138431 & 138819 & 389 & + \\ 
  IGH1D-1 & D & 139381 & 139689 & 309 & + \\ 
  IGH1D-2A & D & 139774 & 140064 & 291 & + \\ 
  IGH1D-3A & D & 140178 & 140489 & 312 & + \\ 
  IGH1D-4A & D & 140572 & 140853 & 282 & + \\ 
  IGH1D-2B & D & 145613 & 145909 & 297 & + \\ 
  IGH1D-3B & D & 146000 & 146311 & 312 & + \\ 
  IGH1D-4B & D & 146398 & 146676 & 279 & + \\ 
  IGH1D-5 & D & 146795 & 147124 & 330 & + \\ 
  IGH1D-6 & D & 147210 & 147527 & 318 & + \\ 
  IGH1D-7 & D & 147598 & 147885 & 288 & + \\ 
  IGH1D-TM1 & D & 148016 & 148164 & 149 & + \\ 
  IGH1D-TM2 & D & 148323 & 148504 & 182 & + \\ 
  IGH2D-TM2 & D & 187624 & 187803 & 180 & - \\ 
  IGH2D-TM1 & D & 187963 & 188111 & 149 & - \\ 
  IGH2D-7 & D & 188658 & 188945 & 288 & - \\ 
  IGH2D-6 & D & 189016 & 189333 & 318 & - \\ 
  IGH2D-5 & D & 189419 & 189748 & 330 & - \\ 
  IGH2D-4B & D & 189867 & 190145 & 279 & - \\ 
  IGH2D-3B & D & 190232 & 190543 & 312 & - \\ 
  IGH2D-2B & D & 190636 & 190932 & 297 & - \\ 
  IGH2D-4A & D & 195644 & 195925 & 282 & - \\ 
  IGH2D-3A & D & 196008 & 196319 & 312 & - \\ 
  IGH2D-2A & D & 196433 & 196723 & 291 & - \\ 
  IGH2D-1 & D & 196808 & 197116 & 309 & - \\ 
  IGH2M-TM2 & M & 198315 & 198506 & 192 & - \\ 
  IGH2M-TM1 & M & 199834 & 199985 & 152 & - \\ 
  IGH2M-4 & M & 200953 & 201425 & 473 & - \\ 
  IGH2M-3 & M & 201536 & 201847 & 312 & - \\ 
  IGH2M-2 & M & 201929 & 202270 & 342 & - \\ 
  IGH2M-1 & M & 203549 & 203845 & 297 & - \\ 
   \bottomrule \end{tabular}

    \label{tab:nfu-ch-coords}
\end{table}

    \begin{landscape}
        \centering
        \vspace*{\fill}
        \scriptsize
		% latex table generated in R 3.5.2 by xtable 1.8-3 package
% Tue Jan 15 17:07:52 2019
\begin{tabular}{lrrrlrlrlrrl}
  \toprule Name & Start & End & Length & Strand & RSS Start & Heptamer & Spacer Length & Nonamer & RSS End & RSS Length & Comment \\ 
  \midrule IGH1V1-01 & 1252 & 1540 & 289 & + & 1541 & CACAGTG & 22 & ACAAAAACC & 1578 & 38 &  \\ 
  IGH1V1-02 & 3365 & 3656 & 292 & + & 3657 & CACAGTG & 22 & ACAAAAACC & 3694 & 38 &  \\ 
  IGH1V2-01 & 5907 & 6201 & 295 & + & 6202 & CACAGAA & 15 & ACAAAAACT & 6232 & 31 &  \\ 
  IGH1V1-03 & 13690 & 13964 & 275 & + & 13965 & CACAGTG & 22 & ACAAAAACC & 14002 & 38 &  \\ 
  IGH1V3-01 & 14862 & 15162 & 301 & + & 15163 & CACAGTG & 23 & ACAAAAACC & 15201 & 39 &  \\ 
  IGH1V2-02 & 17433 & 17730 & 298 & + & 17731 & CACAATG & 23 & ACAAAAACC & 17769 & 39 &  \\ 
  IGH1V4-01p & 24566 & 24837 & 272 & + & 24838 & CGCAGTG & 22 & CCACAAACC & 24875 & 38 & Nonsense mutation \\ 
  IGH1V1-04 & 37305 & 37596 & 292 & + & 37597 & CACAGTG & 22 & ACAAAAACC & 37634 & 38 &  \\ 
  IGH1V2-03 & 48845 & 49139 & 295 & + & 49140 & CACAGTG & 23 & TCAAAAACT & 49178 & 39 &  \\ 
  IGH1V1-05 & 49909 & 50197 & 289 & + & 50198 & CACAGTG & 22 & ACAAAAACC & 50235 & 38 &  \\ 
  IGH1V5-01 & 51710 & 51998 & 289 & + & 51999 & CACAGTG & 22 & ACAAAAACT & 52036 & 38 &  \\ 
  IGH1V2-04 & 56322 & 56616 & 295 & + & 56617 & CACAGTG & 23 & ACAAAAACC & 56655 & 39 &  \\ 
  IGH1V6-01 & 57465 & 57762 & 298 & + & 57763 & CACAGTG & 21 & ACTAAATCT & 57799 & 37 &  \\ 
  IGH1V1-06 & 59678 & 59966 & 289 & + & 59967 & CACAGTG & 22 & ACAAAAACC & 60004 & 38 &  \\ 
  IGH1V4-02p & 68017 & 68288 & 272 & + & 68289 & TGCAGTG & 22 & TCACAAACC & 68326 & 38 & Nonsense mutation \\ 
  IGH1V2-05 & 69787 & 70084 & 298 & + & 70085 & CACAGTG & 23 & ACAAAAACC & 70123 & 39 &  \\ 
  IGH1V1-07 & 155485 & 155763 & 279 & + & 155764 & CACAGTG & 22 & TCAAAACCC & 155801 & 38 &  \\ 
  IGH2V2-02 & 282620 & 282914 & 295 & - & 282915 & CACAGTG & 23 & ACAAAAACC & 282953 & 39 &  \\ 
  IGH2V4-01p & 284404 & 284675 & 272 & - & 284676 & TGCAGTG & 22 & TCACAAACC & 284713 & 38 & Nonsense mutation \\ 
  IGH2V5-01 & 288808 & 289096 & 289 & - & 289097 & CACAGTG & 22 & ACAGAAACT & 289134 & 38 &  \\ 
  IGH2V1-03 & 289977 & 290271 & 295 & - & 290272 & CACAGTG & 22 & ACAAAAACC & 290309 & 38 &  \\ 
  IGH2V1-02 & 293835 & 294126 & 292 & - & 294127 & CACAGTG & 22 & ACAAAAACC & 294164 & 38 &  \\ 
  IGH2V2-01 & 303780 & 304074 & 295 & - & 304075 & CAGGGCC & 24 & AGCACAAAG & 304114 & 40 &  \\ 
  IGH2V1-01 & 304926 & 305204 & 279 & - & 305205 & CACAGTG & 22 & TCAAAACCC & 305242 & 38 &  \\ 
   \bottomrule \end{tabular}

		\normalsize\vspace{1em}
        \captionof{table}{Co-ordinate table of \vh segments in the \nfu \igh{} locus.}
        \label{tab:nfu-vh-coords}
        \vspace*{\fill}
    \end{landscape}

        {\centering
        \captionof{table}{Co-ordinate table of \dh segments in the \nfu \igh{} locus.}\vspace{-0.3em}
        \label{tab:nfu-dh-coords-seg}
        \scriptsize
		% latex table generated in R 3.5.2 by xtable 1.8-3 package
% Tue Jan 15 17:07:52 2019
\begin{tabular}{lrlrrl}
  \toprule Name & Start & NT Sequence & End & Length & Strand \\ 
  \midrule IGH1D01 & 25782 & ATACGTACTTTCGTGGTATATAGAGA & 25807 & 26 & + \\ 
  IGH1D02 & 76700 & GATATCTGGGTGGGGG & 76715 & 16 & + \\ 
  IGH1D03 & 77027 & TGAAATGATTAC & 77038 & 12 & + \\ 
  IGH1D04 & 77476 & TCGCGTAGCGGC & 77487 & 12 & + \\ 
  IGH1D05 & 78717 & GAAACCACGGCAGC & 78730 & 14 & + \\ 
  IGH1D06 & 79049 & TTTATAGCGGCTAC & 79062 & 14 & + \\ 
  IGH1D07 & 80417 & CAGACTGGAGA & 80427 & 11 & + \\ 
  IGH1D08 & 81362 & TTCATGGCAGCCAC & 81375 & 14 & + \\ 
  IGH1D09 & 82067 & CAGACTGGAGC & 82077 & 11 & + \\ 
  IGH1D10 & 84282 & TGGGGTGGCAGC & 84293 & 12 & + \\ 
  IGH2D04 & 263497 & CAGACTGGAGA & 263507 & 11 & - \\ 
  IGH2D03 & 270243 & TTTATAGCGGCTAC & 270256 & 14 & - \\ 
  IGH2D02 & 270878 & GAAACCACGGCAGC & 270891 & 14 & - \\ 
  IGH2D01 & 271749 & GACTTTTACTAC & 271760 & 12 & - \\ 
   \bottomrule \end{tabular}

		\normalsize\vspace{1em}
        \captionof{table}{Co-ordinate table of \dh 5'-RSSs in the \nfu \igh{} locus.}\vspace{-0.3em}
        \label{tab:nfu-dh-coords-rss5}
        \scriptsize
        	% latex table generated in R 3.5.2 by xtable 1.8-3 package
% Fri Jan  4 11:18:28 2019
\begin{tabular}{lrlrlrr}
  \toprule Name & 5'-RSS Start & Nonamer & Spacer Length & Heptamer & 5'-RSS End & Length \\ 
  \midrule IGH1D01 & 25754 & GGTTGTTGT & 12 & CACTGTG & 25781 & 28 \\ 
  IGH1D02 & 76672 & AGTTTTTGA & 12 & CACAGTG & 76699 & 28 \\ 
  IGH1D03 & 76999 & TGTTGTTGT & 12 & CACAGTG & 77026 & 28 \\ 
  IGH1D04 & 77448 & AGTTTTTGT & 12 & CACGGTG & 77475 & 28 \\ 
  IGH1D05 & 78688 & GATGTTTTT & 13 & CACAGTG & 78716 & 29 \\ 
  IGH1D06 & 79021 & TGTTTTTGT & 12 & CGCTGTG & 79048 & 28 \\ 
  IGH1D07 & 80417 & AGTTTTGGT & 12 & CACAGTG & 80444 & 28 \\ 
  IGH1D08 & 81334 & TGTTTTTGT & 12 & CGCTGTG & 81361 & 28 \\ 
  IGH1D09 & 82039 & AGTTTTGGT & 12 & CACAGTG & 82066 & 28 \\ 
  IGH1D10 & 84254 & TCATTCATT & 12 & CACTGTG & 84281 & 28 \\ 
  IGH2D04 & 263497 & AGTTTTGGT & 12 & CACAGTG & 263524 & 28 \\ 
  IGH2D03 & 270215 & TGTTTTTGT & 12 & CGCTGTG & 270242 & 28 \\ 
  IGH2D02 & 270850 & TGTTTTTGT & 12 & CACAGTG & 270877 & 28 \\ 
  IGH2D01 & 271721 & AGTTTTTAT & 12 & CATGGTG & 271748 & 28 \\ 
   \bottomrule \end{tabular}

        	\normalsize\vspace{1em}
        \captionof{table}{Co-ordinate table of \dh 3'-RSSs in the \nfu \igh{} locus.}\vspace{-0.3em}
        \label{tab:nfu-dh-coords-rss3}
        \scriptsize
		% latex table generated in R 3.5.2 by xtable 1.8-3 package
% Tue Jan 15 17:07:52 2019
\begin{tabular}{lrlrlrr}
  \toprule Name & 3'-RSS Start & Heptamer & Spacer Length & Nonamer & 3'-RSS End & Length \\ 
  \midrule IGH1D01 & 25808 & CACAGTG & 12 & ACAAAAACC & 25835 & 28 \\ 
  IGH1D02 & 76716 & CACAGTG & 12 & ACAAAAACC & 76743 & 28 \\ 
  IGH1D03 & 77039 & CACTGTG & 11 & AATATAACC & 77065 & 27 \\ 
  IGH1D04 & 77488 & CACAGCG & 12 & ACATAAAAC & 77515 & 28 \\ 
  IGH1D05 & 78731 & CACAGCG & 12 & ACAAAAGCC & 78758 & 28 \\ 
  IGH1D06 & 79063 & CACTGTG & 12 & ACAAGATCC & 79090 & 28 \\ 
  IGH1D07 & 80428 & CACAACG & 12 & ACAAAAACC & 80455 & 28 \\ 
  IGH1D08 & 81376 & CACTGTG & 12 & ACAAAATCC & 81403 & 28 \\ 
  IGH1D09 & 82078 & CACAATG & 12 & ACAAAAACC & 82105 & 28 \\ 
  IGH1D10 & 84294 & CACAGTG & 12 & ACAAAAACC & 84321 & 28 \\ 
  IGH2D04 & 263508 & CACAACG & 12 & ACAAAAACC & 263535 & 28 \\ 
  IGH2D03 & 270257 & CACTGTG & 12 & ACAAGATCC & 270284 & 28 \\ 
  IGH2D02 & 270892 & CACAGCG & 12 & ACAAAAGCC & 270919 & 28 \\ 
  IGH2D01 & 271761 & CACAATG & 12 & ACAAAAACC & 271788 & 28 \\ 
   \bottomrule \end{tabular}

		\normalsize
		}

    \begin{landscape}
        \centering
        \notsotiny
		% latex table generated in R 3.5.2 by xtable 1.8-3 package
% Tue Jan 15 17:07:52 2019
\begin{tabular}{lrllrrl}
  \toprule Name & Start & NT Sequence & AA Sequence & End & Length & Strand \\ 
  \midrule IGH1J01 & 26187 & GTGCTTTAGACAACTGGGGAAAAGGAACGGAGGTTACTGTTCAACCTG & ALDNWGKGTEVTVQP & 26234 & 48 & + \\ 
  IGH1J02 & 128176 & ATGACTACTTTGACTACTGGGGAAAAGGAACAATGGTGACGGTCACATCAG & DYFDYWGKGTMVTVTS & 128226 & 51 & + \\ 
  IGH1J03 & 128354 & ACCGTGGGGTAAAGGGACAACAGTCACGGTCAAAACAG & PWGKGTTVTVKT & 128391 & 38 & + \\ 
  IGH1J04 & 128533 & ACGGTGCTCTTGACTACTGGGGTAAAGGGACCGCAGTCACTGTAACATCAG & GALDYWGKGTAVTVTS & 128583 & 51 & + \\ 
  IGH1J05 & 128887 & ACAACGCTTTTGACTACTGGGGAAAAGGAACAACGGTCACCGTCACTTCAG & NAFDYWGKGTTVTVTS & 128937 & 51 & + \\ 
  IGH1J06 & 129346 & CTACGATGCTTTTGACTACTGGGGGAAAAGGACGATGGTCACGTCACTTCAG & YDAFDYWGKRTMVTSLQ & 129397 & 52 & + \\ 
  IGH1J07 & 129635 & TTAACTGGGCTTTCGACTACTGGGGAAAAGGGACGATGGTAACGGTGACTTCAG & NWAFDYWGKGTMVTVTS & 129688 & 54 & + \\ 
  IGH1J08 & 129965 & TTACCACGCAGCTTTGGACTACTGGGGAAAAGGGACGACGGTCACCGTCACCTCAG & YHXALDYWGKGTTVTVTS & 130020 & 56 & + \\ 
  IGH1J09 & 130612 & TCTACGCTGCTTTTGACTACTGGGGTAAAGGTACAACGGTAACCGTTTCATCAG & YAAFDYWGKGTTVTVSS & 130665 & 54 & + \\ 
  IGH2J08 & 204031 & TCTACGCTGCTTTTGACTACTGGGGTAAAGGTACAACGGTAACCGTTTCATCAG & YAAFDYWGKGTTVTVSS & 204084 & 54 & - \\ 
  IGH2J07 & 204673 & TTACCACGCAGCTTTGGACTACTGGGGAAAAGGGACGACGGTCACCGTCACCTCAG & YHXALDYWGKGTTVTVTS & 204728 & 56 & - \\ 
  IGH2J06 & 205005 & ATAACTGGGCTTTCGACTACTGGGGAAAAGGGACGATGGTAACGGTGACTTCAG & NWAFDYWGKGTMVTVTS & 205058 & 54 & - \\ 
  IGH2J05 & 205296 & CTACGATGCTTTTGACTACTGGGGGAAAAGGACGATGGTCACGTCACTTCAG & YDAFDYWGKRTMVTSLQ & 205347 & 52 & - \\ 
  IGH2J04 & 205756 & ACAACGCTTTTGACTACTGGGGAAAAGGAACAACGGTCACCGTCACTTCAG & NAFDYWGKGTTVTVTS & 205806 & 51 & - \\ 
  IGH2J03 & 206111 & ATGGTGCTTTTGACTACTGGGGTAAAGGGACCGCAGTCACTGTAACATCAG & GAFDYWGKGTAVTVTS & 206161 & 51 & - \\ 
  IGH2J02 & 206303 & ACCGTGGGGTAAAGGGACAACAGTCACGGTCAAAACAG & PWGKGTTVTVKT & 206340 & 38 & - \\ 
  IGH2J01 & 206466 & ATGACTACTTTGACTACTGGGGAAAAGGAACAATGGTGACGGTCACATCAG & DYFDYWGKGTMVTVTS & 206516 & 51 & - \\ 
   \bottomrule \end{tabular}

		\normalsize\vspace{0.6em}
        \captionof{table}{Co-ordinate table of \jh segments in the \nfu \igh{} locus.}
        \label{tab:nfu-jh-coords-seg}
        \notsotiny
		% latex table generated in R 3.5.2 by xtable 1.8-3 package
% Tue Jan 15 17:07:52 2019
\begin{tabular}{lrlrlrr}
  \toprule Name & RSS Start & Nonamer & Spacer Length & Heptamer & RSS End & RSS Length \\ 
  \midrule IGH1J01 & 26196 & TGTTTTTGT & 23 & CACTGTG & 26186 & 39 \\ 
  IGH1J02 & 128188 & AGTGTTTGT & 23 & CACTGTG & 128175 & 39 \\ 
  IGH1J03 & 128353 & TGTTTATTT & 23 & CACTGTG & 128353 & 39 \\ 
  IGH1J04 & 128545 & GGTTTTTGT & 23 & CACTGTG & 128532 & 39 \\ 
  IGH1J05 & 128899 & GGTTTTAGT & 23 & TACTGTG & 128886 & 39 \\ 
  IGH1J06 & 129360 & TCTTCTTGT & 22 & TACTTTG & 129345 & 38 \\ 
  IGH1J07 & 129650 & AGTTTTTGT & 23 & TACTGTG & 129634 & 39 \\ 
  IGH1J08 & 129983 & AGTTTTAGT & 22 & TACTGTG & 129964 & 38 \\ 
  IGH1J09 & 130628 & CGTTTTTAT & 22 & CACTGTG & 130611 & 38 \\ 
  IGH2J08 & 204047 & CGTTTTTAT & 22 & CACTGTG & 204030 & 38 \\ 
  IGH2J07 & 204691 & AGTTTTAGT & 22 & TACTGTG & 204672 & 38 \\ 
  IGH2J06 & 205020 & AGTTTTTGT & 23 & TACTGTG & 205004 & 39 \\ 
  IGH2J05 & 205310 & TCTTCTTGT & 22 & TACTTTG & 205295 & 38 \\ 
  IGH2J04 & 205768 & GGTTTTAGT & 23 & TACTGTG & 205755 & 39 \\ 
  IGH2J03 & 206123 & GGTTTTTGT & 23 & CACTGTG & 206110 & 39 \\ 
  IGH2J02 & 206302 & TGTTTATTT & 23 & CACTGTG & 206302 & 39 \\ 
  IGH2J01 & 206478 & AGTGTTTGT & 23 & CACTGTG & 206465 & 39 \\ 
   \bottomrule \end{tabular}

		\normalsize\vspace{0.6em}
        \captionof{table}{Co-ordinate table of \jh RSSs in the \nfu \igh{} locus.}
        \label{tab:nfu-dh-coords-rss}
    \end{landscape}

\begin{table}\centering
    \caption{Co-ordinate table of constant-region exons in the \xma \igh{} locus.}
    	% latex table generated in R 3.5.2 by xtable 1.8-3 package
% Tue Jan  8 15:33:35 2019
\begin{tabular}{llrrrl}
  \toprule Name & Isotype & Start & End & Length & Strand \\ 
  \midrule IGHZ1-1 & Z & 3380 & 3667 & 288 & + \\ 
  IGHZ1-2 & Z & 3814 & 4098 & 285 & + \\ 
  IGHZ1-3 & Z & 4195 & 4497 & 303 & + \\ 
  IGHZ1-4 & Z & 4934 & 5263 & 330 & + \\ 
  IGHZ1-S & Z & 5264 & 5459 & 196 & + \\ 
  IGHZ1-TM1 & Z & 6345 & 6490 & 146 & + \\ 
  IGHZ1-TM2 & Z & 6645 & 7043 & 399 & + \\ 
  IGHZ2-1 & Z & 256059 & 256337 & 279 & + \\ 
  IGHZ2-2 & Z & 256453 & 256734 & 282 & + \\ 
  IGHZ2-3 & Z & 256893 & 257171 & 279 & + \\ 
  IGHZ2-4 & Z & 257319 & 257636 & 318 & + \\ 
  IGHZ2-S & Z & 257637 & 257850 & 214 & + \\ 
  IGHZ2-TM1 & Z & 258059 & 258213 & 155 & + \\ 
  IGHZ2-TM2 & Z & 258410 & 258629 & 220 & + \\ 
  IGHM-1 & M & 279664 & 279960 & 297 & + \\ 
  IGHM-2 & M & 280880 & 281224 & 345 & + \\ 
  IGHM-3 & M & 281321 & 281629 & 309 & + \\ 
  IGHM-4 & M & 281789 & 282291 & 503 & + \\ 
  IGHM-TM1 & M & 282910 & 283034 & 125 & + \\ 
  IGHM-TM2 & M & 285028 & 285740 & 713 & + \\ 
  IGHD-1 & D & 285902 & 286219 & 318 & + \\ 
  IGHD-2A & D & 286310 & 286597 & 288 & + \\ 
  IGHD-3A & D & 286814 & 287128 & 315 & + \\ 
  IGHD-4A & D & 287250 & 287534 & 285 & + \\ 
  IGHD-2B & D & 288876 & 289166 & 291 & + \\ 
  IGHD-3B & D & 289262 & 289576 & 315 & + \\ 
  IGHD-4B & D & 289680 & 289964 & 285 & + \\ 
  IGHD-5 & D & 290052 & 290381 & 330 & + \\ 
  IGHD-6 & D & 290472 & 290789 & 318 & + \\ 
  IGHD-7 & D & 290865 & 291152 & 288 & + \\ 
  IGHD-TM1 & D & 291286 & 291434 & 149 & + \\ 
  IGHD-TM2 & D & 291541 & 291642 & 102 & + \\ 
   \bottomrule \end{tabular}

    \label{tab:xma-ch-coords}
\end{table}

    \begin{landscape}
        \centering
        \vspace*{\fill}
        \scriptsize
		% latex table generated in R 3.5.2 by xtable 1.8-3 package
% Tue Jan  8 15:33:33 2019
\begin{tabular}{lrrrlrlllrrl}
  \toprule Name & Start & End & Length & Strand & RSS Start & Heptamer & Spacer Length & Nonamer & RSS End & RSS Length & Comment \\ 
  \midrule IGHV01-01 & 1159 & 1450 & 292 & + & 1451 & CACAGTG & 23 & GTAAAAACC & 1489 & 39 &  \\ 
  IGHV02-01 & 10534 & 10825 & 292 & + & 10826 & CACAGTG & 23 & ACAAAACCC & 10864 & 39 &  \\ 
  IGHV02-02 & 11961 & 12261 & 301 & + & 12262 & CACTGTG & 23 & ACAAAAACT & 12300 & 39 &  \\ 
  IGHV02-03 & 13319 & 13616 & 298 & + & 13617 & CACAGTG & 23 & ACACAAACT & 13655 & 39 &  \\ 
  IGHV03-01 & 15440 & 15734 & 295 & + & 15735 & CACAGTG & 22 & ACAAAAACT & 15772 & 38 &  \\ 
  IGHV02-04 & 16618 & 16908 & 291 & + & 16909 & CACAGTG & 23 & ACAAAAACC & 16947 & 39 &  \\ 
  IGHV02-05 & 17522 & 17822 & 301 & + & 17823 & CACTGTG & 22 & ACAAAAACT & 17860 & 38 &  \\ 
  IGHV02-06 & 18881 & 19178 & 298 & + & 19179 & CACAGTG & 23 & ACACAAACT & 19217 & 39 &  \\ 
  IGHV03-02 & 21000 & 21294 & 295 & + & 21295 & CACAGTG & 22 & ACAAAAACT & 21332 & 38 &  \\ 
  IGHV02-07 & 22179 & 22467 & 289 & + & 22468 & CACAGTG & 23 & ACAAAAACC & 22506 & 39 &  \\ 
  IGHV02-08p & 24234 & 24514 & 281 & + & 24515 & CACAGTG & 23 & ACAAAAACT & 24553 & 39 & Frameshift \\ 
  IGHV04-01 & 25359 & 25659 & 301 & + & 25660 & CACAGTG & 23 & ACAAAAACT & 25698 & 39 &  \\ 
  IGHV04-02 & 27066 & 27366 & 301 & + & 27367 & CACAGTG & 23 & ACAAAAACA & 27405 & 39 &  \\ 
  IGHV02-09 & 28669 & 28958 & 290 & + & 28959 & CACAGTG & 23 & ACAAAAACC & 28997 & 39 &  \\ 
  IGHV02-10p & 30460 & 30741 & 282 & + & 30742 & CACAATG & 23 & ACAAAACTC & 30780 & 39 & Frameshift \\ 
  IGHV02-11 & 32395 & 32681 & 287 & + & 32682 & CACAGTG & 23 & ACAAAAACC & 32720 & 39 &  \\ 
  IGHV03-03 & 33663 & 33957 & 295 & + & 33958 & CACTGTG & 22 & ACAAAAACT & 33995 & 38 &  \\ 
  IGHV02-12 & 35012 & 35299 & 288 & + & 35300 & CACAGTG & 23 & ACAAAAACC & 35338 & 39 &  \\ 
  IGHV03-04 & 36281 & 36575 & 295 & + & 36576 & CACTGTG & 22 & ACAAAAACT & 36613 & 38 &  \\ 
  IGHV02-13 & 37639 & 37931 & 293 & + & 37932 & CACAGTG & 23 & ACAAAAACT & 37970 & 39 &  \\ 
  IGHV02-14 & 39019 & 39311 & 293 & + & 39312 & CACAGTG & 23 & ACAAAAACT & 39350 & 39 &  \\ 
  IGHV03-05 & 41008 & 41302 & 295 & + & 41303 & CACAGTG & 22 & ACAAAAACT & 41340 & 38 &  \\ 
  IGHV02-15 & 42660 & 42952 & 293 & + & 42953 & CACAGTG & 23 & ACAAAAACT & 42991 & 39 &  \\ 
  IGHV03-06 & 45081 & 45375 & 295 & + & 45376 & CACAGTG & 22 & ACAAAAACT & 45413 & 38 &  \\ 
  IGHV02-16 & 46732 & 47024 & 293 & + & 47025 & CACAGTG & 23 & ACAAAAACT & 47063 & 39 &  \\ 
   \bottomrule \end{tabular}

		\normalsize\vspace{1em}
        \captionof{table}{Co-ordinate table of \vh segments in the \xma \igh{} locus, part 1.}
        \label{tab:xma-vh-coords-1}
        \vspace*{\fill}
    \end{landscape}

    \begin{landscape}
        \centering
        \vspace*{\fill}
        \scriptsize
		% latex table generated in R 3.5.2 by xtable 1.8-3 package
% Tue Jan  8 15:33:33 2019
\begin{tabular}{lrrrlrlllrrl}
  \toprule Name & Start & End & Length & Strand & RSS Start & Heptamer & Spacer Length & Nonamer & RSS End & RSS Length & Comment \\ 
  \midrule IGHV03-07 & 48618 & 48912 & 295 & + & 48913 & CACAGTG & 22 & ACAAAAACT & 48950 & 38 &  \\ 
  IGHV02-17 & 50323 & 50611 & 289 & + & 50612 & CACAGTG & 23 & ACAAAAACC & 50650 & 39 &  \\ 
  IGHV03-08 & 51890 & 52184 & 295 & + & 52185 & CACAGTG & 22 & ACAAAAACT & 52222 & 38 &  \\ 
  IGHV03-09p & 53026 & 53274 & 249 & + & 53275 &  &  &  &  &  & 3'-truncated, no RSS \\ 
  IGHV02-18 & 54462 & 54747 & 286 & + & 54748 & CACAGTG & 23 & ACAAAAACC & 54786 & 39 &  \\ 
  IGHV02-19p & 55729 & 55866 & 138 & + & 55867 & CACAGTG & 23 & ACAAAAACC & 55905 & 39 & 3'-truncated \\ 
  IGHV03-10 & 57371 & 57662 & 292 & + & 57663 & CACAGTG & 22 & ACAAAAACT & 57700 & 38 &  \\ 
  IGHV02-20p & 58698 & 58986 & 289 & + & 58987 & CACAGTG & 23 & ATAAAAACC & 59025 & 39 & Nonsense mutation \\ 
  IGHV03-11 & 59940 & 60234 & 295 & + & 60235 & CACAGTG & 22 & ACAAAAACT & 60272 & 38 &  \\ 
  IGHV02-21 & 61249 & 61537 & 289 & + & 61538 & CACAGTG & 23 & ATAAAAACC & 61576 & 39 &  \\ 
  IGHV03-12 & 62491 & 62785 & 295 & + & 62786 & CACAGTG & 22 & ACAAAAACT & 62823 & 38 &  \\ 
  IGHV02-22 & 63801 & 64089 & 289 & + & 64090 & CACAGTG & 23 & ATAAAAACC & 64128 & 39 &  \\ 
  IGHV03-13 & 65043 & 65337 & 295 & + & 65338 & CACAGTG & 22 & ACAAAAACT & 65375 & 38 &  \\ 
  IGHV02-23 & 66354 & 66640 & 287 & + & 66641 & CACAGTG & 23 & ACAAAAACT & 66679 & 39 &  \\ 
  IGHV03-14 & 68452 & 68743 & 292 & + & 68744 & CACTATG & 22 & ACAAAACTC & 68781 & 38 &  \\ 
  IGHV02-24 & 70101 & 70389 & 289 & + & 70390 & CACAGTG & 23 & ACAAAAACC & 70428 & 39 &  \\ 
  IGHV03-15 & 72206 & 72501 & 296 & + & 72502 & CACAGTG & 22 & ACAAAAACT & 72539 & 38 &  \\ 
  IGHV02-25 & 73484 & 73772 & 289 & + & 73773 & CACAGTG & 23 & ACAAAAACC & 73811 & 39 &  \\ 
  IGHV03-16 & 75799 & 76090 & 292 & + & 76091 & CACAGTG & 22 & ACAAAAACT & 76128 & 38 &  \\ 
  IGHV03-17 & 77773 & 78067 & 295 & + & 78068 & CACAGTG & 22 & ACAAAAACT & 78105 & 38 &  \\ 
  IGHV02-26 & 79001 & 79289 & 289 & + & 79290 & CACAGTG & 23 & ACAAAAACC & 79328 & 39 &  \\ 
  IGHV03-18 & 80492 & 80784 & 293 & + & 80785 & CACAGTG & 22 & ACAAAAACT & 80822 & 38 &  \\ 
  IGHV02-27p & 81799 & 82082 & 284 & + & 82083 & CACAGTG & 23 & ACAAAAACC & 82121 & 39 & Frameshift \\ 
  IGHV03-19 & 83736 & 84030 & 295 & + & 84031 & CACAGTG & 22 & ACAAAAACT & 84068 & 38 &  \\ 
  IGHV02-28p & 85093 & 85381 & 289 & + & 85382 & CACAGGG & 23 & GCAAAAACC & 85420 & 39 & Nonsense mutation \\ 
   \bottomrule \end{tabular}

		\normalsize\vspace{1em}
        \captionof{table}{Co-ordinate table of \vh segments in the \xma \igh{} locus, part 2.}
        \label{tab:xma-vh-coords-2}
        \vspace*{\fill}
    \end{landscape}

    \begin{landscape}
        \centering
        \vspace*{\fill}
        \scriptsize
		% latex table generated in R 3.5.2 by xtable 1.8-3 package
% Tue Jan  8 15:33:33 2019
\begin{tabular}{lrrrlrlllrrl}
  \toprule Name & Start & End & Length & Strand & RSS Start & Heptamer & Spacer Length & Nonamer & RSS End & RSS Length & Comment \\ 
  \midrule IGHV02-29 & 86225 & 86505 & 281 & + & 86506 & CACAGTG & 23 & ATAAAAACC & 86544 & 39 &  \\ 
  IGHV03-20 & 87419 & 87713 & 295 & + & 87714 & CACAGTG & 22 & ACAAAAACT & 87751 & 38 &  \\ 
  IGHV03-21 & 94532 & 94826 & 295 & + & 94827 & CACAGTG & 23 & ACAAAAACC & 94865 & 39 &  \\ 
  IGHV03-22 & 96192 & 96489 & 298 & + & 96490 & CACAGTG & 23 & ACAAAAACC & 96528 & 39 &  \\ 
  IGHV03-23 & 98068 & 98368 & 301 & + & 98369 & CACAGTG & 23 & ACAAAAACC & 98407 & 39 &  \\ 
  IGHV03-24 & 99482 & 99779 & 298 & + & 99780 & CACAGTG & 23 & ACAAAAACC & 99818 & 39 &  \\ 
  IGHV03-25 & 101639 & 101936 & 298 & + & 101937 & CACAGTG & 23 & ACAAAAACC & 101975 & 39 &  \\ 
  IGHV05-01p & 102818 & 103096 & 279 & + & 103097 & CAGAAGC & 0 & ACAAAAACT & 103112 & 16 & Frameshift \\ 
  IGHV03-26 & 104098 & 104389 & 292 & + & 104390 & CACAGTG & 23 & ACAAAATCC & 104428 & 39 &  \\ 
  IGHV06-01 & 105551 & 105831 & 281 & + & 105832 & CACAGTG & 23 & ACAAAAACC & 105870 & 39 &  \\ 
  IGHV03-27 & 107274 & 107571 & 298 & + & 107572 & CACAGTG & 23 & ACAAAAACC & 107610 & 39 &  \\ 
  IGHV03-28 & 108775 & 109072 & 298 & + & 109073 & CACAGAG & 23 & ACAAAAACC & 109111 & 39 &  \\ 
  IGHV03-29 & 110372 & 110672 & 301 & + & 110673 & CACAGTG & 23 & ACAAAAACC & 110711 & 39 &  \\ 
  IGHV07-01 & 111565 & 111856 & 292 & + & 111857 & CACAATG & 23 & ACAAAAACT & 111895 & 39 &  \\ 
  IGHV08-01p & 113033 & 113330 & 298 & + & 113331 & CACAGAG & 23 & CCAAGAACC & 113369 & 39 & Nonsense mutation \\ 
  IGHV09-01 & 115512 & 115800 & 289 & + & 115801 & CACAGTG & 22 & ACAAAAACT & 115838 & 38 &  \\ 
  IGHV10-01 & 117078 & 117379 & 302 & + & 117380 & CACAGTG & 22 & ACATAAACT & 117417 & 38 &  \\ 
  IGHV11-01 & 119462 & 119760 & 299 & + & 119761 & CACAGTG & 23 & ACAAAAACT & 119799 & 39 &  \\ 
  IGHV03-30 & 126125 & 126416 & 292 & + & 126417 & CACAGTG & 22 & ACAAAAACC & 126454 & 38 &  \\ 
  IGHV03-31 & 127109 & 127400 & 292 & + & 127401 & CACAGTG & 23 & GCAAAAACC & 127439 & 39 &  \\ 
  IGHV12-01 & 128489 & 128786 & 298 & + & 128787 & CACAGTG & 23 & ACAAAAACC & 128825 & 39 &  \\ 
  IGHV02-30 & 135711 & 136000 & 290 & + & 136001 & CACAGTG & 22 & ACAAAAACA & 136038 & 38 &  \\ 
  IGHV13-01 & 136757 & 137057 & 301 & + & 137058 & CACAGTG & 23 & ACAAAAACT & 137096 & 39 &  \\ 
  IGHV02-31 & 138344 & 138637 & 294 & + & 138638 & CACAGTG & 23 & ACAAAAATC & 138676 & 39 &  \\ 
  IGHV02-32 & 140024 & 140315 & 292 & + & 140316 & CACTGTG & 23 & ACAAAAACT & 140354 & 39 &  \\ 
   \bottomrule \end{tabular}

		\normalsize\vspace{1em}
        \captionof{table}{Co-ordinate table of \vh segments in the \xma \igh{} locus, part 3.}
        \label{tab:xma-vh-coords-3}
        \vspace*{\fill}
    \end{landscape}

    \begin{landscape}
        \centering
        \vspace*{\fill}
        \scriptsize
		% latex table generated in R 3.5.2 by xtable 1.8-3 package
% Tue Jan  8 15:33:33 2019
\begin{tabular}{lrrrlrlllrrl}
  \toprule Name & Start & End & Length & Strand & RSS Start & Heptamer & Spacer Length & Nonamer & RSS End & RSS Length & Comment \\ 
  \midrule IGHV02-33 & 142332 & 142620 & 289 & + & 142621 & CACAGTG & 23 & ACAAAAACA & 142659 & 39 &  \\ 
  IGHV02-34 & 144334 & 144625 & 292 & + & 144626 & CACAGTG & 23 & ACAAAAACT & 144664 & 39 &  \\ 
  IGHV02-35 & 145740 & 146031 & 292 & + & 146032 & CACAGTG & 23 & ACAAAAAAT & 146070 & 39 &  \\ 
  IGHV02-36 & 146903 & 147194 & 292 & + & 147195 & CACAGTG & 23 & ACAAAAACT & 147233 & 39 &  \\ 
  IGHV02-37 & 147839 & 148138 & 300 & + & 148139 & CACAGTG & 23 & ACAAAAATC & 148177 & 39 &  \\ 
  IGHV02-38p & 150504 & 150797 & 294 & + & 150798 & CACAATA & 23 & ACAAAAACC & 150836 & 39 & Nonsense mutation \\ 
  IGHV02-39 & 152249 & 152537 & 289 & + & 152538 & CACAGTA & 23 & ACAAAAACC & 152576 & 39 &  \\ 
  IGHV14-01 & 154075 & 154374 & 300 & + & 154375 & CACAGTG & 23 & ACAAAAAGT & 154413 & 39 &  \\ 
  IGHV02-40 & 155433 & 155709 & 277 & + & 155710 & CACAGTG & 23 & ACAAAAACC & 155748 & 39 &  \\ 
  IGHV02-41 & 156583 & 156870 & 288 & + & 156871 & CACAGTG & 23 & ACAAAAACC & 156909 & 39 &  \\ 
  IGHV02-42 & 163977 & 164269 & 293 & + & 164270 & CACAGTG & 23 & ACAAAACCC & 164308 & 39 &  \\ 
  IGHV03-32 & 165416 & 165708 & 293 & + & 165709 & CACAGTG & 22 & ACAAAAACA & 165746 & 38 &  \\ 
  IGHV02-43 & 166994 & 167293 & 300 & + & 167294 & CACAATG & 23 & ACAGAAACT & 167332 & 39 &  \\ 
  IGHV12-02 & 169602 & 169900 & 299 & + & 169901 & CACAGTG & 23 & ACAAAAACC & 169939 & 39 &  \\ 
  IGHV02-44 & 171452 & 171752 & 301 & + & 171753 & CACTGTG & 23 & GCAAAAACT & 171791 & 39 &  \\ 
  IGHV02-45 & 173096 & 173384 & 289 & + & 173385 & CTCAGTG & 23 & ACAAAAACC & 173423 & 39 &  \\ 
  IGHV02-46 & 174714 & 175009 & 296 & + & 175010 & CACAGTG & 23 & ACAAAAACT & 175048 & 39 &  \\ 
  IGHV02-47 & 176396 & 176697 & 302 & + & 176698 & CACAGTG & 23 & ACAAAAACT & 176736 & 39 &  \\ 
  IGHV12-03 & 178422 & 178719 & 298 & + & 178720 & CACAGTG & 23 & ACAAAAACA & 178758 & 39 &  \\ 
  IGHV12-04 & 181245 & 181543 & 299 & + & 181544 & CACAGTG & 23 & ACAAAAACC & 181582 & 39 &  \\ 
  IGHV02-48p & 182977 & 183236 & 260 & + & 183237 & CACAGGT & 8 & ACAAAAACT & 183260 & 24 & 5'-truncated \\ 
  IGHV02-49p & 184323 & 184611 & 289 & + & 184612 & CACAGTG & 23 & ACAAAAACC & 184650 & 39 & Nonsense mutation \\ 
  IGHV02-50 & 185946 & 186244 & 299 & + & 186245 & CACAGTG & 23 & ACAAAAACT & 186283 & 39 &  \\ 
  IGHV02-51 & 187624 & 187925 & 302 & + & 187926 & CACAGTG & 23 & ACAAAAACT & 187964 & 39 &  \\ 
  IGHV12-05 & 190987 & 191284 & 298 & + & 191285 & CACAGTG & 23 & ACAAAAACA & 191323 & 39 &  \\ 
   \bottomrule \end{tabular}

		\normalsize\vspace{1em}
        \captionof{table}{Co-ordinate table of \vh segments in the \xma \igh{} locus, part 4.}
        \label{tab:xma-vh-coords-4}
        \vspace*{\fill}
    \end{landscape}

    \begin{landscape}
        \centering
        \vspace*{\fill}
        \scriptsize
		% latex table generated in R 3.5.2 by xtable 1.8-3 package
% Tue Jan  8 15:33:35 2019
\begin{tabular}{lrrrlrlllrrp{4cm}}
  \toprule Name & Start & End & Length & Strand & RSS Start & Heptamer & Spacer Length & Nonamer & RSS End & RSS Length & Comment \\ 
  \midrule IGHV02-52 & 192570 & 192868 & 299 & + & 192869 & CACAGTG & 19 & CTGAAAACC & 192903 & 35 &  \\ 
  IGHV12-06 & 193608 & 193906 & 299 & + & 193907 & CACAGTG & 23 & ACAAAAACA & 193945 & 39 &  \\ 
  IGHV02-53 & 195271 & 195572 & 302 & + & 195573 & CACAGTG & 23 & ACAAAAACC & 195611 & 39 &  \\ 
  IGHV15-01 & 204396 & 204693 & 298 & + & 204694 & CACAATC & 23 & ACAAAAACT & 204732 & 39 &  \\ 
  IGHV13-02 & 206203 & 206503 & 301 & + & 206504 & CACAGTG & 23 & ACAAAAACT & 206542 & 39 &  \\ 
  IGHV16-01 & 207726 & 208020 & 295 & + & 208021 & CACAGTG & 22 & ACAAAAACT & 208058 & 38 &  \\ 
  IGHV13-03 & 208477 & 208777 & 301 & + & 208778 & CACAGTA & 23 & ACAAAAACT & 208816 & 39 &  \\ 
  IGHV03-33 & 209921 & 210215 & 295 & + & 210216 & CACGGTG & 22 & ACGAAAACT & 210253 & 38 &  \\ 
  IGHV17-01 & 211322 & 211625 & 304 & + & 211626 & CACAGTA & 23 & ACAAAAACC & 211664 & 39 &  \\ 
  IGHV15-02p & 214600 & 214860 & 261 & + & 214861 &  &  &  &  &  & 3'-truncated, no RSS \\ 
  IGHV18-01 & 215671 & 215962 & 292 & + & 215963 & CACACTG & 23 & ACAAAAACC & 216001 & 39 &  \\ 
  IGHV19-01 & 217874 & 218174 & 301 & + & 218175 & CACAGTG & 23 & ACAAAAACT & 218213 & 39 &  \\ 
  IGHV03-34 & 219368 & 219668 & 301 & + & 219669 & CACAGTG & 23 & ACAAAAACA & 219707 & 39 &  \\ 
  IGHV20-01 & 220329 & 220632 & 304 & + & 220633 & CACAGTG & 23 & ACAAAAATT & 220671 & 39 &  \\ 
  IGHV02-54p & 228547 & 228838 & 292 & + & 228839 & CACACTG & 23 & ACAACCCCC & 228877 & 39 & Nonsense mutation \\ 
  IGHV02-55 & 229963 & 230267 & 305 & + & 230268 & CACAGCG & 23 & ACAAAAAAA & 230306 & 39 &  \\ 
  IGHV03-35 & 231630 & 231928 & 299 & + & 231929 & CACAGTG & 23 & ACAAAAACC & 231967 & 39 &  \\ 
  IGHV21-01p & 233069 & 233230 & 162 & + & 233231 &  &  &  &  &  & Nonsense mutation, 3'-truncated, no RSS \\ 
  IGHV22-01p & 234954 & 235102 & 149 & + & 235103 & CACAGTG & 23 & TCAAAAACT & 235141 & 39 & 5'-truncated \\ 
  IGHV02-56 & 236029 & 236330 & 302 & + & 236331 & CACAGTG & 23 & ACAAATACT & 236369 & 39 &  \\ 
  IGHV03-36p & 238122 & 238413 & 292 & + & 238414 & CACAATG & 23 & ACAGAATCC & 238452 & 39 & Nonsense mutation \\ 
  IGHV11-02p & 240281 & 240579 & 299 & + & 240580 & CACAGTG & 24 & ACAAAAACT & 240619 & 40 & Nonsense mutation \\ 
  IGHV09-02 & 241878 & 242166 & 289 & + & 242167 & CACAGTG & 22 & ACAAAAACT & 242204 & 38 &  \\ 
  IGHV23-01 & 243867 & 244164 & 298 & + & 244165 & CACAGTG & 23 & ACAAAATCC & 244203 & 39 &  \\ 
  IGHV02-57 & 245524 & 245813 & 290 & + & 245814 & CACCATA & 22 & ACAAAATCC & 245851 & 38 &  \\ 
   \bottomrule \end{tabular}

		\normalsize\vspace{1em}
        \captionof{table}{Co-ordinate table of \vh segments in the \xma \igh{} locus, part 5.}
        \label{tab:xma-vh-coords-5}
        \vspace*{\fill}
    \end{landscape}

        {\centering
        \captionof{table}{Co-ordinate table of \dh segments in the \xma \igh{} locus.}\vspace{-0.3em}
        \label{tab:xma-dh-coords-rss3}
        \scriptsize
		% latex table generated in R 3.5.2 by xtable 1.8-3 package
% Tue Jan  8 15:33:35 2019
\begin{tabular}{lrlrrl}
  \toprule Name & Start & NT Sequence & End & Length & Strand \\ 
  \midrule IGHDZ01 & 2243 & GTGGGCAGGAGGCTATGC & 2260 & 18 & + \\ 
  IGHDZ02 & 119768 & AGG & 119770 & 3 & + \\ 
  IGHDZ03 & 128794 & ACTAAAGG & 128801 & 8 & + \\ 
  IGHDZ04 & 129907 & ATCGGG & 129912 & 6 & + \\ 
  IGHDZ05 & 158017 & ATATATGGGGG & 158027 & 11 & + \\ 
  IGHDZ06 & 197791 & ATATACTGGGGTGG & 197804 & 14 & + \\ 
  IGHDZ07 & 222022 & ATGGACTGGGGGG & 222034 & 13 & + \\ 
  IGHDZ08 & 247941 & GTGATTACGGCTACGGGGC & 247959 & 19 & + \\ 
  IGHDZ09 & 249514 & TTATGGGCTGGGGAG & 249528 & 15 & + \\ 
  IGHDZ10 & 253752 & TGGGTGGGGC & 253761 & 10 & + \\ 
  IGHDM01 & 267392 & TATACAGTGGCAAC & 267405 & 14 & + \\ 
  IGHDM02 & 268498 & CAGTATAGCAAC & 268509 & 12 & + \\ 
  IGHDM03 & 268836 & TACAATGGCAAC & 268847 & 12 & + \\ 
  IGHDM04 & 269694 & TAAACAGTGGCTAC & 269707 & 14 & + \\ 
   \bottomrule \end{tabular}

		\normalsize\vspace{1em}
        \captionof{table}{Co-ordinate table of \dh 5'-RSSs in the \xma \igh{} locus.}\vspace{-0.3em}
        \label{tab:xma-dh-coords-seg}
        \scriptsize
        	% latex table generated in R 3.5.2 by xtable 1.8-3 package
% Tue Jan  8 15:33:35 2019
\begin{tabular}{lrlrlrr}
  \toprule Name & 5'-RSS Start & Nonamer & Spacer Length & Heptamer & 5'-RSS End & Length \\ 
  \midrule IGHDZ01 & 2215 & GGTTTTTGT & 12 & CACTGTG & 2242 & 28 \\ 
  IGHDZ02 & 119739 & TGTATTACT & 13 & CACAGTG & 119767 & 29 \\ 
  IGHDZ03 & 128766 & TTTACTTCT & 12 & CACAGTG & 128793 & 28 \\ 
  IGHDZ04 & 129879 & GGTTTTTGT & 12 & CACAGTG & 129906 & 28 \\ 
  IGHDZ05 & 157989 & AGTTTTTGT & 12 & CACAGTG & 158016 & 28 \\ 
  IGHDZ06 & 197763 & GGTTTTTGC & 12 & TACTGTG & 197790 & 28 \\ 
  IGHDZ07 & 221994 & GGTTTTTGT & 12 & CGCTGTG & 222021 & 28 \\ 
  IGHDZ08 & 247913 & TGTTTTTGT & 12 & ATCTGTG & 247940 & 28 \\ 
  IGHDZ09 & 249486 & AGTTTTTGT & 12 & TGTGGTG & 249513 & 28 \\ 
  IGHDZ10 & 253724 & AGTTTTTGT & 12 & TGTAGTG & 253751 & 28 \\ 
  IGHDM01 & 267364 & AGTTTTTGT & 12 & TACAGTG & 267391 & 28 \\ 
  IGHDM02 & 268470 & TGTTTTTGT & 12 & CACAGTG & 268497 & 28 \\ 
  IGHDM03 & 268808 & AGTTTTTGC & 12 & TACTGTG & 268835 & 28 \\ 
  IGHDM04 & 269666 & CGTTTTTGT & 12 & CATTGTG & 269693 & 28 \\ 
   \bottomrule \end{tabular}

        	\normalsize\vspace{1em}
        \captionof{table}{Co-ordinate table of \dh 3'-RSSs in the \xma \igh{} locus.}\vspace{-0.3em}
        \label{tab:xma-dh-coords-rss5}
        \scriptsize
		% latex table generated in R 3.5.2 by xtable 1.8-3 package
% Tue Jan  8 15:33:35 2019
\begin{tabular}{lrlrlrr}
  \toprule Name & 3'-RSS Start & Heptamer & Spacer Length & Nonamer & 3'-RSS End & Length \\ 
  \midrule IGHDZ01 & 2261 & CACTAAG & 12 & ACAAAAAGT & 2288 & 28 \\ 
  IGHDZ02 & 119771 & CAAAATG & 13 & ACAAAAACT & 119799 & 29 \\ 
  IGHDZ03 & 128802 & CAGAGAA & 8 & ACAAAAACC & 128825 & 24 \\ 
  IGHDZ04 & 129913 & CACAATG & 12 & TCAAAAACC & 129940 & 28 \\ 
  IGHDZ05 & 158028 & CACAGAG & 12 & ACAAAAACC & 158055 & 28 \\ 
  IGHDZ06 & 197805 & CACACAG & 12 & ACAAAAACC & 197832 & 28 \\ 
  IGHDZ07 & 222035 & CACAGAG & 12 & ACAAAAACC & 222062 & 28 \\ 
  IGHDZ08 & 247960 & CACAATA & 12 & ACAAAAACC & 247987 & 28 \\ 
  IGHDZ09 & 249529 & CACAATG & 12 & ACAAAAACC & 249556 & 28 \\ 
  IGHDZ10 & 253762 & CACAGTA & 12 & ACAAAAACC & 253789 & 28 \\ 
  IGHDM01 & 267406 & CACAGTG & 12 & GCAAAAACC & 267433 & 28 \\ 
  IGHDM02 & 268510 & CACAGTG & 12 & ACAGAAACC & 268537 & 28 \\ 
  IGHDM03 & 268848 & CACAGTG & 12 & ACAAAAACC & 268875 & 28 \\ 
  IGHDM04 & 269708 & CACTGTG & 12 & ACAAAATCA & 269735 & 28 \\ 
   \bottomrule \end{tabular}

		\normalsize
}

    \begin{landscape}
        \centering
        \notsotiny
		% latex table generated in R 3.5.2 by xtable 1.8-3 package
% Tue Jan  8 15:33:35 2019
\begin{tabular}{lrllrrl}
  \toprule Name & Start & NT Sequence & AA Sequence & End & Length & Strand \\ 
  \midrule IGHJZ01 & 2653 & ATGCCTTAGATTACTGGGGTGAAGGGACCAGAGTCACAGTGACTTCAG & ALDYWGEGTRVTVTS & 2700 & 48 & + \\ 
  IGHJZ02 & 120639 & ATTACGCTCTTGACTACTGGGGAGCAGGAACCAAAGTTACTGTAAAGCCAG & YALDYWGAGTKVTVKP & 120689 & 51 & + \\ 
  IGHJZ03 & 130376 & ACTACGGCTTTGATTACTGGGGAGACGGAACTGAAGTTACTGTTGAACCAG & YGFDYWGDGTEVTVEP & 130426 & 51 & + \\ 
  IGHJZ04 & 158408 & AGATTTAGACTACTGGGGTAATGGAACAACAGTCACGGTTCTACCAG & DLDYWGNGTTVTVLP & 158454 & 47 & + \\ 
  IGHJZ05 & 198186 & ATTATGGTTTTGACTACTGGGGAGACGGAACCACAGTCACTGTTAGTCCAG & YGFDYWGDGTTVTVSP & 198236 & 51 & + \\ 
  IGHJZ06 & 222417 & ATGCTTTTGACGTCTGGGGTAAAGGAACCACAGTTACTGTTGTACCAG & AFDVWGKGTTVTVVP & 222464 & 48 & + \\ 
  IGHJZ07 & 254130 & ATGTTTTTGACTACTGGGGTAAAGGGACTGATGTCACAGTATCTCCAG & VFDYWGKGTDVTVSP & 254177 & 48 & + \\ 
  IGHJM01 & 276014 & ACGGCTACTTCGACTACTGGGGGAAAGGAACACAAGTCACAGTGACTTCTG & GYFDYWGKGTQVTVTS & 276064 & 51 & + \\ 
  IGHJM02 & 276284 & CCACTACTTTGACTACTGGGGAAAAGGAACCACGGTTACCGTCACTTCAG & HYFDYWGKGTTVTVTS & 276333 & 50 & + \\ 
  IGHJM03 & 276654 & ACAATGCTTTTGACTACTGGGGAAAAGGAACTACGGTAACAGTAACATCAG & NAFDYWGKGTTVTVTS & 276704 & 51 & + \\ 
  IGHJM04 & 276999 & ACTACGCTTTTGACTACTGGGGAAAAGGAACAATGGTCACTGTCACTTCAG & YAFDYWGKGTMVTVTS & 277049 & 51 & + \\ 
  IGHJM05 & 277322 & ACAACTGGGCTTTTGACTACTGGGGAGCAGGAACCATGGTAACAGTAACATCAG & NWAFDYWGAGTMVTVTS & 277375 & 54 & + \\ 
  IGHJM06 & 277672 & CTACGGTGCTTTTGACTACTGGGGTAAAGGGACTACAGTCACCGTCACTTCAG & YGAFDYWGKGTTVTVTS & 277724 & 53 & + \\ 
  IGHJM07 & 278150 & CTACGATGCTTTTGACTATTGGGGGAAAGGAACAACAGTCACCGTCATCACTTCAG & YDAFDYWGKGTTVTVITS & 278205 & 56 & + \\ 
  IGHJM08 & 278606 & TTACTACTACGCTTTTGACTATTGGGGAAAAGGGACAATGGTCACCGTCACTTCAG & YYYAFDYWGKGTMVTVTS & 278661 & 56 & + \\ 
   \bottomrule \end{tabular}

		\normalsize\vspace{0.6em}
        \captionof{table}{Co-ordinate table of \jh segments in the \xma \igh{} locus.}
        \label{tab:xma-jh-coords-seg}
        \notsotiny
		% latex table generated in R 3.5.2 by xtable 1.8-3 package
% Tue Jan  8 15:33:35 2019
\begin{tabular}{lrlrlrr}
  \toprule Name & RSS Start & Nonamer & Spacer Length & Heptamer & RSS End & RSS Length \\ 
  \midrule IGHJZ01 & 2662 & TGTTTTTGT & 23 & CACTGTG & 2652 & 39 \\ 
  IGHJZ02 & 120651 & TGTTTTTGT & 23 & CACTGTG & 120638 & 39 \\ 
  IGHJZ03 & 130388 & TGTTTTTGT & 23 & CACCGTG & 130375 & 39 \\ 
  IGHJZ04 & 158416 & GGTTTTTGT & 23 & CACTGTG & 158407 & 39 \\ 
  IGHJZ05 & 198198 & GGTTTTTGT & 23 & CACTGTG & 198185 & 39 \\ 
  IGHJZ06 & 222426 & TGTTTTTGT & 23 & CACTGTG & 222416 & 39 \\ 
  IGHJZ07 & 254139 & GGTTTTTGT & 23 & CACTGTG & 254129 & 39 \\ 
  IGHJM01 & 276026 & TGTATTTGT & 23 & CACTGTG & 276013 & 39 \\ 
  IGHJM02 & 276295 & TATTTTTGC & 23 & CACCGTG & 276283 & 39 \\ 
  IGHJM03 & 276666 & TGTTTTTGT & 23 & TACTGTG & 276653 & 39 \\ 
  IGHJM04 & 277011 & TGTTTTAGT & 23 & TACTGTG & 276998 & 39 \\ 
  IGHJM05 & 277338 & GGTTTTTGT & 22 & TACTGTG & 277321 & 38 \\ 
  IGHJM06 & 277687 & GCTTTTTAT & 22 & CACTGTG & 277671 & 38 \\ 
  IGHJM07 & 278168 & CCTTTTTAC & 22 & CACTGTG & 278149 & 38 \\ 
  IGHJM08 & 278624 & GCTTTTTAA & 22 & CACTGTG & 278605 & 38 \\ 
   \bottomrule \end{tabular}

		\normalsize\vspace{0.6em}
        \captionof{table}{Co-ordinate table of \jh RSSs in the \xma \igh{} locus.}
        \label{tab:xma-dh-coords-rss}
    \end{landscape}

	\begin{landscape}
	\centering
	\vspace*{\fill}
    \scriptsize
    \begin{threeparttable}
    % latex table generated in R 3.5.2 by xtable 1.8-3 package
% Mon Jan 14 12:10:23 2019
\begin{tabular}{>{\itshape}lllllllp{4cm}}
  \toprule \textnormal{\textbf{Species}} & \textbf{Scaffold(s)} & \textbf{Region} & \textbf{Isotype} & \textbf{Known Exons} \tnote{1} & \textbf{Complete?} & \textbf{Pseudo-exons} & \textbf{Comments} \\ 
  \midrule Nothobranchius orthonotus & scf33878 & IGHM1 & M & 1,2,3,TM1 & \textbf{No} & -- & CM4 missing (missing sequence) \\ 
  Nothobranchius orthonotus & scf33878 & IGHD1 & D & 1,2,3,4,2,3,4,5,6,7,TM1 & Yes & -- &  \\ 
  Nothobranchius orthonotus & scf34438 & IGHM2 & M & 1,2,3,4,TM1 & Yes & -- &  \\ 
  Nothobranchius orthonotus & scf34438, scf33917 & IGHD2 & D & 1,2,3,4,2,3,4,5,6,7,TM1 & Yes & -- &  \\ 
  Nothobranchius orthonotus & scf33917 & IGHD3 & D & 1,2,3,4,2,3,4,5,6,7,TM1 & Yes & -- &  \\ 
  Nothobranchius orthonotus & scf33917 & IGHD4 & D & 1,2,3,4,2,3,4,5,6,7,TM1 & Yes & -- &  \\ 
  Nothobranchius orthonotus & scf9255, scf26119, scf33917 & IGHD5 & D & 3,4,2,3,4,5,6,7,TM1 & \textbf{No} & -- & CD1 \& CD2A missing (missing sequence) \\ 
  Nothobranchius orthonotus & scf27951, scf33789 & IGHM3 & M & 1,2,3,4,TM1 & Yes & -- &  \\ 
  Nothobranchius orthonotus & scf27951, 32033 & IGHD6 & D & 1,2,3,4,2,3,4,5,6,7,TM1 & Yes & -- &  \\ 
  Nothobranchius orthonotus & scf32137, scf21286 & IGHM4 & M & 1,2,3,4,TM1 & Yes & -- &  \\ 
  Nothobranchius furzeri & chr6 \tnote{2} & IGH1M & M & 1,2,3,4,TM1 & Yes & -- &  \\ 
  Nothobranchius furzeri & chr6 \tnote{2} & IGH1D & D & 1,2,3,4,2,3,4,5,6,7,TM1 & Yes & -- &  \\ 
  Nothobranchius furzeri & chr6 \tnote{2} & IGH2M & M & 1,2,3,4,TM1 & Yes & -- &  \\ 
  Nothobranchius furzeri & chr6 \tnote{2} & IGH2D & D & 1,2,3,4,2,3,4,5,6,7,TM1 & Yes & -- &  \\ 
  Aphyosemion australe & scf373 & IGHM & M & 1,2,3,4,TM1 & Yes & -- &  \\ 
  Aphyosemion australe & scf373 & IGHD & D & 1,2,3,4,5,6,7,TM1 & Yes & -- &  \\ 
  Callopanchax toddi & scf107 & IGHZ1 & Z & 1,2,3,4,TM1 & Yes & -- &  \\ 
  Callopanchax toddi & scf107 & IGHZ2 & Z & 1,2,3,4,TM1 & Yes & -- &  \\ 
  Callopanchax toddi & scf1209 & IGHZ3 & Z & 1,2,3,4,TM1 & Yes & -- &  \\ 
  Callopanchax toddi & scf1209 & IGHM1 & M & 1 & \textbf{No} & -- & Isolated CM1 exon \\ 
  Callopanchax toddi & scf945 & IGHZ4 & Z & 1,2,3,4,TM1 & Yes & -- &  \\ 
  Callopanchax toddi & scf945 & IGHM2 & M & 1,2,3,4,TM1 & Yes & -- &  \\ 
  Callopanchax toddi & scf945 & IGHD1 & D & 1,2,3,4,5,6,7,TM1 & Yes & 1,4,5 & Frameshift mutations in CD1, CD4 \& CD5 \\ 
  Callopanchax toddi & scf265 & IGHM3 & M & 1,2,3,4,TM1 & Yes & -- &  \\ 
  Callopanchax toddi & scf265 & IGHD2 & D & 1,5,7,TM1 & \textbf{No} & -- & CD2-4 \& CD5-6 missing (not in sequence) \\ 
   \bottomrule \end{tabular}

	\begin{tablenotes}
	\item[1] Excluding TM2 and secretory exons.
	\item[2] Expanded \igh{} locus sequency from \Cref{sec:nfu-locus}.
	\end{tablenotes}
	\end{threeparttable}
	\normalsize\vspace{1em}
    \captionof{table}{\igh{} constant regions in cyprinidontiform fish, part 1.}
	\label{tab:multispecies-ch-regions-1}
    \vspace*{\fill}
    \end{landscape}

	\begin{landscape}
	\centering
	\vspace*{\fill}
    \scriptsize
    \begin{threeparttable}
    \begin{tabular}{>{\itshape}lllllllp{4cm}}
  \toprule \textnormal{\textbf{Species}} & \textbf{Scaffold(s)} & \textbf{Region} & \textbf{Isotype} & \textbf{Known Exons} \tnote{1} & \textbf{Complete?} & \textbf{Pseudo-exons} & \textbf{Comments} \\ 
  \midrule Pachypanchax playfairii & scf547 & IGHZ & Z & 1,2,3,4,TM1 & Yes & -- &  \\ 
  Pachypanchax playfairii & scf125 & IGHM1 & M & 1,2,3,4,TM1 & Yes & -- &  \\ 
  Pachypanchax playfairii & scf125 & IGHD & D & 1,2,3,4,5,6,7,TM1 & Yes & -- &  \\ 
  Pachypanchax playfairii & scf547 & IGHM2 & M & 1 & \textbf{No} & -- & Isolated CM1 exon \\ 
  Austrofundulus limnaeus & NW\_013954375.1 & IGHZ & Z & TM1 & \textbf{No} & TM1 & Isolated TM1 exon with frameshift mutation \\ 
  Austrofundulus limnaeus & NW\_013952673.1 & IGHM & M & 1,2,3,4,TM1 & Yes & -- &  \\ 
  Austrofundulus limnaeus & NW\_013952673.1, NW\_013956335.1 & IGHD & D & 1,2,3,4,5,6,7,TM1 & Yes & -- &  \\ 
  Kryptolebias marmoratus & NW\_016094348.1 & IGHZ1 & Z & 1,2,3,4,TM1 & Yes & -- &  \\ 
  Kryptolebias marmoratus & NW\_016094348.1 & IGHZ2 & Z & 1,4,TM1 & \textbf{No} & -- & CZ2 \& CZ3 missing (not in sequence) \\ 
  Kryptolebias marmoratus & NW\_016094301.1 & IGHM1 & M & 1,2,3,4,TM1 & Yes & -- &  \\ 
  Kryptolebias marmoratus & NW\_016094301.1 & IGHD1 & D & 1,2,3,4,5,6,7,TM1 & Yes & -- &  \\ 
  Kryptolebias marmoratus & NW\_016094277.1 & IGHM2 & M & 1,2,3,4,TM1 & Yes & -- &  \\ 
  Kryptolebias marmoratus & NW\_016094277.1 & IGHD2 & D & 1,2,3,4,5,6,TM1 & \textbf{No} & -- & CD7 missing (not in sequence) \\ 
  Poecilia reticulata & NC\_024338.1 & IGHZ1 & Z & 1,2,3,4 & \textbf{No} & -- & TM1 missing (missing sequence) \\ 
  Poecilia reticulata & NC\_024338.1 & IGHZ2 & Z & 1,2,3,4,TM1 & Yes & -- &  \\ 
  Poecilia reticulata & NC\_024338.1 & IGHM & M & 1,2,3,4,TM1 & Yes & -- &  \\ 
  Poecilia reticulata & NC\_024338.1 & IGHD & D & 1,2,3,4,2,3,4,5,6,7,TM1 & Yes & -- &  \\ 
  Poecilia formosa & NW\_006800081.1 & IGHZ1 & Z & 1,2,3,4,TM1 & Yes & -- &  \\ 
  Poecilia formosa & NW\_006800081.1 & IGHZ2 & Z & 1,2,3,4,TM1 & Yes & -- &  \\ 
  Poecilia formosa & NW\_006800081.1 & IGHZ3 & Z & 1,2,3,4,TM1 & Yes & -- &  \\ 
  Poecilia formosa & NW\_006800081.1 & IGHM & M & 1,2,3,4,TM1 & Yes & -- &  \\ 
  Poecilia formosa & NW\_006800081.1 & IGHD & D & 1,2,3,4,5,6,7,TM1 & Yes & -- &  \\ 
  Xiphophorus maculatus & NC\_036458 & IGHZ1 & Z & 1,2,3,4,TM1 & Yes & -- &  \\ 
  Xiphophorus maculatus & NC\_036458 & IGHZ2 & Z & 1,2,3,4,TM1 & Yes & -- &  \\ 
  Xiphophorus maculatus & NC\_036458 & IGHM & M & 1,2,3,4,TM1 & Yes & -- &  \\ 
   \bottomrule \end{tabular}

	\begin{tablenotes}
	\item[1] Excluding TM2 and secretory exons.
	\end{tablenotes}
	\end{threeparttable}
	\normalsize\vspace{1em}
    \captionof{table}{\igh{} constant regions in cyprinidontiform fish, part 2.}
	\label{tab:multispecies-ch-regions-2}
    \vspace*{\fill}
    \end{landscape}

	\begin{landscape}
	\centering
	\vspace*{\fill}
    \scriptsize
    \begin{threeparttable}
    % latex table generated in R 3.5.2 by xtable 1.8-3 package
% Mon Jan 14 12:10:24 2019
\begin{tabular}{>{\itshape}lllllllp{4cm}}
  \toprule \textnormal{\textbf{Species}} & \textbf{Scaffold(s)} & \textbf{Region} & \textbf{Isotype} & \textbf{Known Exons} \tnote{1} & \textbf{Complete?} & \textbf{Pseudo-exons} & \textbf{Comments} \\ 
  \midrule Xiphophorus maculatus & NC\_036458 & IGHD & D & 1,2,3,4,2,3,4,5,6,7,TM1 & Yes & -- &  \\ 
  Fundulus heteroclitus & NW\_012234561.1 & IGHZ1 & Z & 1,2,3,4,TM1 & Yes & -- &  \\ 
  Fundulus heteroclitus & NW\_012230737.1 & IGHZ2 & Z & 4,TM1 & \textbf{No} & -- & CZ1 to CZ3 missing (missing sequence) \\ 
  Fundulus heteroclitus & NW\_012234542.1 & IGHM & M & 1,2,3,4,TM1 & Yes & -- &  \\ 
  Fundulus heteroclitus & NW\_012234542.1 & IGHD & D & 1,2,3,4,2,3,4,5,6,7,TM1 & Yes & -- &  \\ 
  Cyprinodon variegatus & NW\_015154250.1, NW\_015151047.1 & IGHZ & Z & 1,2,3,4,TM1 & Yes & -- &  \\ 
  Cyprinodon variegatus & NW\_015151047.1 & IGHM & M & 1,2,3,4,TM1 & Yes & -- &  \\ 
  Cyprinodon variegatus & NW\_015151047.1 & IGHD & D & 1,2,3,4,2,3,4,5,6,7,TM1 & Yes & -- &  \\ 
  Oryzias latipes & NC\_019866.2 & IGHM1 & M & 1,2,3,4,TM1 & Yes & -- &  \\ 
  Oryzias latipes & NC\_019866.2 & IGHD1 & D & 1,2,3,4,6,7,TM1 & Yes & 7 & Nonsense mutation in CD7 \\ 
  Oryzias latipes & NC\_019866.2 & IGHM2 & M & 1,2,3,4,TM1 & Yes & -- &  \\ 
  Oryzias latipes & NC\_019866.2 & IGHD2 & D & 1,2,3,4,6,7,TM1 & Yes & -- &  \\ 
  Oryzias latipes & NC\_019866.2 & IGHM3 & M & 1,2,3,4,TM1 & Yes & -- &  \\ 
  Oryzias latipes & NC\_019866.2 & IGHD3 & D & 1,2,3,4,6,7,TM1 & Yes & -- &  \\ 
  Oryzias latipes & NC\_019866.2 & IGHM4 & M & 1,2,3,4,TM1 & Yes & -- &  \\ 
  Oryzias latipes & NC\_019866.2 & IGHD4 & D & 2,7,TM1 & \textbf{No} & -- & CD1 \& CD3-6 missing (not in sequence) \\ 
  Oryzias latipes & NC\_019866.2 & IGHM5 & M & 1,2,3,4,TM1 & Yes & -- &  \\ 
  Oryzias latipes & NC\_019866.2 & IGHD5 & D & 1,2,3,4,6,7,TM1 & Yes & -- &  \\ 
  Oryzias latipes & NC\_019866.2 & IGHM6 & M & 1,2,3,4,TM1 & Yes & -- &  \\ 
  Oryzias latipes & NC\_019866.2 & IGHD6 & D & 1,2,3,4,6,7,TM1 & Yes & -- &  \\ 
  Oryzias latipes & NC\_019866.2 & IGHD7 & D & 1,2,3,6 & \textbf{No} & -- & CD4, CD5, CD7 and TM1 missing (not in sequence) \\ 
   \bottomrule \end{tabular}

	\begin{tablenotes}
	\item[1] Excluding TM2 and secretory exons.
	\end{tablenotes}
	\end{threeparttable}
	\normalsize\vspace{1em}
    \captionof{table}{\igh{} constant regions in cyprinidontiform fish, part 3.}
	\label{tab:multispecies-ch-regions-3}
    \vspace*{\fill}
    \end{landscape}

% 4 - IGSEQ CHAPTER

\begin{table}
\caption{Turquoise killifish used in \igseq validation and ageing experiment. All fish are GRZ-AD strain and male.}
\label{tab:igseq-cohorts-fish}
\begin{threeparttable}
% latex table generated in R 3.5.2 by xtable 1.8-3 package
% Wed Jan 30 11:19:43 2019
\begin{tabular}{rrrrllrr}
  \toprule Group & \# & Fish ID\tnote{1} & Death weight (g) & Hatch date & Sacrifice date & Age (days) & Age (weeks) \\ 
  \midrule 1 & 1 & 4194 & 1.24 & 2016-05-09 & 2016-06-17 &  39 &  5.57 \\ 
  1 & 2 & 4107 & 1.39 & 2016-05-09 & 2016-06-17 &  39 &  5.57 \\ 
  1 & 3 & 4127 & 1.29 & 2016-05-09 & 2016-06-17 &  39 &  5.57 \\ 
  1 & 4 & 4204 & 1.35 & 2016-05-09 & 2016-06-17 &  39 &  5.57 \\ 
  1 & 5 & 4189 & 1.43 & 2016-05-09 & 2016-06-17 &  39 &  5.57 \\ 
  1 & 6 & 4160 & 0.68 & 2016-05-09 & 2016-06-17 &  39 &  5.57 \\ 
  1 & 7 & 4164 & 1.57 & 2016-05-09 & 2016-06-17 &  39 &  5.57 \\ 
  1 & 8 & 4171 & 1.40 & 2016-05-09 & 2016-06-17 &  39 &  5.57 \\ 
  1 & 9 & 4200 & 1.42 & 2016-05-09 & 2016-06-17 &  39 &  5.57 \\ 
  1 & 10 & 4131 & 1.27 & 2016-05-09 & 2016-06-17 &  39 &  5.57 \\\midrule
  2 & 1 & 4159 & 1.37 & 2016-05-09 & 2016-07-04 &  56 &  8.00 \\ 
  2 & 2 & 4179 & 1.47 & 2016-05-09 & 2016-07-04 &  56 &  8.00 \\ 
  2 & 3 & 4152 & 1.33 & 2016-05-09 & 2016-07-04 &  56 &  8.00 \\ 
  2 & 4 & 4132 & 1.35 & 2016-05-09 & 2016-07-04 &  56 &  8.00 \\ 
  2 & 5 & 4177 & 1.22 & 2016-05-09 & 2016-07-04 &  56 &  8.00 \\ 
  2 & 6 & 4158 & 1.51 & 2016-05-09 & 2016-07-04 &  56 &  8.00 \\ 
  2 & 7 & 4182 & 1.12 & 2016-05-09 & 2016-07-04 &  56 &  8.00 \\ 
  2 & 8 & 4202 & 1.54 & 2016-05-09 & 2016-07-04 &  56 &  8.00 \\ 
  2 & 9 & 4143 & 1.28 & 2016-05-09 & 2016-07-04 &  56 &  8.00 \\ 
  2 & 10 & 4201 & 1.55 & 2016-05-09 & 2016-07-04 &  56 &  8.00 \\\midrule
  3 & 1 & 4155 & 2.06 & 2016-05-09 & 2016-07-21 &  73 & 10.43 \\ 
  3 & 2 & 4193 & 1.92 & 2016-05-09 & 2016-07-21 &  73 & 10.43 \\ 
  3 & 3 & 4170 & 1.80 & 2016-05-09 & 2016-07-21 &  73 & 10.43 \\ 
  3 & 4 & 4135 & 1.65 & 2016-05-09 & 2016-07-21 &  73 & 10.43 \\ 
  3 & 5 & 4190 & 1.87 & 2016-05-09 & 2016-07-21 &  73 & 10.43 \\ 
  3 & 6 & 4099 & 1.94 & 2016-05-09 & 2016-07-21 &  73 & 10.43 \\ 
  3 & 7 & 4198 & 1.49 & 2016-05-09 & 2016-07-21 &  73 & 10.43 \\ 
  3 & 8 & 4024 & 1.73 & 2016-05-09 & 2016-07-21 &  73 & 10.43 \\ 
  3 & 9 & 4044 & 1.53 & 2016-05-09 & 2016-07-21 &  73 & 10.43 \\ 
  3 & 10 & 4117 & 1.57 & 2016-05-09 & 2016-07-21 &  73 & 10.43 \\\midrule
  4 & 1 & 4173 & 2.20 & 2016-05-09 & 2016-09-14 & 128 & 18.29 \\ 
  4 & 2 & 4197 & 2.40 & 2016-05-09 & 2016-09-14 & 128 & 18.29 \\ 
   \bottomrule \end{tabular}

\begin{tablenotes}
\item[1] grz-AD\_...\_E
\end{tablenotes}
\end{threeparttable}
\end{table}

