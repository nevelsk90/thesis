\chapter{Hill numbers and antibody repertoire diversity}
\label{app:diversity}

\normalsize

The question of how to measure the diversity of clones or sequences within an antibody repertoire, as well as the degree of divergence in composition between pairs or groups of repertoires, closely parallels related questions in ecology, information theory, and elsewhere. Over time, a great many different conceptions of diversity have been developed in these fields \citep{peet1974diversity}, each with its own cohort of measurement and approximation methods. Broadly speaking, these measures can be grouped into those that consider the diversity of a single, unitary population, and those that consider the diversity \textit{structure} of a group of related populations...

\section{Diversity in unitary populations}
\label{sec:diversity-unitary}

\subsection{Terminology}
\label{sec:diversity-unitary-terminology}

Though in this appendix I use terminology derived from the ecological literature, the concepts and measured discussed here can be applied to any situation in which a set (or \textit{population}) of elements (\textit{individuals}) is divided among some number of mutually-exclusive categories (\textit{species}). Considered abstractly, these terms and denotations could be used to refer to coloured balls in an urn, species in a rainforest, or sequences in a repertoire.

Let $X$ be a unitary population of individuals, each of which is assigned to some species $s$ from a set of possible species $S$. ``Unitary'' denotes that $X$ has no internal structure except the species identity of its constitutent individuals. Let $n_s$ denote the number of individuals in $X$ belonging to $s$, and $N$ denote the total number of individuals in $X$. Then 

\begin{equation}
p_s = \frac{n_s}{N}
\label{eq:species_proportion}
\end{equation}

\noindent denotes the proportion (or relative frequency) of individuals in $X$ belonging to $s$, or equivalently the probability that a randomly-selected individual from $X$ belongs to $s$. The \textit{species richness} of $P$ is the total number of species $|S|$, while the \textit{evenness} of $P$ is the degree to which different species are similar in their $p_s$ values: a population containing one very abundant species and $n-1$ very rare species, for example, is much less even than a population containing $n$ equally-abundant species.

\subsection{Simple diversity indices}
\label{sec:diversity-unitary-simple}

The diversity of a unitary population $X$ of some total size $N$ is generally considered to increase with both the richness and evenness (\Cref{sec:diversity-terminology}) of $X$. Different measures of within-population diversity, however, place different amounts of weight on the richness of $X$ compared to its evenness when evaluating its diversity. At one extreme, the species richness itself is used as a measure of diversity, albeit one that entirely ignores evenness. Other commonly-used diversity measures, such as Simpson's index, Shannon entropy, and the Berger-Parker index % Citations needed
are affected to varying degrees by both richness and evenness.

\subsubsection{Simpson's index}

One of the oldest measures incorporating both species richness and evenness is Simpson's index, which measures the probability that two randomly selected individuals from a population are from the same species \citep{simpson1949diversity}. For a finite population (or sample) $X$ with a set of possible species $S$, Simpson's index is calculated as follows:

\begin{equation}
L(X) = \frac{\sum_{s \in S} n_s(n_s-1)}{N(N-1)}
\label{eq:simpson_finite}
\end{equation} 

\noindent As the population size tends to infinity, this value simplifies to

\begin{equation}
L^*(X) = \frac{\sum_{s \in S} n_s^2}{N^2} = \sum_{s \in S} p_s^2
\label{eq:simpson_finite}
\end{equation} 

\noindent As originally formulated, Simpson's index can be considered a dominance index (also known as a concentration  index \citep{simpson1949diversity}), measuring the degree to which a population is dominated by a small number of species; as such, a higher Simpson's index indicates a less-diverse population. Conversely, the complement $1-L(P)$ of Simpson's index, known as the Gini-Simpson index \citep{jost2006entropy}, represents the ``probability of interspecific encounter" \citep{peet1974diversity}, and is widely used as a diversity index in the literature.

\subsubsection{The Berger-Parker index}

The Berger-Parker index is a very simple measure of diversity, given by the relative frequency $p_x$ of the most abundant species $x$ in the population \parencite{berger1970diversity,caruso2008bergerparker}. Like Simpson's index, the Berger-Parker is a dominance index, measuring the degree to which the population is dominated by the single largest species. Despite its simplicity, the Berger-Parker responds strongly to the evenness of a population, as well as responding weakly to its richness, and can be used to distinguish different diversity structures in real populations \parencite{caruso2008bergerparker}. As it focuses only on the most common species in each population, it also has the particular advantage of being much less vulnerable to sampling bias than other diversity metrics, especially compared to the species richness itself \parencite{berger1970diversity}.

\subsubsection{Entropic diversity indices}
\label{sec:diversity-unitary-simple-entropy}

In information theory, the Shannon entropy $H(Y)$ of a random variable $Y$ provides a measure of the unpredictability of that variable, and therefore of the degree to which its value can be predicted in advance \citep{shannon1948communication1}. A variable which can take only a single value has zero entropy (it is perfectly predictable), while one which can take all values in its state space with equal probability has maximum entropy (it is perfectly unpredictable) for that state space. All else being equal, a variable which can take a larger number of possible values has greater entropy than one which can take fewer values; hence, the entropy of a variable increases with both the richness and evenness of its state space \citep{shannon1948communication1}.

Although the concept of Shannon entropy was developed in the context of electronic communication, it can be extended quite naturally to ecology, where the process of sampling species from a population represents an  information source and each species observation represents an output. In this case, the Shannon entropy represents  ``the amount of uncertainty that exists regarding the species of an individual selected at random from the population" \citep{peet1974diversity}. The Shannon entropy of a population $X$ with possible species $S$ is given by

\begin{equation}
H(X) = -\sum_{s \in S} p_s \cdot \log p_s
\label{eq:shannon_infinite}
\end{equation}

\noindent Any base logarithm can be used, though bases $2$ and $e$ are most common; in these bases, the units of Shannon entropy are the bit (``binary digit") and nat (``natural unit"), respectively.

A generalisation of the Shannon entropy, also used as a diversity measure, is the R\'{e}nyi entropy \parencite{mora2016renyi}:

\begin{equation}
H_q(X) = \frac{1}{1-q}\,\log\right[\sum_{s \in S} p_i^q\right]
\label{eq:renyi}
\end{equation}

\noindent where $q$ is denoted the \textit{order} of the entropy measure. The R\'{e}nyi entropy is undefined at $q = 1$, but reduces to the Shannon entropy (with the same base of logarithm) in the limit as $q \to 1$ \parencite{mora2016renyi}. As $p_i$ is always between 0 and 1, raising the relative frequencies to a power greater than 1 downweight rarer species (with smaller $p_i$) relative to more common ones; as a result, higher-order R\'{e}nyi entropies put less weight on species richness compared to evenness when used as diversity measures. Conversely, R\'{e}nyi entropies with order less than 1 put more weight on richness and less on evenness than the Shannon entropy; at $q = 0$, $H_0$ is simply the logarithm of the species richness $|S|$.

\subsection{Effective species richness and true diversity}
\label{sec:diversity-unitary-hill}

\Cref{sec:diversity-unitary-simple} discusses several commonly-used diversity indices, including Simpson's index, the Gini-Simpson index, Shannon entropy (\& other R\'{e}nyi entropies), and the Berger-Parker index; many other diversity indices are possible. These indices differ importantly in their forms, numeric ranges and biological interpretation, as well as their response behaviours to different changes in population composition \citep{peet1974diversity, jost2006entropy}. For example, Simpson's index and the Berger-Parker index range from 0 to 1, with lower values indicating greater population diversity, while Shannon- and R\'{e}nyi-entropy measures range from 0 to infinity, with higher values indicating greater diversity (in log-scale). The use of different diversity indices can therefore yield importantly different results when used to compare different populations: comparing two populations using only one such measure will capture only part of the diversity structure of those populations, but comparing them using two or more raw indices will often yield results that are difficult to accurately interpret \citep{peet1974diversity, jost2006entropy}.

Fortunately, different diversity indices can be transformed into a common framework by considering, for each index, the number $D$ of \textit{equally-common} species that would be needed to produce the same diversity value. This transformation gives an estimation, for each index, of the ``effective species richness" of the population: if we corrected for differences in species abundance while holding the diversity index constant, how rich would the resulting population be? This effective richness can itself be considered a diversity measure; one that is comparable across diversity indices in a way the raw indices are not \citep{jost2006entropy}.

For many important diversity indices, the equation for the effective richness $D$ takes a common form, known as the Hill number or ``true diversity" for some parameter $q$ \citep{hill1973diversity, jost2006entropy}:

\begin{equation}
^qD(X) = \left(\sum_{s \in S} p_s^q \right)^{\frac{1}{1-q}}
\label{eq:hill_index}
\end{equation}

\noindent Indices whose effective richnesses take this form include species richness ($q=0$), R\'{e}nyi entropy in base $e$ ($q=q$), Shannon entropy in base $e$ ($q \to 1$), Simpson's index ($q = 2$) and the Berger-Parker index ($q \to \infty$) among others \citep{peet1974diversity, hill1973diversity, jost2006entropy, miho2018strategies}; see \autoref{tab:diversity} for more details of their respective transformation functions. As with the R\'{e}nyi entropy, the parameter $q$, also known here as the \textit{diversity order}, describes the degree to which rare species are downweighted compared to common ones when calculating the Hill number: at one extreme ($q=0$, the species richness), all species are given equal weight regardless of their frequency, while at the other ($q \to \infty$, the reciprocal Berger-Parker index) only the most common species in the sample is considered. As a result, higher diversity orders are proportionally less sensitive to species richness and more sensitive to evenness when computing the diversity of a population. % TODO: Check this in the literature; cite it or drop it

The use of Hill numbers in diversity measurement enables many different, widely-used diversity measurements to be considered and compared in a common framework, giving a description of population diversity that is easy to interpret biologically and compare across different populations \parencite{jost2006entropy}. Typically, the limit at $q \to 1$ is substituted for the (undefined) value at $q = 1$, giving a function that is continuous and monotonically decreasing:

\begin{equation}
^qD(X) = \begin{cases} \left(\displaystyle\sum_{s \in S} p_s^q \right)^{\frac{1}{1-q}} & q \neq 1\\
\exp\left(-\displaystyle\sum_{s \in S}p_s \cdot \ln p_s\right) & q = 1 \end{cases}
\label{eq:hill_index_continuous}
\end{equation}

\noindent This equantion can then be easily used to construct diversity profiles (or \textit{spectra}) spanning many different orders of diversity \citep{miho2018strategies}; since each value of $q$ captures a different aspect of a population's diversity structure, these profiles can be much more informative than any single metric when analysing and comparing the diversity structure of populations.

\begin{table}
\centering
\caption{Summary of effective richness measures for some common diversity indices (adapted and expanded from \citep{jost2006entropy}}
\begin{tabular}{lll}\toprule
Diversity index $f(X)$ & Effective species richness~$^qD(X)$ & Diversity order $q$\\\midrule
Species richness & $^0D(X) = f(X)$ & 0 \\
Shannon entropy (base $e$) & $\lim_{q \to 1}~^qD(X) = \exp\,f(X)$ & 1\\
Simpson index & $^2D(X) = \frac{1}{f(X)}$ & 2\\
Gini-Simpson index & $^2D(X) = \frac{1}{1-f(X)}$ & 2\\
Renyi entropy (base $e$, order $q$) & $^qD(X) = \exp\,f(X)$ & $q$\\
Berger-Parker index & $\lim_{q \to \infty}~^qD(X) = \frac{1}{f(X)}$ & $\infty$\\
\bottomrule
\end{tabular}
\label{tab:diversity}
\end{table}

Finally, it is worth noting here that diversity measures of different orders differ in their ... % TODO: Discuss sample-size issues at lower orders

\section{Diversity in structured populations}
\label{sec:diversity-structured}

Section \ref{sec:diversity-unitary} discusses methods for analysing the diversity of a single, unitary population. In many cases, however, we are interested in groups of related populations, and the relative amount of variability within and between the populations in each group. In order to apply the mathematical framework of diversity measurement, and especially that of true diversities and Hill spectra, to this more complex case...

%, each of which shares with $Y$ the set of possible species $S$ (though some may contain zero individuals of some or many species in $S$). 
%
% in terms, ultimately, of the effective number of equally-abundant species it contains. These metrics, in addition to measuring the diversity of individual populations, can also be used to compare the diversity of different populations. This direct sample-by-sample comparison of diversity measures is often of relevance in repertoire sequencing, where the diversity of a repertoire library provides important information about clonal-expansion dynamics or the underlying generative process. 
%
%While the metrics discussed in \ref{sec:within} can be used to compare the \textit{diversity} of different populations, they are not sufficient for comparing the \textit{composition} of these populations or measuring the degree of similarity between them. % Overlap, homogeneity, ...

\subsection{Terminology}
\label{sec:diversity-structured-terminology}

In a \textit{structured} population $C$, the individuals in $C$ can be partitioned into some number $M$ of disjoint unitary subpopulations $X_1, X_2, \dots$, such that:

\begin{itemize} % Itemize or enumerate?
\item Each subpopulation $X$ has size $N_X$, with the total size of the whole population given by $N = \sum_{X \in C}~N_X$.
\item Each individual in a subpopulation $X$ is assigned to a species $s$ drawn from a set of possible species $S_X$, with the total species set for the population given by $S = \bigcup_{X \in C}~S_X$.
\item Each subpopulation can be assigned a relative statistical weight $w_X$; this could be equal for all populations ($w_X = 1/M$), proportional to each population's relative size $w_X = \frac{N_X}{N}$, or proportional to some other measure of each population's ``importance'' to the system.
\item The relative frequency of a species $s$ in a subpopulation $X$ is given by $p_{X,s} = \frac{n_{X,s}}{N_X}$, where $n_{X,s}$ is the number of individuals in $X$ belonging to $s$.
\end{itemize}

\noindent What is the diversity of $C$? There are several possible answers, depending on which features of the makeup of $C$ are most salient:

\begin{enumerate}
\item The \textbf{gamma diversity} of $C$ is the \textit{total} diversity of the population when its subpopulations are pooled according to their weights; it represents the species diversity across the whole population, ignoring the subpopulation membership of individuals.
\item The \textbf{alpha diversity} of $C$ is the diversity arising from differences in species identity among individuals \textit{within} each subpopulation, and is given by a weighted \textit{average} of the unitary diversities of those subpopulations; the appropriate weighting function depends on the order of diversity under consideration. In some sense, the alpha-diversity of $C$ can be thought of as the expected diversity of a single population drawn from $C$.
\item The \textbf{beta diversity} of $C$ is the diversity arising from variability in species composition \textit{among} the subpopulations in $C$; it is lowest when all subpopulations have identical species compositions and highest when they have no species in common.
\end{enumerate}

\noindent The alpha and beta diversities of a population are independent; two different structured populations can have identical alpha- and very different beta-diversities, or vice versa, depending on the exact species compositions of their subpopulations. The alpha and beta diversity of a structured population also completely determine its gamma-diversity \citep{jost2007partitioning}:

\begin{equation}
D_\gamma(C) = D_\alpha(C) \times D_\beta(C)
\label{eq:diversity_relationship_gamma}
\end{equation}

and therefore 

\begin{equation}
D_\beta = \frac{D_\gamma}{D_\alpha}
\label{eq:diversity_relationship_beta}
\end{equation}

\noindent This (\autoref{eq:diversity_relationship_beta}) is typically the easiest way of computing the beta-diversity of a given collection of populations.

In the rest of this section, I will primarily consider only subpopulations with equal sizes $N_X = \frac{N}{M}$ and equal weights $w = \frac{1}{M}$, as this is how they are used in the repertoire-diversity methods discussed in \Cref{sec:methods_comp_igdownstream_spectra,chap:igseq}. Equal-sized and -weighted populations can be produced by downsampling each subpopulation to the same number of individuals (in the \igseq case, by downsampling each repertoire to the same number of unique sequences) prior to calculating the diversity of the population.
 
\subsubsection{Calculating alpha, beta, and gamma}
\label{sec:diversity-structured-calc}

The alpha-, beta- and gamma-diversities of structured populations can be calculated analogously to the unitary true diversities discussed in \Cref{sec:diversity-unitary-hill}. In this framework, the alpha-diversity represents the effective number of equally-abundant species present in the average subpopulation drawn from $C$, while the gamma-diversity represents the effective number of such species present in the population as a whole (ignoring subpopulation membership). Beta-diversity also represents an effective number of groupings, but in this case the unit is subpopulations rather than species: given an alpha-diversity value for $C$, the beta-diversity gives the number of equally-weighted subpopulations, with no species in common, that would give rise to the gamma-diversity of $C$.

Under this framework, the diversities of order $q$ for a structured population $C$ are given by

\begin{equation}
^qD_\alpha(C)
= \left[\frac{\displaystyle\sum_{X \in C} w_X^q \left(\sum_{s \in S_X} p_{X,s}^q\right)}{\displaystyle\sum_{X \in C} w_X^q}\right]^\frac{1}{1-q}
= \left[\frac{\displaystyle\sum_{X \in C}\sum_{s \in S_X} (w_Xp_{X,s})^q}{\displaystyle\sum_{X \in C} w_X^q}\right]^\frac{1}{1-q}
\label{eq:diversity_alpha}
\end{equation}

\begin{equation}
^qD_\gamma(C)
= \left[\sum_{s \in S} \left(\frac{\displaystyle\sum_{X \in C} w_Xp_{X,s}}{\displaystyle\sum_{X \in C} w_X}\right)^q\right]^\frac{1}{1-q}
= \left[\frac{\displaystyle\sum_{s \in S}\left(\sum_{X \in C} w_Xp_{X,s}\right)^q}{\displaystyle\left(\sum_{X \in C} w_X\right)^q}\right]^\frac{1}{1-q}
\label{eq:diversity_gamma}
\end{equation}

\begin{equation}
^qD_\beta(C) = \frac{^qD_\gamma(C)}{^qD_\alpha(C)}
= \left[
\frac{\displaystyle\sum_{s \in S}\left(\sum_{X \in C} w_Xp_{X,s}\right)^q}
{\displaystyle\sum_{X \in C}\sum_{s \in S_X} (w_Xp_{X,s})^q}
\times
\frac{\displaystyle\sum_{X \in C} w_X^q}
{\displaystyle\left(\sum_{X \in C} w_X\right)^q}
\right]^\frac{1}{1-q}
\label{eq:diversity_beta}
\end{equation}

When all the subpopulations are equally-weighted (i.e. $w_X = w = \frac{1}{M}$ for all subpopulations), these formulae simplify considerably:

\begin{equation}
^qD_\alpha(C)
= \left[\frac{\displaystyle\sum_{X \in C}\sum_{s \in S_X} (wp_{X,s})^q}{\displaystyle\sum_{X \in C} w^q}\right]^\frac{1}{1-q}
= \left[\frac{\displaystyle w^q\sum_{X \in C}\sum_{s \in S_X} (p_{X,s})^q}{\displaystyle M w^q}\right]^\frac{1}{1-q}
= \left[\frac{\displaystyle \sum_{X \in C}\sum_{s \in S_X} (p_{X,s})^q}{\displaystyle M}\right]^\frac{1}{1-q}
\label{eq:diversity_alpha_even}
\end{equation}

\begin{equation}
^qD_\gamma(C)
= \left[\frac{\displaystyle\sum_{s \in S}\left(\sum_{X \in C} wp_{X,s}\right)^q}{\displaystyle\left(\sum_{X \in C} w\right)^q}\right]^\frac{1}{1-q}
= \left[\frac{\displaystyle w^q\sum_{s \in S}\left(\sum_{X \in C} p_{P,s}\right)^q}{\displaystyle M^qw^q}\right]^\frac{1}{1-q}
= \left[\frac{\displaystyle\sum_{s \in S}\left(\sum_{X \in C} p_{X,s}\right)^q}{\displaystyle M^q}\right]^\frac{1}{1-q}
\label{eq:diversity_gamma_even}
\end{equation}

\begin{equation}
^qD_\beta(C) = \frac{^qD_\gamma(C)}{^qD_\alpha(C)}
= \left[
\frac{\displaystyle\sum_{s \in S}\left(\sum_{X \in C} p_{X,s}\right)^q}{\displaystyle \sum_{X \in C}\sum_{s \in S_X} (p_{X,s})^q}
\times
\frac{\displaystyle M}{\displaystyle M^q}
\right]^\frac{1}{1-q}
= \left[
\frac{\displaystyle\sum_{s \in S}\left(\sum_{X \in C} p_{X,s}\right)^q}{\displaystyle M^{q-1} \sum_{X \in C}\sum_{s \in S_X} (p_{X,s})^q}
\right]^\frac{1}{1-q}
\label{eq:diversity_beta_even}
\end{equation}

These equations are valid for all values of $q \in \mathbb{R}$, providing a spectrum of alpha- or beta-diversity measures analogous to the diversity spectra provided for unitary populations in \Cref{sec:diversity-unitary-hill}. As in the unitary case, a special case needs to be made for $q = 1$ in order to make these functions continuous\footnote{Note that, when $w = \frac{1}{M}$ and $N_X = \frac{N}{M}$ for all populations, $\displaystyle\sum_{X \in C}wp_{X,s} = \sum_{X \in C}\frac{1}{M}\frac{n_{X,s}}{N_X} = \frac{1}{M}\frac{M \sum_{X \in C} n_{X,s}}{N} = \frac{n_{s}}{N} = p_s$}:
% TODO: Citation needed

\begin{equation}
\begin{split}
^1D_\alpha(C) & = \lim_{q \to 1} {^qD}_\alpha(C)
= \exp\left[-\sum_{X \in C}w\sum_{s \in S_X}(p_{X,s}\cdot\ln p_{X,s})\right]\\
 & = \exp\left[\frac{1}{M}\sum_{X \in C}\left(-\sum_{s \in S_X}(p_{X,s}\cdot\ln p_{X,s})\right)\right] = \exp\left[\frac{1}{M}\sum_{X \in C}\ln\,^1D(X)\right]
\end{split}
\label{eq:diversity_alpha_q1}
\end{equation}

\begin{equation}
\begin{split}
^1D_\gamma(C) & = \lim_{q \to 1} {^qD}_\gamma(C)
= \exp\left[-\sum_{s \in S}\left(\left[\sum_{X \in C}wp_{X,s}\right]\cdot\ln \left[\sum_{X \in C}wp_{X,s}\right]\right)\right]\\
& = \exp\left(-\sum_{s \in S}p_s \cdot\ln p_s\right) =\:^1D(C)
\end{split}
\label{eq:diversity_gamma_q1}
\end{equation}

\begin{equation}
\begin{split}
^1D_\beta(C) & = \frac{^1D_\gamma(C)}{^1D_\alpha(C)} = \frac{\exp\left[-\sum_{s \in S}p_s \cdot\ln p_s\right]}{\exp\left[\frac{1}{M}\sum_{X \in C}\left(-\sum_{s \in S_X}(p_{X,s}\cdot\ln p_{X,s})\right)\right]} = \frac{\exp\left[\frac{1}{M}\sum_{X \in C}\ln\,^1D(X)\right]}{^1D(C)}
\end{split}
\label{eq:diversity_beta_q1}
\end{equation}

\subsubsection{Rescaling beta-diversity}
\label{sec:diversity-structured-rescaling}

% TODO: Discuss limits of alpha and gamma: 0 to ??

As discussed in \Cref{sec:diversity-structured-calc}, while alpha- and gamma-diversity are expressed in terms of an effective number of species (in an average subpopulation and the entire structured population, respectively), beta-diversity is expressed in terms of an effective number of subpopulations. Since the effective number of subpopulations is determined in part by the actual number of subpopulations, this means that the beta-diversity, unlike alpha- and gamma-diversity, is directly dependent on the number of subpopulations $M$. If two different structured populations contain different numbers of subpopulations, it is therefore not possible to compare their beta-diversity values directly; rather, the beta-spectra of the populations must be rescaled to a common range before such a comparison is performed.

The \textit{minimum} beta-diversity of a structured population obtains when all subpopulations have identical species competition (i.e. for all species $s$ and subpopulations $X$, $p_{X,s} = p_{s}$). In this case, the beta-diversity for the structured population is given by: % TODO: Explain this better

\begin{equation}
^qD_{\beta\min}(C) = \left[
\frac{\displaystyle\sum_{s \in S}\left(\sum_{X \in C} p_{X,s}\right)^q}{\displaystyle M^{q-1} \sum_{X \in C}\sum_{s \in S_X} (p_{X,s})^q}
\right]^\frac{1}{1-q}
= \left[
\frac{\displaystyle\sum_{s \in S}\left(Mp_s\right)^q}{\displaystyle M^{q-1}\sum_{s \in S_X} Mp_s^q}
\right]^\frac{1}{1-q}
= \left[
\frac{M^q\displaystyle\sum_{s \in S}p_s^q}{M^q\displaystyle\sum_{s \in S_X} p_s^q}
\right]^\frac{1}{1-q} = 1
\label{eq:diversity_beta_min}
\end{equation}

\noindent The \textit{maximum} beta-diversity, meanwhile, obtains when there is no overlap in species between populations. In this case, $\left(\sum_{X \in C} p_{X,s}\right)^q = \sum_{X \in C} (p_{X,s})^q$, and the beta-diversity is therefore given by:

\begin{equation}\begin{split}
^qD_{\beta\max}(C) &
= \left[\frac{\displaystyle\sum_{s \in S}\left(\sum_{X \in C} p_{X,s}\right)^q}{\displaystyle M^{q-1} \sum_{X \in C}\sum_{s \in S_X} (p_{X,s})^q}\right]^\frac{1}{1-q}
= \left[\frac{\displaystyle\sum_{s \in S}\sum_{X \in C} (p_{X,s})^q}{\displaystyle M^{q-1} \sum_{X \in C}\sum_{s \in S_X} (p_{X,s})^q}\right]^\frac{1}{1-q}
= \left[\frac{1}{M^{q-1}}\right]^\frac{1}{1-q} = \boldsymbol{M}
\end{split}
\label{eq:diversity_beta_max}
\end{equation}

\noindent The same identities hold for the special case when $q = 1$:

\begin{equation}
\begin{split}
^1D_{\beta\min}(C) & = \frac{\exp\left[-\sum_{s \in S}p_s \cdot\ln p_s\right]}{\exp\left[\frac{1}{M}\sum_{X \in C}\left(-\sum_{s \in S_X}(p_{X,s}\cdot\ln p_{X,s})\right)\right]} = \frac{\exp\left[-\sum_{s \in S}p_s \cdot\ln p_s\right]}{\exp\left[\frac{1}{M}\sum_{X \in C}\left(-\sum_{s \in S}(p_{s}\cdot\ln p_{s})\right)\right]}\\
& = \frac{\exp\left[-\sum_{s \in S}p_s \cdot\ln p_s\right]}{\exp\left[\frac{M}{M}\left(-\sum_{s \in S}p_{s}\cdot\ln p_{s}\right)\right]} = 1
\end{split}
\label{eq:diversity_beta_q1_min}
\end{equation}

\begin{equation}\begin{split}
^1D_{\beta\max}(C)
& = \exp\left[\sum_{X \in C}w\sum_{s \in S_X}(p_{X,s}\cdot\ln p_{X,s})-\sum_{s \in S_X}\left(\sum_{X \in C}wp_{X,s}\right)\cdot\ln \left(\sum_{X \in C}wp_{X,s}\right)\right]\\
& = \exp\left[\sum_{s \in S_X}\left(\sum_{X \in C}w \cdot p_{X,s}\cdot\ln p_{X,s} - \sum_{X \in C} w \cdot p_{X,s} \cdot\ln \left(wp_{X,s}\right)\right)\right]\\
& = \exp\left[\sum_{X \in C}\sum_{s \in S_X}\left(w \cdot p_{X,s}\cdot\ln p_{X,s} - w \cdot p_{X,s} \cdot\ln w - w \cdot p_{X,s} \cdot\ln p_{X,s}\right)\right]\\
& = \exp\left[-\sum_{X \in C}\sum_{s \in S_X} w \cdot p_{X,s} \cdot \ln w \right]
 = \exp\left[-\sum_{X \in C} w \cdot \ln w \cdot \left(\sum_{s \in S_X} p_{X,s}\right)\right]\\
& = \exp\left[-\sum_{X \in C} w \cdot \ln w \right]
= \exp\left[-M \frac{1}{M} \cdot \ln \frac{1}{M} \right]
= \exp[\ln M] = M
\end{split}
\label{eq:diversity_beta_q1_max}
\end{equation} % TODO: Redo this for new same-size-group version

The beta-diversity for a structured population with $M$ subpopulations therefore ranges between 1 (minimum) and $M$ (maximum). The beta-diversities of this population can thus be transformed onto a new scale from 0 (minimum) to 1 (maximum) as follows:

\begin{equation}
^qD_{\beta\,\mathrm{rescaled}} = \frac{^qD_\beta -\:^qD_{\beta\min}}{^qD_{\beta\max} -\: ^qD_{\beta\min}} = \frac{^qD_\beta - 1}{M - 1}
\label{eq:diversity_beta_rescale}
\end{equation}

By transforming the beta-diversity spectra of different structured populations onto this common scale, the inter-subpopulation variability of those populations can be meaningfully compared, even if they differ in the number of subpopulations they contain.