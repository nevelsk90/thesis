% Chapter 1 summary
% Should fit on chapter title page

\section*{Summary} % Fits one one page if 1.5-spaced, but not at double spacing

Faced with the pervasive risk of parasitic infection, vertebrates have evolved highly complex immune systems capable of effectively combating a wide range of pathogenic threats. Among the most sophisticated of these adaptations is the adaptive immunity provided by B- and T-lymphocytes, which utilise a range of specialised genome-editing mechanisms to generate a vast array of distinct antigen-binding proteins. These remarkable systems, however, undergo severe systemic decline with ageing, leading to greatly increased rates of infection-related morbidity in older individuals. Comparable immunosenescent phenotypes have been widely observed in mammals, birds, fish and elsewhere, and appear to be broadly conserved across the vertebrate lineage.

To understand and counter the complex changes that occur in the adaptive immune system with age, it is necessary to analyse the whole population of lymphocyte antigen-receptor sequences present in an individual. Such a top-down approach was impossible until relatively recently, when the advent of modern high-throughput sequencing technologies enabled the development of specialised protocols for immune-repertoire sequencing and analysis. Since then, the field of immune-repertoire studies has developed rapidly, providing a new and more powerful method for interrogating the changes ocurring in adaptive immune repertoires in a wide variety of contexts, including ageing. However, while initial human studies have indicated a decline in the diversity of these repertoires in older people, there remains a need for further research in this area. % This is weak, fill in once you've done your lit review.

As the shortest-lived vertebrate to be bred in captivity, the African turquoise killifish (\textit{Nothobranchius furzeri}) represents a powerful model for studying vertebrate-specific ageing processes, including immunosenescence of the adaptive immune system. Though the killifish has seen rapid development as a model system for ageing research, little was known about its adaptive immune system prior to the work described in this thesis. % In this chapter...? But it's a summary, not an introduction.