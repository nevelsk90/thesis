%!TEX root = ../thesis.tex
% TODO: Change this?

\chapter{Materials and methods}  
%s\doublespacing
\onehalfspacing

% Chapter summary (should fit on title page)

%% Chapter 1 summary
% Should fit on chapter title page

\section*{Summary} % Fits one one page if 1.5-spaced, but not at double spacing

Faced with the pervasive risk of parasitic infection, vertebrates have evolved highly complex immune systems capable of effectively combating a wide range of pathogenic threats. Among the most sophisticated of these adaptations is the adaptive immunity provided by B- and T-lymphocytes, which utilise a range of specialised genome-editing mechanisms to generate a vast array of distinct antigen-binding proteins. These remarkable systems, however, undergo severe systemic decline with ageing, leading to greatly increased rates of infection-related morbidity in older individuals. Comparable immunosenescent phenotypes have been widely observed in mammals, birds, fish and elsewhere, and appear to be broadly conserved across the vertebrate lineage.

To understand and counter the complex changes that occur in the adaptive immune system with age, it is necessary to analyse the whole population of lymphocyte antigen-receptor sequences present in an individual. Such a top-down approach was impossible until relatively recently, when the advent of modern high-throughput sequencing technologies enabled the development of specialised protocols for immune-repertoire sequencing and analysis. Since then, the field of immune-repertoire studies has developed rapidly, providing a new and more powerful method for interrogating the changes ocurring in adaptive immune repertoires in a wide variety of contexts, including ageing. However, while initial human studies have indicated a decline in the diversity of these repertoires in older people, there remains a need for further research in this area. % This is weak, fill in once you've done your lit review.

As the shortest-lived vertebrate to be bred in captivity, the African turquoise killifish (\textit{Nothobranchius furzeri}) represents a powerful model for studying vertebrate-specific ageing processes, including immunosenescence of the adaptive immune system. Though the killifish has seen rapid development as a model system for ageing research, little was known about its adaptive immune system prior to the work described in this thesis. % In this chapter...? But it's a summary, not an introduction.

\pagebreak

% Sections

\section{Killifish husbandry and sample preparation methods}
\label{sec:methods_husbandry}

Male turquoise killifish (\nfu, GRZ-AD strain) from a single hatching were raised under standard husbandry conditions \parencite{dodzian2018husbandry} and housed from four weeks post-hatching in individual \lt{2.8} tanks connected to a water recirculation system. Fish received \hr{12} of light per day on a regular light/dark cycle, and were fed blood
worm larvae and brine shrimp nauplii twice a day during the week and once a day during the weekend \parencite{dodzian2018husbandry,smith2017microbiota}.

Sacrificed fish (\Cref{tab:igseq-cohorts-fish}) were killed by anaesthetisation in \gl{1.5} Tricaine solution in room-temperature tank water \parencite{carter2011tricaine}, then flash-frozen in liquid nitrogen and ground to a homogenous powder with a pestle in a liquid-nitrogen-filled mortar. The powder was mixed thoroughly and stored at \degC{-80} prior to RNA isolation.
\section{Biochemistry and molecular biology methods}

\subsection{Standard methods}

\subsubsection{PCR}
\label{sec:methods_molec_standard_pcr}

\newabbreviation{PCR}{PCR}{polymerase chain reaction}
\newabbreviation{DNA}{DNA}{deoxyribonucleic acid}
\newabbreviation{HiFi}{HiFi}{hot-start (PCR)}
\newabbreviation{ul}{\textmu{}l}{microlitre(s)}
\newabbreviation{s}{s}{second(s)}
\newabbreviation[sort=T_a]{ta}{$\boldsymbol{T_a}$}{annealing temperature (PCR)}
\newabbreviation[sort=t_{ext}]{t_ext}{$\boldsymbol{t_{ext}}$}{extension time (PCR)}
\newabbreviation[sort=n_c]{n_c}{$\boldsymbol{n_c}$}{cycle number (PCR)}

The polymerase chain reaction is a well-established method for rapid amplification of a DNA sequence through repeated cycles of denaturation, priming and replication by a high-temperature-tolerant DNA polymerase enzyme \parencite{paul2010hotstartpcr}. Unless otherwise specified, all PCRs in this chapter were performed using \x{2} Kapa HiFi HotStart ReadyMix PCR Kit (\Cref{app:solutions_enzymes}) according to the manufacturer's instructions. Briefly, for a \ul{25} reaction, \ul{12.5} Kapa ReadyMix was combined with \ul{12.5} total of template, nuclease-free water, and primers; these volumes were scaled linearly for reactions of different volumes. The mixture was then heated in a thermocycler as follows:

\begin{center}
\begin{threeparttable}
\begin{tabular}{cccc}\toprule
\textbf{Step} & \textbf{Temperature [\degC{}]} & \textbf{Duration [\secs{}]} & \textbf{Cycles}\\\midrule
Initial denaturation & 95 & 180 & 1 \\\midrule
Denaturation & 98 & 20 & \multirow{3}{*}{$n_c$\tnote{1}}\\
Annealing & $T_a$\tnote{a} \tnote{} & 15 & \\
Extension & 72 & $t_{ext}$\tnote{a} & \\\midrule
Final extension & 72 & $t_{ext} \times 4$\tnote{a} & 1\\
\bottomrule\end{tabular}
\begin{tablenotes}
\item[a] Annealing temperature ($T_a$), extension time ($t_{ext}$) and cycle number ($n_c$) determined separately for each reaction.
\end{tablenotes}
\end{threeparttable}
\end{center}

\subsubsection{Nucleic-acid purification with SeraSure magnetic beads}
\label{sec:methods_molec_standard_serasure}

\newabbreviation{IgSeq}{IgSeq}{immunoglobulin sequencing}
\newabbreviation{SPRI}{SPRI}{solid-phase reversible immobilisation}
\newabbreviation{PEG}{PEG}{polyethylene glycol}
\newabbreviation{ml}{ml}{millilitre(s)}
\newabbreviation{TET}{TET}{Tris:EDTA:Tween (buffer)}
\newabbreviation{iSB}{iSB}{incomplete SeraBind (buffer)}
\newabbreviation{pc}{\%}{percent; per cent}
\newabbreviation{wv}{w/v}{weight/volume}
\newabbreviation{min}{min}{minute(s)}
\newabbreviation{EB}{EB}{elution buffer}

Nucleic-acid isolation, size-selection and concentration in the IgSeq library preparation protocol (and elsewhere where necessary) was performed using SeraSure SPRI (solid phase reversible immobilization) bead preparations \parencite{hawkins1994spri,deangelis1995spri,lennon2010cleanup,fisher2011cleanup}. In SPRI, paramagnetic beads bind DNA in the presence of polyethylene glycol (PEG), with the affinity of the beads for DNA depending on the concentration of PEG in the binding buffer. As a result, the range of nucleic-acid sequence lengths retained by SPRI bead purification depends primarily on the concentration of PEG, which in turn depends on the relative volume of SeraSure bead suspension added to a sample; the higher the concentration, the shorter the minimum fragment length retained during the purification process. In combination with a magnetic rack to remove the DNA-bound beads from suspension, this allows DNA of the desired size range to be isolated from a solution and resuspended in the desired volume of fresh buffer.

To prepare \ml{50} of SeraSure bead suspension for DNA (or DNA:RNA heteroduplex) isolation, a stock of SeraMag beads (\Cref{app:solutions_reagents}) was vortexed thoroughly, then \ml{1} was transferred to a new tube. This tube was then transferred to a magnetic rack and incubated at room temperature for \mins{1}, then the supernatant was removed and replaced with \ml{1} TET buffer (\Cref{app:solutions_buffers}) and the tube was removed from the rack and vortexed thoroughly. This washing process was repeated twice more, for a total of three washes in TET. A fourth cycle was used to replace the TET with incomplete SeraBind buffer (iSB, \Cref{app:solutions_buffers}). The vortexed \ml{1} aliquot of beads in iSB was then transferred to a conical tube containing \ml{28} iSB and mixed by inversion. To add the PEG, \ml{20} \pc{50} (w/v) PEG 8000 solution was dispensed slowly down the side of the conical tube, bringing the total volume to \ml{49}. Finally, this was brought to \ml{50} by adding \ul{250} \pc{10} (w/v) Tween 20 solution and \ul{750} autoclaved water to complete the SeraSure bead suspension.

To perform a bead cleanup, an aliquot of prepared SeraBind solution was vortexed thoroughly to completely resuspend the beads, then the appropriate relative volume of SeraSure suspension was added to a sample, mixing thoroughly by gentle pipetting. The sample was incubated at room temperature for \mins{5} to allow the beads to bind the DNA, then transferred to a magnetic rack and incubated for a further \mins{5} to draw as many beads as possible out of suspension. The supernatant was removed and discarded and replaced with \pc{80} ethanol, to a volume sufficient to completely submerge the bead pellet. The sample was incubated for 0.5-\mins{1}, then the ethanol was replaced and incubated for a further 0.5-\mins{1}. The second ethanol wash was removed, and the tube left on the rack until the bead pellet was almost, but not completely, dry, after which it was removed from the rack. The bead pellet was resuspended in a suitable volume of elution buffer (EB, \Cref{app:solutions_buffers}) then incubated at room temperature for at least 5 minutes to allow the nucleic-acid molecules to elute from the beads.

Unless otherwise specified, the beads from a cleanup were left in a sample during subsequent applications. To remove beads from a sample, the sample was mixed gently but thoroughly to resuspend the beads, incubated for an extended time period (at least \mins{10}) to maximise nucleic-acid elution, then transferred to a magnetic rack and incubated for 2-\mins{5} to remove the beads from suspension. The supernatant (containing the eluted nucleic-acid molecules) was then transferred to a new tube, and the beads discarded.

\subsubsection{Phenol-chloroform extraction and ethanol precipitation of DNA}
\label{sec:methods_molec_standard_phenol}

\newabbreviation{PCI}{PCI}{phenol:chloroform:isoamyl alcohol mixture}
\newabbreviation{degrees}{\textdegree{}}{degrees (angle)}

To clean up RNA and protein from isolated DNA samples, each sample was diluted to \ul{500} in nuclease-free water and mixed with \ul{500} of equilibrated phenol:chloroform:isoamyl alcohol (PCI) mixture (\Cref{app:solutions_reagents}) in a fume hood. The sample/PCI mixture was shaken vigorously by had for \secs{15} to thoroughly mix the different components, then centrifuged in a benchtop centrifuge (\mins{5}, room temperature, top speed). Again in a fume hood, the mixed sample was angled at \degrees{45}, and the upper aqueous phase containing the DNA was removed and transferred to a new tube while the lower organic phase was discarded. A second aliquot of \ul{500} PCI was added and the sample was mixed, centrifuged and separated as before. Finally, in order to remove residual phenol in the sample, \ul{500} pure chloroform was added to the newly-separated aqueous phase and the sample was once again mixed, centrifuged and separated. 

Following this final round of separation, the DNA in the aqueous phase was precipitated by addition of 0.1 volumes of \mol{3} sodium acetate solution, followed by 2.5 volumes of fresh \pc{100} ethanol. The mixture was mixed gently by inversion, then incubated for 1-\hr{3} at \degC{-80} or at \degC{-20} overnight. The suspension of precipitated DNA was pelleted through centrifugation in a benchtop centrifuge (\mins{30}, \degC{4}, top speed). The supernatant was discarded and replaced with \ul{500} chilled \pc{70} ethanol, and the sample centrifuged again (\mins{5}, \degC{4}, top speed). After this, the supernatant was again discarded, and the samples allowed to air-dry before being resuspended in 30-\ul{50} EB (\Cref{app:solutions_buffers}). 

\subsubsection{Guanidinium thiocyanate-phenol-chloroform extraction of RNA}
\label{sec:methods_molec_standard_qiazol}

\newabbreviation[sort=g]{g}{$g$}{relative centrifugal force}
\newabbreviation{BR}{BR}{broad-range (Qubit assay)}
\newabbreviation{HS}{HS}{high-sensitivity (Qubit assay)}

To isolate total RNA from homogenised killifish tissues, \ml{1} of QIAzol lysis reagent (containing acid phenol and guanidinium thiocyanate) was added to \g{0.1} of tissue, mixed gently but thoroughly by inversion, then incubated at room temperature for \mins{5} to allow the QIAzol to penetrate the tissue. \vol{0.2} of chloroform was added and the mixture was shaken vigorously for \secs{15}, then incubated at room temperature for \mins{3}. The mixture was then centrifuged (\mins{15}, \degC{4}, \g{12000}). Angling the tube at \degrees{45}, the upper aqueous phase containing the RNA was removed and transferred to a new tube, while the lower organic phase was discarded.

Following phase separation, the RNA was precipitated by adding \vol{0.5} room-temperature isopropanol, mixing gently by inversion and incubating for \mins{5} at room temperature. The suspension was centrifuged (\mins{10}, \degC{4}, \g{12000}) and the supernatant discarded. \vol{1} freshly prepared \pc{75} ethanol was added and the tube was vortexed briefly and centrifuged again (\mins{5}, \degC{4}, \g{7500}). The supernatant was discarded and the RNA pellet allowed to air-dry for 5-\mins{10}, then resuspended in \ul{50} EB (\Cref{app:solutions_buffers}). The concentration and quality of the resulting total-RNA solution were assayed with the Qubit 2.0 flourometer (RNA BR assay kit) and TapeStation 4200 (RNA tape), respectively, according to the manufacturer's instructions.

% TODO(?): Cite Qubit and TapeStation operation manuals, here and elsewhere (Dario)

\subsection{Library size-selection with the BluePippin}
\label{sec:methods_molec_standard_bluepippin}

\newabbreviation{bp}{bp}{base pairs}

The BluePippin is a DNA size-selection system based on agarose gel electrophoresis, which uses timed switching between positively-charged electrodes at a forked gel channel in an agarose cassette to redirect DNA of a desired size range into a separate lane from the rest of the sample \parencite{sage2016bluepippin}. The timing of switching is determined based on the size range input by the user and calibrated using flourescent internal standards, which are added to the sample during the sample preparation process and designed to run well ahead of the possible size ranges for that cassette type. The combination of the choice of cassette and the choice of standards determines which fragment lengths can be effectively isolated using the machine.

For the experiments described in this thesis, a \pc{1.5} cassette with R2 markers were used, enabling size selection of targets in the range of 250--\bp{1500} \parencite{sage2016bluepippin}. Machine calibration and testing, cassette preparation, and protocol design were performed in accordance with the BluePippin documentation and instructions given by the machine software. During this process, the elution wells of the lanes to be used in the size-selection run were emptied and refilled with \ul{40} of electrophoresis buffer (\Cref{app:solutions_reagents}), then sealed for the duration of the run, and a broad-range size-selection protocol with a target range of 400 to \bp{800} was specified. \ul{30} of sample was then combined with \ul{10} of loading solution (\Cref{app:solutions_reagents}) and vortexed to mix, then \ul{40} of buffer was removed from the appropriate loading well and replaced, slowly, with the prepared sample mixture. The protocol was started and run until the final elution was complete. Finally, the eluted samples were removed from the elution wells of the appropriate lanes, and the unused lanes of the cassette were re-sealed for future use.

\subsection{BAC isolation and sequencing}
\label{sec:methods_molec_bacs}

\newabbreviation{BAC}{BAC}{bacterial artificial chromosome}
\newabbreviation{FLI}{FLI}{Leibniz Institute on Aging – Fritz Lipmann Institute (Jena, Germany)}
\newabbreviation[sort=E. coli]{ecoli}{\textit{E. coli}}{\textit{Escherichia coli}}
\newabbreviation{LB}{LB}{Lysogeny broth}
\newabbreviation{M}{M}{molar concentration (\cu{\mole\per\liter}{}{})}
\newabbreviation{mol}{mol}{mole(s) (unit of amount of substance)}

All BAC clones that were sequenced for this research were provided by the FLI in Jena as plate or stab cultures of transformed \textit{E. coli}, which were replated and stored at \degC{4} when not in use. Prior to isolation, the clones of interest were cultured overnight in at least \ml{100} LB medium to produce a large liquid culture. The cultures were then transferred to \ml{50} conical tubes and centrifuged (10-\mins{25}, \degC{4}, \g{3500}) to pellet the cells. After pelleting, the supernatant was carefully discarded and the cells were resuspended in \ml{18} buffer P1.

After resuspension, the cultures underwent alkaline lysis to release the BAC DNA and precipitate genomic DNA and cellular debris. \ml{18} lysis buffer P2 was added to each tube, which was then mixed gently but thoroughly by inversion and incubated at room temperature for \mins{5}. \ml{10} ice-chilled neutralisation buffer P3 was added and each tube was mixed gently but thoroughly by inversion and incubated on ice for \mins{15}. The tubes were then centrifuged (20-\mins{30}, \degC{4}, \g{12000}) to pellet cellular debris and the supernatant transferred to new conical tubes. This process was repeated at least two more times, until no more debris was visible in any tube; this repeated pelleting was necessary to minimise contamination in each sample, as the normal column- or paper-based filtering steps used during alkaline lysis protocols result in the loss of the BAC DNA.

Following alkaline lysis, the BAC (and residual genomic) DNA in each sample underwent isopropanol precipitation: 0.6 volumes of room-temperature isopropanol was added to the clean supernatant in each tube, followed by 0.1 volumes of \mol{3} sodium acetate solution. Each tube was mixed well by inversion, incubated for 10-\mins{15} at room temperature, then centrifuged (\mins{30}, \degC{4}, \g{12000}) to pellet the DNA. The supernatant was discarded and the resulting DNA smear was ``resuspended" in \ml{1} \pc{100} ethanol and transferred to a \ml{1.5} tube, which was re-centrifuged (\mins{5}, \degC{4}, top speed) to obtain a concentrated pellet. Finally, the peletted samples were resuspended in EB (\Cref{app:solutions_buffers}) and purified of proteins and RNA using phenol:chloroform extraction and ethanol precipitation (\Cref{sec:methods_molec_standard_phenol}).

The resuspended BAC isolates were sent to the Cologne Center for Genomics, where they underwent Illumina Nextera XT library preparation and were sequenced on an Illumina MiSeq sequencing machine (MiSeq Reagent Kit v3, 2x300bp reads).

\subsection{Immunoglobulin sequencing of killifish samples}
\label{sec:methods_molec_igseq}

\subsubsection{RNA template quantification and quality control}
\label{sec:methods_molec_igseq_template}

Total RNA from whole-body killifish samples was isolated as described in \Cref{sec:methods_molec_standard_qiazol}; gut RNA from microbiota transfer experiments \parencite{smith2017microbiota} was already prepared and available. Quantification of RNA samples was performed with the Qubit 2.0 flourometer (RNA BR assay kit), while quality control and integrity measurement was performed using the TapeStation 4200 (RNA tape), both according to the manufacturer's instructions.

\subsubsection{Reverse-transcription and template switching}
\label{sec:methods_molec_igseq_rt}

\newabbreviation{ng}{ng}{nanogram(s)}
\newabbreviation{uM}{\textmu{}M}{micromolar concentration (\cu{\micro\mole\per\liter}{}{})}
\newabbreviation{times}{$\times$}{multiples of standard concentration}
\newabbreviation{GSP}{GSP}{gene-specific primer (for reverse-transcription)}
\newabbreviation{umol}{\textmu{}mol}{micromole(s)}
\newabbreviation{TSA}{TSA}{template-switch adaptor}
\newabbreviation{DTT}{DTT}{dithiothreitol (reducing agent)}
\newabbreviation{mM}{mM}{millimolar concentration (\cu{\milli\mole\per\liter}{}{})}
\newabbreviation{mmol}{mmol}{millimole(s)}
\newabbreviation{U}{U}{unit(s) of enzyme activity}
\newabbreviation{dNTP}{dNTP}{deoxyribonucleoside triphosphate mix (dATP, dGTP, dCTP and dTTP)}
\newabbreviation{dATP}{dATP}{deoxyadenosine triphosphate}
\newabbreviation{dGTP}{dGTP}{deoxyguanosine triphosphate}
\newabbreviation{dCTP}{dCTP}{deoxycytidine triphosphate}
\newabbreviation{dTTP}{dTTP}{deoxythymidine triphosphate}
\newabbreviation{UDG}{UDG}{uracil DNA glycosylase}


Reverse transcription of total RNA and template switching for \igseq library preparation was performed using SMARTScribe Reverse Transcriptase, in line with the protocol specified in \parencite{turchaninova2016igprep} (Procedure, steps 5-9). Briefly, \ng{750} total RNA from a killifish sample was combined with \ul{2} \umol{10} gene-specific primer (GSP), homologous with the second \ch exon of \Nfu \igh{M} (\Cref{app:oligos_primers}). The reaction volume was brought to a total of \ul{8} with nuclease-free water, and the resulting mixture was incubated for 2 minutes at \degC{70} to denature the RNA, then cooled to \degC{42} to anneal the GSP. 

Following annealing, the RNA-primer mixture was combined with \ul{12} of reverse-transcription master-mix (\Cref{tab:methods_rt_mm}), including the reverse-transcriptase enzyme and template-switch adapter (\Cref{app:oligos_tsa}). The complete reaction mixture was incubated at for \hr{1} at \degC{42} for the reverse-transcription reaction, then mixed with \ul{1} of uracil DNA glycosylase (UDG) and incubated for a further \mins{40} at \degC{37} to digest the template-switch adapter. Finally, the reaction product was purified using SeraSure beads (\Cref{sec:methods_molec_standard_serasure}) at \x{0.7} concentration, eluting in \ul{16.5} clean elution buffer.

\begin{table}[h]
\begin{center}\small
\begin{threeparttable}
\caption{Master-mix components for SMARTScribe reverse transcription, per sample}
\begin{tabular}{llll}\toprule
\textbf{Volume [\ul{}]} & \textbf{Component} & \textbf{Concentration} & \textbf{Reference}\\\midrule
2 & SMARTScribe reverse transcriptase & \unitsul{100} & \Cref{app:solutions_enzymes} \\
4 & SMARTScribe first-strand buffer & \x{5} & \Cref{app:solutions_reagents} \\
2 & SmartNNNa barcoded TSA & \umol{10} & \Cref{app:oligos_tsa}\\
2 & DTT & \mmol{20} & \Cref{app:solutions_reagents}\\
2 & dNTP mix & \umol{10} per nucleotide & \Cref{app:solutions_reagents}\\
0.5 & RNasin RNase inhibitor & \unitsul{40} & \Cref{app:solutions_enzymes}\\\bottomrule
\end{tabular}
\label{tab:methods_rt_mm}
\end{threeparttable}
\end{center}
\end{table}

\subsubsection{PCR amplification and adaptor addition} 
\label{sec:methods_molec_igseq_pcr}

\newabbreviation[sort=H2O]{H2O}{H\textsubscript{2}O}{Water}

Following reverse-transcription, UDG digestion, and cleanup, the reaction mixture underwent three successive rounds of Kapa PCR (\Cref{sec:methods_molec_standard_pcr}, \Cref{tab:methods_igseq_pcr}) each of which was followed by a further round of bead cleanups (\Cref{sec:methods_molec_standard_serasure}, \Cref{tab:methods_igseq_beads}). The first of these PCR reactions added a second strand to the reverse-transcribed cDNA and amplified the resulting DNA molecules; the second added partial Illumina sequencing adaptors and further amplified the library, and the third added complete Illumina adaptors (including i5 and i7 indices \parencite{illumina2018adaptors}).

\begin{table}[h]
\def\arraystretch{1.3}
\centering\small
\begin{threeparttable}
\caption{Details of PCR protocols for \Nfu immunoglobulin sequencing}
\begin{tabular}{c|ccc|cc|ccccc}\toprule
\multirow{2}{*}{\textbf{PCR}} & \multicolumn{3}{c|}{\textbf{Protocol details}} & \multicolumn{2}{c|}{\textbf{Primers}} & \multicolumn{4}{c}{\textbf{Volumes (\ul{})\tnote{b}}}\\
 & \# cycles & $T_m$ (\degC{}) & $t_\mathrm{ext}$ (\secs{}) & F & R & Template & Primers (each) & Kapa & H\textsubscript{2}O \\\midrule
1 & 18 & 65 & 15 & IGHC-B & M1SS & 10.5 & 1 (\x{}2) & 12.5 & 0 \\\midrule
2 & 13 & 65 & 15 & M1S+P2 & IGHC-C+P1 & 1 & 0.5 (\x{}2) & 12.5 & 10.5 \\\midrule
3 & 7 to 9 & \textbf{68} & 15 & D50*\tnote{a} & D7**\tnote{a} & 2 & \textbf{0.75} (\x{}2) & 12.5 & 9 \\
\bottomrule % TODO: Correct cycle numbers (Ola, Michael)
\end{tabular}
\begin{tablenotes}
\item[a] PCR3 primers selected as appropriate for each library's assigned indices; see \Cref{app:oligos_illumina} for more information.
\item[b] If the number of samples to be sequenced was small, all volumes of PCR 3 were doubled for a \ul{50} total PCR volume.
\end{tablenotes}
\label{tab:methods_igseq_pcr}
\end{threeparttable}
\end{table}

\begin{table}[h]
\def\arraystretch{1.5}
\centering\small
\caption{Details of bead cleanups during \Nfu immunoglobulin sequencing}
\begin{threeparttable}
\begin{tabular}{l|c|cc|c}\toprule
\multirow{2}{*}{\textbf{Stage}\tnote{a}} & \multirow{2}{*}{\textbf{Sample volume}} & \multicolumn{2}{c|}{\textbf{Beads volume (\ul{})}} & \multirow{2}{*}{\textbf{Elution volume (\ul{})}}\\
& & \ul{} & \x{}\tnote{b} & \\\midrule
RT & 21 & 14.7 & 0.7 & 16.5\\
PCR 1 & 25 & 17.5 & 0.7 & 25\\
PCR 2 & 25 & 17.5 & 0.7 & 15\\
PCR 3 & 25\tnote{c} & 20\tnote{c} & 0.8 & 15\tnote{c}\\
Pooling & Varies & Varies & 2.5 & 35\\ 
\bottomrule
\end{tabular}
\begin{tablenotes}
\item[a] Each bead cleanup takes place immediately \textit{after} its corresponding stage.
\item[b] Bead volumes are usually given as multiples of the sample volume.
\item[c] If PCR 3 reaction volume differs from \ul{25}, bead and elution volumes are rescaled proportionally to sample volume as appropriate.
\end{tablenotes}
\label{tab:methods_igseq_beads}
\end{threeparttable}
\end{table}

% TODO: Table of experiment-specific cycle numbers, sequencing conditions, et cetera. (Ola, Michael)

\subsubsection{Library pooling, size selection and sequencing} 
\label{sec:methods_molec_igseq_seq}

\newabbreviation{nM}{nM}{nanomolar concentration (\cu{\nano\mole\per\liter}{}{})}
\newabbreviation{nmol}{nmol}{nanomole(s)}
\newabbreviation{PhiX}{PhiX}{PhiX174 bacteriophage genome}


Following PCR3 and its attendant bead cleanup, the total concentration of each library was assayed with the Qubit 2.0 (DNA HS assay kit), while the size distribution of each library was obtained using the TapeStation 4200 (D1000 tape). To obtain the concentration of complete library molecules in each case (as opposed to primer dimers or other off-target bands), the ratio between the concentration of the desired library band (c. 620-\bp{680}) and the total assayed concentration of the sample was calculated for each TapeStation lane, and the total concentration of each library as measured by the Qubit was multiplied by this number to obtain an estimate of the desired figure:

\begin{equation}
\mathrm{Library~concentration} = \mathrm{Qubit~concentration} \times \frac{\mathrm{TapeStation~concentration~[main~band]}}{\mathrm{TapeStation~concentration~[total]}}
\label{eq:library-conc}
\end{equation}

All the libraries for a given experiment were then pooled, such that the estimated concentration of each library in the final pooled sample was equal and the total mass of nucleic acid in the pooled sample was at least \ng{240}. The pooled libraries underwent a final bead cleanup (\Cref{sec:methods_molec_standard_serasure}, \Cref{tab:methods_igseq_beads}) to concentrate the samples, then underwent size selection with the BluePippin (\Cref{sec:methods_molec_standard_bluepippin}, \pc{1.5} DF Marker R2, broad 400-800bp). The size-selected pooled samples underwent a final round of quality control (as above) to confirm their collective concentration (at least \nmol{1.5}) and size distribution (one peak at c. 620-\bp{680}). Finally, the pooled and size-selected libraries were sequenced externally on an Illumina MiSeq System (MiSeq Reagent Kit v3, 2x300bp reads, 30\% PhiX spike-in). % TODO: Specify sequencing facility for each experiment? (Dario)

\newpage
\section{Computational and analytic methods}
\label{sec:methods_comp}

\subsection{General data processing and pipeline structure}
\label{sec:methods_comp_general}

Unless otherwise specified, processing and analysis of biological data was performed using standard Bioconductor \parencite{huber2015bioconductor} packages: \program[R]{Biostrings} \parencite{pages2017biostrings} and \program[R]{BSgenome} \parencite{pages2018bsgenome} for biological sequence data, \program[R]{GenomicRanges} \parencite{lawrence2013genomicranges} for sequence ranges, and \program[R]{genbankr} \parencite{becker2018genbankr} and \program[R]{rentrez} \parencite{winter2017rentrez} for GenBank files. 

Smith-Waterman and Needleman-Wunsch exhaustive alignments \parencite{needleman1970alignment,waterman1981alignment} were performed using the \snippet{pairwiseAlignment} function from \program[R]{Biostrings}; percentage sequence identities were computed using the \snippet{pid} function from the same package.

Processing of tabular data was performed using the Tidyverse suite of tools, especially \program[R]{readr} \parencite{wickham2018readr}, \program[R]{dplyr} \parencite{wickham2018dplyr}, \program[R]{tidyr} \parencite{wickham2018tidyr} and \program[R]{stringr} \parencite{wickham2018stringr}. \program{snakemake} \parencite{koster2012snakemake} was used to design and run data-processing pipelines.

\subsection{Data visualisation}
\label{sec:methods_comp_visualisation}

Unless otherwise specified, data were visualised using \program[R]{ggplot2} \parencite{wickham2016ggplot2}. Chromosome ideograms, locus structure visualisations, and sashimi plots were constructed using \program[R]{Gviz} \parencite{hahne2016gviz}. Cluster dendrograms and phylogenetic trees were drawn with \program[R]{ggtree} \parencite{guangchuang2018ggtree}, using utilities from \program[R]{ape} \parencite{paradis2018ape} and \program[R]{tidytree} \parencite{guangchuang2018tidytree}. Sequence logos were drawn with \program[R]{ggseqlogo} \parencite{wagih2017ggseqlogo}. 

\subsection{BAC insert assembly}
\label{sec:methods_comp_bacs}

\subsubsection{Identifying BAC candidates for the \nfu \igh{} locus}
\label{sec:methods_comp_bacs_ident}

The first group of candidate BAC clones to be used in the \Nfu locus assembly was identified by searching for scaffolds in a previous assembly of the \Nfu genome (\texttt{NotFur1}, GenBank accession GCA\_000878545.1 \parencite{valenzano2015genome}) that contained either \textit{IGH} gene fragments (\texttt{GapFilledScaffold\_8761}, \texttt{8571}, \texttt{16121}) or genes homologous to those flanking the \textit{IGH} locus in stickleback and medaka (\texttt{GapFilledScaffold\_2443}, \texttt{292}). Subsequences from these scaffolds were sent to Kathrin Reichwald at the FLI in Jena, who identified four BAC clones (193A03, 276N03, 209K12, 181N10) with sequenced ends close to the query sequences.

Following sequencing and assembly of these BAC inserts, a further group of BACs was identified using a second, independent genome assembly (GenBank accession 	GCA\_001465895.2, \parencite{reichwald2015genome}) and the database of BAC end sequences, which by then were publicly available. The assembled BAC sequences were found to map within or near a large, gapped region on synteny group 3 of this genome assembly, and BACs were selected that either intruded into this gapped region or had end sequences that mapped to another scaffold aligning to the assembled BAC inserts (scaffold01427, scaffold02214, scaffold01820). In total, 11 further BACs were sequenced and assembled in this second round (223M21, 162F04, 220O06, 248A22, 165M01, 206K13, 154G24, 208A08, 277J10, 109B21, 216D12).

\subsubsection{Sequence trimming, filtering and correction}
\label{sec:methods_comp_bacs_trim}

Demultiplexed, adaptor-trimmed MiSeq sequencing data were uploaded by the sequencing provider to Illumina BaseSpace and accessed via \program{basemount}. Reads from each library were trimmed with \lstinline{Trimmomatic} \parencite{bolger2014trimmomatic} to remove adaptor sequences, trim low-quality sequence, and discard any trimmed reads below a minimum lengh:

\begin{lstlisting}
trimmomatic PE -phred33 <forward_reads_fastq> <reverse_reads_fastq> <output_paths> ILLUMINACLIP:<adaptor_directory>/TruSeq3-PE.fa:2:30:10 LEADING:20 TRAILING:20 SLIDINGWINDOW:4:30 MINLEN:36
\end{lstlisting}

Following this, the trimmed reads were filtered to remove \textit{E. coli} genomic DNA and other contaminants by aligning them using \lstinline{Bowtie2} \parencite{langmead2012bowtie2} and retaining read pairs that did not align concordantly:

\begin{lstlisting}
bowtie2 --very-sensitive-local --local --reorder --un-conc <output_prefix> -x <ecoli_genome_index_path> -1 <forward_reads_fastq> -2 <reverse_reads_fastq> -S <sam_file_prefix>
\end{lstlisting}

% TODO: Note about incomplete filtering?

Before sequence assembly, the filtered reads then underwent correction, to reduce the impact of errors occurring during the library preparation and sequencing process. In order to increase the reliability of the resulting scaffolds and reduce the impact of ideosyncracies of any given correction tool, the reads were corrected in parallel using two different programs; \program{QuorUM} \parencite{marcais2015quorum}:

\begin{lstlisting}
quorum -d -q "33" -p <output_path> <interleaved_reads_files>
\end{lstlisting}

\noindent and \program{BayesHammer} (the built-in correction tool of the \program{SPAdes} genome-assembly software \parencite{bankevich2012spades,nikolenko2013bayeshammer}):

\begin{lstlisting}
spades.py -1 <forward_reads_fastq> -2 <reverse_reads_fastq> -o <output_path> --disable-gzip-output --only-error-correction --careful --cov-cutoff auto -k 21,33,55,77,99,127 --phred-offset 33
\end{lstlisting}

\subsubsection{Sequence assembly and scaffolding}
\label{sec:methods_comp_bacs_assembly}

Each pair of independently-corrected reads files was then passed to \program{SPAdes} \parencite{bankevich2012spades} for \textit{de novo} genome assembly:

\begin{lstlisting}
spades.py -1 <forward_reads_fastq> -2 <reverse_reads_fastq> -o <output_path> --disable-gzip-output --only-assembler --careful --cov-cutoff auto -k 21,33,55,77,99,127 --phred-offset 33
\end{lstlisting}

Following assembly, any \textit{E. coli} scaffolds resulting from residual contaminating reads were identified by aligning scaffolds to the \textit{E. coli} genome using \program{BLASTN} \parencite{altschul1990blast,altschul1997blast}, and scaffolds containing significant matches were discarded. The remaining scaffolds were then scaffolded using \program{SSPACE} \parencite{boetzer2011sspace}, using jumping libraries from the Stanford \parencite{valenzano2015genome} and Jena \parencite{reichwald2015genome} killifish genome assemblies: % TODO: Get genome accessions

\begin{lstlisting}
SSPACE_Standard_v3.0.pl -x 0 -k 5 -a 0.7 -n 15 -z 200 -g 1 -p 0 -l <jumping_library_config_file> -s <spades_scaffolds_file>
\end{lstlisting}

In order to guarantee the reliability of the assembled scaffolds, the assemblies produced with \program{BayesHammer}- and \program{QuorUM}-corrected reads were compared, and scaffolds were broken into segments whose contiguity was agreed on between both assemblies. To integrate these fragments into a contiguous insert assembly, points of agreement between BAC assemblies from the same genomic region (e.g. two scaffolds from one assembly aligning concordantly to one scaffold from another) and between BAC assemblies and genome scaffolds, were used to combine scaffolds where possible. Any still-unconnected scaffolds were assembled together through pairwise end-to-end PCR (\Cref{sec:methods_molec_pcr}, with one primer each on the end of each scaffold) and Sanger sequencing (Eurofins).

\subsection{Locus characterisation and assembly}
\label{sec:methods_comp_locus}

\subsubsection{Collating reference sequences}
\label{sec:methods_comp_locus_reference}

\newabbreviation{HSP}{HSP}{high-scoring segment pair (\program{BLAST})}

Most publications presenting characterisations of \igh{} loci do not provide easy-to-use databases of trimmed and curated gene segments, and the data that is available is often partial and heterogeneous between publications. In order to obtain standardised databases, further analysis was performed on publically-available data from three reference species with previously-characterised \igh{} loci: medaka (\species{Oryzias}{latipes}) \parencite{magadan2011medaka}, zebrafish (\species{Danio}{rerio}) \parencite{danilova2005zebrafish} and three-spined stickleback (\species{Gasterosteus}{aculeatus}) \parencite{bao2010stickleback,gambondeza2011stickleback}, as described below. Following automatic sequence extraction, the reference sequences were checked manually for any severely pathological (e.g. out-of-frame) sequences and edited before being used for inference in novel loci.

\subsubsubsection{Medaka} % TODO: Disable indent after subsubsub

GenBank files of the annotated medaka \igh{} locus were downloaded from the supplementary information of the medaka locus paper (\parencite{magadan2011medaka}, additional file 6) and corrected to make them parsable by \program[R]{genbankr}. Locus sequence and annotation ranges were extracted from these GenBank files into \fmt{FASTA} and tab-separated tabular formats, respectively, and segment annotations were renamed to match the naming conventions used in other species. \vh, \dh, \jh and constant-region exon nucleotide sequences were extracted from the locus sequence using these annotations. Amino-acid sequences for \vh, \jh and constant-region sequences were obtained automatically by identifying the reading frames which minimised the number of STOP codons in each sequence.

\subsubsubsection{Stickleback}

Limited sequence information on the \igh{} locus in stickleback, including \vh segments and bulk (non-exon-separated) constant regions was provided in a GenBank file in the locus characterisation paper for medaka (\parencite{magadan2011medaka}, additional file 6), while additional sequence information (including \dh and \jh nucleic-acid sequences and amino-acid sequences of constant-region exons) was extracted manually from one of the stickleback locus papers (\parencite{bao2010stickleback},  Figure S1 to S4) into \fmt{FASTA} files. As with medaka, the GenBank reference file was downloaded, corrected and parsed to yield a \fmt{FASTA} file of the locus sequence and tab-separated tabular files of annotation ranges. \vh sequences were extracted from the locus sequence using these annotation ranges and translated as specified for medaka above; \jh sequences provided by \parencite{bao2010stickleback} were translated such that the final nucleotide formed the last position of the final codon.

To obtain nucleic-acid sequences of the constant-region exons, the amino-acid sequences from \parencite{bao2010stickleback} were aligned to the locus sequence with \program{TBLASTN} \parencite{gertz2006tblastn}, with a query coverage threshold of 40\% and a maximum of three HSPs per query sequence:

\begin{lstlisting}
tblastn -query <ch_aa_fasta> -subject <gac_locus_fasta> -qcov_hsp_perc 40 -max_hsps 3 -outfmt '<output_format>' > <output_path>
\end{lstlisting}

\noindent with the following standardised tabular output format: 

\begin{lstlisting}
6 qseqid sseqid pident qcovhsp length mismatch gapopen gaps sstrand qstart qend sstart send evalue bitscore qlen slen
\end{lstlisting}

To filter out alignments across subloci, any alignment of an exon upstream of the annotated boundaries of its corresponding bulk constant region (whose ranges were specified in the GenBank file) was discarded; the alignment with the highest score for each exon was then used to extract the corresponding nucleic-acid sequence from the locus. In order to control for any errors, either during manual extraction of locus sequences from the paper or in the paper itself, these nucleic-acid sequences were then re-translated to generate new amino-acid sequences, again using the translation frame producing the fewest STOP codons; these sequences were then used in place of the reference files in downstream applications.

\subsubsubsection{Zebrafish}

GenBank files corresponding to the zebrafish \igh{} locus were provided (without segment annotations) on GenBank by \parencite{danilova2005zebrafish}; this publication also provided detailed co-ordinates for the \vh, \dh and \jh segments (but not constant exons) on these sequences. Aligned amino-acid sequences were provided for the exons of \igh{M} and \igh{Z}, but no detailed information about \igh{D} exons could be found for these sequences; as a result, reference information about \igh{D} was not used from this species.

As with stickleback, the amino-acid sequences provided were aligned to the locus sequences  with \program{TBLASTN} to identify and extract exon nucleic-acid sequences, which were then translated using the frame yielding the fewest STOP codons for each sequence. \vh sequences were obtained using the ranges provided in \parencite{danilova2005zebrafish} and translated in the same manner. \dh and \jh nucleotide sequences were obtained directly from \parencite{danilova2005zebrafish}; as with stickleback, \jh amino-acid sequences were obtained by translating the nucleotide sequences in the frame such that the final nucleotide formed the last position of the final codon.

\subsubsection{Identifying putative locus sequences}
\label{sec:methods_comp_locus_scaffolds}

\newabbreviation{E-value}{E-value}{Expect value (\program{BLAST})}
\newabbreviation{kb}{kb}{kilobase(s)}

In order to identify sequences in a genome assembly potentially containing part of an \igh{} locus, reference \vh, \jh and constant-region nucleotide and amino-acid sequences were mapped to the assembly using \program{BLAST} \parencite{altschul1990blast,altschul1997blast}. Nucleotide sequences were aligned to the locus using the relatively permissive \program{blastn} algorithm (as opposed to e.g. \program{megablast} or \program{dc-megablast}):

\begin{lstlisting}
blastn -tastk blastn -query <reference_exon_fasta> -subject <locus_fasta> -outfmt "<output_format>"
\end{lstlisting}

Protein sequences, meanwhile, were aligned using the standard \program{blastp} algorithm:

\begin{lstlisting}
blastp -query <reference_exon_fasta> -subject <locus_fasta> -outfmt '<output_format>'
\end{lstlisting}

In both cases, the same tabular output format specified in \Cref{sec:methods_comp_locus_reference} was used, to provide a predictable format for downstream processing of \program{BLAST} alignment tables.

Following alignment of reference sequences, overlapping alignments to reference segments of the same segment type, isotype (if applicable) and exon number (if applicable) were collapsed together, keeping track of the number of collapsed alignments and the best E-values and bitscores obtained for each alignment group. Alignment groups with a very poor maximum E-value ($> 0.001$) were discarded, as were groups consisting of fewer than two alignments and groups overlapping with a much better alignments to a different sequence type, where ``much better" was defined as a bitscore difference of at least 33. Following resolution of conflicts, \vh and \ch alignments underwent a second filtering step of increased stringency, requiring a minimum E-value of $10^{-10}$ to be retained. 

Following alignment filtering, scaffolds containing surviving alignments to at least two distinct segment types (where \vh, \jh, and each type of constant-region exon each counted as one segment type), or alignments to one segment type covering at least 1\% of the scaffold's total length were retained as potential locus scaffolds. To reduce computational runtime spent processing irrelevant sequence, each candidate scaffold so identified was trimmed to 100kb before the first putative gene segment and 100kb after the last one; in the case of \nfu and \xma, these ranges were further reduced following more thorough segment characterisation (\Cref{sec:methods_comp_locus_segments}).

The exact reference sequences used for this extraction process differed depending on the genome being analysed. For \nfu (\Cref{sec:nfu-locus}), the reference sequences extracted from medaka, stickleback and zebrafish (\Cref{sec:methods_comp_locus_reference}) were used; for \xma (\Cref{sec:xma-locus}), gene segments inferred for \Nfu were also included; and for other species (\Cref{sec:locus_comparative}), the reference sequences plus those inferred for both \Nfu and \Xma were used.

\subsubsection{Locus sequence finalisation}
\label{sec:methods_comp_locus_final}

In the case of both \nfu and \xma, a single chromosome (chromosome 6 in \Nfu, chromosome 16 in \Xma) was identified as bearing the \igh{} locus in that species. In the case of \Xma, this was the only segment-bearing scaffold identified in the genome, and the completed locus sequence was obtained by simply trimming the chromosomal sequence at either end of the segment-bearing region. In contrast, multiple scaffolds from the \Nfu genome were also identified as bearing at least one potential \igh{} segment (\Cref{tab:nfu-locus-scaffolds}). In order to identify which of these were in fact part of the locus and integrate them into a contiguous sequence, BAC candidates identified and assembled as described in \Cref{sec:methods_comp_bacs} were incorporated into the assembly.

To do this, all assembled BAC inserts were screened for \igh{} locus segments in the same manner described for genome scaffolds (\Cref{sec:methods_comp_locus_scaffolds}). Passing BACs (\Cref{tab:nfu-locus-bacs}) were aligned to the candidate genome scaffolds with \program{BLASTN} and integrated manually together, giving priority in the event of a sequence conflict to (i) any sequence containing a gene segment missing from the other, and (ii) the genome scaffold sequence if neither sequence contained such a segment. BACs and scaffolds which could not be integrated into the locus sequence in this way were discarded as orphons.

\subsubsection{Locus segment characterisation}
\label{sec:methods_comp_locus_segments}

\newabbreviation{HMM}{HMM}{Hidden Markov Model}
\newabbreviation[sort=X. maculatus]{xma}{\Xma}{\xma}
\newabbreviation{SA}{SA}{suffix array}

Detailed characterisation of \igh{} gene segments was performed on finished \igh{} locus sequences for \xma and \nfu, and on isolated candidate scaffolds for other species, using the same reference segment databases used to identify candidate scaffolds for that species in \Cref{sec:methods_comp_locus_scaffolds}. % For Xma they are the same.
The specific methods used depended on segment type.

\subsubsubsection{\vh}

To identify \vh segments on newly characterised loci, reference \vh segments were used to construct a multiple-sequence alignment with \program{PRANK} \parencite{loytynoja2014prank}:

\begin{lstlisting}[language=bash]
prank -d=<reference_vh_db> -o=<output_path> -gaprate=0.00001 -gapext=0.00001 -F -termgap
\end{lstlisting}

The resulting alignment was used as an input to \program{NHMMER} \parencite{wheeler2013nhmmer,eddy2011hmm,eddy2009homology,eddy2008alignment}, which constructs a Hidden Markov Model from a multiple-sequence alignment and uses it to identify matching sequences in a reference sequence:

\begin{lstlisting}[language=bash]
nhmmer --dna --notextw --tblout <output_path> -T 80 <vh_alignment> <locus_sequence_path>
\end{lstlisting}

The resulting match table was used to identify candidate ranges in the locus sequence corresponding to \vh segments; these ranges were extended by 9bp at either end to account for boundary errors, and the corresponding nucleotide sequences were extracted to a FASTA file. Each sequence was then checked and refined manually: 3' ends were identified by the start of the RSS heptamer sequence (consensus \sequence{CACAGTG}), if present, while 5' ends and FR/CDR boundaries were identified using IMGT/DomainGapAlign \parencite{ehrenmann2011domaingapalign} with the default settings. Where necessary, IMGT/DomainGapAlign was also used to IMGT-gap the \vh segments in accordance with the IMGT unique numbering \parencite{lefranc2003vnumbering}.

An initial amino-acid sequence for each \vh segment was produced automatically from the extracted nucleotide sequence by identifying the reading frame which minimised the number of STOP codons in the sequence; this worked well for most segments. \vh amino-acid sequences were then refined (and in a few cases re-translated) from the manually-refined nucleotide sequences, including end-refinement and FR/CDR boundary identification.

Following extraction and manual curation, \vh segments were grouped into families based on their pairwise sequence identity. In order to assign segments to families, the nucleotide sequence of each \vh segment in a locus was aligned to each other segment using Needleman-Wunsch global alignment, and the resulting matrix of pairwise sequence identities was used to perform single-linkage heirarchical clustering on the \vh segments, and the resulting dendrogram was cut at 80\% sequence identity to obtain \vh families. These families were then numbered based on the order of the first-occurring \vh segment from that family in the first-occurring sublocus in the parent locus, and each \vh segment was named based on its parent sublocus, its family, and its order among elements of that family in that sublocus.

\subsubsubsection{\jh}

As with \vh segments, \jh segments were identified by building a multiple-sequence alignment with \program{PRANK} and using it to construct an HMM with \program{nhmmer}; the parameters used were the same as for \vh segments, except that there was no minimum score for \program{nhmmer} to report a sequence match (\snippet{-T 0} instead of \snippet{-T 80}). The resulting sequence ranges were extended by \bp{20} on either end and extracted into FASTA format. These sequences were then trimmed automatically by identifying the RSS heptamer sequence at the 5' end and the splice junction motif (\texttt{GTA}) at the 3' end. The \jh nucleotide sequences were then checked and refined manually.

\begin{wraptable}{r}{5.5cm}
\caption{Regex patterns used to search for conserved W118 residues in \jh sequences during AUX file generation}\label{tab:jh-aux-patterns}
\begin{tabular}{r>{\ttseries}l}\toprule  
\# & Pattern \\\midrule
1 & TGGGBNNNNGBN\\
2 & TGGGBNNNGBN\\
3 & TGGGBNNNNNGBN\\
4 & TGGGBNNNNNNGBN\\
5 & TGGGBN\\\bottomrule
\end{tabular}
\end{wraptable}

\program{IgBLAST} \parencite{ye2013igblast} identifies CDR3 boundaries for recombined IGH VDJ sequences using an AUX file specifying the reading frame of each \jh segment, along with the co-ordinate of the conserved \texttt{TGG} codon (corresponding to the conserved W118 residue in the recombined sequence \parencite{lefranc2014immunoglobulins}) marking the CDR3/FR4 boundary. An AUX file for the inferred \jh segments was generated automatically by searching for the conserved sequence using a series of regular-expression patterns of decreasing stringency (\Cref{tab:jh-aux-patterns}), taking the first match in each sequence as the desired residue; this determined both the reading frame and the W118 sequence co-ordinate. Once generated, the AUX file was then used to determine the reading frame for automatically translated the \jh sequences; both the AUX file and amino-acid FASTA file were then edited to incorporate any manual refinements made to the \jh nucleotide sequences.

Curated \jh sequences were named based on their order within their parent sublocus and, where applicable, on whether they were upstream of IGHZ or IGHM constant regions. 

\subsubsubsection{\dh}

Unlike \vh and \jh gene segments, \dh segments are too short and variable to be found effectively using an HMM-based search strategy. Instead, \dh segments in assembled loci were located using their distinctive pattern of flanking recombination signal sequences: an antisense RSS in 5', then a short D-segment, then a sense RSS in 3'. Potential matches to this pattern were searched for using \program{FUZZNUC} from the EMBOSS collection of bioinformatics tools \parencite{rice2000emboss}, with a high error tolerance to account for deviations from the conserved sequence in either or both of the RSSs:

\begin{lstlisting}
fuzznuc -pattern 'GGTTTTTGTN(10,14)CACTGTGN(1,25)CACAGTGN(10,14)ACAAAAACC' -pmismatch 8 -rformat gff -outfile <output_path> <locus_sequence_path>
\end{lstlisting}

This generated a GFF file \parencite{stein2010generic} of permissive matches, representing potential \dh segments; these were then grouped by sequence co-ordinate, and higher-mismatch candidates overlapping with a lower-mismatch alternative were discarded.

Orientation of \dh segments based on their own sequence is challenging, as the segments themselves have no clear conserved structure and the flanking RSSs are rotationally symmetric. To overcome this problem and orientate the \dh segments on the locus, the table of \dh candidate ranges was combined with previously-identified (and easier to orientate) \vh and \jh ranges. Each \dh candidate was then orientated based on the orientations of its flanking segments: segments with an oriented segment immediately upstream or downstream adopted the orientation of that segment, while segments with contradictory orientation information were discarded. This process was repeated until all \dh segments had either been orientated or discarded.

After orientation, the \dh ranges were used to extract \dh sequences in FASTA format from the locus sequence; these sequences then underwent a second, more stringent filtering step, in which sequences lacking the most conserved positions in each RSS were discarded \parencite{grep}:
% TODO: Citation for RSS consensus sequences

\begin{lstlisting}
grep -B 1 '[ACTG]\{{25,27\}}TG[ACTG]\{{1,25\}}CA[ACTG]\{{25,27\}}' <dh_fasta> | sed '/^--$'/d > <output_fasta>
\end{lstlisting}

Finally, the identified \dh candidates were checked manually, candidates without good RSS sequences were discarded, and flanking RSS sequences were trimmed to obtain \dh segment sequences. As with \jh, these were numbered based on their order within their parent sublocus and, when applicable, on whether they were upstream of IGHZ or IGHM constant regions.

\subsubsubsection{CH}

To detect and identify constant-region exons in the characterised loci, constant-region nucleotide and protein sequences from reference species were mapped to the locus sequence using \program{BLAST} \parencite{altschul1990blast,altschul1997blast}, in the same manner described for putative locus scaffolds in \Cref{sec:methods_comp_locus_scaffolds}.
Following alignment of reference sequences, overlapping alignments to reference segments of the same isotype and exon number were collapsed together, keeping track of the number of collapsed alignments and the best E-values and bitscores obtained for each alignment groups. Alignment groups with a very poor maximum E-value ($> 0.001$) were discarded, as were groups overlapping with a much better alignments to a different isotype or exon type, where ``much better'' was defined as a bitscore difference of at least 16.5. Where conflicting alignments to different isotypes or exon types co-occurred without a sufficiently large difference in bitscore, both alignment groups were retained for manual resolution of exon identity.

Following resolution of conflicts, alignment groups underwent a second filtering step of increased stringency, requiring a minimum E-value of $10^{-8}$ and at least two aligned reference exons over all reference species to be retained. Each surviving alignment group was then converted to a sequence range, extended by \bp{10} at each end to account for truncated alignments failing to cover the ends of the exon, and used to extract the corresponding exon sequence into \format{FASTA} format. These sequences then underwent manual curation to resolve conflicting exon identities, assign exon names and perform initial end refinement based on putative splice junctions.

In order to validate intron/exon boundaries and investigate splicing behaviour among \textit{IGH} constant-region exons in \Nfu and \Xma, published RNA-sequencing data (\Cref{tab:rnaseq-sources}) were aligned to the annotated locus using STAR (\parencite{dobin2013star}, version 2.5.2b). In both cases, reads files from multiple individuals were concatenated and aligned together, in order to make the intron/exon boundary changes in mapping behaviour as clear as possible. % TODO: Add more software versions, or a table at the end

Before aligning the RNA-seq reads, each locus underwent basic repeat masking, using the built-in zebrafish repeat parameters from \program{RepeatMasker} \parencite{smith2016repeatmasker}:

\begin{lstlisting}[language=bash]
RepeatMasker -species danio -dir <masked_locus_dir> -s <unmasked_locus_path>
\end{lstlisting}

\noindent After masking, a \program{STAR} genome index was generated from each locus:

\begin{lstlisting}[language=bash]
STAR --runMode genomeGenerate --genomeDir <star_index_directory_path> --genomeFastaFiles <masked_locus_path> --genomeSAindexNbases <sa_index>
\end{lstlisting}

\noindent where the \snippet{--genomeSAindexNbases} option determines the size of the suffix-array index and is dependent on the length of the reference sequence being indexed : 

\begin{equation}
\mathrm{SA~index~size~(bits)} = \frac{\log_2(\mathrm{length~of~reference~sequence})}{2} - 1
\label{eq:sa_index}
\end{equation}

Following index generation, the RNA-seq reads were mapped to the generated index as follows:

\begin{lstlisting}[language=bash]
STAR --genomeDir <star_index_directory_path> --readFilesIn <input_reads> --outFilterMultimapNmax 5 --alignIntronMax 10000 --alignMatesGapMax 10000
\end{lstlisting}

\noindent where the \snippet{--outFilterMultimapNmax} option excludes read pairs mapping to more than five distinct co-ordinates in the reference sequence and the \snippet{--alignIntronMax} option excludes read pairs spanning predicted introns of more than \kb{10}, and the \snippet{--alignMatesGapMax} option excludes read pairs mapping more than \kb{10} apart. Following alignment, the resulting \format{SAM} files were processed into sorted, indexed \format{BAM} files using \program{SAMtools} \parencite{li2009samtools} and visualised with Integrated Genomics Viewer (IGV, \parencite{robinson2011igv,thorvaldsdottir2013igv}) to determine intron/exon boundaries of predicted exons, as well as the major splice isoforms present in each dataset.

In order to reduce time and memory requirements for generating alignment figures (\Cref{fig:fig:nfu-locus-sashimi,fig:xma-locus-sashimi}), secondary alignments were performed on truncated loci consisting only of the IGHM/D or (where present) IGHZ constant regions, plus a few flanking kilobases on each side. In these cases, the additional parameters constraining multimapping, intron length and mate distance were not necessary due to the much shorter and less repetitive reference sequence.

For species other than \Nfu or \Xma, intron/exon boundaries were predicted manually based on BLASTN and BLASTP alignments to closely-related species and the presence of conserved splice-site motifs (\texttt{AG} at the 5' end of the intron, \texttt{GT} at the 3' end \parencite{shapiro1987splice}). In cases where no 3' splice site was expected to be present (e.g. for CM4 or TM2 exons), the nucleotide exon sequence was terminated at the first canonical polyadenylation site (\texttt{AATAAA} if present, otherwise one of \texttt{ATTAAA}, \texttt{AGTAAA} or \texttt{TATAAA} \parencite{ulitsky2012polya}), while the amino-acid sequence was terminated at the first STOP codon. In many cases, it was not possible to locate a TM2 exon due to its very short (two-amino-acid-residue) conserved coding sequence.

\subsubsection{Synteny analysis}
\label{sec:methods_comp_locus_synteny}

Synteny between subloci in the \Nfu locus was analysed using \program[R]{DECIPHER}'s standard synteny pipeline \parencite{wright2016decipher}, which searches for chains of exact $k$-mer matches within two sequences:

\begin{lstlisting}[language=R]
DBPath <- tempfile()
DBConn <- dbConnect(SQLite(), DBPath)

Seqs2DB(seqs = <sublocus_1_sequence>, type = "XStringSet", dbFile = DBConn, identifier = "IGH1", verbose = FALSE)
Seqs2DB(seqs = <sublocus_2_sequence>, type = "XStringSet", dbFile = DBConn, identifier = "IGH2", verbose = FALSE)

dbDisconnect(DBConn)

SyntenyObject <- FindSynteny(dbFile = DBPath, verbose = FALSE)
\end{lstlisting}

\noindent Cross-locus sequence comparisons between gene segments were performed analogously to the comparisons involved in \vh family assignment, with \snippet{pairwiseAlignment} and \snippet{pid} from \program[R]{Biostrings}.


\subsection{Phylogenetic trees}
\label{sec:methods_comp_trees}


\subsubsection{Species tree construction and annotation}
\label{sec:methods_comp_trees_species}

Information about the interrelationships of most of the teleost taxa discussed in this thesis was obtained from the comprehensive teleost phylogeny of Hughes \textit{et al.} \parencite{hughes2018teleostphylo}, while additional, higher-resolution information on the interrelationships of African killifishes missing from that tree was provided by Cui \textit{et al.} \parencite{cui2019annual}. As no single published tree covered all the species of interest, a simple cladogram of relatioships was constructed manually from the information provided by these two sources. Annotations (e.g. of clade membership or isotype status) were added using \program[R]{tidytree} \parencite{guangchuang2018tidytree}.

\subsubsection{Phylogenetic inference on \igh{} locus sequences}
\label{sec:methods_comp_trees_phylo}

Three phylogenetic trees were inferred from molecular data of \igh{Z} gene segments: one on the \vh segments on \nfu and \xma, one on \ch exons from all species, and one on \igh{Z} constant-regions of \igh{Z} bearing species. In all cases, a sequence \format{FASTA} database was assembled from the relevant species. As identical sequences can cause problems during phylogenetic analysis, entries with completely identical sequences  were then collapsed together into a single \fast{FASTA} sequence, which was relabelled with the names of all its parent sequences. 

A multiple-sequence alignment of the remaining sequences was then constructed with \program{PRANK}:

\begin{lstlisting}
prank -d=<ch_fasta> -o=<output_prefix> DNA -termgap
\end{lstlisting}

The resulting alignment was passed to the maximum-likelihood phylogenetic inference program \lstinline{RAxML} (\parencite{stamatakis2005raxml3,stamatakis2006raxml6,stamatakis2014raxml8}, version 8.2.12), using the SSE3-enabled parallelised version of the software, the standard GTR-Gamma nucleotide substitution model, and built-in rapid bootstrapping:

\begin{lstlisting}
raxmlHPC-PTHREADS-SSE3 -f a -m GTRGAMMA -s <ch_prank_alignment> -w <output_dir> -N <n_bootstrap_replicates> -x <bootstrap_seed> -p <parsimony_seed> -n <output_suffix>
\end{lstlisting} 

Finally, the bootstrap-annotated \snippet{RAxML_bipartitions} file was inspected and rooted manually in \program{Figtree} \parencite{rambaut2012figtree}, before being annotated and visualised in \program{R} with \program[R]{tidytree} and \program[R]{ggtree}, respectively.

\subsubsubsection{\vh-segment tree}

In order to build a phylogenetic tree of \vh segments from the \Nfu and \Xma \igh{} loci, all \vh sequences from those loci were labelled with their origin species and concatenated together. Sequences with more than 25\% missing characters were discarded prior to \program{PRANK} alignment. During tree-inference with \program{RAxML}, 100 bootstrap replicates were used.

\subsubsubsection{\ch exon tree}

To build a phylogenetic tree of \ch exons, nucleotide sequences of constant exons from all species involved in this study were labelled with their origin species and concatenated into a single \format{FASTA} file, which was then filtered to discard transmembrane exons, secretory tails, and sequences with more than 25\% missing characters. In addition, CM4 nucleotide sequences were trimmed to the coding region, removing the 3'-UTR. As with the \vh-segment tree described above, 100 bootstrap replicates were used during tree-inference with \program{RAxML}. As the outgroup among \ch exon groups is unknown, the tree was visualised in unrooted format.

\subsubsubsection{\igh{Z} tree}

To investigate the evolution of \igh{Z}, the \cz{1-4} exons from each \igh{Z} constant region found in any of the analysed genomes were concatenated together into a single sequence and labelled with the source species and constant region. In the event of partial constant regions missing one or more \cz{} exons, the remaining exons were concatenated together in the usual order. Following database processing and alignment, \program{RAxML} tree-inference was run using 1000 bootstrap replicates, in order to increase the reliability and precision of the support values obtained.

\subsection{Ig-Seq data pre-processing}
\label{sec:methods_comp_igpreproc}

Unless otherwise specified, pre-processing utilities used in the following sections are provided by the pRESTO \parencite{vanderheiden2014presto} and Change-O \parencite{gupta2015changeo} suites of Ig-Seq processing tools.

\subsubsection{Sequence uploading and annotation}
\label{sec:methods_comp_igpreproc_annot}

Demultiplexed, adaptor-trimmed MiSeq sequencing data were uploaded by the sequencing provider to Illumina BaseSpace and accessed via \program{basemount}. Library annotation information (fish ID, sex, strain, age at death, death weight, etc.) were added to the read headers of each library:

\begin{lstlisting}
ParseHeaders add -f <field_keys> -u <field_values> -s <input_reads_file>
\end{lstlisting}

User-specified combinations of annotations that together uniquely specified the individual and replicate identity of each library were then combined into a single annotation in addition to the separate source annotations:

\begin{lstlisting}
ParseHeaders.py merge -f <field_keys> -k INDIVIDUAL --act cat -s <annotated_reads_file>
ParseHeaders.py merge -f <field_keys> -k REPLICATE --act cat -s <annotated_reads_file>
\end{lstlisting}

Following annotations, reads from different libraries were pooled together, then split by replicate identity:

\begin{lstlisting}
SplitSeq.py group -s {input} -f REPLICATE s <pooled_reads_file>
\end{lstlisting}

This pooling and re-splitting process enables all reads considered to be a single replicate to be processed together even if sequenced separately, maximising the effectiveness of UMI-based pre-processing while also allowing all replicates to be processed in parallel.

\subsubsection{Read quality control}
\label{sec:methods_comp_igpreproc_filter}


After pooling and re-splitting, the raw read set underwent quality control, discarding any read with an average Phred score of less than 20:
% TODO: Citation needed for Phred score

\begin{lstlisting}
FilterSeq quality -q 20 -s <input_reads_file>
\end{lstlisting}

\subsubsection{Primer masking and UMI extraction}
\label{sec:methods_comp_igpreproc_mask}

\newabbreviation{ID}{ID}{identity}
\newabbreviation{UMI}{UMI}{unique molecular identifier}
\newabbreviation{MIG}{MIG}{molecular identifier group}

Following quality filtering, the reads underwent processing to identify and remove invariant primer sequences. To do this, known primer sequences were aligned to each read from a fixed starting position, and the best match on each read was identified and trimmed. Initially, the primer sequences from the third PCR step of the library prep protocol were used, trimming off primer sequences corresponding to part of the constant \cm{1} exon and the 5' invariant part of the template-switch adapter:

\begin{lstlisting}
MaskPrimers score --mode cut --start 0 -s <3prime_read_file> -p <CM1_primer_file>;
MaskPrimers score --mode cut --start 0 -s <5prime_read_file> -p <TSA_primer_file>
\end{lstlisting}

Following this, the 5' reads underwent a second round of masking using the 3' invariant part of the TSA sequence (\sequence{CTTGGGG}), and the intervening 16 bases were extracted and recorded in each read header as that read's unique molecular identifier (UMI):

\begin{lstlisting}
MaskPrimers score --mode cut --barcode --start 16 --maxerror 0.5 -s <masked_5prime_read_file> -p <TSA_3prime_sequence_file>
\end{lstlisting}

As the match sequence for this second round of masking is shorter and more error-prone than the primer sequences used in the first round, an increased mismatch tolerance was used to increase the number of reads with successfully-extracted UMIs.

\subsubsection{Barcode error handling}
\label{sec:methods_comp_igpreproc_correct}

%The use of UMI sequences enables biases and errors in library insert sequences to be corrected by taking the consensus sequence of all reads sharing a given UMI (a molecular identifier group, or MIG). However, PCR and sequencing errors can also affect the sequence of the UMI itself, in which case reads that in fact belong to a single MIG will be spuriously separated during pre-processing; this can result in spuriously low MIG read counts, spuriously high numbers of unique sequences, and avoidable loss of sequencing data due to reads with erroneous barcodes being discarded (as low-quality, low-read-count unique sequences) at various points in the pre-processing pipeline.
%
%In addition to these barcode \textit{errors}, barcode \textit{collisions} can occur, in which multiple distinct sequences are labelled with the same UMI sequence and spuriously grouped together during UMI grouping. This can lead to spuriously large MIGs and spuriously low numbers of unique sequences, and in extreme cases lead to the rejection and loss of entire MIGs due to an insufficiently high level of sequence identity during consensus generation (see below). % TODO: Move to intro or results

In order to reduce the level of barcode errors in each dataset, primer-masked IgSeq reads underwent barcode clustering, in which reads with the same replicate identity and highly similar UMI sequences were grouped together into the same molecular identifier group (MIG). To do this, 5'-reads were clustered by UMI sequence using  \program{CD-HIT-EST} \parencite{li2006cdhit,fu2012cdhit} with a 90\% sequence identity cutoff, with cluster identities being recorded in a new CLUSTER field in each read header:

\begin{lstlisting}
SplitSeq group -s <masked_reads_file> -f REPLICATE;
ClusterSets barcode -f BARCODE -k CLUSTER --cluster cd-hit-est --prefix B --ident 0.9 -s <split_reads_file>
\end{lstlisting}

In order to split any geniunuly distinct MIGs accidentally united by this process, as well as to reduce the level of ``natural" barcode collisions, the reads then underwent a second round of clustering, this time separately on the read sequences within each barcode cluster using \program{VSEARCH} (an open-source alternative to \program{USEARCH} \parencite{edgar2010usearch,rognes2016vsearch}).
This time, the cluster dendrogram was cut at 75\% total sequence identity, and each subcluster was separated into its own distinct MIG:

\begin{lstlisting}
ClusterSets set -f CLUSTER -k CLUSTER --cluster vsearch --prefix S --ident 0.75 -s <clustered_reads_file>
\end{lstlisting}

These clustering thresholds (90\% for barcode clustering, 75\% for barcode splitting) were identified empirically as the values that maximise the number of reads passing downstream quality checks and included in the final preprocessed dataset.

The cluster annotations from the two clustering steps were combined into a single annotation uniquely identifying each MIG in each replicate. These annotations were further modified to designate the replicate identity of each read, giving each MIG a unique annotation across the entire dataset. These annotations were copied to the 3' reads such that each pair had a matching MIG annotation, and reads without a mate (due to differential processing of the two reads files) were discarded:

\begin{lstlisting}
ParseHeaders collapse -s <clustered_5prime_reads> -f CLUSTER --act cat;
ParseHeaders merge -f REPLICATE CLUSTER -k RCLUSTER --act set -s <annotated_5prime_reads>;
PairSeq -1 <reannotated_5prime_reads> -2 <3prime_reads> --1f BARCODE CLUSTER RCLUSTER --coord illumina
\end{lstlisting}

\subsubsection{Consensus-read generation and pair merging}
\label{sec:methods_comp_igpreproc_consensus}

Following barcode clustering, the 5' and 3' IgSeq reads were separately grouped based on cluster identity, and the reads in each cluster grouping were aligned and collapsed into a consensus read sequence:

\begin{lstlisting}
BuildConsensus --bf RCLUSTER --cf <header_fields> --act <copy_actions> --maxerror 0.1 --maxgap 0.5 -s <annotated_reads_file>
\end{lstlisting}

Positions at which at least half the aligned reads in the MIG had a gap character were deleted from the consensus (\snippet{--maxgap 0.5}), while MIGs with a mismatch rate from the consensus of more than 10\% were discarded from the dataset (\snippet{--maxerror 0.1}). The resulting \format{FASTQ} file contained a single consensus sequence for each cluster annotation, labelled with its CONSCOUNT (the number of reads contributing to that consensus sequence),  the number of reads allocated to each barcode in the cluster, and various header fields (\snippet{<header_fields>}) propagated from the contributing reads by summing or concatenating the values from each contributing read (\snippet{<copy_actions>}). 

After consensus-read generation had been performed for both 5' and 3' reads, the annotations attached to each read were unified across read pairs with matching cluster identities, and consensus reads without a mate of the same cluster identity were dropped:

\begin{lstlisting}
PairSeq -1 <5prime_consensus_reads> -2 <3prime_consensus_reads> -coord presto
\end{lstlisting}

Following consensus-read generation and annotation unification, consensus-read pairs with matching cluster annotations were aligned and merged into a single contiguous sequence. Where possible, this was done by simply aligning the two mate sequences against each other; where this was not possible (e.g. due to the lack of a significant sequence overlap) the consensus reads were instead aligned with \program{BLASTN} \parencite{altschul1990blast,altschul1997blast} to a reference database of \vh sequences to generate a merged sequence, with \sequence{N} characters used to separate pairs that aligned in a non-overlapping manner on the same \vh segment:

\begin{lstlisting}
AssemblePairs sequential --coord presto --scanrev --aligner blastn --rc tail --1f <header_fields> -1 <5prime_consensus_reads> -2 <3prime_consensus_reads> -r <vh_fasta_file>
\end{lstlisting}

In either case, annotation fields were copied to the new merged sequence from the forward consensus read, with the fields to be copied specified by \snippet{--1f <header_fields>}. Sequence pairs for which both alignment approaches failed were discarded.

\subsubsection{Collapsing identical sequences and singleton removal}
\label{sec:methods_comp_igpreproc_collapse}

To quantify the abundance of each unique sequence present in each sample, merged consensus sequences with identical insert sequences but distinct MIG assignments were collapsed together into a single \format{FASTQ} entry, recording the number, size and UMI makeup of contributing MIGs in each case in the sequence header alongside any existing annotation information:

\begin{lstlisting}
CollapseSeq --inner --cf <header_fields> --act <copy_actions> -n 20 -s <merged_consensus_seqs>
\end{lstlisting}

The collapsed sequences from each replicate identity in the dataset were then concatenated into a single file for easier downstream processing. 

As sequences represented by only a single read across all MIGs in the dataset (so-called \textit{singleton} sequences, with a \snippet{CONSCOUNT} of no more than 1) could not be corrected by UMI clustering or consensus building, they are not considered reliable for downstream processing and analysis; as a result, they were here identified and separated from the other collapsed sequences:

\begin{lstlisting}
SplitSeq group -f CONSCOUNT --num 2 -s <collapsed_consensus_seqs>
\end{lstlisting}

Finally, the non-singleton sequences so identified were converted into \format{FASTA} format with \program{seqtk} \parencite{li2016seqtk} for downstream processing:

\begin{lstlisting}
seqtk seq -a <non_singleton_consensus_seqs> > <presto_fasta_output>
\end{lstlisting}

\subsubsection{Assigning VDJ identities with IgBLAST}
\label{sec:methods_comp_igpreproc_igblast}

%The \program{pRESTO} pipeline described in preceding sections converts raw IgSeq reads into corrected, collapsed consensus sequences, with each sequence representing a unique sequence present in the original RNA sample. Before analysing the diversity, structure, or other properties of the sequenced repertoires, further processing is required to assign V/D/J identities, clonotype membership, and germline sequences to each sequence in the dataset. These processing steps were here performed using \program{pRESTO}'s sister program \program{Change-O} and the ... program \program{IgBLAST}. % TODO: Describe and cite IgBLAST; cite Change-O % TODO: Move to results

To assign \vh, \dh and \jh identities to the corrected, collapsed consensus sequences produced by the \program{pRESTO} pipeline, gene segment databases were aligned to the \format{FASTA} output from \Cref{sec:methods_comp_igpreproc_collapse} with \program{IgBLAST} \parencite{ye2013igblast}. To do this, each reference file was converted into a \program{BLAST} database with \snippet{makeblastdb}, and the output \format{FASTA} file was aligned to these databases with \snippet{igblastn}:

\begin{lstlisting}
makeblastdb -parse_seqids -dbtype nucl -in <vh_reference_fasta> -out <vh_db_prefix>;
makeblastdb -parse_seqids -dbtype nucl -in <dh_reference_fasta> -out <dh_db_prefix>;
makeblastdb -parse_seqids -dbtype nucl -in <jh_reference_fasta> -out <jh_db_prefix>;
igblastn -ig_seqtype Ig -domain_system imgt -query <presto_fasta_output> -out <igblast_output> -germline_db_V <vh_db_prefix> -germline_db_D <dh_db_prefix> -germline_db_J <jh_db_prefix> -auxiliary_data <jh_aux_file> -outfmt '7 std qseq sseq btop'
\end{lstlisting}

A \jh auxiliary file was used to indicate the reading frame and CDR3 boundary co-ordinate of each \jh sequence in the reference database (\snippet{-auxiliary_data <jh_aux_file>}); for more information on this file and how it was generated, see \Cref{sec:methods_comp_locus_segments}.

\subsubsection{Clonotype inference with Change-O}
\label{sec:methods_comp_igpreproc_clones}

Following V/D/J identity assignment, the output \format{FASTA} file from \Cref{sec:methods_comp_igpreproc_collapse}, raw reference segment databases from \Cref{sec:methods_comp_locus_segments} and segment assignments from \Cref{sec:methods_comp_igpreproc_igblast} were used to construct a tab-delimited \program{Change-O} sequence database:

\begin{lstlisting}
MakeDb igblast --regions --scores --failed --partial --asis-calls --cdr3 -i <igblast_output> -s <presto_fasta_output> -r <vh_reference_fasta> <dh_reference_fasta> <jh_reference_fasta>
\end{lstlisting}

\noindent where \snippet{--failed} indicates that invalid sequences should be included in a separate database rather than discarded outright, \snippet{--regions} and \snippet{--cdr3} indicate that the database should include FR and CDR annotations, \snippet{--scores} indicates that the database should include alignment score metrics, \snippet{--partial} indicates that sequences with incomplete V/D/J alignments (e.g. those without an unambiguous V- or J-assignment) should not automaticall qualify as failed, and \snippet{--asis-calls} instructs the program to accept assignments to V/D/J databases with non-standard name formatting. Following database construction, each entry was given a unique name on the basis of its replicate identity and ordering. 

In order to compute the appropriate distance threshold for clonotype assignment, each sequence was assigned a nearest-neighbour Hamming distance within the repertoire, using the related \program{R} packages \program[R]{SHazaM} and \program[R]{Alakazam}:

\begin{lstlisting}[language=R]
tab <- readChangeoDb(<named_db_path>) %>% mutate(ROW = seq(n()))
dist_pass <- distToNearest(tab, model = "ham", normalize = "len", fields = <group_field>, first = FALSE)
dist_fail <- tab %>% filter(! ROW %in% dist_pass$ROW) %>%·mutate(DIST_NEAREST = NA) 
writeChangeoDb(bind_rows(dist_pass, dist_fail), <db_output_path>)
\end{lstlisting}

\noindent where \snippet{model = "ham"} indicates that a single-nucleotide Hamming distance metric is to be used, \snippet{normalize = "len"} that distances should be normalised by total sequence length, \snippet{first = FALSE} determines how to handle ambiguous V/J calls, and \snippet{fields = <group_field>} that only distances within an entry group (see below) should be considered.

Following assignment of nearest-neighbour distances, a distance threshold for clonotyping was computed by fitting a pair of unimodal distributions to the nearest-neighbour distribution over all sequences and selecting the threshold that minimises the maximises the average of sensitivity and specificity when assigning a point to one of these distributions (for more information, see \Cref{sec:igseq_pilot_clones} and \parencite{nouri2018threshold}). As with nearest-neighbour distance assignment, this was done in \program{R} using \program[R]{SHazaM} and \program[R]{Alakazam}; all four possible models (fitting either a normal or gamma distribution to each of the two peaks) were tried, and the one with the highest maximum likelihood was used to compute the threshold value:

\begin{lstlisting}[language=R]
tab <- readChangeoDb(<changeo_db_path_with_distances>)
models <- c("gamma-gamma", "gamma-norm", "norm-gamma", "norm-norm")
thresholds <- numeric(length(models))
likelihoods <- numeric(length(models))
for(n in 1:length(models)){
  obj <- tryCatch(findThreshold(as.numeric(tab$DIST_NEAREST), method = "gmm", model = "hmm", cutoff = "opt"), error = function(e) return(e$message), warning = function(w) return(w$message))
  thresholds[n] <- ifelse(isS4(obj), obj@threshold, NA)
  likelihoods[n] <- ifelse(isS4(obj), obj@loglk, NA)
}
if (!all(is.na(thresholds))) write(thresholds[last(which(likelihoods == max(likelihoods, na.rm = TRUE))], <threshold_output_path>)
\end{lstlisting} 

Following inference of the correct distance threshold, clonotype inference was performed on the sequence database by grouping sequences by V- and J-assignment and CDR3 length, computing pairwise Hamming distances between each pair of sequences in each group, and ... : %TODO: Finish describing clonotyping algorithm, briefly here and in more detail elsewhere

%Once the database was constructed, the sequences it contained underwent clonotype assignment, with sequences predicted to have arisen from a common na\"{i}ve B-cell ancestor annotated as belonging to the same B-cell clone. To do this, sequences from the same individual with compatible V- and J-assignments and the same CDR3 length were grouped together, and each of these groups underwent single-linkage clustering based on length-normalised pairwise Hamming distances between sequences: % TODO: Citations for various things in here % TODO: To results

\begin{lstlisting}
DefineClones --act set --model ham --sym min --norm len --failed -d <changeo_db> --dist <cluster_distance_threshold> --gf INDIVIDUAL
\end{lstlisting}

\noindent where \snippet{--act set} tells the program how to handle ambiguous V/D/J assignments, \snippet{--model ham} specifies the clustering metric as the pairwise Hamming distance; \snippet{--norm len} indicates that the Hamming distances should be normalised by sequence length; \snippet{--sym min} specifies that, in the event of asymmetric \texttt{A -> B} and \texttt{B -> A} distances (e.g. arising from length normalisation) the minimum distance should be used; and \snippet{--dist} specifies the distance threshold at which to cut the clustering dendrogram.

Finally, the clonotype numbers assigned to each group % individual? replicate?
were combined with the group ID of each clonotype to give a unique ID for each clonotype in the dataset.

\subsubsection{Germline inference with Change-O}
\label{sec:methods_comp_igpreproc_clones}

After threshold determination and clonotyping, a so-called ``full-length germline sequence" is constructed for each sequence. To do this for a given sequence, germline V/J sequences are trimmed of deleted positions and concatenated together, separated by a masked region of length corresponding to the inserted nucleotides and remaining \dh sequence:

\begin{lstlisting}
CreateGermlines -g dmask --cloned -d <clonotyped_changeo_db> -r <vh_reference_fasta> <dh_reference_fasta> <jh_reference_fasta> --failed
\end{lstlisting}

where \snippet{-g dmask} indicates that \dh sequences should be masked as well as insert sequences and \snippet{--failed} indicates that sequences that fail germline assignment should be retained in a separate database. Importantly, \snippet{--cloned} indicates that sequences from the same clone should recieve the same germline assignment, based on a simple majority rule among sequences in the clone; this process also enables assignment of unambiguous segment identities to ambiguously-assigned sequences within larger clones, improving segment calls in the dataset.

%Finally, the clonotyped and germline-inferred sequence database was split on the basis of functionality, with nonfunctional (frame-shifted, nonsense-mutated, or lacking V/J assignments) sequences moved into a separate database file before downstream analysis:
%
%\begin{lstlisting}
%ParseDb split -f FUNCTIONAL -d <germlined_changeo_db>
%\end{lstlisting} % TODO: Restore once pipeline finalised

% \subsection{Downstream analysis of IgSeq data}
% [or split into multiple sections?]

% ... 

\subsection{Downstream analysis of antibody repertoires}
\label{sec:methods_comp_igdownstream}

% ...

\subsection{Zipf approximation of rank:frequency distributions}
\label{sec:methods_comp_igdownstream_zipf}

\subsection{Repertoire Dissimilarity Index (RDI) measurements}
\label{sec:methods_comp_igdownstream_rdi}

\newabbreviation{RDI}{RDI}{Repertoire Dissimilarity Index}

The Repertoire Dissimilarity Index (RDI) \parencite{bolen2017rdi} is a method for computing a distance between any two repertoires on the basis of their V(D)J segment composition, based on the Euclidean distance between their respective V(D)J-abundance vectors. As such, it provides a pairwise distance metric which can be used to compare, cluster and visualise the relationship between different antibody repertoires.

\subsection{Computing diversity spectra}
\label{sec:methods_comp_igdownstream_spectra}





%!TEX root = ../thesis.tex
%*******************************************************************************
%*********************************** First Chapter *****************************
%*******************************************************************************

\chapter{Introduction}  %Title of the First Chapter
%s\doublespacing
\onehalfspacing

% Chapter summary (should fit on title page)

% Chapter 1 summary
% Should fit on chapter title page

\section*{Summary} % Fits one one page if 1.5-spaced, but not at double spacing

Faced with the pervasive risk of parasitic infection, vertebrates have evolved highly complex immune systems capable of effectively combating a wide range of pathogenic threats. Among the most sophisticated of these adaptations is the adaptive immunity provided by B- and T-lymphocytes, which utilise a range of specialised genome-editing mechanisms to generate a vast array of distinct antigen-binding proteins. These remarkable systems, however, undergo severe systemic decline with ageing, leading to greatly increased rates of infection-related morbidity in older individuals. Comparable immunosenescent phenotypes have been widely observed in mammals, birds, fish and elsewhere, and appear to be broadly conserved across the vertebrate lineage.

To understand and counter the complex changes that occur in the adaptive immune system with age, it is necessary to analyse the whole population of lymphocyte antigen-receptor sequences present in an individual. Such a top-down approach was impossible until relatively recently, when the advent of modern high-throughput sequencing technologies enabled the development of specialised protocols for immune-repertoire sequencing and analysis. Since then, the field of immune-repertoire studies has developed rapidly, providing a new and more powerful method for interrogating the changes ocurring in adaptive immune repertoires in a wide variety of contexts, including ageing. However, while initial human studies have indicated a decline in the diversity of these repertoires in older people, there remains a need for further research in this area. % This is weak, fill in once you've done your lit review.

As the shortest-lived vertebrate to be bred in captivity, the African turquoise killifish (\textit{Nothobranchius furzeri}) represents a powerful model for studying vertebrate-specific ageing processes, including immunosenescence of the adaptive immune system. Though the killifish has seen rapid development as a model system for ageing research, little was known about its adaptive immune system prior to the work described in this thesis. % In this chapter...? But it's a summary, not an introduction.

\pagebreak

% Sections

\section{The vertebrate adaptive immune system} % Humoral/B-cell/antibody immune system?

All organisms exist in a condition of intense competition for resources, with predators, peers, and parasites all competing, in some way, for the nutrients and energy consumed and used by an individual. Among the most insidious of these competitors are parasites who attempt to colonise an organism's own body, consuming its stores of nutrients and energy and turning its internal mechanisms to their own advantage. When an organism falls prey to one of these parasites and manifests the symptoms of its exploitation, we call it disease. When the organism utilises adaptations to prevent this exploitation, through excluding or killing the parasites, we call it immunity.

Given the extreme selective pressure to protect their fitness from parasitic exploitation, it is perhaps unsurprising that so many different organisms have evolved immune systems of great complexity and effectiveness. Nevertheless, the intricacy of the vertebrate immune system has proven one of the most enduringly fascinating aspects of vertebrate biology, comparable to vertebrate neural systems in its complexity. Indeed, the vertebrate immune system shows many parallels with the nervous system, being the only other system capable of complex information processing and memory. This complexity, which is fundamental to the effectiveness of the vertebrate immune system, rests on the interplay between the two traditional wings of vertebrate immunology: the innate immune system, and the adaptive.

Innate immunity refers to a large collection of mechanisms designed to exclude, sequester, or kill invading pathogens (disease-causing organisms) in a rapid and nonspecific manner. Many innate systems combat pathogens in ways that are either physically difficult to circumvent (such as external barriers, or engulfment by phagocytic cells) or which target aspects of pathogen biology that are difficult to alter without catastrophic loss of function (such as this wonderful example %example
). As such, the great majority of possible pathogenic threats are dealt with rapidly and effectively by the innate immune system, either by keeping parasites from accessing vulnerable parts of the organism or by rapidly and nonspecifically eliminating them once there.

Despite its speed, power and impressive generality, the innate immune system suffers from severe limitations. The first is that it is helpless in the face of evolutionarily novel threats to which its existing defences do not apply. The second, perhaps more fundamental, problem is that many common pathogens are capable of evolving at speeds vastly exceeding that of vertebrates. For example, many bacteria have generation times of much less than an hour, compared to months or years for most vertebrates, while also exhibiting much higher per-cell-division rates of mutation. And even this rapid rate of evolution is far exceeded by the highly volatile genomes of many viruses.

This capacity for parasitic organisms to out-evolve their hosts represents a serious problem for vertebrates, who cannot hope to effectively respond to these threats through the generation of new and improved innate immune mechanisms via selection. Instead, what is needed is a mechanisms by which vertebrates can dynamically learn to respond to novel immune threats within the lifespan of a single organism. That mechanism is the adaptive immune system.

% Need to rapidly focus onto B-cell immunity in particular, I know nothing of T-cells

The mechanisms used to generate this sequence diversity in the B-cell population are themselves diverse, and have been discovered progressively over the past decades. The most fundamental, and well-known, such mechanism is so-called V(D)J recombination, first discovered quite some time ago. %Find out more about history of this

In V(D)J recombination, a number of

A canonical immunoglobulin heavy chain (IgH) gene locus consists of clusters of variable (V), diversity (D) and joining (J) regions in series, followed by some number of larger constant-region exons. During B-cell maturation, a single V, D and J region are selected and the intervening DNA regions are excised to produce a single, contiguous VDJ sequence. As part of this process, nontemplated nucleotides are inserted and deleted at the V/D and D/J junctions, a process known as junctional diversification. Following transcription, the sequence between the VDJ sequence and the first constant-region exon is removed by splicing to produce a mature IgH transcript, with its characteristig VDJC sequence structure.

The manner of constant region selection in B-cells differs significantly between mammals and teleosts. In the former, a number of distinct constant regions are present in series on the chromo

\newpage 
\section{Structure and diversification of the antibody heavy chain}

% Pretend light chains and TCRs don't exist for now; you can generalise later as needed

The great majority of antibodies produced by the gnathostome adaptive immune system share a canonical tetrameric structure: two heavy chains and two light chains, arranged into a roughly Y-shaped configuration. This structure comprises three important functional domains: two antigen-binding domains, formed by the N-terminal portions of the heavy and light chains, and one effector domain formed by the C-termini of the heavy chains. As a result of these distinct functionalities, the two ends of an antibody heavy-chain protein have very different properties when considered as a population. The N-terminal variable domain is highly variable in amino-acid sequence, with unrelated\footnote{i.e. not derived from the same plasma-cell clone} B-cells typically displaying different and often unique sequences. The C-terminal constant region, meanwhile, shows very limited sequence diversity, with all B-cells in the body producing one of a small number of distinct effector classes. This combination of a highly-diverse antigen-binding domain and a limited range of distinct effector domains enables antibodies to simultaneously recognise and contact a vast array of potential antigens, while simultaneously interacting with the rest of the immune system in a predictable manner.

The constant-region sequence of an antibody heavy chain (or its mRNA) is known as the class or isotype of that antibody, while its variable-region sequence is known as its idiotype. % Definitions box?

The variable region of an antibody can be further subdivided into three complementarity-determining-regions (CDR1-3), which form part of the antigen-binding site and directly contact the antigen, and four framework regions (FR1-4), which do not. Sequence variability in the variable region is concentrated in the CDRs, with CDR3 showing by far the greatest variability for a variety of reasons discussed below. % Figure for this? E.g. align Vs and mark regions of non-conservation
This sequence variability is generated by several highly-specialised genome-editing mechanisms, which together produce a distinct idiotypic sequence on the chromosome of each developing B-cell. Broadly speaking, these diversification mechanisms can be divided into three categories, each of which has a distinct effect on repertoire sequence diversity; these categories are V(D)J recombination, junctional diversification, and somatic hypermutation.

\subsection{Mechanisms of heavy-chain diversification I: V(D)J recombination}


% FIGURES AND TABLES:
% - Schematic of native locus state -> VDJR -> recombined locus
% - RAG structure and DNA Binding
% - Schematic of RAG action and DNA excision
% - Schematic of RSS structure
% - Sequence logos of human/mouse/other RSSs
% - Schematic of V/D/J coverage on protein chain
% - Length distributions of human/mouse/other V/D/J segments

The structure of the immunoglobulin heavy chain (\textit{IGH}) locus differs markedly between its native germline configuration and that adopted by a mature B-cell. In the native state, most \textit{IGH} loci comprise large numbers of isolated gene segments separated by non-coding DNA. These segments can be divided into three classes, variable (V), diversity (D) and joining (J), with distinct sequence properties and length distributions (see Box~\dots). % FIGURE: VDJ length distributions in humans etc; schematic of V/D/J positioning in peptide chain
During B-cell development, site-specific recombination reactions result in the rearrangement of individual V, D and J segments into a continuous VDJ sequence, with the intervening genomic regions permanently excised and degraded. % Citation about what happens to excised sequences
This irreversible genomic maturation process is known as V(D)J recombination.

VDJ recombination is carried out by the RAG endonuclease complex, an enzyme formed by the association of \textbf{r}ecombination-\textbf{a}ctivating \textbf{g}enes 1 \& 2 \citep{jung2006vdjr}. % more detailed citation, roles of RAG1 vs RAG2
This complex, which is expressed specifically in developing lymphocytes, % citation needed
introduces recognises specialised recombination sequences (RSSs) at the ends of IGH gene segments and introduces targeted double-strand breaks between two segments and their respective RSSs \citep{jung2006vdjr}; these DSBs are then repaired by non-homologous end joining, resulting in a continuous coding sequence spanning both segments. The excised DNA ... % Citation for this

VDJ recombination in the \textit{IGH} locus is highly structured, and occurs in a specific order. First, a D and a J segment are selected and recombined to produce a DJ sequence. Then, a V region is selected and recombined with the DJ to produce a continuous VDJ sequence constituting the variable region of the heavy chain. The complete protein sequence is produced during later transcriptional splicing, which joins this variable-region sequence to downstream constant-region exons to produce a mature \textit{IGH} mRNA. This strict ordering of V/D/J segments, which obtains in the vast majority of recombined sequences observed, is produced through the combination of a variety of regulatory mechanisms. The most basic of these is the structure of the RSSs, which comprise a conserved heptamer and nonamer sequence separated by a relatively unconserved spacer region of either 12 or 23bp \citep{jung2006vdjr}, corresponding to either one or two turns of the DNA helix. V and J segments in the IGH locus are flanked by RSSs with 23bp spacer regions, while those flanking D-regions have 12bp spacers %citation needed
As the RAG recombinase specifically recognises pairs of RSSs with dissimilar spacer lengths (a restriction known as the 12/23 or one-turn/two-turn rule), direct V-to-J recombination events are excluded \citep{jung2006vdjr}. % Better citation if possible.

In addition to the restrictions imposed by the 12/23 rule, additional limitations on VDJ recombination are imposed by the requirement that RAG binding be preceded by transcription from the to-be-recombined segments, apparently in order to open the chromatin structure in that region \citep{jung2006vdjr}. Both V and D segments, but not J segments, are preceded by upstream promoter regions, which are involved in transcriptional initiation during specific stages of B-cell development...

Complete VDJ recombination places the V-region promoter in close proximity to a highly-conserved enhancer element (known as iE$\mu$) lying between the last J segment and the first constant-region exon \citep{jung2006vdjr}; this enhancer is important for strong expression of the mature IGH mRNA from the pre-B-cell stage onwards. 

\begin{itemize}
\item \textbf{Variable (V) segments} are the longest class of \textit{IGH} gene segment, with functional segments ranging from \dots to \dots nucleotides in humans and mice. % Figure and citation for this; add amino-acid lengths
This includes the coding sequence for the bulk of the variable region of the heavy chain, including all of FR1, CDR1, FR2, CDR2 and FR3, as well as the 5'/N-terminal part of CDR3 \citep{jung2006vdjr}. Each V segment also includes an N-terminal signal peptide sequence for antibody trafficking % citation needed
, and is preceded by its own 5'-UTR and promoter region. % citation for V promoters?
\item \textbf{Diversity (D) segments} are the shortest class of \textit{IGH} gene segment, ranging from \dots to \dots nucleotides in humans. %!
They form the middle part of the heavy chain CDR3. Like V-segments, D-segments are flanked by upstream promoter regions, which are involved in initiating the transcription required for DJ recombination. \citep{jung2006vdjr}
\item{Joining (J) segments} are of intermediate length, \dots % length data
. They form the 3'/C-terminal part of heavy chain CDR3 and the 5'/N-terminal part of FR4. Each J-segment is succeeded on the chromosome by a splice donor site, which is used to splice the recombined variable region to the constant region following transcription of IGH mRNA.
\end{itemize}

In its native germline state, the antibody heavy chain locus occupies a highly unusual configuration



The immunoglobulin heavy chain (\textit{IGH}) locus has a highly distinctive structure, whose broad outlines are shared only by other adaptive-immune antigen receptors...

The human genome contains seven loci that follow these general patterns of diversification: the \textit{IGH} locus, two light chain loci, and four T-cell receptor loci \citep{jung2006vdjr}.

% First explain structure of antibody protein chain
% -> mRNA
% -> unrecombined locus



An antibody protein chain can be divided into two regions with distinctive roles in the immune system: an N-terminal antigen-binding domain

. The C-terminal part of an antibody, known as the constant region, 

The N-terminal portion, by contrast, 

While the first two complementarity-determining regions of the heavy chain are contained entirely within the V segment, the heavy-chain CDR3 is formed by the combination of the V, D, and J segments selected for rearrangement \citep{jung2006vdjr}. As a result, the bulk of sequence diversity in the heavy chain falls within the CDR3, which is responsible for the greater part of antigen-binding variability among antibodies. % Citation for this
Nevertheless, the combinatorial sequence diversity produced by rearrangements of V/D/J segments alone is necessarily limited, with a maximum of $a \times b \times c$ possible sequences for a locus containing $a$ V, $b$ D and $c$ J sequences; far below the diversity of possible antigens. % citation needed 
Fortunately, the potential sequence diversity of the na\:{i}ve repertoire is powerfully augmented by a second diversification mechanism, taking place alongside VDJ recombination during B-cell development: junctional diversification.

\subsection{Mechanisms of heavy-chain diversification II: junctional diversification}

...

The net number of inserted nucleotides is essentially random, but the reading frame of the IGH transcript is fixed by the positions of the ATG initiation codon in the V-segment and the J-segment/constant-region splice junction. As a result, junctional diversification results in a large number of frameshift mutations, in addition to STOP codons that prematurely truncate the protein sequence. % Are these STOP mutations also counted as non-productive, or just frameshifts
Such loci, which are unable to produce a functional heavy chain protein, are termed non-productive, while those without such mutations are termed productive ; at a first approximation, roughly one-third of VDJ recombination events are expected to produce a productive sequence \citep{jung2006vdjr}. % Derivation and graphical representation of this; probability of two discrete random variables being congruent modulo three?

While a given recombination event only has about a one-third chance of producing a productive rearrangement, B-cells, like other somatic cells in vertebrates, are diploid. As a result, while a given recombination event only has about a one-third chance of producing a productive rearrangement, a B-cell which undergoes a non-productive rearrangement on one chromosome can make a second attempt on the other. Given the one-third approximation given above, this means that about 55\% of B-cells will achieve a productive rearrangement on one or the other chromosome; the other 45\%, unable to produce a functional heavy chain from either chromosome, die by apoptosis during B-cell development \citep{jung2006vdjr}. % Adapt figure 3 from \citep{jung2006vdjr}
This loss of almost half of all developing B-cells, a substantial cost, demonstrates the profound selective value of the additional antigen-binding diversity provided by junctional diversification.

Of those B-cells undergoing a productive heavy-chain rearrangement (and therefore surviving this process), roughly three-fifths are expected to bear one rearranged and one unrearraranged locus, with the remaining two-fifths bearing two rearranged loci, one of which in unproductive. This 60/40 ratio is roughly borne out by empirical data 


\subsection{Structure of mammalian \textit{IGH} loci} % And teleosts/others as available

% Human locus

The mouse \textit{IGH} locus also adopts a translocon structure, in this case with an enormous V-region, comprising 150 or more V-segments (depending on the strain) spanning 2.7Mb \citep{jung2006vdjr}. The 12-13 murine D segments occupy a region of approximately 50kb, while the four J segments cover about 2kb. This is followed by 200kb of constant region exons. 

% Mouse locus (depending on strain): 150+ V, 12-13 D, 4 J. (\citep{jung2006vdjr})
% Total length of murine locus: ~3Mb near telomeric end of chr12

\subsection{Antibody effector function and isotype diversity}



\newpage
\section{The African turquoise killifish as a model for vertebrate ageing}

% POSSIBLE FIGURES:
% - Nothobranchius genus range, photos of different notho males
% * Photo of male and female GRZ TK
% * Lifespan curves of male and female GRZ-AD
% - Diapause schematic
% - Photograph of ephemeral pools in Zimbabwe
% - Something showing correlation between aridity and lifespan in nothos
% - Schematic comparing GRZ lifespan to other common ageing models, along with important present/absent human-like systems
% - Phylogeny of TK within teleosts, with divergence times and human outgroup

% POSSIBLE TABLES:
% - List of notho species used in research with approx mean/max lifespans
% - List of ageing phenotypes observed/not observed in GRZ / wild-derived strains
% - Table of TK genome assemblies (JENA/STFD/CLGN) with metrics
% - Comparison of genome metrics (size, repetitiveness, etc) between TK and other models
% - Life history table for GRZ, MZM, other nothos, other fish models

The genus \textit{Nothobranchius} comprises a broad group of annual freshwater fishes distributed across equatorial and subequatorial Africa \citep{valdesalici2003lifespan}, with species diversity concentrated in the south-east of the continent \citep{genade2005annual}. Members of this genus share a suite of adaptations to life in ephemeral pools and rivers, most notably the production of desiccation-resistant embryos capable of surviving through the dry season in a diapause state \citep{genade2005annual}. Fish from this genus have been known for several decades to exhibit very rapid growth and short lifespans, consistent with their evolving under conditions of very high extrinsic mortality \citep{valdesalici2003lifespan}, with many species exhibiting a median lifespan of less than one year. Nevertheless, there is wide variation within the genus in body size, growth rate and lifespan, with species from less arid regions tending to show slower growth and longer median lifespans \citep{genade2005annual}.

Like other \textit{Nothobranchius} species, the turquoise killifish (\textit{Nothobranchius furzeri}) is a medium-sized annual fish first isolated from ephemeral freshwater pools -- in this case, from a relatively arid region of southeastern Zimbabwe \citep{jubb1971new,genade2005annual}. Even by the standards of the \textit{Nothobranchius} genus, \textit{N. furzeri} exhibits extremely rapid growth, maturation, and ageing, with the most widely-used laboratory strain (GRZ) exhibiting a median lifespan of just 9-14 weeks \citep{valdesalici2003lifespan,genade2005annual,terzibasi2008strains,kirschner2012map,valenzano2015genome} % Update range as other papers get different results
-- the shortest lifespan of any captive vertebrate. Moreover, while all turquoise killifish strains are very short-lived by vertebrate standards, different strains of this species have been found to differ several-fold in their median and maximum lifespans \citep{terzibasi2008strains,kirschner2012map}, providing a rare opportunity for intraspecific comparative ageing studies \citep{terzibasi2008strains,terzibasi2009dr,hartmann2009telomeres} and mapping the genetic underpinnings of lifespan \citep{kirschner2012map}. This combination of extremely short-lived laboratory strains and a wide range of lifespan phenotypes within a single species makes the turquoise killifish an extremely promising model organism for ageing research, especially when combined with the presence of vertebrate-specific adaptations absent in short-lived invertebrate ageing models.


Despite its very short lifespan, \textit{N. furzeri} has been found to show a wide range of senescent phenotypes in even the shortest-lived strains, including lipofuscin deposition \citep{genade2005annual};  accumulation of senescence markers \citep{genade2005annual};  increased neurodegenaration \citep{valenzano2006resveratrol1,valenzano2006resveratrol2}; impaired learning and behavioural phenotypes  \citep{genade2005annual,valenzano2006resveratrol1}; and a high incidence of degenerative and neoplastic lesions \citep{dicicco2011histopathology}. These diverse phenotypes indicate that the short lifespan of the turquoise killifish is the result of an accelerated general ageing process, rather than the specific failure of a particular organ or system.
Moreover, established anti-ageing interventions such as resveratrol treatment \citep{valenzano2006resveratrol1}, reduction in ambient temperature \citep{valenzano2006temperature} and dietary restriction \citep{terzibasi2009dr} also extend lifespan in the turquoise killifish, indicating a strong analogy with the ageing phenotypes observed in canonical model systems.


% (though others, such as shortened telomeres \citep{hartmann2009telomeres} and gonadal fibrosis \citep{dicicco2011histopathology}, are only observed in longer-lived strains of the species) % Add more as you read them
% Other ageing phenotypes, including
% telomere shortening \citep{hartmann2009telomeres}, gonadal fibrosis \citep{dicicco2011histopathology}, % not observed in GRZ
% mtDNA depletion \citep{hartmann2011mitochondria}, and bioenergetic impairment \citep{hartmann2011mitochondria} % Not studied in GRZ
% have been observed in longer-lived \textit{N. furzeri} strains, but were either not observed (telomeres, gonads) or not studied (mitochondrial phenotypes) in the short-lived GRZ strain.
%.  

Due primarily to its potential as a model organism for ageing research, the turquoise killifish has also seen rapid development as a genetic model. The short-lived GRZ strain has been bred in captivity for fifty years and at least a hundred generations \citep{terzibasi2007review} and exibits a very high degree of homozygosity \citep{reichwald2009genome,valenzano2009map,kirschner2012map}, providing a uniform genetic background for experimental interventions. A variety of effective transgenesis and mutagenesis methods have been developed for this species, including \textit{tol2} transgenesis \citep{valenzano2011tol2,hartmann2012tol2,allard2013inducible} %TALENS?
and CRISPR \citep{harel2015crispr,harel2016crispr}; furthermore, a number of important genetic resources are now available, including linkage maps \citep{valenzano2009map,kirschner2012map,valenzano2015genome}, a transcript catalogue \citep{petzold2013transcriptome}, an miRNAome % citation needed
and a mitochondrial genome \citep{hartmann2011mitochondria}. Most importantly, a high-quality nuclear genome of the short-lived GRZ strain is now available
% with a recent study integrating evidence from multiple independent studies to produce a very high-quality assembly \cite{ray's genome paper, plus others}. 
\citep{reichwald2015genome,valenzano2015genome}. % Add more here
The progressive release of improved genome assemblies for the turquoise killifish has been particularly important for this project, and will be discussed in more detail in Section ??. %!
The genome itself is unusually large and repetitive by teleost standards, potentially contributing to these fishes' unusually short lifespan. % citation needed

% Needed: oxidation levels, telomere dysfunction, more senescence markers

% Ageing-related genes under positive selection (both genome papers)


%The killifish genome is unusually large by teleost standards, % TODO: figure for this? update figures and add citations as you read newer sources
%with an estimated size of 1.5 to 2 gigabases \citep{reichwald2009genome,reichwald2015genome}. This large size is primarily accounted for by the exceptionally high repeat content of the killifish genome, with at least 21\% and 24\% composed of tandem and other types of repeats, respectively \citep{reichwald2009genome}. GC content is also relatively high, at 44\%, though this average is affected somewhat by the presence of a distinct fraction of highly GC-rich tandem repeats \citep{reichwald2009genome}. Karyotyping \citep{reichwald2009genome} and sequencing analysis \citep{reichwald2015genome} both indicate a chromosome number of $2n = 38$, corresponding to 19 distinct linkage groups in the haploid genome. 

%They are strongly sexually dimorphic, with an XY sex-determination system \citep{reichwald2015genome,valenzano2009map}. %Photo

Phylogenetically, the genus \textit{Nothobranchius} falls within the Cyprinodontiformes, and the turquoise killifish is therefore closely related to a number of fish species that have been the subject of extensive research, including guppies, platyfish, medaka, sticklebacks, and pufferfish  \citep{terzibasi2007review} %TODO: cite an up-to-date phylogeny, add estimated divergency times, add a phylogeny figure.
; the genetic resources available for these species, in combination with the high degree of synteny shown across this group of teleosts \citep{terzibasi2007review} have been of great utility in a number of killifish projects, including this one. On the other hand, the most well-developed teleost model organism, the zebrafish, is relatively distantly-related, making the use in a killifish context of genetic and experimental resources developed for the zebrafish more challenging.


% TODO: Add a paragraph on the killifish immune system to segue into teleost immunity stuff

\section{Misc.}

\subsection{Ageing of the antibody repertoire}

% Notes from de Bourcey et al. (2017)

The vertebrate adaptive immune system has long been observed to undergo a severe decline with age in multiple species, with notable changes in humans including decreased lymphocyte proliferation \citep{debourcy2017ageing} and defects in antibody production \citep{debourcy2017ageing}. Changes in the human antibody repertoire observed with age include restricted

In studies of peripheral blood repertoires taken before and after influenza vaccination, older individuals have been observed to show reduced within-individual and increased between-individual repertoire diversity \citep{debourcy2017ageing}, suggesting that repertoires become increasingly (and divergently) specialised with age. Older repertoires also show less change in composition following vaccination, showing a reduced capacity to adapt to new information from the pathogenic environment \citep{debourcy2017ageing}. An oligoclonal phenotype, in which one or a few memory B-cell lineages occupy a disproportionate share of repertoire diversity, has been reported in a subset of older individuals in multiple studies \citep{debourcy2017ageing}; these expanded clones appear to be resistant to immunogenic interventions such as vaccination. 

Within similarly-sized lineages, the repertoires of older individuals have been found to show reduced per-nucleotide sequence diversity, indicating an impairment in secondary diversification through somatic hypermutation; this effect is especially pronounced in larger clones \citep{debourcy2017ageing}. % Fig 3D

A subset of older repertoires also showed a greater prevalence of sequences bearing premature stop codons, indicating... % Also reduced "radical" mutations

This loss in repertoire diversity with age is observed in both naive and antigen-experienced subsets of the repertoire \citep{debourcy2017ageing}, but is strongest in the former, indicating an increase in the relative prevalence of the memory compartment within the repertoire. This is consistent with a model of the aged immune system as impaired by the accumulation of stubborn immune memory.

An important feature of these studies is that CMV infection has been found to have an important influence of certain aspects of repertoire ageing, including...


% Does this need to be double spaced?

%
%\ifpdf
%    \graphicspath{{Chapter1/Figs/Raster/}{Chapter1/Figs/PDF/}{Chapter1/Figs/}}
%\else
%    \graphicspath{{Chapter1/Figs/Vector/}{Chapter1/Figs/}}
%\fi

\section*{Summary} % Fits one one page if 1.5-spaced, but not at double spacing

The vertebrate adaptive immune system is among the most impressive achievements of animal evolution. In the billion-year-old arms race between pathogens and their hosts, it represents a paradigm change in defensive capability, enabling large, slow-breeding vertebrates to effectively fight off the vast majority of assaults from much-faster-evolving pathogenic attackers. By dynamically recombining their antigen-receptor genes during immune-cell development, vertebrates are capable of producing receptors with an almost unlimited variety of antigen-specificities, enabling B- and T-lymphocytes with the correct receptors to respond specifically to pathogens that neither that individual nor its ancestors have ever encountered. The selective value of such a system is evidenced by the high costs selection is apparently willing to accept to retain it: despite the very high levels of cell wastage it necessitates and frequent autoimmune infections it causes, the adaptive immune system forms a key part of the immune system of all known jawed vertebrates.

Among the different adaptive antigen-receptor proteins present in gnathostomes, antibodies stand out in the biochemical diversity of their cognate antigens and their ability to act independently of their generating B-cell.

% If the adaptive immune repertoire is so great, there should be some evidence that vertebrates are more resistant to infections than other species that lack it. Is this the case?

Faced with the pervasive risk of parasitic infection, vertebrates have evolved highly complex immune systems capable of effectively combating a wide range of pathogenic threats. Among the most sophisticated of these adaptations is the adaptive immunity provided by B- and T-lymphocytes, which utilise a range of specialised genome-editing mechanisms to generate a vast array of distinct antigen-binding proteins. These remarkable systems, however, undergo severe systemic decline with ageing, leading to greatly increased rates of infection-related morbidity in older individuals. Comparable immunosenescent phenotypes have been widely observed in mammals, birds, fish and elsewhere, and appear to be broadly conserved across the vertebrate lineage.

To understand and counter the complex changes that occur in the adaptive immune system with age, it is necessary to analyse the whole population of lymphocyte antigen-receptor sequences present in an individual. Such a top-down approach was impossible until relatively recently, when the advent of modern high-throughput sequencing technologies enabled the development of specialised protocols for immune-repertoire sequencing and analysis. Since then, the field of immune-repertoire studies has developed rapidly, providing a new and more powerful method for interrogating the changes ocurring in adaptive immune repertoires in a wide variety of contexts, including ageing. However, while initial human studies have indicated a decline in the diversity of these repertoires in older people, there remains a need for further research in this area. % This is weak, fill in once you've done your lit review.

As the shortest-lived vertebrate to be bred in captivity, the African turquoise killifish (\textit{Nothobranchius furzeri}) represents a powerful model for studying vertebrate-specific ageing processes, including immunosenescence of the adaptive immune system. Though the killifish has seen rapid development as a model system for ageing research, little was known about its adaptive immune system prior to the work described in this thesis. % In this chapter...? But it's a summary, not an introduction. 

\pagebreak

\section{The vertebrate adaptive immune system}

All organisms exist in a condition of intense competition for resources, with predators, peers, and parasites all competing, in some way, for the nutrients and energy consumed and used by an individual. Among the most insidious of these competitors are parasites who attempt to colonise an organism's own body, consuming its stores of nutrients and energy and turning its internal mechanisms to their own advantage. When an organism falls prey to one of these parasites and manifests the symptoms of its exploitation, we call it disease. When the organism utilises adaptations to prevent this exploitation, through excluding or killing the parasites, we call it immunity.

Given the extreme selective pressure to protect their fitness from parasitic exploitation, it is perhaps unsurprising that so many different organisms have evolved immune systems of great complexity and effectiveness. Nevertheless, the intricacy of the vertebrate immune system has proven one of the most enduringly fascinating aspects of vertebrate biology, comparable to vertebrate neural systems in its complexity. Indeed, the vertebrate immune system shows many parallels with the nervous system, being the only other system capable of complex information processing and memory. This complexity, which is fundamental to the effectiveness of the vertebrate immune system, rests on the interplay between the two traditional wings of vertebrate immunology: the innate immune system, and the adaptive.

Innate immunity refers to a large collection of mechanisms designed to exclude, sequester, or kill invading pathogens (disease-causing organisms) in a rapid and nonspecific manner. Many innate systems combat pathogens in ways that are either physically difficult to circumvent (such as external barriers, or engulfment by phagocytic cells) or which target aspects of pathogen biology that are difficult to alter without catastrophic loss of function (such as this wonderful example %example
). As such, the great majority of possible pathogenic threats are dealt with rapidly and effectively by the innate immune system, either by keeping parasites from accessing vulnerable parts of the organism or by rapidly and nonspecifically eliminating them once there.

Despite its speed, power and impressive generality, the innate immune system suffers from severe limitations. The first is that it is helpless in the face of evolutionarily novel threats to which its existing defences do not apply. The second, perhaps more fundamental, problem is that many common pathogens are capable of evolving at speeds vastly exceeding that of vertebrates. For example, many bacteria have generation times of much less than an hour, compared to months or years for most vertebrates, while also exhibiting much higher per-cell-division rates of mutation. And even this rapid rate of evolution is far exceeded by the highly volatile genomes of many viruses.

This capacity for parasitic organisms to out-evolve their hosts represents a serious problem for vertebrates, who cannot hope to effectively respond to these threats through the generation of new and improved innate immune mechanisms via selection. Instead, what is needed is a mechanisms by which vertebrates can dynamically learn to respond to novel immune threats within the lifespan of a single organism. That mechanism is the adaptive immune system.

% Need to rapidly focus onto B-cell immunity in particular, I know nothing of T-cells

The mechanisms used to generate this sequence diversity in the B-cell population are themselves diverse, and have been discovered progressively over the past decades. The most fundamental, and well-known, such mechanism is so-called V(D)J recombination, first discovered quite some time ago. %Find out more about history of this

In V(D)J recombination, a number of

A canonical immunoglobulin heavy chain (IgH) gene locus consists of clusters of variable (V), diversity (D) and joining (J) regions in series, followed by some number of larger constant-region exons. During B-cell maturation, a single V, D and J region are selected and the intervening DNA regions are excised to produce a single, contiguous VDJ sequence. As part of this process, nontemplated nucleotides are inserted and deleted at the V/D and D/J junctions, a process known as junctional diversification. Following transcription, the sequence between the VDJ sequence and the first constant-region exon is removed by splicing to produce a mature IgH transcript, with its characteristig VDJC sequence structure.

The manner of constant region selection in B-cells differs significantly between mammals and teleosts. In the former, a number of distinct constant regions are present in series on the chromo

\newpage 
\section{Structure and diversification of the antibody heavy chain}

% Pretend light chains and TCRs don't exist for now; you can generalise later as needed

The great majority of antibodies produced by the gnathostome adaptive immune system share a canonical tetrameric structure: two heavy chains and two light chains, arranged into a roughly Y-shaped configuration. This structure comprises three important functional domains: two antigen-binding domains, formed by the N-terminal portions of the heavy and light chains, and one effector domain formed by the C-termini of the heavy chains. As a result of these distinct functionalities, the two ends of an antibody heavy-chain protein have very different properties when considered as a population. The N-terminal variable domain is highly variable in amino-acid sequence, with unrelated\footnote{i.e. not derived from the same plasma-cell clone} B-cells typically displaying different and often unique sequences. The C-terminal constant region, meanwhile, shows very limited sequence diversity, with all B-cells in the body producing one of a small number of distinct effector classes. This combination of a highly-diverse antigen-binding domain and a limited range of distinct effector domains enables antibodies to simultaneously recognise and contact a vast array of potential antigens, while simultaneously interacting with the rest of the immune system in a predictable manner.

The constant-region sequence of an antibody heavy chain (or its mRNA) is known as the class or isotype of that antibody, while its variable-region sequence is known as its idiotype. % Definitions box?

The variable region of an antibody can be further subdivided into three complementarity-determining-regions (CDR1-3), which form part of the antigen-binding site and directly contact the antigen, and four framework regions (FR1-4), which do not. Sequence variability in the variable region is concentrated in the CDRs, with CDR3 showing by far the greatest variability for a variety of reasons discussed below. % Figure for this? E.g. align Vs and mark regions of non-conservation
This sequence variability is generated by several highly-specialised genome-editing mechanisms, which together produce a distinct idiotypic sequence on the chromosome of each developing B-cell. Broadly speaking, these diversification mechanisms can be divided into three categories, each of which has a distinct effect on repertoire sequence diversity; these categories are V(D)J recombination, junctional diversification, and somatic hypermutation.

\subsection{Mechanisms of heavy-chain diversification I: V(D)J recombination}


% FIGURES AND TABLES:
% - Schematic of native locus state -> VDJR -> recombined locus
% - RAG structure and DNA Binding
% - Schematic of RAG action and DNA excision
% - Schematic of RSS structure
% - Sequence logos of human/mouse/other RSSs
% - Schematic of V/D/J coverage on protein chain
% - Length distributions of human/mouse/other V/D/J segments

The structure of the immunoglobulin heavy chain (\textit{IGH}) locus differs markedly between its native germline configuration and that adopted by a mature B-cell. In the native state, most \textit{IGH} loci comprise large numbers of isolated gene segments separated by non-coding DNA. These segments can be divided into three classes, variable (V), diversity (D) and joining (J), with distinct sequence properties and length distributions (see Box~\dots). % FIGURE: VDJ length distributions in humans etc; schematic of V/D/J positioning in peptide chain
During B-cell development, site-specific recombination reactions result in the rearrangement of individual V, D and J segments into a continuous VDJ sequence, with the intervening genomic regions permanently excised and degraded. % Citation about what happens to excised sequences
This irreversible genomic maturation process is known as V(D)J recombination.

VDJ recombination is carried out by the RAG endonuclease complex, an enzyme formed by the association of \textbf{r}ecombination-\textbf{a}ctivating \textbf{g}enes 1 \& 2 \citep{jung2006vdjr}. % more detailed citation, roles of RAG1 vs RAG2
This complex, which is expressed specifically in developing lymphocytes, % citation needed
introduces recognises specialised recombination sequences (RSSs) at the ends of IGH gene segments and introduces targeted double-strand breaks between two segments and their respective RSSs \citep{jung2006vdjr}; these DSBs are then repaired by non-homologous end joining, resulting in a continuous coding sequence spanning both segments. The excised DNA ... % Citation for this

VDJ recombination in the \textit{IGH} locus is highly structured, and occurs in a specific order. First, a D and a J segment are selected and recombined to produce a DJ sequence. Then, a V region is selected and recombined with the DJ to produce a continuous VDJ sequence constituting the variable region of the heavy chain. The complete protein sequence is produced during later transcriptional splicing, which joins this variable-region sequence to downstream constant-region exons to produce a mature \textit{IGH} mRNA. This strict ordering of V/D/J segments, which obtains in the vast majority of recombined sequences observed, is produced through the combination of a variety of regulatory mechanisms. The most basic of these is the structure of the RSSs, which comprise a conserved heptamer and nonamer sequence separated by a relatively unconserved spacer region of either 12 or 23bp \citep{jung2006vdjr}, corresponding to either one or two turns of the DNA helix. V and J segments in the IGH locus are flanked by RSSs with 23bp spacer regions, while those flanking D-regions have 12bp spacers %citation needed
As the RAG recombinase specifically recognises pairs of RSSs with dissimilar spacer lengths (a restriction known as the 12/23 or one-turn/two-turn rule), direct V-to-J recombination events are excluded \citep{jung2006vdjr}. % Better citation if possible.

In addition to the restrictions imposed by the 12/23 rule, additional limitations on VDJ recombination are imposed by the requirement that RAG binding be preceded by transcription from the to-be-recombined segments, apparently in order to open the chromatin structure in that region \citep{jung2006vdjr}. Both V and D segments, but not J segments, are preceded by upstream promoter regions, which are involved in transcriptional initiation during specific stages of B-cell development...

Complete VDJ recombination places the V-region promoter in close proximity to a highly-conserved enhancer element (known as iE$\mu$) lying between the last J segment and the first constant-region exon \citep{jung2006vdjr}; this enhancer is important for strong expression of the mature IGH mRNA from the pre-B-cell stage onwards. 

\begin{itemize}
\item \textbf{Variable (V) segments} are the longest class of \textit{IGH} gene segment, with functional segments ranging from \dots to \dots nucleotides in humans and mice. % Figure and citation for this; add amino-acid lengths
This includes the coding sequence for the bulk of the variable region of the heavy chain, including all of FR1, CDR1, FR2, CDR2 and FR3, as well as the 5'/N-terminal part of CDR3 \citep{jung2006vdjr}. Each V segment also includes an N-terminal signal peptide sequence for antibody trafficking % citation needed
, and is preceded by its own 5'-UTR and promoter region. % citation for V promoters?
\item \textbf{Diversity (D) segments} are the shortest class of \textit{IGH} gene segment, ranging from \dots to \dots nucleotides in humans. %!
They form the middle part of the heavy chain CDR3. Like V-segments, D-segments are flanked by upstream promoter regions, which are involved in initiating the transcription required for DJ recombination. \citep{jung2006vdjr}
\item{Joining (J) segments} are of intermediate length, \dots % length data
. They form the 3'/C-terminal part of heavy chain CDR3 and the 5'/N-terminal part of FR4. Each J-segment is succeeded on the chromosome by a splice donor site, which is used to splice the recombined variable region to the constant region following transcription of IGH mRNA.
\end{itemize}

In its native germline state, the antibody heavy chain locus occupies a highly unusual configuration



The immunoglobulin heavy chain (\textit{IGH}) locus has a highly distinctive structure, whose broad outlines are shared only by other adaptive-immune antigen receptors...

The human genome contains seven loci that follow these general patterns of diversification: the \textit{IGH} locus, two light chain loci, and four T-cell receptor loci \citep{jung2006vdjr}.

% First explain structure of antibody protein chain
% -> mRNA
% -> unrecombined locus



An antibody protein chain can be divided into two regions with distinctive roles in the immune system: an N-terminal antigen-binding domain

. The C-terminal part of an antibody, known as the constant region, 

The N-terminal portion, by contrast, 

While the first two complementarity-determining regions of the heavy chain are contained entirely within the V segment, the heavy-chain CDR3 is formed by the combination of the V, D, and J segments selected for rearrangement \citep{jung2006vdjr}. As a result, the bulk of sequence diversity in the heavy chain falls within the CDR3, which is responsible for the greater part of antigen-binding variability among antibodies. % Citation for this
Nevertheless, the combinatorial sequence diversity produced by rearrangements of V/D/J segments alone is necessarily limited, with a maximum of $a \times b \times c$ possible sequences for a locus containing $a$ V, $b$ D and $c$ J sequences; far below the diversity of possible antigens. % citation needed 
Fortunately, the potential sequence diversity of the na\:{i}ve repertoire is powerfully augmented by a second diversification mechanism, taking place alongside VDJ recombination during B-cell development: junctional diversification.

\subsection{Mechanisms of heavy-chain diversification II: junctional diversification}

...

The net number of inserted nucleotides is essentially random, but the reading frame of the IGH transcript is fixed by the positions of the ATG initiation codon in the V-segment and the J-segment/constant-region splice junction. As a result, junctional diversification results in a large number of frameshift mutations, in addition to STOP codons that prematurely truncate the protein sequence. % Are these STOP mutations also counted as non-productive, or just frameshifts
Such loci, which are unable to produce a functional heavy chain protein, are termed non-productive, while those without such mutations are termed productive ; at a first approximation, roughly one-third of VDJ recombination events are expected to produce a productive sequence \citep{jung2006vdjr}. % Derivation and graphical representation of this; probability of two discrete random variables being congruent modulo three?

While a given recombination event only has about a one-third chance of producing a productive rearrangement, B-cells, like other somatic cells in vertebrates, are diploid. As a result, while a given recombination event only has about a one-third chance of producing a productive rearrangement, a B-cell which undergoes a non-productive rearrangement on one chromosome can make a second attempt on the other. Given the one-third approximation given above, this means that about 55\% of B-cells will achieve a productive rearrangement on one or the other chromosome; the other 45\%, unable to produce a functional heavy chain from either chromosome, die by apoptosis during B-cell development \citep{jung2006vdjr}. % Adapt figure 3 from \citep{jung2006vdjr}
This loss of almost half of all developing B-cells, a substantial cost, demonstrates the profound selective value of the additional antigen-binding diversity provided by junctional diversification.

Of those B-cells undergoing a productive heavy-chain rearrangement (and therefore surviving this process), roughly three-fifths are expected to bear one rearranged and one unrearraranged locus, with the remaining two-fifths bearing two rearranged loci, one of which in unproductive. This 60/40 ratio is roughly borne out by empirical data 


\subsection{Structure of mammalian \textit{IGH} loci} % And teleosts/others as available

% Human locus

The mouse \textit{IGH} locus also adopts a translocon structure, in this case with an enormous V-region, comprising 150 or more V-segments (depending on the strain) spanning 2.7Mb \citep{jung2006vdjr}. The 12-13 murine D segments occupy a region of approximately 50kb, while the four J segments cover about 2kb. This is followed by 200kb of constant region exons. 

% Mouse locus (depending on strain): 150+ V, 12-13 D, 4 J. (\citep{jung2006vdjr})
% Total length of murine locus: ~3Mb near telomeric end of chr12

\subsection{Antibody effector function and isotype diversity}



\newpage
\section{The African turquoise killifish as a model for vertebrate ageing}

% POSSIBLE FIGURES:
% - Nothobranchius genus range, photos of different notho males
% * Photo of male and female GRZ TK
% * Lifespan curves of male and female GRZ-AD
% - Diapause schematic
% - Photograph of ephemeral pools in Zimbabwe
% - Something showing correlation between aridity and lifespan in nothos
% - Schematic comparing GRZ lifespan to other common ageing models, along with important present/absent human-like systems
% - Phylogeny of TK within teleosts, with divergence times and human outgroup

% POSSIBLE TABLES:
% - List of notho species used in research with approx mean/max lifespans
% - List of ageing phenotypes observed/not observed in GRZ / wild-derived strains
% - Table of TK genome assemblies (JENA/STFD/CLGN) with metrics
% - Comparison of genome metrics (size, repetitiveness, etc) between TK and other models
% - Life history table for GRZ, MZM, other nothos, other fish models

The genus \textit{Nothobranchius} comprises a broad group of annual freshwater fishes distributed across equatorial and subequatorial Africa \citep{valdesalici2003lifespan}, with species diversity concentrated in the south-east of the continent \citep{genade2005annual}. Members of this genus share a suite of adaptations to life in ephemeral pools and rivers, most notably the production of desiccation-resistant embryos capable of surviving through the dry season in a diapause state \citep{genade2005annual}. Fish from this genus have been known for several decades to exhibit very rapid growth and short lifespans, consistent with their evolving under conditions of very high extrinsic mortality \citep{valdesalici2003lifespan}, with many species exhibiting a median lifespan of less than one year. Nevertheless, there is wide variation within the genus in body size, growth rate and lifespan, with species from less arid regions tending to show slower growth and longer median lifespans \citep{genade2005annual}.

Like other \textit{Nothobranchius} species, the turquoise killifish (\textit{Nothobranchius furzeri}) is a medium-sized annual fish first isolated from ephemeral freshwater pools -- in this case, from a relatively arid region of southeastern Zimbabwe \citep{jubb1971new,genade2005annual}. Even by the standards of the \textit{Nothobranchius} genus, \textit{N. furzeri} exhibits extremely rapid growth, maturation, and ageing, with the most widely-used laboratory strain (GRZ) exhibiting a median lifespan of just 9-14 weeks \citep{valdesalici2003lifespan,genade2005annual,terzibasi2008strains,kirschner2012map,valenzano2015genome} % Update range as other papers get different results
-- the shortest lifespan of any captive vertebrate. Moreover, while all turquoise killifish strains are very short-lived by vertebrate standards, different strains of this species have been found to differ several-fold in their median and maximum lifespans \citep{terzibasi2008strains,kirschner2012map}, providing a rare opportunity for intraspecific comparative ageing studies \citep{terzibasi2008strains,terzibasi2009dr,hartmann2009telomeres} and mapping the genetic underpinnings of lifespan \citep{kirschner2012map}. This combination of extremely short-lived laboratory strains and a wide range of lifespan phenotypes within a single species makes the turquoise killifish an extremely promising model organism for ageing research, especially when combined with the presence of vertebrate-specific adaptations absent in short-lived invertebrate ageing models.


Despite its very short lifespan, \textit{N. furzeri} has been found to show a wide range of senescent phenotypes in even the shortest-lived strains, including lipofuscin deposition \citep{genade2005annual};  accumulation of senescence markers \citep{genade2005annual};  increased neurodegenaration \citep{valenzano2006resveratrol1,valenzano2006resveratrol2}; impaired learning and behavioural phenotypes  \citep{genade2005annual,valenzano2006resveratrol1}; and a high incidence of degenerative and neoplastic lesions \citep{dicicco2011histopathology}. These diverse phenotypes indicate that the short lifespan of the turquoise killifish is the result of an accelerated general ageing process, rather than the specific failure of a particular organ or system.
Moreover, established anti-ageing interventions such as resveratrol treatment \citep{valenzano2006resveratrol1}, reduction in ambient temperature \citep{valenzano2006temperature} and dietary restriction \citep{terzibasi2009dr} also extend lifespan in the turquoise killifish, indicating a strong analogy with the ageing phenotypes observed in canonical model systems.


% (though others, such as shortened telomeres \citep{hartmann2009telomeres} and gonadal fibrosis \citep{dicicco2011histopathology}, are only observed in longer-lived strains of the species) % Add more as you read them
% Other ageing phenotypes, including
% telomere shortening \citep{hartmann2009telomeres}, gonadal fibrosis \citep{dicicco2011histopathology}, % not observed in GRZ
% mtDNA depletion \citep{hartmann2011mitochondria}, and bioenergetic impairment \citep{hartmann2011mitochondria} % Not studied in GRZ
% have been observed in longer-lived \textit{N. furzeri} strains, but were either not observed (telomeres, gonads) or not studied (mitochondrial phenotypes) in the short-lived GRZ strain.
%.  

Due primarily to its potential as a model organism for ageing research, the turquoise killifish has also seen rapid development as a genetic model. The short-lived GRZ strain has been bred in captivity for fifty years and at least a hundred generations \citep{terzibasi2007review} and exibits a very high degree of homozygosity \citep{reichwald2009genome,valenzano2009map,kirschner2012map}, providing a uniform genetic background for experimental interventions. A variety of effective transgenesis and mutagenesis methods have been developed for this species, including \textit{tol2} transgenesis \citep{valenzano2011tol2,hartmann2012tol2,allard2013inducible} %TALENS?
and CRISPR \citep{harel2015crispr,harel2016crispr}; furthermore, a number of important genetic resources are now available, including linkage maps \citep{valenzano2009map,kirschner2012map,valenzano2015genome}, a transcript catalogue \citep{petzold2013transcriptome}, an miRNAome % citation needed
and a mitochondrial genome \citep{hartmann2011mitochondria}. Most importantly, a high-quality nuclear genome of the short-lived GRZ strain is now available
% with a recent study integrating evidence from multiple independent studies to produce a very high-quality assembly \cite{ray's genome paper, plus others}. 
\citep{reichwald2015genome,valenzano2015genome}. % Add more here
The progressive release of improved genome assemblies for the turquoise killifish has been particularly important for this project, and will be discussed in more detail in Section ??. %!
The genome itself is unusually large and repetitive by teleost standards, potentially contributing to these fishes' unusually short lifespan. % citation needed

% Needed: oxidation levels, telomere dysfunction, more senescence markers

% Ageing-related genes under positive selection (both genome papers)


%The killifish genome is unusually large by teleost standards, % TODO: figure for this? update figures and add citations as you read newer sources
%with an estimated size of 1.5 to 2 gigabases \citep{reichwald2009genome,reichwald2015genome}. This large size is primarily accounted for by the exceptionally high repeat content of the killifish genome, with at least 21\% and 24\% composed of tandem and other types of repeats, respectively \citep{reichwald2009genome}. GC content is also relatively high, at 44\%, though this average is affected somewhat by the presence of a distinct fraction of highly GC-rich tandem repeats \citep{reichwald2009genome}. Karyotyping \citep{reichwald2009genome} and sequencing analysis \citep{reichwald2015genome} both indicate a chromosome number of $2n = 38$, corresponding to 19 distinct linkage groups in the haploid genome. 

%They are strongly sexually dimorphic, with an XY sex-determination system \citep{reichwald2015genome,valenzano2009map}. %Photo

Phylogenetically, the genus \textit{Nothobranchius} falls within the Cyprinodontiformes, and the turquoise killifish is therefore closely related to a number of fish species that have been the subject of extensive research, including guppies, platyfish, medaka, sticklebacks, and pufferfish  \citep{terzibasi2007review} %TODO: cite an up-to-date phylogeny, add estimated divergency times, add a phylogeny figure.
; the genetic resources available for these species, in combination with the high degree of synteny shown across this group of teleosts \citep{terzibasi2007review} have been of great utility in a number of killifish projects, including this one. On the other hand, the most well-developed teleost model organism, the zebrafish, is relatively distantly-related, making the use in a killifish context of genetic and experimental resources developed for the zebrafish more challenging.


% TODO: Add a paragraph on the killifish immune system to segue into teleost immunity stuff

\section{Misc.}

\subsection{Ageing of the antibody repertoire}

% Notes from de Bourcey et al. (2017)

The vertebrate adaptive immune system has long been observed to undergo a severe decline with age in multiple species, with notable changes in humans including decreased lymphocyte proliferation \citep{debourcy2017ageing} and defects in antibody production \citep{debourcy2017ageing}. Changes in the human antibody repertoire observed with age include restricted

In studies of peripheral blood repertoires taken before and after influenza vaccination, older individuals have been observed to show reduced within-individual and increased between-individual repertoire diversity \citep{debourcy2017ageing}, suggesting that repertoires become increasingly (and divergently) specialised with age. Older repertoires also show less change in composition following vaccination, showing a reduced capacity to adapt to new information from the pathogenic environment \citep{debourcy2017ageing}. An oligoclonal phenotype, in which one or a few memory B-cell lineages occupy a disproportionate share of repertoire diversity, has been reported in a subset of older individuals in multiple studies \citep{debourcy2017ageing}; these expanded clones appear to be resistant to immunogenic interventions such as vaccination. 

Within similarly-sized lineages, the repertoires of older individuals have been found to show reduced per-nucleotide sequence diversity, indicating an impairment in secondary diversification through somatic hypermutation; this effect is especially pronounced in larger clones \citep{debourcy2017ageing}. % Fig 3D

A subset of older repertoires also showed a greater prevalence of sequences bearing premature stop codons, indicating... % Also reduced "radical" mutations

This loss in repertoire diversity with age is observed in both naive and antigen-experienced subsets of the repertoire \citep{debourcy2017ageing}, but is strongest in the former, indicating an increase in the relative prevalence of the memory compartment within the repertoire. This is consistent with a model of the aged immune system as impaired by the accumulation of stubborn immune memory.

An important feature of these studies is that CMV infection has been found to have an important influence of certain aspects of repertoire ageing, including...
