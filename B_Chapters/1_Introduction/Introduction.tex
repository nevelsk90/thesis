%!TEX root = ../thesis.tex
%*******************************************************************************
%*********************************** First Chapter *****************************
%*******************************************************************************

\chapter{Introduction}  %Title of the First Chapter
\pagebreak
%s\doublespacing
\onehalfspacing

\section{Overview}

All organisms exist in a state of intense competition for resources. For many organisms, among the most dangerous competitors are parasites, which attempt to colonise the host's own body and co-opt its internal resources and systems to their own advantage. The evolutionary arms race between parasites and hosts is ancient, and has led to the development of a stunning variety of complex offensive and defensive adaptations on each side. % Mention faster evolution of parasites relative to hosts

Among the most complex and sophisticated systems developed as part of this ancient host/parasite conflict is the vertebrate adaptive immune system. By dynamically recombining their own DNA within specialised lymphocyte cells, jawed vertebrates are capable of producing an almost unlimited variety of different \textit{de novo} antigen-receptor proteins, and hence of responding effectively to entirely novel immune threats. In addition, by producing long-lived memory cells in response to antigenic stimulation, the adaptive immune system can retain the ability to respond to recurrent immune threats years or even decades after they were first encountered by an individual. This combination of dynamic adaptability to novel immune threats and persistent immune memory enables vertebrates to progressively improve their protection against predictable aspects of their immune environment, while also coping effectively with the rapid evolution which is one of the main advantages of bacterial and viral pathogens over their slower-evolving hosts.

Among the different branches of vertebrate adaptive immunity, the humoral immune system is unique in both the breadth of antigenic compounds it can respond to and its ability to produce secreted antigen-receptor proteins capable of acting independently of the cells that produced them. Whereas the T-lymphocytes of the cellular adaptive immune system can respond only to processed peptide antigens expressed on the surface of antigen-presenting cells, the antibodies produced by B-lymphocytes are capable of responding to almost any molecular structure on the surface of a cell or protein. Secreted antibodies in serum and mucosal secretions play a number of essential roles in vertebrate immunity, including opsonisation (recruitment of phagocytic cells), activation of complement, and inactivation and aggregation of antigens and pathogens, while membrane-bound antibodies (also known as B-cell receptors or BCRs) orchestrate B-cell development, and response to antigen exposure. An effective humoral immune system is essential to immune and organismal function in vertebrates, and mutations that disable humoral adaptive immunity in ...% species?
can lead to severe immunopathies and severely curtailed life expectancy. % TODO: Focus on importance here, move details to another section.

% TODO: Discuss antibodies specifically

Despite all its sophistication, however, the functionality of the vertebrate adaptive immune system declines dramatically with age, leaving older individuals increasingly vulnerable to infectious disease. In humans and other species, the decline in immune functionality with age manifests as a dramatic increase in deaths from infection in older people, as well as increasing levels of infection-associated morbidity and a decline in the effectiveness of pro-immune interventions such as vaccination. The molecular and physiological changes underlying this immunosenescent phenotype are wide-ranging, implicating many different parts of the immune system; however, a particularly important contributor is the systemic decline in the effectiveness of the adaptive immune system in % identifying and neutralising novel immune threats?
For the humoral adaptive immune system, established immunosenescent phenotypes include... % decline in naive B-cells, reduction of antibody avidity, increase in autoantibody production
% TODO: Recast in terms of humoral immunity *as part of* general immune phenotype

While fair amount is known about the cellular and physiological changes that take place in the humoral adaptive immune system with age in some species, much less is known about how these changes translate to alterations in the high-level diversity, clonal makeup, and other structural aspects of the overall antibody repertoire in older individuals. Such a high-level systemic approach to antibody diversity could potentially reveal a great deal about how ageing and other processes affect adaptive immune functionality in an organism. Specialised, high-throughput, quantitative approaches have been used to assess the structure, diversity and health of adaptive immune repertoires in various contexts, including development, disease, and ageing; in the latter case, this pre-existing work has revealed ... . However, much remains to be discovered about ... .

When it comes to investigating the ageing of vertebrate-specific adaptations like the adaptive immune system, research is made more difficult by a relative lack of well-suited model organisms. Most established short-lived model organisms used in ageing research (e.g. yeast, nematode worms, or fruit flies) are not vertebrates, while most established vertebrate models (e.g. zebrafish or \species{Xenopus}{laevis}) were selected for properties other than their lifespan (such as rapid development) and are too long-lived for use as ageing models in many contexts. Even mouse, the most widely-used vertebrate model organism for both ageing and other biomedical applications, has a median lifespan of several years in most commonly-used laboratory strains, making many ageing studies prohibitively expensive. 

In this context, the recent emergence of the turquoise killifish (\nfu), the shortest-lived vertebrate species currently bred in captivity, as a model organism for ageing research represents a highly promising development for the study of adaptive immunosenescence; ...

In this thesis, therefore, I establish the turquoise killifish as a model for the study of comparative immunology and humoral adaptive immunosenescence. 
, by characterising 

Using a combination of existing genomic assemblies and new sequence data, I assembled and characterised the immunoglobulin heavy chain (\igh{}) gene locus of the turquoise killifish and compared it to other newly-assembled loci from closely-related species, revealing a complex and rapidly-evolving ... with a number of surprisingly ideosyncratic features (\Cref{sec:locus}). Using the sequences from this newly-characterised locus, I established the first working immunoglobulin sequencing protocol in this species, which I used to investigate the diversity and complexity of heavy-chain immune repertoires in adult killifish and how this diversity changes with age in the whole body and the gut. The results of these investigations demonstrate that the turquoise killifish possesses a complex, diverse and individualised antibody repertoire, which undergoes a rapid decline in within-individual diversity and increase in between-individual variability with age. This phenomenon is particularly strong in the gut, with ..., and is likely to be an important contributor to the changes in gut microbiota structure and diversity observed in aged killifish. % TODO: Discuss lack of change from transfer?



\section{The vertebrate adaptive immune system} % Humoral? Teleost?

\subsection{Antibody structure and function}

The lineage of lymphocytic white blood cells known as B-cells is ancient, with its origins predating any modern gnathostome (jawed-vertebrate) lineage and possibly the development of antibodies themselves. In modern jawed vertebrates, B-lymphocytes are responsible for the production, diversification and secretion of the antigen-receptor proteins known as antibodies, as well as diverse other roles that vary between taxa. In almost all species examined to date, antibodies share a common, and highly distinctive, tetrameric structure, with two identical immunoglobulin heavy chains (IGH) and two light chains (IGL) linked by disulfide bonds into a roughly Y-shaped configuration. The sequence and structure of these chains, and the corresponding regions of their underlying gene loci, is divided into an N-terminal (5') variable region and a C-terminal (3') constant region; together, the variable regions of each light/heavy-chain pair determines the antigen-binding specificity of the antibody, while the constant regions, particularly that of the heavy chain, determine its structure, functional properties, and interactions with the rest of the immune system. Unsurprisingly, the sequence diversity of the constant region is far lower than that of the variable region, with most species expressing only a few distinct constant-region classes (or \textit{isotypes}) but an almost unlimited variety of variable region sequences (or \textit{ideotypes}).

As members of the immunoglobulin superfamily of proteins, the tertiary structure of antibody chains consists primarily of a species of immunoglobulin fold domains; most light chains consist of two such domains, while the number in heavy chains varies substantially with isotype. In both heavy and light chains, the most N-terminal immunoglobulin fold comprises the variable region, while the rest make up the constant region; in some taxa (but not in teleost fishes), one of the constant-region immunoglobulin folds is replaced with a flexible hinge domain in some isotypes to increase the flexibility of the heavy chain. The three loops of the variable-region immunoglobulin fold facing the antigen-binding site are labelled H1, H2 and H3 in the heavy chain (L1, L2 and L3 in the light chain) and are principally responsible for determining the antigen-binding specificity of the antibody; their corresponding gene regions are known as complementarity-determining regions (CDRs) and are the main focus of sequence variability among ideotypes, while the other parts of the antibody sequence are known as framework regions (FRs or FWRs) and exhibit less variability.

Most antibody isotypes can be expressed in secreted or membrane-bound form; in the latter case they are also known as B-cell receptors (BCRs). The choice between secreted and membrane-bound forms of an antibody is made through alternative splicing; typically, the transmembrane domain and cytosolic tail of the BCR are expressed via two additional exons (TM1 and TM2). Other constant-region exons are referred to collectively as CH exons, and are numbered by their occurence in the protein chain from the variable region to the C-terminus. In secreted antibodies, the standard four-chain configuration described above is known as an antibody monomer, while multiple four-chain antibodies bound together are known as antibody multimers: dimers for two connected antibodies, tetramers for four, \etc. These multimeric antibody supercomplexes can be bound together in various ways: by intermolecular interactions, covalent disulfide bonds, or ... % TODO: Describe J-chain in tetrapods

The number and variety of antibody heavy-chain classes available to B-cells in an organism, and the mechanism by which the isotype of an antibody is determined, vary substantially by species; in tetrapods, the isotype of the antibodies produced by a given B-cell can be modified by a specialised class-switch recombination (CSR) process which is absent in teleost fishes. Different isotypes vary in their length, flexibility, multimerisation behaviour (see below) and effector functions. In teleost fishes, three constant-region classes have been observed to date, two of which are primitive and universal across teleost species and one of which is telesot-specific and present in most, but not all, characterised teleost \igh{} loci:

\begin{itemize}
\item \textbf{Immunoglobulin M} (\igh{M}) was the first IgH isotype to be identified in teleosts, and is homologous to the isotype of the same name found in mammals and other jawed vertebrates. It is expressed in both secreted and transmembrane form; in most teleosts, the transcript of the secreted form comprises four CH exons (\cm{1-4}), while the transmembrane form comprises three CH exons and two TM exons. In contrast to mammals, in which secreted IGHM is primarily found as a pentamer connected by ..., in teleosts it is typically found as a tetramer connected by disulfide bonds between heavy chains. In those fish species which have been tested, secreted IGHM is the main form of antibody found in serum.
\item Like \igh{M}, \textbf{immunoglobulin D} (\igh{D}) is a primitive isoform present in most lineages of jawed vertebrates, with teleost \igh{D} homologous to the mammalian isotype of the same name. The size and structure of \igh{D} varies dramatically between teleost species, with the number of CH exons varying more than twofold, from roughly seven (\cd{1-7}) in some species to seventeen in zebrafish.
All teleost \igh{D} transcripts to date have possessed a chimeric \cm{1} exon from \igh{M}, a configuration almost unknown in mammals \citep{fillatreau2013astonishing}; teleost \igh{D} also lacks the flexible hinge region present in mammalian \igh{D} \citep{fillatreau2013astonishing}. A minority of teleost species have known secretory forms of \igh{D}, though the mechanism of producing them varies between species; ... . % TODO: Explain sIGHD mechanisms
in other species, only transmembrane isoforms have been observed. In teleosts as in mammals, transmembrane \igh{D} is usually co-expressed with \igh{M}; however, its role in the adaptive immune system remains unclear.
%This size and variablity arises from frequent tandem duplications of C$\delta$ loci in the IgD constant region, especially of the C$\delta$2--C$\delta$3--C$\delta$4 exon block: for example, this block occurs three times in succession in channel catfish IgD, three to four times in Atlantic salmon, and four times in zebrafish \citep{fillatreau2013astonishing}. Other C$\delta$ tandem duplications have also been observed, sometimes accompanied by deletions of other C$\delta$ exons. As a result, teleost IgD can vary from seven to over 17 exons in length \citep{fillatreau2013astonishing}, with proportional variation in the size and weight of the resulting protein chain. % TODO: Add table of IgD sizes in different teleosts?
\item Unlike \igh{M} and \igh{D}, \textbf{immunoglobulin Z} (\igh{Z}, also known as \igh{T}, \igh{Z/T} and \igh{Z/T}) is unique to teleost fishes. Also unlike \igh{M} and \igh{D}, \igh{Z} is not found universally among teleost loci -- of those IgH loci characterised to date, IgZ is missing in those of medaka %citation needed
and channel catfish %citation needed
\citep{fillatreau2013astonishing} , having apparently been lost independently in these species. In those species in which it is present, IgZ appears to act as a specialised mucosal antibody class, with elevated levels observed in mucosal secretions compared to the level in serum \citep{fillatreau2013astonishing} % better citation needed
. Unlike teleost \igh{M}, secretory \igh{Z} in serum is predominantly monomeric % Clarify that this means one full tetramer, not only one chain
, while in mucosal secretions it is found primarily as a tetramer bound together by noncovalent intermolecular bonds between heavy chains.
In most species, \igh{Z} comprises four CH exons (\cz{1-4}) and two TM exons, though some species have fewer -- for example, stickleback \igh{Z} has only three CH exons, while fugu \igh{Z} has only two.
\end{itemize}

\subsection{Antibody locus structure and sequence diversification}

The immune environment encountered by a vertebrate organism contains an enormous number of different potential pathogenic threats, each of which has its own antigenic signatures and many of which are capable of evolving much more rapidly than the vertebrate host. In order for the adaptive immune system to cope with this huge diversity of different threats, it must be able to produce antibodies with a correspondingly large diversity of different antigen specificities. The greater the potential diversity of antibody sequences available to the adaptive immune system, the greater its capacity to respond effectively to novel immune threats. In reality, the mechanisms employed by the vertebrate adaptive immune system to diversify its antigen receptors enable an almost unlimited diversity of potential antibody sequences, with a correspondingly vast array of potential antigen specificities.

The mechanisms by which the adaptive immune system produces this diversity are dramatic, and rely on a highly unusual underlying gene structure and a very high level of cellular wastage. In the humoral immune system, antibody diversification takes place during B-cell development in the primary lymphopoietic organs (bone marrow in mammals, anterior kidney in teleosts). Prior to this process, the native antibody loci in B-progenitor cells (and other cell types) is highly fragmented, with numerous fragmentary variable-region sequences present in series on the chromosome upstream of the constant-region exons. In the heavy chain locus, these variable-region gene segments can be divided into three categories:

\begin{itemize}
\item \textbf{Variable (\vh) segments} are the longest class of gene segment, at roughly \bp{300} in length. Each V-segment codes for the majority of the variable region of an antibody, including the entirety of the first three framework regions (FWR1-3) and the first two complementarity-determining regions (CDR1-2) as well as the 5' part of CDR3 \parencite{jung2006vdjr}. They are therefore highly structured, and include several highly-conserved positions present in virtually all functional V-segments in nearly all species, including two conserved cysteine residues (which code for an intra-domain disulfide bond) and a conserved tryptophan residue following CDR1. Each V-segment is also associated with its own promoter sequence and 5'-UTR, as well as a 5'/C-terminal leader peptide containing... % What does the leader do anyway? Trafficking signal?
\item \textbf{Diversity (\dh) segments} are the shortest class of segment, typically on the order of \bp{10}, and are the least structured. They form the middle part of CDR3.  % Like V-segments, D-segments are flanked by upstream promoter regions, which are involved in initiating the transcription required for DJ recombination. \citep{jung2006vdjr}
\item \textbf{Joining (\jh) segments} are of intermediate length, typically 50-\bp{60}. They form the 3'/C-terminal part of heavy chain CDR3 and the 5'/N-terminal part of FR4. Each J-segment is succeeded on the chromosome by a splice donor site, which is used to join the variable region of the antibody sequence to the constant region via RNA splicing, following transcription of \igh{} mRNA. Like \vh segments, \jh segments can be identified from their conserved structure, particularly the conserved tryptophan residue marking the end of CDR3.
\end{itemize}

In the simplest translocon configuration of the \igh{} locus, blocks of repeated \vh, \dh and \jh segments are present in series on the chromosome in contiguous V-, D- and J-regions. During B-cell development, a single \vd, \dh and \jh segment are selected, and the intervening genomic regions are permanently excised from the genome to produce a single contiguous VDJ sequence coding for the complete variable region of an antibody. The mechanism by which this excision occurs is called VDJ recombination, and is fundamental to antibody sequence diversification. This process relies on recombination signal sequences (RSSs) flanking each variable gene segment in the unrecombined locus, which are recognised by a specialised recombinase complex formed by recombination-activating genes 1 and 2 (\gene{RAG1} and \gene{RAG2}). These RSSs possess a distinctive structure, with highly conserved heptamer and nonamer sequences separated by a spacer sequence of relatively unconserved sequence but conserved length (either 12 or \bp{23}, corresponding respectively to one or two turns of the DNA helix). Each functional \vh segment is succeeded by a \bp{23}-spacer RSS in 5'-3' orientation, while each \jh segment is preceded by a \bp{23}-spacer RSS in 3'-5' orientation; each \dh segment, meanwhile, is flanked by \bp{12}-spacer RSSs in 5'-3' and 3'-5' orientation, respectively. The recombinase complex binds pairs of RSSs with opposite orientation and dissimilar spacer lengths; as a result, D-J and V-D joins are permitted while V-J joins are prevented. % TODO: Add brief explanation of how join is made

% TODO: Order of recombinations: in mammals, in fish

VDJ recombination provides a basic combinatorial sequence diversity, with a number of possible sequences in the simplest case equal to the product of the numbers of \vh, \dh and \jh segments in the \igh{} gene locus. The number of potential \igh{} variable-region sequences in most species, however, is far higher than this, with an estimated ... possible sequences in humans and ... in mice. The difference arises as a consequence of the fact that the V/D and D/J joins produced by VDJ recombination are not exact, but instead involve a degree of terminal deletion and insertion of untemplated nucleotides at the segmental boundaries... % TODO: Explain insertions and deletions during VDJr

The insertions and deletions that occur at the V/D and D/J junctional boundaries during VDJ recombination vastly increase the potential sequence diversity of the heavy-chain CDR3 region, which as a consequence is by far the most important single region in determining antigen specificity. This junctional diversity, however, comes at a substantial cost. The indel mutations introduced by VDJ recombination are not constrained to occur in multiples of three or in the same number at each junction, resulting in a high rate of recombinations in which the \vh and \jh segments are out of frame with each other; still more loci are rendered nonfunctional through the introduction of \textit{de novo} STOP codons. As the sequence changes introduced by VDJ recombination are irreversible, many developing B-cells are left with permanently disrupted \igh{} loci on both chromosomes, and are left with no recourse but programmed cell death. In addition, the huge and untemplated sequence diversity introduced in this way means that some B-cells whose \igh{} loci do successfully undergo VDJ recombination are left with BCRs that either cannot effectively bind any antigen or strongly bind self-antigen, resulting in useless antibodies in the first case and a dangerous risk of autoimmunity in the second; these B-cells must be removed through a process of antigenic selection before \naive B-cells are permitted to exit the primary lymphopoietic organs, resulting in still more wastage as cells with nonfunctional or self-binding antigens undergo programmed cell death; if these selective processes are weakened by age or mutation, the rates of low-affinity or self-reactive antigens in the periphery will increase. % TODO: More detail on selection mechanisms
% TODO: Discuss primary B-cell selection to remove autoantibodies

% TODO: Antigen exposure and affinity maturation (in this section or separately?)

% TODO: IGH locus structures in teleosts

% TODO: Primary and secondary immune repertoires and immune repertoire sequencing

% TODO: Adaptive immunosenescence in vertebrates and ageing of antibody repertoires

% TODO: \section{The turquoise killifish as a model for vertebrate ageing}

%% Chapter summary (should fit on title page)
%
%% Chapter 1 summary
% Should fit on chapter title page

\section*{Summary} % Fits one one page if 1.5-spaced, but not at double spacing

Faced with the pervasive risk of parasitic infection, vertebrates have evolved highly complex immune systems capable of effectively combating a wide range of pathogenic threats. Among the most sophisticated of these adaptations is the adaptive immunity provided by B- and T-lymphocytes, which utilise a range of specialised genome-editing mechanisms to generate a vast array of distinct antigen-binding proteins. These remarkable systems, however, undergo severe systemic decline with ageing, leading to greatly increased rates of infection-related morbidity in older individuals. Comparable immunosenescent phenotypes have been widely observed in mammals, birds, fish and elsewhere, and appear to be broadly conserved across the vertebrate lineage.

To understand and counter the complex changes that occur in the adaptive immune system with age, it is necessary to analyse the whole population of lymphocyte antigen-receptor sequences present in an individual. Such a top-down approach was impossible until relatively recently, when the advent of modern high-throughput sequencing technologies enabled the development of specialised protocols for immune-repertoire sequencing and analysis. Since then, the field of immune-repertoire studies has developed rapidly, providing a new and more powerful method for interrogating the changes ocurring in adaptive immune repertoires in a wide variety of contexts, including ageing. However, while initial human studies have indicated a decline in the diversity of these repertoires in older people, there remains a need for further research in this area. % This is weak, fill in once you've done your lit review.

As the shortest-lived vertebrate to be bred in captivity, the African turquoise killifish (\textit{Nothobranchius furzeri}) represents a powerful model for studying vertebrate-specific ageing processes, including immunosenescence of the adaptive immune system. Though the killifish has seen rapid development as a model system for ageing research, little was known about its adaptive immune system prior to the work described in this thesis. % In this chapter...? But it's a summary, not an introduction.
%
%\pagebreak
%
%% Sections
%
%\section{The vertebrate adaptive immune system} % Humoral/B-cell/antibody immune system?

All organisms exist in a condition of intense competition for resources, with predators, peers, and parasites all competing, in some way, for the nutrients and energy consumed and used by an individual. Among the most insidious of these competitors are parasites who attempt to colonise an organism's own body, consuming its stores of nutrients and energy and turning its internal mechanisms to their own advantage. When an organism falls prey to one of these parasites and manifests the symptoms of its exploitation, we call it disease. When the organism utilises adaptations to prevent this exploitation, through excluding or killing the parasites, we call it immunity.

Given the extreme selective pressure to protect their fitness from parasitic exploitation, it is perhaps unsurprising that so many different organisms have evolved immune systems of great complexity and effectiveness. Nevertheless, the intricacy of the vertebrate immune system has proven one of the most enduringly fascinating aspects of vertebrate biology, comparable to vertebrate neural systems in its complexity. Indeed, the vertebrate immune system shows many parallels with the nervous system, being the only other system capable of complex information processing and memory. This complexity, which is fundamental to the effectiveness of the vertebrate immune system, rests on the interplay between the two traditional wings of vertebrate immunology: the innate immune system, and the adaptive.

Innate immunity refers to a large collection of mechanisms designed to exclude, sequester, or kill invading pathogens (disease-causing organisms) in a rapid and nonspecific manner. Many innate systems combat pathogens in ways that are either physically difficult to circumvent (such as external barriers, or engulfment by phagocytic cells) or which target aspects of pathogen biology that are difficult to alter without catastrophic loss of function (such as this wonderful example %example
). As such, the great majority of possible pathogenic threats are dealt with rapidly and effectively by the innate immune system, either by keeping parasites from accessing vulnerable parts of the organism or by rapidly and nonspecifically eliminating them once there.

Despite its speed, power and impressive generality, the innate immune system suffers from severe limitations. The first is that it is helpless in the face of evolutionarily novel threats to which its existing defences do not apply. The second, perhaps more fundamental, problem is that many common pathogens are capable of evolving at speeds vastly exceeding that of vertebrates. For example, many bacteria have generation times of much less than an hour, compared to months or years for most vertebrates, while also exhibiting much higher per-cell-division rates of mutation. And even this rapid rate of evolution is far exceeded by the highly volatile genomes of many viruses.

This capacity for parasitic organisms to out-evolve their hosts represents a serious problem for vertebrates, who cannot hope to effectively respond to these threats through the generation of new and improved innate immune mechanisms via selection. Instead, what is needed is a mechanisms by which vertebrates can dynamically learn to respond to novel immune threats within the lifespan of a single organism. That mechanism is the adaptive immune system.

% Need to rapidly focus onto B-cell immunity in particular, I know nothing of T-cells

The mechanisms used to generate this sequence diversity in the B-cell population are themselves diverse, and have been discovered progressively over the past decades. The most fundamental, and well-known, such mechanism is so-called V(D)J recombination, first discovered quite some time ago. %Find out more about history of this

In V(D)J recombination, a number of

A canonical immunoglobulin heavy chain (IgH) gene locus consists of clusters of variable (V), diversity (D) and joining (J) regions in series, followed by some number of larger constant-region exons. During B-cell maturation, a single V, D and J region are selected and the intervening DNA regions are excised to produce a single, contiguous VDJ sequence. As part of this process, nontemplated nucleotides are inserted and deleted at the V/D and D/J junctions, a process known as junctional diversification. Following transcription, the sequence between the VDJ sequence and the first constant-region exon is removed by splicing to produce a mature IgH transcript, with its characteristig VDJC sequence structure.

The manner of constant region selection in B-cells differs significantly between mammals and teleosts. In the former, a number of distinct constant regions are present in series on the chromo

\newpage 
\section{Structure and diversification of the antibody heavy chain}

% Pretend light chains and TCRs don't exist for now; you can generalise later as needed

The great majority of antibodies produced by the gnathostome adaptive immune system share a canonical tetrameric structure: two heavy chains and two light chains, arranged into a roughly Y-shaped configuration. This structure comprises three important functional domains: two antigen-binding domains, formed by the N-terminal portions of the heavy and light chains, and one effector domain formed by the C-termini of the heavy chains. As a result of these distinct functionalities, the two ends of an antibody heavy-chain protein have very different properties when considered as a population. The N-terminal variable domain is highly variable in amino-acid sequence, with unrelated\footnote{i.e. not derived from the same plasma-cell clone} B-cells typically displaying different and often unique sequences. The C-terminal constant region, meanwhile, shows very limited sequence diversity, with all B-cells in the body producing one of a small number of distinct effector classes. This combination of a highly-diverse antigen-binding domain and a limited range of distinct effector domains enables antibodies to simultaneously recognise and contact a vast array of potential antigens, while simultaneously interacting with the rest of the immune system in a predictable manner.

The constant-region sequence of an antibody heavy chain (or its mRNA) is known as the class or isotype of that antibody, while its variable-region sequence is known as its idiotype. % Definitions box?

The variable region of an antibody can be further subdivided into three complementarity-determining-regions (CDR1-3), which form part of the antigen-binding site and directly contact the antigen, and four framework regions (FR1-4), which do not. Sequence variability in the variable region is concentrated in the CDRs, with CDR3 showing by far the greatest variability for a variety of reasons discussed below. % Figure for this? E.g. align Vs and mark regions of non-conservation
This sequence variability is generated by several highly-specialised genome-editing mechanisms, which together produce a distinct idiotypic sequence on the chromosome of each developing B-cell. Broadly speaking, these diversification mechanisms can be divided into three categories, each of which has a distinct effect on repertoire sequence diversity; these categories are V(D)J recombination, junctional diversification, and somatic hypermutation.

\subsection{Mechanisms of heavy-chain diversification I: V(D)J recombination}


% FIGURES AND TABLES:
% - Schematic of native locus state -> VDJR -> recombined locus
% - RAG structure and DNA Binding
% - Schematic of RAG action and DNA excision
% - Schematic of RSS structure
% - Sequence logos of human/mouse/other RSSs
% - Schematic of V/D/J coverage on protein chain
% - Length distributions of human/mouse/other V/D/J segments

The structure of the immunoglobulin heavy chain (\textit{IGH}) locus differs markedly between its native germline configuration and that adopted by a mature B-cell. In the native state, most \textit{IGH} loci comprise large numbers of isolated gene segments separated by non-coding DNA. These segments can be divided into three classes, variable (V), diversity (D) and joining (J), with distinct sequence properties and length distributions (see Box~\dots). % FIGURE: VDJ length distributions in humans etc; schematic of V/D/J positioning in peptide chain
During B-cell development, site-specific recombination reactions result in the rearrangement of individual V, D and J segments into a continuous VDJ sequence, with the intervening genomic regions permanently excised and degraded. % Citation about what happens to excised sequences
This irreversible genomic maturation process is known as V(D)J recombination.

VDJ recombination is carried out by the RAG endonuclease complex, an enzyme formed by the association of \textbf{r}ecombination-\textbf{a}ctivating \textbf{g}enes 1 \& 2 \citep{jung2006vdjr}. % more detailed citation, roles of RAG1 vs RAG2
This complex, which is expressed specifically in developing lymphocytes, % citation needed
introduces recognises specialised recombination sequences (RSSs) at the ends of IGH gene segments and introduces targeted double-strand breaks between two segments and their respective RSSs \citep{jung2006vdjr}; these DSBs are then repaired by non-homologous end joining, resulting in a continuous coding sequence spanning both segments. The excised DNA ... % Citation for this

VDJ recombination in the \textit{IGH} locus is highly structured, and occurs in a specific order. First, a D and a J segment are selected and recombined to produce a DJ sequence. Then, a V region is selected and recombined with the DJ to produce a continuous VDJ sequence constituting the variable region of the heavy chain. The complete protein sequence is produced during later transcriptional splicing, which joins this variable-region sequence to downstream constant-region exons to produce a mature \textit{IGH} mRNA. This strict ordering of V/D/J segments, which obtains in the vast majority of recombined sequences observed, is produced through the combination of a variety of regulatory mechanisms. The most basic of these is the structure of the RSSs, which comprise a conserved heptamer and nonamer sequence separated by a relatively unconserved spacer region of either 12 or 23bp \citep{jung2006vdjr}, corresponding to either one or two turns of the DNA helix. V and J segments in the IGH locus are flanked by RSSs with 23bp spacer regions, while those flanking D-regions have 12bp spacers %citation needed
As the RAG recombinase specifically recognises pairs of RSSs with dissimilar spacer lengths (a restriction known as the 12/23 or one-turn/two-turn rule), direct V-to-J recombination events are excluded \citep{jung2006vdjr}. % Better citation if possible.

In addition to the restrictions imposed by the 12/23 rule, additional limitations on VDJ recombination are imposed by the requirement that RAG binding be preceded by transcription from the to-be-recombined segments, apparently in order to open the chromatin structure in that region \citep{jung2006vdjr}. Both V and D segments, but not J segments, are preceded by upstream promoter regions, which are involved in transcriptional initiation during specific stages of B-cell development...

Complete VDJ recombination places the V-region promoter in close proximity to a highly-conserved enhancer element (known as iE$\mu$) lying between the last J segment and the first constant-region exon \citep{jung2006vdjr}; this enhancer is important for strong expression of the mature IGH mRNA from the pre-B-cell stage onwards. 

\begin{itemize}
\item \textbf{Variable (V) segments} are the longest class of \textit{IGH} gene segment, with functional segments ranging from \dots to \dots nucleotides in humans and mice. % Figure and citation for this; add amino-acid lengths
This includes the coding sequence for the bulk of the variable region of the heavy chain, including all of FR1, CDR1, FR2, CDR2 and FR3, as well as the 5'/N-terminal part of CDR3 \citep{jung2006vdjr}. Each V segment also includes an N-terminal signal peptide sequence for antibody trafficking % citation needed
, and is preceded by its own 5'-UTR and promoter region. % citation for V promoters?
\item \textbf{Diversity (D) segments} are the shortest class of \textit{IGH} gene segment, ranging from \dots to \dots nucleotides in humans. %!
They form the middle part of the heavy chain CDR3. Like V-segments, D-segments are flanked by upstream promoter regions, which are involved in initiating the transcription required for DJ recombination. \citep{jung2006vdjr}
\item{Joining (J) segments} are of intermediate length, \dots % length data
. They form the 3'/C-terminal part of heavy chain CDR3 and the 5'/N-terminal part of FR4. Each J-segment is succeeded on the chromosome by a splice donor site, which is used to splice the recombined variable region to the constant region following transcription of IGH mRNA.
\end{itemize}

In its native germline state, the antibody heavy chain locus occupies a highly unusual configuration



The immunoglobulin heavy chain (\textit{IGH}) locus has a highly distinctive structure, whose broad outlines are shared only by other adaptive-immune antigen receptors...

The human genome contains seven loci that follow these general patterns of diversification: the \textit{IGH} locus, two light chain loci, and four T-cell receptor loci \citep{jung2006vdjr}.

% First explain structure of antibody protein chain
% -> mRNA
% -> unrecombined locus



An antibody protein chain can be divided into two regions with distinctive roles in the immune system: an N-terminal antigen-binding domain

. The C-terminal part of an antibody, known as the constant region, 

The N-terminal portion, by contrast, 

While the first two complementarity-determining regions of the heavy chain are contained entirely within the V segment, the heavy-chain CDR3 is formed by the combination of the V, D, and J segments selected for rearrangement \citep{jung2006vdjr}. As a result, the bulk of sequence diversity in the heavy chain falls within the CDR3, which is responsible for the greater part of antigen-binding variability among antibodies. % Citation for this
Nevertheless, the combinatorial sequence diversity produced by rearrangements of V/D/J segments alone is necessarily limited, with a maximum of $a \times b \times c$ possible sequences for a locus containing $a$ V, $b$ D and $c$ J sequences; far below the diversity of possible antigens. % citation needed 
Fortunately, the potential sequence diversity of the na\:{i}ve repertoire is powerfully augmented by a second diversification mechanism, taking place alongside VDJ recombination during B-cell development: junctional diversification.

\subsection{Mechanisms of heavy-chain diversification II: junctional diversification}

...

The net number of inserted nucleotides is essentially random, but the reading frame of the IGH transcript is fixed by the positions of the ATG initiation codon in the V-segment and the J-segment/constant-region splice junction. As a result, junctional diversification results in a large number of frameshift mutations, in addition to STOP codons that prematurely truncate the protein sequence. % Are these STOP mutations also counted as non-productive, or just frameshifts
Such loci, which are unable to produce a functional heavy chain protein, are termed non-productive, while those without such mutations are termed productive ; at a first approximation, roughly one-third of VDJ recombination events are expected to produce a productive sequence \citep{jung2006vdjr}. % Derivation and graphical representation of this; probability of two discrete random variables being congruent modulo three?

While a given recombination event only has about a one-third chance of producing a productive rearrangement, B-cells, like other somatic cells in vertebrates, are diploid. As a result, while a given recombination event only has about a one-third chance of producing a productive rearrangement, a B-cell which undergoes a non-productive rearrangement on one chromosome can make a second attempt on the other. Given the one-third approximation given above, this means that about 55\% of B-cells will achieve a productive rearrangement on one or the other chromosome; the other 45\%, unable to produce a functional heavy chain from either chromosome, die by apoptosis during B-cell development \citep{jung2006vdjr}. % Adapt figure 3 from \citep{jung2006vdjr}
This loss of almost half of all developing B-cells, a substantial cost, demonstrates the profound selective value of the additional antigen-binding diversity provided by junctional diversification.

Of those B-cells undergoing a productive heavy-chain rearrangement (and therefore surviving this process), roughly three-fifths are expected to bear one rearranged and one unrearraranged locus, with the remaining two-fifths bearing two rearranged loci, one of which in unproductive. This 60/40 ratio is roughly borne out by empirical data 


\subsection{Structure of mammalian \textit{IGH} loci} % And teleosts/others as available

% Human locus

The mouse \textit{IGH} locus also adopts a translocon structure, in this case with an enormous V-region, comprising 150 or more V-segments (depending on the strain) spanning 2.7Mb \citep{jung2006vdjr}. The 12-13 murine D segments occupy a region of approximately 50kb, while the four J segments cover about 2kb. This is followed by 200kb of constant region exons. 

% Mouse locus (depending on strain): 150+ V, 12-13 D, 4 J. (\citep{jung2006vdjr})
% Total length of murine locus: ~3Mb near telomeric end of chr12

\subsection{Antibody effector function and isotype diversity}



\newpage
\section{The African turquoise killifish as a model for vertebrate ageing}

% POSSIBLE FIGURES:
% - Nothobranchius genus range, photos of different notho males
% * Photo of male and female GRZ TK
% * Lifespan curves of male and female GRZ-AD
% - Diapause schematic
% - Photograph of ephemeral pools in Zimbabwe
% - Something showing correlation between aridity and lifespan in nothos
% - Schematic comparing GRZ lifespan to other common ageing models, along with important present/absent human-like systems
% - Phylogeny of TK within teleosts, with divergence times and human outgroup

% POSSIBLE TABLES:
% - List of notho species used in research with approx mean/max lifespans
% - List of ageing phenotypes observed/not observed in GRZ / wild-derived strains
% - Table of TK genome assemblies (JENA/STFD/CLGN) with metrics
% - Comparison of genome metrics (size, repetitiveness, etc) between TK and other models
% - Life history table for GRZ, MZM, other nothos, other fish models

The genus \textit{Nothobranchius} comprises a broad group of annual freshwater fishes distributed across equatorial and subequatorial Africa \citep{valdesalici2003lifespan}, with species diversity concentrated in the south-east of the continent \citep{genade2005annual}. Members of this genus share a suite of adaptations to life in ephemeral pools and rivers, most notably the production of desiccation-resistant embryos capable of surviving through the dry season in a diapause state \citep{genade2005annual}. Fish from this genus have been known for several decades to exhibit very rapid growth and short lifespans, consistent with their evolving under conditions of very high extrinsic mortality \citep{valdesalici2003lifespan}, with many species exhibiting a median lifespan of less than one year. Nevertheless, there is wide variation within the genus in body size, growth rate and lifespan, with species from less arid regions tending to show slower growth and longer median lifespans \citep{genade2005annual}.

Like other \textit{Nothobranchius} species, the turquoise killifish (\textit{Nothobranchius furzeri}) is a medium-sized annual fish first isolated from ephemeral freshwater pools -- in this case, from a relatively arid region of southeastern Zimbabwe \citep{jubb1971new,genade2005annual}. Even by the standards of the \textit{Nothobranchius} genus, \textit{N. furzeri} exhibits extremely rapid growth, maturation, and ageing, with the most widely-used laboratory strain (GRZ) exhibiting a median lifespan of just 9-14 weeks \citep{valdesalici2003lifespan,genade2005annual,terzibasi2008strains,kirschner2012map,valenzano2015genome} % Update range as other papers get different results
-- the shortest lifespan of any captive vertebrate. Moreover, while all turquoise killifish strains are very short-lived by vertebrate standards, different strains of this species have been found to differ several-fold in their median and maximum lifespans \citep{terzibasi2008strains,kirschner2012map}, providing a rare opportunity for intraspecific comparative ageing studies \citep{terzibasi2008strains,terzibasi2009dr,hartmann2009telomeres} and mapping the genetic underpinnings of lifespan \citep{kirschner2012map}. This combination of extremely short-lived laboratory strains and a wide range of lifespan phenotypes within a single species makes the turquoise killifish an extremely promising model organism for ageing research, especially when combined with the presence of vertebrate-specific adaptations absent in short-lived invertebrate ageing models.


Despite its very short lifespan, \textit{N. furzeri} has been found to show a wide range of senescent phenotypes in even the shortest-lived strains, including lipofuscin deposition \citep{genade2005annual};  accumulation of senescence markers \citep{genade2005annual};  increased neurodegenaration \citep{valenzano2006resveratrol1,valenzano2006resveratrol2}; impaired learning and behavioural phenotypes  \citep{genade2005annual,valenzano2006resveratrol1}; and a high incidence of degenerative and neoplastic lesions \citep{dicicco2011histopathology}. These diverse phenotypes indicate that the short lifespan of the turquoise killifish is the result of an accelerated general ageing process, rather than the specific failure of a particular organ or system.
Moreover, established anti-ageing interventions such as resveratrol treatment \citep{valenzano2006resveratrol1}, reduction in ambient temperature \citep{valenzano2006temperature} and dietary restriction \citep{terzibasi2009dr} also extend lifespan in the turquoise killifish, indicating a strong analogy with the ageing phenotypes observed in canonical model systems.


% (though others, such as shortened telomeres \citep{hartmann2009telomeres} and gonadal fibrosis \citep{dicicco2011histopathology}, are only observed in longer-lived strains of the species) % Add more as you read them
% Other ageing phenotypes, including
% telomere shortening \citep{hartmann2009telomeres}, gonadal fibrosis \citep{dicicco2011histopathology}, % not observed in GRZ
% mtDNA depletion \citep{hartmann2011mitochondria}, and bioenergetic impairment \citep{hartmann2011mitochondria} % Not studied in GRZ
% have been observed in longer-lived \textit{N. furzeri} strains, but were either not observed (telomeres, gonads) or not studied (mitochondrial phenotypes) in the short-lived GRZ strain.
%.  

Due primarily to its potential as a model organism for ageing research, the turquoise killifish has also seen rapid development as a genetic model. The short-lived GRZ strain has been bred in captivity for fifty years and at least a hundred generations \citep{terzibasi2007review} and exibits a very high degree of homozygosity \citep{reichwald2009genome,valenzano2009map,kirschner2012map}, providing a uniform genetic background for experimental interventions. A variety of effective transgenesis and mutagenesis methods have been developed for this species, including \textit{tol2} transgenesis \citep{valenzano2011tol2,hartmann2012tol2,allard2013inducible} %TALENS?
and CRISPR \citep{harel2015crispr,harel2016crispr}; furthermore, a number of important genetic resources are now available, including linkage maps \citep{valenzano2009map,kirschner2012map,valenzano2015genome}, a transcript catalogue \citep{petzold2013transcriptome}, an miRNAome % citation needed
and a mitochondrial genome \citep{hartmann2011mitochondria}. Most importantly, a high-quality nuclear genome of the short-lived GRZ strain is now available
% with a recent study integrating evidence from multiple independent studies to produce a very high-quality assembly \cite{ray's genome paper, plus others}. 
\citep{reichwald2015genome,valenzano2015genome}. % Add more here
The progressive release of improved genome assemblies for the turquoise killifish has been particularly important for this project, and will be discussed in more detail in Section ??. %!
The genome itself is unusually large and repetitive by teleost standards, potentially contributing to these fishes' unusually short lifespan. % citation needed

% Needed: oxidation levels, telomere dysfunction, more senescence markers

% Ageing-related genes under positive selection (both genome papers)


%The killifish genome is unusually large by teleost standards, % TODO: figure for this? update figures and add citations as you read newer sources
%with an estimated size of 1.5 to 2 gigabases \citep{reichwald2009genome,reichwald2015genome}. This large size is primarily accounted for by the exceptionally high repeat content of the killifish genome, with at least 21\% and 24\% composed of tandem and other types of repeats, respectively \citep{reichwald2009genome}. GC content is also relatively high, at 44\%, though this average is affected somewhat by the presence of a distinct fraction of highly GC-rich tandem repeats \citep{reichwald2009genome}. Karyotyping \citep{reichwald2009genome} and sequencing analysis \citep{reichwald2015genome} both indicate a chromosome number of $2n = 38$, corresponding to 19 distinct linkage groups in the haploid genome. 

%They are strongly sexually dimorphic, with an XY sex-determination system \citep{reichwald2015genome,valenzano2009map}. %Photo

Phylogenetically, the genus \textit{Nothobranchius} falls within the Cyprinodontiformes, and the turquoise killifish is therefore closely related to a number of fish species that have been the subject of extensive research, including guppies, platyfish, medaka, sticklebacks, and pufferfish  \citep{terzibasi2007review} %TODO: cite an up-to-date phylogeny, add estimated divergency times, add a phylogeny figure.
; the genetic resources available for these species, in combination with the high degree of synteny shown across this group of teleosts \citep{terzibasi2007review} have been of great utility in a number of killifish projects, including this one. On the other hand, the most well-developed teleost model organism, the zebrafish, is relatively distantly-related, making the use in a killifish context of genetic and experimental resources developed for the zebrafish more challenging.


% TODO: Add a paragraph on the killifish immune system to segue into teleost immunity stuff

\section{Misc.}

\subsection{Ageing of the antibody repertoire}

% Notes from de Bourcey et al. (2017)

The vertebrate adaptive immune system has long been observed to undergo a severe decline with age in multiple species, with notable changes in humans including decreased lymphocyte proliferation \citep{debourcy2017ageing} and defects in antibody production \citep{debourcy2017ageing}. Changes in the human antibody repertoire observed with age include restricted

In studies of peripheral blood repertoires taken before and after influenza vaccination, older individuals have been observed to show reduced within-individual and increased between-individual repertoire diversity \citep{debourcy2017ageing}, suggesting that repertoires become increasingly (and divergently) specialised with age. Older repertoires also show less change in composition following vaccination, showing a reduced capacity to adapt to new information from the pathogenic environment \citep{debourcy2017ageing}. An oligoclonal phenotype, in which one or a few memory B-cell lineages occupy a disproportionate share of repertoire diversity, has been reported in a subset of older individuals in multiple studies \citep{debourcy2017ageing}; these expanded clones appear to be resistant to immunogenic interventions such as vaccination. 

Within similarly-sized lineages, the repertoires of older individuals have been found to show reduced per-nucleotide sequence diversity, indicating an impairment in secondary diversification through somatic hypermutation; this effect is especially pronounced in larger clones \citep{debourcy2017ageing}. % Fig 3D

A subset of older repertoires also showed a greater prevalence of sequences bearing premature stop codons, indicating... % Also reduced "radical" mutations

This loss in repertoire diversity with age is observed in both naive and antigen-experienced subsets of the repertoire \citep{debourcy2017ageing}, but is strongest in the former, indicating an increase in the relative prevalence of the memory compartment within the repertoire. This is consistent with a model of the aged immune system as impaired by the accumulation of stubborn immune memory.

An important feature of these studies is that CMV infection has been found to have an important influence of certain aspects of repertoire ageing, including...

%
%% Does this need to be double spaced?
%
%%
%%\ifpdf
%%    \graphicspath{{Chapter1/Figs/Raster/}{Chapter1/Figs/PDF/}{Chapter1/Figs/}}
%%\else
%%    \graphicspath{{Chapter1/Figs/Vector/}{Chapter1/Figs/}}
%%\fi
%
%\section*{Summary} % Fits one one page if 1.5-spaced, but not at double spacing
%
%The vertebrate adaptive immune system is among the most impressive achievements of animal evolution. In the billion-year-old arms race between pathogens and their hosts, it represents a paradigm change in defensive capability, enabling large, slow-breeding vertebrates to effectively fight off the vast majority of assaults from much-faster-evolving pathogenic attackers. By dynamically recombining their antigen-receptor genes during immune-cell development, vertebrates are capable of producing receptors with an almost unlimited variety of antigen-specificities, enabling B- and T-lymphocytes with the correct receptors to respond specifically to pathogens that neither that individual nor its ancestors have ever encountered. The selective value of such a system is evidenced by the high costs selection is apparently willing to accept to retain it: despite the very high levels of cell wastage it necessitates and frequent autoimmune infections it causes, the adaptive immune system forms a key part of the immune system of all known jawed vertebrates.
%
%Among the different adaptive antigen-receptor proteins present in gnathostomes, antibodies stand out in the biochemical diversity of their cognate antigens and their ability to act independently of their generating B-cell.
%
%% If the adaptive immune repertoire is so great, there should be some evidence that vertebrates are more resistant to infections than other species that lack it. Is this the case?
%
%Faced with the pervasive risk of parasitic infection, vertebrates have evolved highly complex immune systems capable of effectively combating a wide range of pathogenic threats. Among the most sophisticated of these adaptations is the adaptive immunity provided by B- and T-lymphocytes, which utilise a range of specialised genome-editing mechanisms to generate a vast array of distinct antigen-binding proteins. These remarkable systems, however, undergo severe systemic decline with ageing, leading to greatly increased rates of infection-related morbidity in older individuals. Comparable immunosenescent phenotypes have been widely observed in mammals, birds, fish and elsewhere, and appear to be broadly conserved across the vertebrate lineage.
%
%To understand and counter the complex changes that occur in the adaptive immune system with age, it is necessary to analyse the whole population of lymphocyte antigen-receptor sequences present in an individual. Such a top-down approach was impossible until relatively recently, when the advent of modern high-throughput sequencing technologies enabled the development of specialised protocols for immune-repertoire sequencing and analysis. Since then, the field of immune-repertoire studies has developed rapidly, providing a new and more powerful method for interrogating the changes ocurring in adaptive immune repertoires in a wide variety of contexts, including ageing. However, while initial human studies have indicated a decline in the diversity of these repertoires in older people, there remains a need for further research in this area. % This is weak, fill in once you've done your lit review.
%
%As the shortest-lived vertebrate to be bred in captivity, the African turquoise killifish (\textit{Nothobranchius furzeri}) represents a powerful model for studying vertebrate-specific ageing processes, including immunosenescence of the adaptive immune system. Though the killifish has seen rapid development as a model system for ageing research, little was known about its adaptive immune system prior to the work described in this thesis. % In this chapter...? But it's a summary, not an introduction. 
%
%\pagebreak
%
%\section{The vertebrate adaptive immune system}
%
%All organisms exist in a condition of intense competition for resources, with predators, peers, and parasites all competing, in some way, for the nutrients and energy consumed and used by an individual. Among the most insidious of these competitors are parasites who attempt to colonise an organism's own body, consuming its stores of nutrients and energy and turning its internal mechanisms to their own advantage. When an organism falls prey to one of these parasites and manifests the symptoms of its exploitation, we call it disease. When the organism utilises adaptations to prevent this exploitation, through excluding or killing the parasites, we call it immunity.
%
%Given the extreme selective pressure to protect their fitness from parasitic exploitation, it is perhaps unsurprising that so many different organisms have evolved immune systems of great complexity and effectiveness. Nevertheless, the intricacy of the vertebrate immune system has proven one of the most enduringly fascinating aspects of vertebrate biology, comparable to vertebrate neural systems in its complexity. Indeed, the vertebrate immune system shows many parallels with the nervous system, being the only other system capable of complex information processing and memory. This complexity, which is fundamental to the effectiveness of the vertebrate immune system, rests on the interplay between the two traditional wings of vertebrate immunology: the innate immune system, and the adaptive.
%
%Innate immunity refers to a large collection of mechanisms designed to exclude, sequester, or kill invading pathogens (disease-causing organisms) in a rapid and nonspecific manner. Many innate systems combat pathogens in ways that are either physically difficult to circumvent (such as external barriers, or engulfment by phagocytic cells) or which target aspects of pathogen biology that are difficult to alter without catastrophic loss of function (such as this wonderful example %example
%). As such, the great majority of possible pathogenic threats are dealt with rapidly and effectively by the innate immune system, either by keeping parasites from accessing vulnerable parts of the organism or by rapidly and nonspecifically eliminating them once there.
%
%Despite its speed, power and impressive generality, the innate immune system suffers from severe limitations. The first is that it is helpless in the face of evolutionarily novel threats to which its existing defences do not apply. The second, perhaps more fundamental, problem is that many common pathogens are capable of evolving at speeds vastly exceeding that of vertebrates. For example, many bacteria have generation times of much less than an hour, compared to months or years for most vertebrates, while also exhibiting much higher per-cell-division rates of mutation. And even this rapid rate of evolution is far exceeded by the highly volatile genomes of many viruses.
%
%This capacity for parasitic organisms to out-evolve their hosts represents a serious problem for vertebrates, who cannot hope to effectively respond to these threats through the generation of new and improved innate immune mechanisms via selection. Instead, what is needed is a mechanisms by which vertebrates can dynamically learn to respond to novel immune threats within the lifespan of a single organism. That mechanism is the adaptive immune system.
%
%% Need to rapidly focus onto B-cell immunity in particular, I know nothing of T-cells
%
%The mechanisms used to generate this sequence diversity in the B-cell population are themselves diverse, and have been discovered progressively over the past decades. The most fundamental, and well-known, such mechanism is so-called V(D)J recombination, first discovered quite some time ago. %Find out more about history of this
%
%In V(D)J recombination, a number of
%
%A canonical immunoglobulin heavy chain (IgH) gene locus consists of clusters of variable (V), diversity (D) and joining (J) regions in series, followed by some number of larger constant-region exons. During B-cell maturation, a single V, D and J region are selected and the intervening DNA regions are excised to produce a single, contiguous VDJ sequence. As part of this process, nontemplated nucleotides are inserted and deleted at the V/D and D/J junctions, a process known as junctional diversification. Following transcription, the sequence between the VDJ sequence and the first constant-region exon is removed by splicing to produce a mature IgH transcript, with its characteristig VDJC sequence structure.
%
%The manner of constant region selection in B-cells differs significantly between mammals and teleosts. In the former, a number of distinct constant regions are present in series on the chromo
%
%\newpage 
%\section{Structure and diversification of the antibody heavy chain}
%
%% Pretend light chains and TCRs don't exist for now; you can generalise later as needed
%
%The great majority of antibodies produced by the gnathostome adaptive immune system share a canonical tetrameric structure: two heavy chains and two light chains, arranged into a roughly Y-shaped configuration. This structure comprises three important functional domains: two antigen-binding domains, formed by the N-terminal portions of the heavy and light chains, and one effector domain formed by the C-termini of the heavy chains. As a result of these distinct functionalities, the two ends of an antibody heavy-chain protein have very different properties when considered as a population. The N-terminal variable domain is highly variable in amino-acid sequence, with unrelated\footnote{i.e. not derived from the same plasma-cell clone} B-cells typically displaying different and often unique sequences. The C-terminal constant region, meanwhile, shows very limited sequence diversity, with all B-cells in the body producing one of a small number of distinct effector classes. This combination of a highly-diverse antigen-binding domain and a limited range of distinct effector domains enables antibodies to simultaneously recognise and contact a vast array of potential antigens, while simultaneously interacting with the rest of the immune system in a predictable manner.
%
%The constant-region sequence of an antibody heavy chain (or its mRNA) is known as the class or isotype of that antibody, while its variable-region sequence is known as its idiotype. % Definitions box?
%
%The variable region of an antibody can be further subdivided into three complementarity-determining-regions (CDR1-3), which form part of the antigen-binding site and directly contact the antigen, and four framework regions (FR1-4), which do not. Sequence variability in the variable region is concentrated in the CDRs, with CDR3 showing by far the greatest variability for a variety of reasons discussed below. % Figure for this? E.g. align Vs and mark regions of non-conservation
%This sequence variability is generated by several highly-specialised genome-editing mechanisms, which together produce a distinct idiotypic sequence on the chromosome of each developing B-cell. Broadly speaking, these diversification mechanisms can be divided into three categories, each of which has a distinct effect on repertoire sequence diversity; these categories are V(D)J recombination, junctional diversification, and somatic hypermutation.
%
%\subsection{Mechanisms of heavy-chain diversification I: V(D)J recombination}
%
%
%% FIGURES AND TABLES:
%% - Schematic of native locus state -> VDJR -> recombined locus
%% - RAG structure and DNA Binding
%% - Schematic of RAG action and DNA excision
%% - Schematic of RSS structure
%% - Sequence logos of human/mouse/other RSSs
%% - Schematic of V/D/J coverage on protein chain
%% - Length distributions of human/mouse/other V/D/J segments
%
%The structure of the immunoglobulin heavy chain (\textit{IGH}) locus differs markedly between its native germline configuration and that adopted by a mature B-cell. In the native state, most \textit{IGH} loci comprise large numbers of isolated gene segments separated by non-coding DNA. These segments can be divided into three classes, variable (V), diversity (D) and joining (J), with distinct sequence properties and length distributions (see Box~\dots). % FIGURE: VDJ length distributions in humans etc; schematic of V/D/J positioning in peptide chain
%During B-cell development, site-specific recombination reactions result in the rearrangement of individual V, D and J segments into a continuous VDJ sequence, with the intervening genomic regions permanently excised and degraded. % Citation about what happens to excised sequences
%This irreversible genomic maturation process is known as V(D)J recombination.
%
%VDJ recombination is carried out by the RAG endonuclease complex, an enzyme formed by the association of \textbf{r}ecombination-\textbf{a}ctivating \textbf{g}enes 1 \& 2 \citep{jung2006vdjr}. % more detailed citation, roles of RAG1 vs RAG2
%This complex, which is expressed specifically in developing lymphocytes, % citation needed
%introduces recognises specialised recombination sequences (RSSs) at the ends of IGH gene segments and introduces targeted double-strand breaks between two segments and their respective RSSs \citep{jung2006vdjr}; these DSBs are then repaired by non-homologous end joining, resulting in a continuous coding sequence spanning both segments. The excised DNA ... % Citation for this
%
%VDJ recombination in the \textit{IGH} locus is highly structured, and occurs in a specific order. First, a D and a J segment are selected and recombined to produce a DJ sequence. Then, a V region is selected and recombined with the DJ to produce a continuous VDJ sequence constituting the variable region of the heavy chain. The complete protein sequence is produced during later transcriptional splicing, which joins this variable-region sequence to downstream constant-region exons to produce a mature \textit{IGH} mRNA. This strict ordering of V/D/J segments, which obtains in the vast majority of recombined sequences observed, is produced through the combination of a variety of regulatory mechanisms. The most basic of these is the structure of the RSSs, which comprise a conserved heptamer and nonamer sequence separated by a relatively unconserved spacer region of either 12 or 23bp \citep{jung2006vdjr}, corresponding to either one or two turns of the DNA helix. V and J segments in the IGH locus are flanked by RSSs with 23bp spacer regions, while those flanking D-regions have 12bp spacers %citation needed
%As the RAG recombinase specifically recognises pairs of RSSs with dissimilar spacer lengths (a restriction known as the 12/23 or one-turn/two-turn rule), direct V-to-J recombination events are excluded \citep{jung2006vdjr}. % Better citation if possible.
%
%In addition to the restrictions imposed by the 12/23 rule, additional limitations on VDJ recombination are imposed by the requirement that RAG binding be preceded by transcription from the to-be-recombined segments, apparently in order to open the chromatin structure in that region \citep{jung2006vdjr}. Both V and D segments, but not J segments, are preceded by upstream promoter regions, which are involved in transcriptional initiation during specific stages of B-cell development...
%
%Complete VDJ recombination places the V-region promoter in close proximity to a highly-conserved enhancer element (known as iE$\mu$) lying between the last J segment and the first constant-region exon \citep{jung2006vdjr}; this enhancer is important for strong expression of the mature IGH mRNA from the pre-B-cell stage onwards. 
%
%
%In its native germline state, the antibody heavy chain locus occupies a highly unusual configuration
%
%
%
%The immunoglobulin heavy chain (\textit{IGH}) locus has a highly distinctive structure, whose broad outlines are shared only by other adaptive-immune antigen receptors...
%
%The human genome contains seven loci that follow these general patterns of diversification: the \textit{IGH} locus, two light chain loci, and four T-cell receptor loci \citep{jung2006vdjr}.
%
%% First explain structure of antibody protein chain
%% -> mRNA
%% -> unrecombined locus
%
%
%
%An antibody protein chain can be divided into two regions with distinctive roles in the immune system: an N-terminal antigen-binding domain
%
%. The C-terminal part of an antibody, known as the constant region, 
%
%The N-terminal portion, by contrast, 
%
%While the first two complementarity-determining regions of the heavy chain are contained entirely within the V segment, the heavy-chain CDR3 is formed by the combination of the V, D, and J segments selected for rearrangement \citep{jung2006vdjr}. As a result, the bulk of sequence diversity in the heavy chain falls within the CDR3, which is responsible for the greater part of antigen-binding variability among antibodies. % Citation for this
%Nevertheless, the combinatorial sequence diversity produced by rearrangements of V/D/J segments alone is necessarily limited, with a maximum of $a \times b \times c$ possible sequences for a locus containing $a$ V, $b$ D and $c$ J sequences; far below the diversity of possible antigens. % citation needed 
%Fortunately, the potential sequence diversity of the na\:{i}ve repertoire is powerfully augmented by a second diversification mechanism, taking place alongside VDJ recombination during B-cell development: junctional diversification.
%
%\subsection{Mechanisms of heavy-chain diversification II: junctional diversification}
%
%...
%
%The net number of inserted nucleotides is essentially random, but the reading frame of the IGH transcript is fixed by the positions of the ATG initiation codon in the V-segment and the J-segment/constant-region splice junction. As a result, junctional diversification results in a large number of frameshift mutations, in addition to STOP codons that prematurely truncate the protein sequence. % Are these STOP mutations also counted as non-productive, or just frameshifts
%Such loci, which are unable to produce a functional heavy chain protein, are termed non-productive, while those without such mutations are termed productive ; at a first approximation, roughly one-third of VDJ recombination events are expected to produce a productive sequence \citep{jung2006vdjr}. % Derivation and graphical representation of this; probability of two discrete random variables being congruent modulo three?
%
%While a given recombination event only has about a one-third chance of producing a productive rearrangement, B-cells, like other somatic cells in vertebrates, are diploid. As a result, while a given recombination event only has about a one-third chance of producing a productive rearrangement, a B-cell which undergoes a non-productive rearrangement on one chromosome can make a second attempt on the other. Given the one-third approximation given above, this means that about 55\% of B-cells will achieve a productive rearrangement on one or the other chromosome; the other 45\%, unable to produce a functional heavy chain from either chromosome, die by apoptosis during B-cell development \citep{jung2006vdjr}. % Adapt figure 3 from \citep{jung2006vdjr}
%This loss of almost half of all developing B-cells, a substantial cost, demonstrates the profound selective value of the additional antigen-binding diversity provided by junctional diversification.
%
%Of those B-cells undergoing a productive heavy-chain rearrangement (and therefore surviving this process), roughly three-fifths are expected to bear one rearranged and one unrearraranged locus, with the remaining two-fifths bearing two rearranged loci, one of which in unproductive. This 60/40 ratio is roughly borne out by empirical data 
%
%
%\subsection{Structure of mammalian \textit{IGH} loci} % And teleosts/others as available
%
%% Human locus
%
%The mouse \textit{IGH} locus also adopts a translocon structure, in this case with an enormous V-region, comprising 150 or more V-segments (depending on the strain) spanning 2.7Mb \citep{jung2006vdjr}. The 12-13 murine D segments occupy a region of approximately 50kb, while the four J segments cover about 2kb. This is followed by 200kb of constant region exons. 
%
%% Mouse locus (depending on strain): 150+ V, 12-13 D, 4 J. (\citep{jung2006vdjr})
%% Total length of murine locus: ~3Mb near telomeric end of chr12
%
%\subsection{Antibody effector function and isotype diversity}
%
%
%
%\newpage
%\section{The African turquoise killifish as a model for vertebrate ageing}
%
%% POSSIBLE FIGURES:
%% - Nothobranchius genus range, photos of different notho males
%% * Photo of male and female GRZ TK
%% * Lifespan curves of male and female GRZ-AD
%% - Diapause schematic
%% - Photograph of ephemeral pools in Zimbabwe
%% - Something showing correlation between aridity and lifespan in nothos
%% - Schematic comparing GRZ lifespan to other common ageing models, along with important present/absent human-like systems
%% - Phylogeny of TK within teleosts, with divergence times and human outgroup
%
%% POSSIBLE TABLES:
%% - List of notho species used in research with approx mean/max lifespans
%% - List of ageing phenotypes observed/not observed in GRZ / wild-derived strains
%% - Table of TK genome assemblies (JENA/STFD/CLGN) with metrics
%% - Comparison of genome metrics (size, repetitiveness, etc) between TK and other models
%% - Life history table for GRZ, MZM, other nothos, other fish models
%
%The genus \textit{Nothobranchius} comprises a broad group of annual freshwater fishes distributed across equatorial and subequatorial Africa \citep{valdesalici2003lifespan}, with species diversity concentrated in the south-east of the continent \citep{genade2005annual}. Members of this genus share a suite of adaptations to life in ephemeral pools and rivers, most notably the production of desiccation-resistant embryos capable of surviving through the dry season in a diapause state \citep{genade2005annual}. Fish from this genus have been known for several decades to exhibit very rapid growth and short lifespans, consistent with their evolving under conditions of very high extrinsic mortality \citep{valdesalici2003lifespan}, with many species exhibiting a median lifespan of less than one year. Nevertheless, there is wide variation within the genus in body size, growth rate and lifespan, with species from less arid regions tending to show slower growth and longer median lifespans \citep{genade2005annual}.
%
%Like other \textit{Nothobranchius} species, the turquoise killifish (\textit{Nothobranchius furzeri}) is a medium-sized annual fish first isolated from ephemeral freshwater pools -- in this case, from a relatively arid region of southeastern Zimbabwe \citep{jubb1971new,genade2005annual}. Even by the standards of the \textit{Nothobranchius} genus, \textit{N. furzeri} exhibits extremely rapid growth, maturation, and ageing, with the most widely-used laboratory strain (GRZ) exhibiting a median lifespan of just 9-14 weeks \citep{valdesalici2003lifespan,genade2005annual,terzibasi2008strains,kirschner2012map,valenzano2015genome} % Update range as other papers get different results
%-- the shortest lifespan of any captive vertebrate. Moreover, while all turquoise killifish strains are very short-lived by vertebrate standards, different strains of this species have been found to differ several-fold in their median and maximum lifespans \citep{terzibasi2008strains,kirschner2012map}, providing a rare opportunity for intraspecific comparative ageing studies \citep{terzibasi2008strains,terzibasi2009dr,hartmann2009telomeres} and mapping the genetic underpinnings of lifespan \citep{kirschner2012map}. This combination of extremely short-lived laboratory strains and a wide range of lifespan phenotypes within a single species makes the turquoise killifish an extremely promising model organism for ageing research, especially when combined with the presence of vertebrate-specific adaptations absent in short-lived invertebrate ageing models.
%
%
%Despite its very short lifespan, \textit{N. furzeri} has been found to show a wide range of senescent phenotypes in even the shortest-lived strains, including lipofuscin deposition \citep{genade2005annual};  accumulation of senescence markers \citep{genade2005annual};  increased neurodegenaration \citep{valenzano2006resveratrol1,valenzano2006resveratrol2}; impaired learning and behavioural phenotypes  \citep{genade2005annual,valenzano2006resveratrol1}; and a high incidence of degenerative and neoplastic lesions \citep{dicicco2011histopathology}. These diverse phenotypes indicate that the short lifespan of the turquoise killifish is the result of an accelerated general ageing process, rather than the specific failure of a particular organ or system.
%Moreover, established anti-ageing interventions such as resveratrol treatment \citep{valenzano2006resveratrol1}, reduction in ambient temperature \citep{valenzano2006temperature} and dietary restriction \citep{terzibasi2009dr} also extend lifespan in the turquoise killifish, indicating a strong analogy with the ageing phenotypes observed in canonical model systems.
%
%
%% (though others, such as shortened telomeres \citep{hartmann2009telomeres} and gonadal fibrosis \citep{dicicco2011histopathology}, are only observed in longer-lived strains of the species) % Add more as you read them
%% Other ageing phenotypes, including
%% telomere shortening \citep{hartmann2009telomeres}, gonadal fibrosis \citep{dicicco2011histopathology}, % not observed in GRZ
%% mtDNA depletion \citep{hartmann2011mitochondria}, and bioenergetic impairment \citep{hartmann2011mitochondria} % Not studied in GRZ
%% have been observed in longer-lived \textit{N. furzeri} strains, but were either not observed (telomeres, gonads) or not studied (mitochondrial phenotypes) in the short-lived GRZ strain.
%%.  
%
%Due primarily to its potential as a model organism for ageing research, the turquoise killifish has also seen rapid development as a genetic model. The short-lived GRZ strain has been bred in captivity for fifty years and at least a hundred generations \citep{terzibasi2007review} and exibits a very high degree of homozygosity \citep{reichwald2009genome,valenzano2009map,kirschner2012map}, providing a uniform genetic background for experimental interventions. A variety of effective transgenesis and mutagenesis methods have been developed for this species, including \textit{tol2} transgenesis \citep{valenzano2011tol2,hartmann2012tol2,allard2013inducible} %TALENS?
%and CRISPR \citep{harel2015crispr,harel2016crispr}; furthermore, a number of important genetic resources are now available, including linkage maps \citep{valenzano2009map,kirschner2012map,valenzano2015genome}, a transcript catalogue \citep{petzold2013transcriptome}, an miRNAome % citation needed
%and a mitochondrial genome \citep{hartmann2011mitochondria}. Most importantly, a high-quality nuclear genome of the short-lived GRZ strain is now available
%% with a recent study integrating evidence from multiple independent studies to produce a very high-quality assembly \cite{ray's genome paper, plus others}. 
%\citep{reichwald2015genome,valenzano2015genome}. % Add more here
%The progressive release of improved genome assemblies for the turquoise killifish has been particularly important for this project, and will be discussed in more detail in Section ??. %!
%The genome itself is unusually large and repetitive by teleost standards, potentially contributing to these fishes' unusually short lifespan. % citation needed
%
%% Needed: oxidation levels, telomere dysfunction, more senescence markers
%
%% Ageing-related genes under positive selection (both genome papers)
%
%
%%The killifish genome is unusually large by teleost standards, % TODO: figure for this? update figures and add citations as you read newer sources
%%with an estimated size of 1.5 to 2 gigabases \citep{reichwald2009genome,reichwald2015genome}. This large size is primarily accounted for by the exceptionally high repeat content of the killifish genome, with at least 21\% and 24\% composed of tandem and other types of repeats, respectively \citep{reichwald2009genome}. GC content is also relatively high, at 44\%, though this average is affected somewhat by the presence of a distinct fraction of highly GC-rich tandem repeats \citep{reichwald2009genome}. Karyotyping \citep{reichwald2009genome} and sequencing analysis \citep{reichwald2015genome} both indicate a chromosome number of $2n = 38$, corresponding to 19 distinct linkage groups in the haploid genome. 
%
%%They are strongly sexually dimorphic, with an XY sex-determination system \citep{reichwald2015genome,valenzano2009map}. %Photo
%
%Phylogenetically, the genus \textit{Nothobranchius} falls within the Cyprinodontiformes, and the turquoise killifish is therefore closely related to a number of fish species that have been the subject of extensive research, including guppies, platyfish, medaka, sticklebacks, and pufferfish  \citep{terzibasi2007review} %TODO: cite an up-to-date phylogeny, add estimated divergency times, add a phylogeny figure.
%; the genetic resources available for these species, in combination with the high degree of synteny shown across this group of teleosts \citep{terzibasi2007review} have been of great utility in a number of killifish projects, including this one. On the other hand, the most well-developed teleost model organism, the zebrafish, is relatively distantly-related, making the use in a killifish context of genetic and experimental resources developed for the zebrafish more challenging.
%
%
%% TODO: Add a paragraph on the killifish immune system to segue into teleost immunity stuff
%
%\section{Misc.}
%
%\subsection{Ageing of the antibody repertoire}
%
%% Notes from de Bourcey et al. (2017)
%.
%The vertebrate adaptive immune system has long been observed to undergo a severe decline with age in multiple species, with notable changes in humans including decreased lymphocyte proliferation \citep{debourcy2017ageing} and defects in antibody production \citep{debourcy2017ageing}. Changes in the human antibody repertoire observed with age include restricted
%
%In studies of peripheral blood repertoires taken before and after influenza vaccination, older individuals have been observed to show reduced within-individual and increased between-individual repertoire diversity \citep{debourcy2017ageing}, suggesting that repertoires become increasingly (and divergently) specialised with age. Older repertoires also show less change in composition following vaccination, showing a reduced capacity to adapt to new information from the pathogenic environment \citep{debourcy2017ageing}. An oligoclonal phenotype, in which one or a few memory B-cell lineages occupy a disproportionate share of repertoire diversity, has been reported in a subset of older individuals in multiple studies \citep{debourcy2017ageing}; these expanded clones appear to be resistant to immunogenic interventions such as vaccination. 
%
%Within similarly-sized lineages, the repertoires of older individuals have been found to show reduced per-nucleotide sequence diversity, indicating an impairment in secondary diversification through somatic hypermutation; this effect is especially pronounced in larger clones \citep{debourcy2017ageing}. % Fig 3D
%
%A subset of older repertoires also showed a greater prevalence of sequences bearing premature stop codons, indicating... % Also reduced "radical" mutations
%
%This loss in repertoire diversity with age is observed in both naive and antigen-experienced subsets of the repertoire \citep{debourcy2017ageing}, but is strongest in the former, indicating an increase in the relative prevalence of the memory compartment within the repertoire. This is consistent with a model of the aged immune system as impaired by the accumulation of stubborn immune memory.
%
%An important feature of these studies is that CMV infection has been found to have an important influence of certain aspects of repertoire ageing, including...
