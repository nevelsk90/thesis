%!TEX root = ../thesis.tex
%*******************************************************************************
%*********************************** First Chapter *****************************
%*******************************************************************************

\chapter{Introduction}  %Title of the First Chapter
\doublespacing
%\onehalfspacing

% Does this need to be double spaced?

%
%\ifpdf
%    \graphicspath{{Chapter1/Figs/Raster/}{Chapter1/Figs/PDF/}{Chapter1/Figs/}}
%\else
%    \graphicspath{{Chapter1/Figs/Vector/}{Chapter1/Figs/}}
%\fi

\section*{Summary} % Fits one one page if 1.5-spaced, but not at double spacing

Faced with the pervasive risk of parasitic infection, vertebrates have evolved highly complex immune systems capable of effectively combating a wide range of pathogenic threats. Among the most sophisticated of these adaptations is the adaptive immunity provided by B- and T-lymphocytes, which utilise a range of specialised genome-editing mechanisms to generate a vast array of distinct antigen-binding proteins. These remarkable systems, however, undergo severe systemic decline with ageing, leading to greatly increased rates of infection-related morbidity in older individuals. Comparable immunosenescent phenotypes have been widely observed in mammals, birds, fish and elsewhere, and appear to be broadly conserved across the vertebrate lineage.

To understand and counter the complex changes that occur in the adaptive immune system with age, it is necessary to analyse the whole population of lymphocyte antigen-receptor sequences present in an individual. Such a top-down approach was impossible until relatively recently, when the advent of modern high-throughput sequencing technologies enabled the development of specialised protocols for immune-repertoire sequencing and analysis. Since then, the field of immune-repertoire studies has developed rapidly, providing a new and more powerful method for interrogating the changes ocurring in adaptive immune repertoires in a wide variety of contexts, including ageing. However, while initial human studies have indicated a decline in the diversity of these repertoires in older people, there remains a need for further research in this area. % This is weak, fill in once you've done your lit review.

As the shortest-lived vertebrate to be bred in captivity, the African turquoise killifish (\textit{Nothobranchius furzeri}) represents a powerful model for studying vertebrate-specific ageing processes, including immunosenescence of the adaptive immune system. Though the killifish has seen rapid development as a model system for ageing research, little was known about its adaptive immune system prior to the work described in this thesis. % In this chapter...? But it's a summary, not an introduction. 

\pagebreak

\section{The vertebrate adaptive immune system}

All organisms exist in a condition of intense competition for resources, with predators, peers, and parasites all competing, in some way, for the nutrients and energy consumed and used by an individual. Among the most insidious of these competitors are parasites who attempt to colonise an organism's own body, consuming its stores of nutrients and energy and turning its internal mechanisms to their own advantage. When an organism falls prey to one of these parasites and manifests the symptoms of its exploitation, we call it disease. When the organism utilises adaptations to prevent this exploitation, through excluding or killing the parasites, we call it immunity.

Given the extreme selective pressure to protect their fitness from parasitic exploitation, it is perhaps unsurprising that so many different organisms have evolved immune systems of great complexity and effectiveness. Nevertheless, the intricacy of the vertebrate immune system has proven one of the most enduringly fascinating aspects of vertebrate biology, comparable to vertebrate neural systems in its complexity. Indeed, the vertebrate immune system shows many parallels with the nervous system, being the only other system capable of complex information processing and memory. This complexity, which is fundamental to the effectiveness of the vertebrate immune system, rests on the interplay between the two traditional wings of vertebrate immunology: the innate immune system, and the adaptive.

Innate immunity refers to a large collection of mechanisms designed to exclude, sequester, or kill invading pathogens (disease-causing organisms) in a rapid and nonspecific manner. Many innate systems combat pathogens in ways that are either physically difficult to circumvent (such as external barriers, or engulfment by phagocytic cells) or which target aspects of pathogen biology that are difficult to alter without catastrophic loss of function (such as this wonderful example %example
). As such, the great majority of possible pathogenic threats are dealt with rapidly and effectively by the innate immune system, either by keeping parasites from accessing vulnerable parts of the organism or by rapidly and nonspecifically eliminating them once there.

Despite its speed, power and impressive generality, the innate immune system suffers from severe limitations. The first is that it is helpless in the face of evolutionarily novel threats to which its existing defences do not apply. The second, perhaps more fundamental, problem is that many common pathogens are capable of evolving at speeds vastly exceeding that of vertebrates. For example, many bacteria have generation times of much less than an hour, compared to months or years for most vertebrates, while also exhibiting much higher per-cell-division rates of mutation. And even this rapid rate of evolution is far exceeded by the highly volatile genomes of many viruses.

This capacity for parasitic organisms to out-evolve their hosts represents a serious problem for vertebrates, who cannot hope to effectively respond to these threats through the generation of new and improved innate immune mechanisms via selection. Instead, what is needed is a mechanisms by which vertebrates can dynamically learn to respond to novel immune threats within the lifespan of a single organism. That mechanism is the adaptive immune system.

% Need to rapidly focus onto B-cell immunity in particular, I know nothing of T-cells

The mechanisms used to generate this sequence diversity in the B-cell population are themselves diverse, and have been discovered progressively over the past decades. The most fundamental, and well-known, such mechanism is so-called V(D)J recombination, first discovered quite some time ago. %Find out more about history of this

In V(D)J recombination, a number of

A canonical immunoglobulin heavy chain (IgH) gene locus consists of clusters of variable (V), diversity (D) and joining (J) regions in series, followed by some number of larger constant-region exons. During B-cell maturation, a single V, D and J region are selected and the intervening DNA regions are excised to produce a single, contiguous VDJ sequence. As part of this process, nontemplated nucleotides are inserted and deleted at the V/D and D/J junctions, a process known as junctional diversification. Following transcription, the sequence between the VDJ sequence and the first constant-region exon is removed by splicing to produce a mature IgH transcript, with its characteristig VDJC sequence structure.

The manner of constant region selection in B-cells differs significantly between mammals and teleosts. In the former, a number of distinct constant regions are present in series on the chromo


\section{The African turquoise killifish as a model for vertebrate ageing}

The genus \textit{Nothobranchius} comprises a broad group of killifish species distributed across equatorial and subequatorial Africa \citep{valdesalici2003lifespan}, with diversity concentrated in the south-east of the continent. Members of this genus 

The African turquoise killifish (\textit{Nothobranchius furzeri}) is a small, freshwater teleost fish from southeastern Africa, with 

Like other members of the genus \textit{Nothobranchius}, \textit{N. furzeri} is an annual fish, inhabiting transient ponds and rivers which fill during the rainy season and dry rapidly thereafter. In order to survive in this harsh and unforgiving environment, the annual African killifishes have developed a highly sophisticated reproductive cycle, with embryos surviving between rainy seasons in a desiccated diapause state. The very high extrinsic mortality implied by this extreme lifestyle exerts strong selective pressure for rapid growth and development, while rapidly deprecating the selective value of later ages. Consequently, fish from the genus \textit{Nothobranchius} have been known for several decades to exhibit rapid sexual maturation and highly limited intrinsic lifespans, with many species living for less than a year in aquarium conditions. 

The development of the turquoise killifish \textbf{N. furzeri} began over a decade ago, when a captive line of this species was found to have an extremely short lifespan even by the standards of this short-lived genus: initial experiments suggested a median laboratory lifespan of just 12 weeks. Despite this short lifespan, the turquoise killifish was found to show relatively normal (albeit accelerated) kinetics of mortality over its lifespan, and to respond to interventions such as resveratrol treatment or dietary restriction that have long been known to increase lifespan in diverse animal models. 

Later research on the turquoise killifish found evidence of broad-spectrum functional decline with age in a wide variety of metrics and systems, including increases in cancer, neurodegeneration, lipofuscin accumulation, and kidney pathology % add and cite more here as found
, with accompanying loss of pigmentation, spontaneous locomotion, reproductive output and the diversity of the intestinal microbiota. This simultaneous, broad-spectrum decline in a wide range of systems indicates that the short lifespan of the turquoise killifish is due not to the localised failure of a particular organ or system, but to a general ageing phenotype, albeit one carried out at an extremely rapid pace. This further supported the potential value of the species as a model for ageing research, especially into those parts of the ageing process that are specific to vertebrate systems.

Unlike other short-lived ageing models (most prominently \textit{Drosophila melanogaster} and \textit{Caenorhabditis elegans}), the turquoise killifish possesses many vertebrate-specific adaptations which play an important role in the human ageing process, including bone, blood, and a highly-diverse intestinal microbiota. Crucially, they also possess a broadly mammal-like adaptive immune system, including many specialised immune cell types that are only found in vertebrates.

The use of the turquoise killifish as an ageing model has exploded in recent years, with many more research groups making use of the model. This has led to the development of many of the tools and techniques required in a canonical model organism, including high-quality genome and transcriptome assemblies, a variety of successful transgenesis tools, and standardised protocols for feeding and husbandry. Many of these tools only have been published since the outset of this project, with the consequence that some of them were unavailable for parts of the work presented here and available -- and used -- for other parts. Most importantly, the turquoise killifish genome has been through a number of revisions and republications since the first publications appeared. This progressive improvement in available tools and techniques is par for the course in any work developing a new model system; the work presented here itself prevents tools and techniques previously unavailable to the killifish community, which can now be employed more widely in further research. Nevertheless, it explains why some parts of the work presented here, especially that on the assembly of the killifish immunoglobulin loci, went through so many iterations and revisions before completion.

