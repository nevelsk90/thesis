\chapter*{Conclusion}  
\label{chap:conclusion}
\addcontentsline{toc}{chapter}{Conclusion}
\onehalfspacing
%\pagebreak

One of the most remarkable features of the adaptive immune system is its sheer proteanism: at all levels of vertebrate biology and evolution, it is constantly changing, shifting in response to both stochastic processes and the adaptive pressures imposed by pathogens, internal selection mechanisms and the rest of the immune system. At the level of a single lymphocyte clone, the dynamic processes of VDJ recombination and junctional diversity give rise to a controlled yet hugely variable diversity of gene sequences, while affinity maturation results in repeated flowerings of even more sequence diversity from single \naive B-cell ancestors (\Cref{sec:intro_antibodies}). At the level of the organism, the constant production of new antigen-receptor sequences combined with the rapid evolution of lymphocyte lineages gives rise to a constantly fluctuating population of breathtaking complexity, exhibiting both a huge \naive sequence diversity and constantly-accumulating layers of long-term immune memory. Over the entirety of an individual's lifespan, the effects of ageing add additional levels of change to this already-mercurial population, resulting in immune repertoires in old age which differ systematically from those expressed in the young (\Cref{sec:intro_immunosenescence,sec:igseq_ageing}). And at the level of a species lineage, the rapid genomic evolution of antigen-receptor genes results in still more variability, with closely-related species and even conspecific individuals \parencite{corcoran2016igdiscover} differing in their locus organisation, their \vh segments, and even the constant regions they possess (\Cref{chap:locus}).

In this thesis, I have made use of an important emerging model organism, the turquoise killifish (\nfu), and its close relatives to investigate several of these levels of variabilty in one of the most important and diverse antigen-receptor genes in the adaptive immune system, the immunoglobulin heavy chain (\igh{}). I have shown that this gene is highly variable in its composition and organisation in the Cyprinidontiform lineage of teleost fishes, with large differences in its sublocus organisation and V/D/J segments and even repeated gains and losses of an important antibody isoform, the mucosally-specialised \igh{Z} (\Cref{chap:locus}). Focusing on the turquoise killifish itself, I have shown that this species exhibits an intact and highly diverse heavy-chain repertoire, with thousands of unique clones and billions of potential unique antibody sequences per individual and highly individualised V/J segment usage (\Cref{sec:igseq_pilot}). Consistent with the extremely short lifespan and rapid ageing phenotypes of the turquoise killifish, its secondary antibody repertoire exhibits a rapid age-associated decline in diversity over a timescale of weeks, both in the whole-body repertoire and the specialised mucosal repertoire of the gut, as well as increased divergence in repertoire composition between older individuals (\Cref{sec:igseq_ageing,sec:igseq_gut}). These changes, however, appear to be concentrated in the expanded clones of the repertoire, with only non-significant declines in diversity observed in the small, \naive clones that make up most of the clonal richness of the repertoire. Whether this distinction represents a genuine preservation of primary-repertoire diversity with age, a confounding of falling primary diversity by increasing body size, or some other phenomenon remains to be seen; in my opinion, any of these cases would prove to be an interesting discovery.

In \Cref{chap:intro}, I described the objective of this thesis as to ``establish the turquoise killifish as a model for the study of comparative immunology and humoral adaptive immunosenescence''. My goal in these pages has always been to start something new, not finish it. To my knowledge, no previously-published research has ever investigated \igh{} locus organisation in so many closely-related species (\Cref{sec:locus_comparative}), characterised antibody-repertoire ageing in such a short-lived species (\Cref{sec:igseq_ageing}), or looked at repertoire ageing specifically in a mucosal immune organ (\Cref{sec:igseq_gut}). In all cases, however, I've only scratched the surface of what can be learned with these models, methods and mindset. Despite everything that we've learned about the adaptive immune system and its ageing, the sheer distributed complexity of both immunity and ageing makes really understanding the intersection of the two a daunting task. My hope is that the work I've done here can contribute to a better understanding of this intersection, not just in humans and mice but in all organisms possessing this unique and truly remarkable suite of adaptations we call adaptive immunity. 