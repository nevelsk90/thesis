\newpage
\section{Computational and analytic methods}
\label{sec:methods_comp}

\subsection{General data processing and pipeline structure}
\label{sec:methods_comp_general}

Unless otherwise specified, processing and analysis of biological data was performed using standard Bioconductor \parencite{huber2015bioconductor} packages: \program[R]{Biostrings} \parencite{pages2017biostrings} and \program[R]{BSgenome} \parencite{pages2018bsgenome} for biological sequence data, \program[R]{GenomicRanges} \parencite{lawrence2013genomicranges} for sequence ranges, and \program[R]{genbankr} \parencite{becker2018genbankr} and \program[R]{rentrez} \parencite{winter2017rentrez} for GenBank files. 

Smith-Waterman and Needleman-Wunsch exhaustive alignments \parencite{needleman1970alignment,waterman1981alignment} were performed using the \snippet{pairwiseAlignment} function from \program[R]{Biostrings}; percentage sequence identities were computed using the \snippet{pid} function from the same package.

Processing of tabular data was performed using the Tidyverse suite of tools, especially \program[R]{readr} \parencite{wickham2018readr}, \program[R]{dplyr} \parencite{wickham2018dplyr}, \program[R]{tidyr} \parencite{wickham2018tidyr} and \program[R]{stringr} \parencite{wickham2018stringr}. \program{snakemake} \parencite{koster2012snakemake} was used to design and run data-processing pipelines.

\subsection{Data visualisation}
\label{sec:methods_comp_visualisation}

Unless otherwise specified, data were visualised using \program[R]{ggplot2} \parencite{wickham2016ggplot2}. Chromosome ideograms, locus structure visualisations, and sashimi plots were constructed using \program[R]{Gviz} \parencite{hahne2016gviz}. Cluster dendrograms and phylogenetic trees were drawn with \program[R]{ggtree} \parencite{guangchuang2018ggtree}, using utilities from \program[R]{ape} \parencite{paradis2018ape} and \program[R]{tidytree} \parencite{guangchuang2018tidytree}. Sequence logos were drawn with \program[R]{ggseqlogo} \parencite{wagih2017ggseqlogo}. 

\subsection{BAC insert assembly}
\label{sec:methods_comp_bacs}

\subsubsection{Identifying BAC candidates for the \nfu \igh{} locus}
\label{sec:methods_comp_bacs_ident}

The first group of candidate BAC clones to be used in the \Nfu locus assembly was identified by searching for scaffolds in a previous assembly of the \Nfu genome (\texttt{NotFur1}, GenBank accession GCA\_000878545.1 \parencite{valenzano2015genome}) that contained either \textit{IGH} gene fragments (\texttt{GapFilledScaffold\_8761}, \texttt{8571}, \texttt{16121}) or genes homologous to those flanking the \textit{IGH} locus in stickleback and medaka (\texttt{GapFilledScaffold\_2443}, \texttt{292}). Subsequences from these scaffolds were sent to Kathrin Reichwald at the FLI in Jena, who identified four BAC clones (193A03, 276N03, 209K12, 181N10) with sequenced ends close to the query sequences.

Following sequencing and assembly of these BAC inserts, a further group of BACs was identified using a second, independent genome assembly (GenBank accession 	GCA\_001465895.2, \parencite{reichwald2015genome}) and the database of BAC end sequences, which by then were publicly available. The assembled BAC sequences were found to map within or near a large, gapped region on synteny group 3 of this genome assembly, and BACs were selected that either intruded into this gapped region or had end sequences that mapped to another scaffold aligning to the assembled BAC inserts (scaffold01427, scaffold02214, scaffold01820). In total, 11 further BACs were sequenced and assembled in this second round (223M21, 162F04, 220O06, 248A22, 165M01, 206K13, 154G24, 208A08, 277J10, 109B21, 216D12).

\subsubsection{Sequence trimming, filtering and correction}
\label{sec:methods_comp_bacs_trim}

Demultiplexed, adaptor-trimmed MiSeq sequencing data were uploaded by the sequencing provider to Illumina BaseSpace and accessed via \program{basemount}. Reads from each library were trimmed with \lstinline{Trimmomatic} \parencite{bolger2014trimmomatic} to remove adaptor sequences, trim low-quality sequence, and discard any trimmed reads below a minimum lengh:

\begin{lstlisting}
trimmomatic PE -phred33 <forward_reads_fastq> <reverse_reads_fastq> <output_paths> ILLUMINACLIP:<adaptor_directory>/TruSeq3-PE.fa:2:30:10 LEADING:20 TRAILING:20 SLIDINGWINDOW:4:30 MINLEN:36
\end{lstlisting}

Following this, the trimmed reads were filtered to remove \textit{E. coli} genomic DNA and other contaminants by aligning them using \lstinline{Bowtie2} \parencite{langmead2012bowtie2} and retaining read pairs that did not align concordantly:

\begin{lstlisting}
bowtie2 --very-sensitive-local --local --reorder --un-conc <output_prefix> -x <ecoli_genome_index_path> -1 <forward_reads_fastq> -2 <reverse_reads_fastq> -S <sam_file_prefix>
\end{lstlisting}

% TODO: Note about incomplete filtering?

Before sequence assembly, the filtered reads then underwent correction, to reduce the impact of errors occurring during the library preparation and sequencing process. In order to increase the reliability of the resulting scaffolds and reduce the impact of ideosyncracies of any given correction tool, the reads were corrected in parallel using two different programs; \program{QuorUM} \parencite{marcais2015quorum}:

\begin{lstlisting}
quorum -d -q "33" -p <output_path> <interleaved_reads_files>
\end{lstlisting}

\noindent and \program{BayesHammer} (the built-in correction tool of the \program{SPAdes} genome-assembly software \parencite{bankevich2012spades,nikolenko2013bayeshammer}):

\begin{lstlisting}
spades.py -1 <forward_reads_fastq> -2 <reverse_reads_fastq> -o <output_path> --disable-gzip-output --only-error-correction --careful --cov-cutoff auto -k 21,33,55,77,99,127 --phred-offset 33
\end{lstlisting}

\subsubsection{Sequence assembly and scaffolding}
\label{sec:methods_comp_bacs_assembly}

Each pair of independently-corrected reads files was then passed to \program{SPAdes} \parencite{bankevich2012spades} for \textit{de novo} genome assembly:

\begin{lstlisting}
spades.py -1 <forward_reads_fastq> -2 <reverse_reads_fastq> -o <output_path> --disable-gzip-output --only-assembler --careful --cov-cutoff auto -k 21,33,55,77,99,127 --phred-offset 33
\end{lstlisting}

Following assembly, any \textit{E. coli} scaffolds resulting from residual contaminating reads were identified by aligning scaffolds to the \textit{E. coli} genome using \program{BLASTN} \parencite{altschul1990blast,altschul1997blast}, and scaffolds containing significant matches were discarded. The remaining scaffolds were then scaffolded using \program{SSPACE} \parencite{boetzer2011sspace}, using jumping libraries from the Stanford \parencite{valenzano2015genome} and Jena \parencite{reichwald2015genome} killifish genome assemblies: % TODO: Get genome accessions

\begin{lstlisting}
SSPACE_Standard_v3.0.pl -x 0 -k 5 -a 0.7 -n 15 -z 200 -g 1 -p 0 -l <jumping_library_config_file> -s <spades_scaffolds_file>
\end{lstlisting}

In order to guarantee the reliability of the assembled scaffolds, the assemblies produced with \program{BayesHammer}- and \program{QuorUM}-corrected reads were compared, and scaffolds were broken into segments whose contiguity was agreed on between both assemblies. To integrate these fragments into a contiguous insert assembly, points of agreement between BAC assemblies from the same genomic region (e.g. two scaffolds from one assembly aligning concordantly to one scaffold from another) and between BAC assemblies and genome scaffolds, were used to combine scaffolds where possible. Any still-unconnected scaffolds were assembled together through pairwise end-to-end PCR (\Cref{sec:methods_molec_pcr}, with one primer each on the end of each scaffold) and Sanger sequencing (Eurofins).

\subsection{Locus characterisation and assembly}
\label{sec:methods_comp_locus}

\subsubsection{Collating reference sequences}
\label{sec:methods_comp_locus_reference}

\newabbreviation{HSP}{HSP}{high-scoring segment pair (\program{BLAST})}

Most publications presenting characterisations of \igh{} loci do not provide easy-to-use databases of trimmed and curated gene segments, and the data that is available is often partial and heterogeneous between publications. In order to obtain standardised databases, further analysis was performed on publically-available data from three reference species with previously-characterised \igh{} loci: medaka (\species{Oryzias}{latipes}) \parencite{magadan2011medaka}, zebrafish (\species{Danio}{rerio}) \parencite{danilova2005zebrafish} and three-spined stickleback (\species{Gasterosteus}{aculeatus}) \parencite{bao2010stickleback,gambondeza2011stickleback}, as described below. Following automatic sequence extraction, the reference sequences were checked manually for any severely pathological (e.g. out-of-frame) sequences and edited before being used for inference in novel loci.

\subsubsubsection{Medaka} % TODO: Disable indent after subsubsub

GenBank files of the annotated medaka \igh{} locus were downloaded from the supplementary information of the medaka locus paper (\parencite{magadan2011medaka}, additional file 6) and corrected to make them parsable by \program[R]{genbankr}. Locus sequence and annotation ranges were extracted from these GenBank files into \fmt{FASTA} and tab-separated tabular formats, respectively, and segment annotations were renamed to match the naming conventions used in other species. \vh, \dh, \jh and constant-region exon nucleotide sequences were extracted from the locus sequence using these annotations. Amino-acid sequences for \vh, \jh and constant-region sequences were obtained automatically by identifying the reading frames which minimised the number of STOP codons in each sequence.

\subsubsubsection{Stickleback}

Limited sequence information on the \igh{} locus in stickleback, including \vh segments and bulk (non-exon-separated) constant regions was provided in a GenBank file in the locus characterisation paper for medaka (\parencite{magadan2011medaka}, additional file 6), while additional sequence information (including \dh and \jh nucleic-acid sequences and amino-acid sequences of constant-region exons) was extracted manually from one of the stickleback locus papers (\parencite{bao2010stickleback},  Figure S1 to S4) into \fmt{FASTA} files. As with medaka, the GenBank reference file was downloaded, corrected and parsed to yield a \fmt{FASTA} file of the locus sequence and tab-separated tabular files of annotation ranges. \vh sequences were extracted from the locus sequence using these annotation ranges and translated as specified for medaka above; \jh sequences provided by \parencite{bao2010stickleback} were translated such that the final nucleotide formed the last position of the final codon.

To obtain nucleic-acid sequences of the constant-region exons, the amino-acid sequences from \parencite{bao2010stickleback} were aligned to the locus sequence with \program{TBLASTN} \parencite{gertz2006tblastn}, with a query coverage threshold of 40\% and a maximum of three HSPs per query sequence:

\begin{lstlisting}
tblastn -query <ch_aa_fasta> -subject <gac_locus_fasta> -qcov_hsp_perc 40 -max_hsps 3 -outfmt '<output_format>' > <output_path>
\end{lstlisting}

\noindent with the following standardised tabular output format: 

\begin{lstlisting}
6 qseqid sseqid pident qcovhsp length mismatch gapopen gaps sstrand qstart qend sstart send evalue bitscore qlen slen
\end{lstlisting}

To filter out alignments across subloci, any alignment of an exon upstream of the annotated boundaries of its corresponding bulk constant region (whose ranges were specified in the GenBank file) was discarded; the alignment with the highest score for each exon was then used to extract the corresponding nucleic-acid sequence from the locus. In order to control for any errors, either during manual extraction of locus sequences from the paper or in the paper itself, these nucleic-acid sequences were then re-translated to generate new amino-acid sequences, again using the translation frame producing the fewest STOP codons; these sequences were then used in place of the reference files in downstream applications.

\subsubsubsection{Zebrafish}

GenBank files corresponding to the zebrafish \igh{} locus were provided (without segment annotations) on GenBank by \parencite{danilova2005zebrafish}; this publication also provided detailed co-ordinates for the \vh, \dh and \jh segments (but not constant exons) on these sequences. Aligned amino-acid sequences were provided for the exons of \igh{M} and \igh{Z}, but no detailed information about \igh{D} exons could be found for these sequences; as a result, reference information about \igh{D} was not used from this species.

As with stickleback, the amino-acid sequences provided were aligned to the locus sequences  with \program{TBLASTN} to identify and extract exon nucleic-acid sequences, which were then translated using the frame yielding the fewest STOP codons for each sequence. \vh sequences were obtained using the ranges provided in \parencite{danilova2005zebrafish} and translated in the same manner. \dh and \jh nucleotide sequences were obtained directly from \parencite{danilova2005zebrafish}; as with stickleback, \jh amino-acid sequences were obtained by translating the nucleotide sequences in the frame such that the final nucleotide formed the last position of the final codon.

\subsubsection{Identifying putative locus sequences}
\label{sec:methods_comp_locus_scaffolds}

\newabbreviation{E-value}{E-value}{Expect value (\program{BLAST})}
\newabbreviation{kb}{kb}{kilobase(s)}

In order to identify sequences in a genome assembly potentially containing part of an \igh{} locus, reference \vh, \jh and constant-region nucleotide and amino-acid sequences were mapped to the assembly using \program{BLAST} \parencite{altschul1990blast,altschul1997blast}. Nucleotide sequences were aligned to the locus using the relatively permissive \program{blastn} algorithm (as opposed to e.g. \program{megablast} or \program{dc-megablast}):

\begin{lstlisting}
blastn -tastk blastn -query <reference_exon_fasta> -subject <locus_fasta> -outfmt "<output_format>"
\end{lstlisting}

Protein sequences, meanwhile, were aligned using the standard \program{blastp} algorithm:

\begin{lstlisting}
blastp -query <reference_exon_fasta> -subject <locus_fasta> -outfmt '<output_format>'
\end{lstlisting}

In both cases, the same tabular output format specified in \Cref{sec:methods_comp_locus_reference} was used, to provide a predictable format for downstream processing of \program{BLAST} alignment tables.

Following alignment of reference sequences, overlapping alignments to reference segments of the same segment type, isotype (if applicable) and exon number (if applicable) were collapsed together, keeping track of the number of collapsed alignments and the best E-values and bitscores obtained for each alignment group. Alignment groups with a very poor maximum E-value ($> 0.001$) were discarded, as were groups consisting of fewer than two alignments and groups overlapping with a much better alignments to a different sequence type, where ``much better" was defined as a bitscore difference of at least 33. Following resolution of conflicts, \vh and \ch alignments underwent a second filtering step of increased stringency, requiring a minimum E-value of $10^{-10}$ to be retained. 

Following alignment filtering, scaffolds containing surviving alignments to at least two distinct segment types (where \vh, \jh, and each type of constant-region exon each counted as one segment type), or alignments to one segment type covering at least 1\% of the scaffold's total length were retained as potential locus scaffolds. To reduce computational runtime spent processing irrelevant sequence, each candidate scaffold so identified was trimmed to 100kb before the first putative gene segment and 100kb after the last one; in the case of \nfu and \xma, these ranges were further reduced following more thorough segment characterisation (\Cref{sec:methods_comp_locus_segments}).

The exact reference sequences used for this extraction process differed depending on the genome being analysed. For \nfu (\Cref{sec:nfu-locus}), the reference sequences extracted from medaka, stickleback and zebrafish (\Cref{sec:methods_comp_locus_reference}) were used; for \xma (\Cref{sec:xma-locus}), gene segments inferred for \Nfu were also included; and for other species (\Cref{sec:locus_comparative}), the reference sequences plus those inferred for both \Nfu and \Xma were used.

\subsubsection{Locus sequence finalisation}
\label{sec:methods_comp_locus_final}

In the case of both \nfu and \xma, a single chromosome (chromosome 6 in \Nfu, chromosome 16 in \Xma) was identified as bearing the \igh{} locus in that species. In the case of \Xma, this was the only segment-bearing scaffold identified in the genome, and the completed locus sequence was obtained by simply trimming the chromosomal sequence at either end of the segment-bearing region. In contrast, multiple scaffolds from the \Nfu genome were also identified as bearing at least one potential \igh{} segment (\Cref{tab:nfu-locus-scaffolds}). In order to identify which of these were in fact part of the locus and integrate them into a contiguous sequence, BAC candidates identified and assembled as described in \Cref{sec:methods_comp_bacs} were incorporated into the assembly.

To do this, all assembled BAC inserts were screened for \igh{} locus segments in the same manner described for genome scaffolds (\Cref{sec:methods_comp_locus_scaffolds}). Passing BACs (\Cref{tab:nfu-locus-bacs}) were aligned to the candidate genome scaffolds with \program{BLASTN} and integrated manually together, giving priority in the event of a sequence conflict to (i) any sequence containing a gene segment missing from the other, and (ii) the genome scaffold sequence if neither sequence contained such a segment. BACs and scaffolds which could not be integrated into the locus sequence in this way were discarded as orphons.

\subsubsection{Locus segment characterisation}
\label{sec:methods_comp_locus_segments}

\newabbreviation{HMM}{HMM}{Hidden Markov Model}
\newabbreviation[sort=X. maculatus]{xma}{\Xma}{\xma}
\newabbreviation{SA}{SA}{suffix array}

Detailed characterisation of \igh{} gene segments was performed on finished \igh{} locus sequences for \xma and \nfu, and on isolated candidate scaffolds for other species, using the same reference segment databases used to identify candidate scaffolds for that species in \Cref{sec:methods_comp_locus_scaffolds}. % For Xma they are the same.
The specific methods used depended on segment type.

\subsubsubsection{\vh}

To identify \vh segments on newly characterised loci, reference \vh segments were used to construct a multiple-sequence alignment with \program{PRANK} \parencite{loytynoja2014prank}:

\begin{lstlisting}[language=bash]
prank -d=<reference_vh_db> -o=<output_path> -gaprate=0.00001 -gapext=0.00001 -F -termgap
\end{lstlisting}

The resulting alignment was used as an input to \program{NHMMER} \parencite{wheeler2013nhmmer,eddy2011hmm,eddy2009homology,eddy2008alignment}, which constructs a Hidden Markov Model from a multiple-sequence alignment and uses it to identify matching sequences in a reference sequence:

\begin{lstlisting}[language=bash]
nhmmer --dna --notextw --tblout <output_path> -T 80 <vh_alignment> <locus_sequence_path>
\end{lstlisting}

The resulting match table was used to identify candidate ranges in the locus sequence corresponding to \vh segments; these ranges were extended by 9bp at either end to account for boundary errors, and the corresponding nucleotide sequences were extracted to a FASTA file. Each sequence was then checked and refined manually: 3' ends were identified by the start of the RSS heptamer sequence (consensus \sequence{CACAGTG}), if present, while 5' ends and FR/CDR boundaries were identified using IMGT/DomainGapAlign \parencite{ehrenmann2011domaingapalign} with the default settings. Where necessary, IMGT/DomainGapAlign was also used to IMGT-gap the \vh segments in accordance with the IMGT unique numbering \parencite{lefranc2003vnumbering}.

An initial amino-acid sequence for each \vh segment was produced automatically from the extracted nucleotide sequence by identifying the reading frame which minimised the number of STOP codons in the sequence; this worked well for most segments. \vh amino-acid sequences were then refined (and in a few cases re-translated) from the manually-refined nucleotide sequences, including end-refinement and FR/CDR boundary identification.

Following extraction and manual curation, \vh segments were grouped into families based on their pairwise sequence identity. In order to assign segments to families, the nucleotide sequence of each \vh segment in a locus was aligned to each other segment using Needleman-Wunsch global alignment, and the resulting matrix of pairwise sequence identities was used to perform single-linkage heirarchical clustering on the \vh segments, and the resulting dendrogram was cut at 80\% sequence identity to obtain \vh families. These families were then numbered based on the order of the first-occurring \vh segment from that family in the first-occurring sublocus in the parent locus, and each \vh segment was named based on its parent sublocus, its family, and its order among elements of that family in that sublocus.

\subsubsubsection{\jh}

As with \vh segments, \jh segments were identified by building a multiple-sequence alignment with \program{PRANK} and using it to construct an HMM with \program{nhmmer}; the parameters used were the same as for \vh segments, except that there was no minimum score for \program{nhmmer} to report a sequence match (\snippet{-T 0} instead of \snippet{-T 80}). The resulting sequence ranges were extended by \bp{20} on either end and extracted into FASTA format. These sequences were then trimmed automatically by identifying the RSS heptamer sequence at the 5' end and the splice junction motif (\texttt{GTA}) at the 3' end. The \jh nucleotide sequences were then checked and refined manually.

\begin{wraptable}{r}{5.5cm}
\caption{Regex patterns used to search for conserved W118 residues in \jh sequences during AUX file generation}\label{tab:jh-aux-patterns}
\begin{tabular}{r>{\ttseries}l}\toprule  
\# & Pattern \\\midrule
1 & TGGGBNNNNGBN\\
2 & TGGGBNNNGBN\\
3 & TGGGBNNNNNGBN\\
4 & TGGGBNNNNNNGBN\\
5 & TGGGBN\\\bottomrule
\end{tabular}
\end{wraptable}

\program{IgBLAST} \parencite{ye2013igblast} identifies CDR3 boundaries for recombined IGH VDJ sequences using an AUX file specifying the reading frame of each \jh segment, along with the co-ordinate of the conserved \texttt{TGG} codon (corresponding to the conserved W118 residue in the recombined sequence \parencite{lefranc2014immunoglobulins}) marking the CDR3/FR4 boundary. An AUX file for the inferred \jh segments was generated automatically by searching for the conserved sequence using a series of regular-expression patterns of decreasing stringency (\Cref{tab:jh-aux-patterns}), taking the first match in each sequence as the desired residue; this determined both the reading frame and the W118 sequence co-ordinate. Once generated, the AUX file was then used to determine the reading frame for automatically translated the \jh sequences; both the AUX file and amino-acid FASTA file were then edited to incorporate any manual refinements made to the \jh nucleotide sequences.

Curated \jh sequences were named based on their order within their parent sublocus and, where applicable, on whether they were upstream of IGHZ or IGHM constant regions. 

\subsubsubsection{\dh}

Unlike \vh and \jh gene segments, \dh segments are too short and variable to be found effectively using an HMM-based search strategy. Instead, \dh segments in assembled loci were located using their distinctive pattern of flanking recombination signal sequences: an antisense RSS in 5', then a short D-segment, then a sense RSS in 3'. Potential matches to this pattern were searched for using \program{FUZZNUC} from the EMBOSS collection of bioinformatics tools \parencite{rice2000emboss}, with a high error tolerance to account for deviations from the conserved sequence in either or both of the RSSs:

\begin{lstlisting}
fuzznuc -pattern 'GGTTTTTGTN(10,14)CACTGTGN(1,25)CACAGTGN(10,14)ACAAAAACC' -pmismatch 8 -rformat gff -outfile <output_path> <locus_sequence_path>
\end{lstlisting}

This generated a GFF file \parencite{stein2010generic} of permissive matches, representing potential \dh segments; these were then grouped by sequence co-ordinate, and higher-mismatch candidates overlapping with a lower-mismatch alternative were discarded.

Orientation of \dh segments based on their own sequence is challenging, as the segments themselves have no clear conserved structure and the flanking RSSs are rotationally symmetric. To overcome this problem and orientate the \dh segments on the locus, the table of \dh candidate ranges was combined with previously-identified (and easier to orientate) \vh and \jh ranges. Each \dh candidate was then orientated based on the orientations of its flanking segments: segments with an oriented segment immediately upstream or downstream adopted the orientation of that segment, while segments with contradictory orientation information were discarded. This process was repeated until all \dh segments had either been orientated or discarded.

After orientation, the \dh ranges were used to extract \dh sequences in FASTA format from the locus sequence; these sequences then underwent a second, more stringent filtering step, in which sequences lacking the most conserved positions in each RSS were discarded \parencite{grep}:
% TODO: Citation for RSS consensus sequences

\begin{lstlisting}
grep -B 1 '[ACTG]\{{25,27\}}TG[ACTG]\{{1,25\}}CA[ACTG]\{{25,27\}}' <dh_fasta> | sed '/^--$'/d > <output_fasta>
\end{lstlisting}

Finally, the identified \dh candidates were checked manually, candidates without good RSS sequences were discarded, and flanking RSS sequences were trimmed to obtain \dh segment sequences. As with \jh, these were numbered based on their order within their parent sublocus and, when applicable, on whether they were upstream of IGHZ or IGHM constant regions.

\subsubsubsection{CH}

To detect and identify constant-region exons in the characterised loci, constant-region nucleotide and protein sequences from reference species were mapped to the locus sequence using \program{BLAST} \parencite{altschul1990blast,altschul1997blast}, in the same manner described for putative locus scaffolds in \Cref{sec:methods_comp_locus_scaffolds}.
Following alignment of reference sequences, overlapping alignments to reference segments of the same isotype and exon number were collapsed together, keeping track of the number of collapsed alignments and the best E-values and bitscores obtained for each alignment groups. Alignment groups with a very poor maximum E-value ($> 0.001$) were discarded, as were groups overlapping with a much better alignments to a different isotype or exon type, where ``much better'' was defined as a bitscore difference of at least 16.5. Where conflicting alignments to different isotypes or exon types co-occurred without a sufficiently large difference in bitscore, both alignment groups were retained for manual resolution of exon identity.

Following resolution of conflicts, alignment groups underwent a second filtering step of increased stringency, requiring a minimum E-value of $10^{-8}$ and at least two aligned reference exons over all reference species to be retained. Each surviving alignment group was then converted to a sequence range, extended by \bp{10} at each end to account for truncated alignments failing to cover the ends of the exon, and used to extract the corresponding exon sequence into \format{FASTA} format. These sequences then underwent manual curation to resolve conflicting exon identities, assign exon names and perform initial end refinement based on putative splice junctions.

In order to validate intron/exon boundaries and investigate splicing behaviour among \textit{IGH} constant-region exons in \Nfu and \Xma, published RNA-sequencing data (\Cref{tab:rnaseq-sources}) were aligned to the annotated locus using STAR (\parencite{dobin2013star}, version 2.5.2b). In both cases, reads files from multiple individuals were concatenated and aligned together, in order to make the intron/exon boundary changes in mapping behaviour as clear as possible. % TODO: Add more software versions, or a table at the end

Before aligning the RNA-seq reads, each locus underwent basic repeat masking, using the built-in zebrafish repeat parameters from \program{RepeatMasker} \parencite{smith2016repeatmasker}:

\begin{lstlisting}[language=bash]
RepeatMasker -species danio -dir <masked_locus_dir> -s <unmasked_locus_path>
\end{lstlisting}

\noindent After masking, a \program{STAR} genome index was generated from each locus:

\begin{lstlisting}[language=bash]
STAR --runMode genomeGenerate --genomeDir <star_index_directory_path> --genomeFastaFiles <masked_locus_path> --genomeSAindexNbases <sa_index>
\end{lstlisting}

\noindent where the \snippet{--genomeSAindexNbases} option determines the size of the suffix-array index and is dependent on the length of the reference sequence being indexed : 

\begin{equation}
\mathrm{SA~index~size~(bits)} = \frac{\log_2(\mathrm{length~of~reference~sequence})}{2} - 1
\label{eq:sa_index}
\end{equation}

Following index generation, the RNA-seq reads were mapped to the generated index as follows:

\begin{lstlisting}[language=bash]
STAR --genomeDir <star_index_directory_path> --readFilesIn <input_reads> --outFilterMultimapNmax 5 --alignIntronMax 10000 --alignMatesGapMax 10000
\end{lstlisting}

\noindent where the \snippet{--outFilterMultimapNmax} option excludes read pairs mapping to more than five distinct co-ordinates in the reference sequence and the \snippet{--alignIntronMax} option excludes read pairs spanning predicted introns of more than \kb{10}, and the \snippet{--alignMatesGapMax} option excludes read pairs mapping more than \kb{10} apart. Following alignment, the resulting \format{SAM} files were processed into sorted, indexed \format{BAM} files using \program{SAMtools} \parencite{li2009samtools} and visualised with Integrated Genomics Viewer (IGV, \parencite{robinson2011igv,thorvaldsdottir2013igv}) to determine intron/exon boundaries of predicted exons, as well as the major splice isoforms present in each dataset.

In order to reduce time and memory requirements for generating alignment figures (\Cref{fig:fig:nfu-locus-sashimi,fig:xma-locus-sashimi}), secondary alignments were performed on truncated loci consisting only of the IGHM/D or (where present) IGHZ constant regions, plus a few flanking kilobases on each side. In these cases, the additional parameters constraining multimapping, intron length and mate distance were not necessary due to the much shorter and less repetitive reference sequence.

For species other than \Nfu or \Xma, intron/exon boundaries were predicted manually based on BLASTN and BLASTP alignments to closely-related species and the presence of conserved splice-site motifs (\texttt{AG} at the 5' end of the intron, \texttt{GT} at the 3' end \parencite{shapiro1987splice}). In cases where no 3' splice site was expected to be present (e.g. for CM4 or TM2 exons), the nucleotide exon sequence was terminated at the first canonical polyadenylation site (\texttt{AATAAA} if present, otherwise one of \texttt{ATTAAA}, \texttt{AGTAAA} or \texttt{TATAAA} \parencite{ulitsky2012polya}), while the amino-acid sequence was terminated at the first STOP codon. In many cases, it was not possible to locate a TM2 exon due to its very short (two-amino-acid-residue) conserved coding sequence.

\subsubsection{Synteny analysis}
\label{sec:methods_comp_locus_synteny}

Synteny between subloci in the \Nfu locus was analysed using \program[R]{DECIPHER}'s standard synteny pipeline \parencite{wright2016decipher}, which searches for chains of exact $k$-mer matches within two sequences:

\begin{lstlisting}[language=R]
DBPath <- tempfile()
DBConn <- dbConnect(SQLite(), DBPath)

Seqs2DB(seqs = <sublocus_1_sequence>, type = "XStringSet", dbFile = DBConn, identifier = "IGH1", verbose = FALSE)
Seqs2DB(seqs = <sublocus_2_sequence>, type = "XStringSet", dbFile = DBConn, identifier = "IGH2", verbose = FALSE)

dbDisconnect(DBConn)

SyntenyObject <- FindSynteny(dbFile = DBPath, verbose = FALSE)
\end{lstlisting}

\noindent Cross-locus sequence comparisons between gene segments were performed analogously to the comparisons involved in \vh family assignment, with \snippet{pairwiseAlignment} and \snippet{pid} from \program[R]{Biostrings}.


\subsection{Phylogenetic trees}
\label{sec:methods_comp_trees}


\subsubsection{Species tree construction and annotation}
\label{sec:methods_comp_trees_species}

Information about the interrelationships of most of the teleost taxa discussed in this thesis was obtained from the comprehensive teleost phylogeny of Hughes \textit{et al.} \parencite{hughes2018teleostphylo}, while additional, higher-resolution information on the interrelationships of African killifishes missing from that tree was provided by Cui \textit{et al.} \parencite{cui2019annual}. As no single published tree covered all the species of interest, a simple cladogram of relatioships was constructed manually from the information provided by these two sources. Annotations (e.g. of clade membership or isotype status) were added using \program[R]{tidytree} \parencite{guangchuang2018tidytree}.

\subsubsection{Phylogenetic inference on \igh{} locus sequences}
\label{sec:methods_comp_trees_phylo}

Three phylogenetic trees were inferred from molecular data of \igh{Z} gene segments: one on the \vh segments on \nfu and \xma, one on \ch exons from all species, and one on \igh{Z} constant-regions of \igh{Z} bearing species. In all cases, a sequence \format{FASTA} database was assembled from the relevant species. As identical sequences can cause problems during phylogenetic analysis, entries with completely identical sequences  were then collapsed together into a single \fast{FASTA} sequence, which was relabelled with the names of all its parent sequences. 

A multiple-sequence alignment of the remaining sequences was then constructed with \program{PRANK}:

\begin{lstlisting}
prank -d=<ch_fasta> -o=<output_prefix> DNA -termgap
\end{lstlisting}

The resulting alignment was passed to the maximum-likelihood phylogenetic inference program \lstinline{RAxML} (\parencite{stamatakis2005raxml3,stamatakis2006raxml6,stamatakis2014raxml8}, version 8.2.12), using the SSE3-enabled parallelised version of the software, the standard GTR-Gamma nucleotide substitution model, and built-in rapid bootstrapping:

\begin{lstlisting}
raxmlHPC-PTHREADS-SSE3 -f a -m GTRGAMMA -s <ch_prank_alignment> -w <output_dir> -N <n_bootstrap_replicates> -x <bootstrap_seed> -p <parsimony_seed> -n <output_suffix>
\end{lstlisting} 

Finally, the bootstrap-annotated \snippet{RAxML_bipartitions} file was inspected and rooted manually in \program{Figtree} \parencite{rambaut2012figtree}, before being annotated and visualised in \program{R} with \program[R]{tidytree} and \program[R]{ggtree}, respectively.

\subsubsubsection{\vh-segment tree}

In order to build a phylogenetic tree of \vh segments from the \Nfu and \Xma \igh{} loci, all \vh sequences from those loci were labelled with their origin species and concatenated together. Sequences with more than 25\% missing characters were discarded prior to \program{PRANK} alignment. During tree-inference with \program{RAxML}, 100 bootstrap replicates were used.

\subsubsubsection{\ch exon tree}

To build a phylogenetic tree of \ch exons, nucleotide sequences of constant exons from all species involved in this study were labelled with their origin species and concatenated into a single \format{FASTA} file, which was then filtered to discard transmembrane exons, secretory tails, and sequences with more than 25\% missing characters. In addition, CM4 nucleotide sequences were trimmed to the coding region, removing the 3'-UTR. As with the \vh-segment tree described above, 100 bootstrap replicates were used during tree-inference with \program{RAxML}. As the outgroup among \ch exon groups is unknown, the tree was visualised in unrooted format.

\subsubsubsection{\igh{Z} tree}

To investigate the evolution of \igh{Z}, the \cz{1-4} exons from each \igh{Z} constant region found in any of the analysed genomes were concatenated together into a single sequence and labelled with the source species and constant region. In the event of partial constant regions missing one or more \cz{} exons, the remaining exons were concatenated together in the usual order. Following database processing and alignment, \program{RAxML} tree-inference was run using 1000 bootstrap replicates, in order to increase the reliability and precision of the support values obtained.

\subsection{Ig-Seq data pre-processing}
\label{sec:methods_comp_igpreproc}

Unless otherwise specified, pre-processing utilities used in the following sections are provided by the pRESTO \parencite{vanderheiden2014presto} and Change-O \parencite{gupta2015changeo} suites of Ig-Seq processing tools.

\subsubsection{Sequence uploading and annotation}
\label{sec:methods_comp_igpreproc_annot}

Demultiplexed, adaptor-trimmed MiSeq sequencing data were uploaded by the sequencing provider to Illumina BaseSpace and accessed via \program{basemount}. Library annotation information (fish ID, sex, strain, age at death, death weight, etc.) were added to the read headers of each library:

\begin{lstlisting}
ParseHeaders add -f <field_keys> -u <field_values> -s <input_reads_file>
\end{lstlisting}

User-specified combinations of annotations that together uniquely specified the individual and replicate identity of each library were then combined into a single annotation in addition to the separate source annotations:

\begin{lstlisting}
ParseHeaders.py merge -f <field_keys> -k INDIVIDUAL --act cat -s <annotated_reads_file>
ParseHeaders.py merge -f <field_keys> -k REPLICATE --act cat -s <annotated_reads_file>
\end{lstlisting}

Following annotations, reads from different libraries were pooled together, then split by replicate identity:

\begin{lstlisting}
SplitSeq.py group -s {input} -f REPLICATE s <pooled_reads_file>
\end{lstlisting}

This pooling and re-splitting process enables all reads considered to be a single replicate to be processed together even if sequenced separately, maximising the effectiveness of UMI-based pre-processing while also allowing all replicates to be processed in parallel.

\subsubsection{Read quality control}
\label{sec:methods_comp_igpreproc_filter}


After pooling and re-splitting, the raw read set underwent quality control, discarding any read with an average Phred score of less than 20:
% TODO: Citation needed for Phred score

\begin{lstlisting}
FilterSeq quality -q 20 -s <input_reads_file>
\end{lstlisting}

\subsubsection{Primer masking and UMI extraction}
\label{sec:methods_comp_igpreproc_mask}

\newabbreviation{ID}{ID}{identity}
\newabbreviation{UMI}{UMI}{unique molecular identifier}
\newabbreviation{MIG}{MIG}{molecular identifier group}

Following quality filtering, the reads underwent processing to identify and remove invariant primer sequences. To do this, known primer sequences were aligned to each read from a fixed starting position, and the best match on each read was identified and trimmed. Initially, the primer sequences from the third PCR step of the library prep protocol were used, trimming off primer sequences corresponding to part of the constant \cm{1} exon and the 5' invariant part of the template-switch adapter:

\begin{lstlisting}
MaskPrimers score --mode cut --start 0 -s <3prime_read_file> -p <CM1_primer_file>;
MaskPrimers score --mode cut --start 0 -s <5prime_read_file> -p <TSA_primer_file>
\end{lstlisting}

Following this, the 5' reads underwent a second round of masking using the 3' invariant part of the TSA sequence (\sequence{CTTGGGG}), and the intervening 16 bases were extracted and recorded in each read header as that read's unique molecular identifier (UMI):

\begin{lstlisting}
MaskPrimers score --mode cut --barcode --start 16 --maxerror 0.5 -s <masked_5prime_read_file> -p <TSA_3prime_sequence_file>
\end{lstlisting}

As the match sequence for this second round of masking is shorter and more error-prone than the primer sequences used in the first round, an increased mismatch tolerance was used to increase the number of reads with successfully-extracted UMIs.

\subsubsection{Barcode error handling}
\label{sec:methods_comp_igpreproc_correct}

%The use of UMI sequences enables biases and errors in library insert sequences to be corrected by taking the consensus sequence of all reads sharing a given UMI (a molecular identifier group, or MIG). However, PCR and sequencing errors can also affect the sequence of the UMI itself, in which case reads that in fact belong to a single MIG will be spuriously separated during pre-processing; this can result in spuriously low MIG read counts, spuriously high numbers of unique sequences, and avoidable loss of sequencing data due to reads with erroneous barcodes being discarded (as low-quality, low-read-count unique sequences) at various points in the pre-processing pipeline.
%
%In addition to these barcode \textit{errors}, barcode \textit{collisions} can occur, in which multiple distinct sequences are labelled with the same UMI sequence and spuriously grouped together during UMI grouping. This can lead to spuriously large MIGs and spuriously low numbers of unique sequences, and in extreme cases lead to the rejection and loss of entire MIGs due to an insufficiently high level of sequence identity during consensus generation (see below). % TODO: Move to intro or results

In order to reduce the level of barcode errors in each dataset, primer-masked IgSeq reads underwent barcode clustering, in which reads with the same replicate identity and highly similar UMI sequences were grouped together into the same molecular identifier group (MIG). To do this, 5'-reads were clustered by UMI sequence using  \program{CD-HIT-EST} \parencite{li2006cdhit,fu2012cdhit} with a 90\% sequence identity cutoff, with cluster identities being recorded in a new CLUSTER field in each read header:

\begin{lstlisting}
SplitSeq group -s <masked_reads_file> -f REPLICATE;
ClusterSets barcode -f BARCODE -k CLUSTER --cluster cd-hit-est --prefix B --ident 0.9 -s <split_reads_file>
\end{lstlisting}

In order to split any geniunuly distinct MIGs accidentally united by this process, as well as to reduce the level of ``natural" barcode collisions, the reads then underwent a second round of clustering, this time separately on the read sequences within each barcode cluster using \program{VSEARCH} (an open-source alternative to \program{USEARCH} \parencite{edgar2010usearch,rognes2016vsearch}).
This time, the cluster dendrogram was cut at 75\% total sequence identity, and each subcluster was separated into its own distinct MIG:

\begin{lstlisting}
ClusterSets set -f CLUSTER -k CLUSTER --cluster vsearch --prefix S --ident 0.75 -s <clustered_reads_file>
\end{lstlisting}

These clustering thresholds (90\% for barcode clustering, 75\% for barcode splitting) were identified empirically as the values that maximise the number of reads passing downstream quality checks and included in the final preprocessed dataset.

The cluster annotations from the two clustering steps were combined into a single annotation uniquely identifying each MIG in each replicate. These annotations were further modified to designate the replicate identity of each read, giving each MIG a unique annotation across the entire dataset. These annotations were copied to the 3' reads such that each pair had a matching MIG annotation, and reads without a mate (due to differential processing of the two reads files) were discarded:

\begin{lstlisting}
ParseHeaders collapse -s <clustered_5prime_reads> -f CLUSTER --act cat;
ParseHeaders merge -f REPLICATE CLUSTER -k RCLUSTER --act set -s <annotated_5prime_reads>;
PairSeq -1 <reannotated_5prime_reads> -2 <3prime_reads> --1f BARCODE CLUSTER RCLUSTER --coord illumina
\end{lstlisting}

\subsubsection{Consensus-read generation and pair merging}
\label{sec:methods_comp_igpreproc_consensus}

Following barcode clustering, the 5' and 3' IgSeq reads were separately grouped based on cluster identity, and the reads in each cluster grouping were aligned and collapsed into a consensus read sequence:

\begin{lstlisting}
BuildConsensus --bf RCLUSTER --cf <header_fields> --act <copy_actions> --maxerror 0.1 --maxgap 0.5 -s <annotated_reads_file>
\end{lstlisting}

Positions at which at least half the aligned reads in the MIG had a gap character were deleted from the consensus (\snippet{--maxgap 0.5}), while MIGs with a mismatch rate from the consensus of more than 10\% were discarded from the dataset (\snippet{--maxerror 0.1}). The resulting \format{FASTQ} file contained a single consensus sequence for each cluster annotation, labelled with its CONSCOUNT (the number of reads contributing to that consensus sequence),  the number of reads allocated to each barcode in the cluster, and various header fields (\snippet{<header_fields>}) propagated from the contributing reads by summing or concatenating the values from each contributing read (\snippet{<copy_actions>}). 

After consensus-read generation had been performed for both 5' and 3' reads, the annotations attached to each read were unified across read pairs with matching cluster identities, and consensus reads without a mate of the same cluster identity were dropped:

\begin{lstlisting}
PairSeq -1 <5prime_consensus_reads> -2 <3prime_consensus_reads> -coord presto
\end{lstlisting}

Following consensus-read generation and annotation unification, consensus-read pairs with matching cluster annotations were aligned and merged into a single contiguous sequence. Where possible, this was done by simply aligning the two mate sequences against each other; where this was not possible (e.g. due to the lack of a significant sequence overlap) the consensus reads were instead aligned with \program{BLASTN} \parencite{altschul1990blast,altschul1997blast} to a reference database of \vh sequences to generate a merged sequence, with \sequence{N} characters used to separate pairs that aligned in a non-overlapping manner on the same \vh segment:

\begin{lstlisting}
AssemblePairs sequential --coord presto --scanrev --aligner blastn --rc tail --1f <header_fields> -1 <5prime_consensus_reads> -2 <3prime_consensus_reads> -r <vh_fasta_file>
\end{lstlisting}

In either case, annotation fields were copied to the new merged sequence from the forward consensus read, with the fields to be copied specified by \snippet{--1f <header_fields>}. Sequence pairs for which both alignment approaches failed were discarded.

\subsubsection{Collapsing identical sequences and singleton removal}
\label{sec:methods_comp_igpreproc_collapse}

To quantify the abundance of each unique sequence present in each sample, merged consensus sequences with identical insert sequences but distinct MIG assignments were collapsed together into a single \format{FASTQ} entry, recording the number, size and UMI makeup of contributing MIGs in each case in the sequence header alongside any existing annotation information:

\begin{lstlisting}
CollapseSeq --inner --cf <header_fields> --act <copy_actions> -n 20 -s <merged_consensus_seqs>
\end{lstlisting}

The collapsed sequences from each replicate identity in the dataset were then concatenated into a single file for easier downstream processing. 

As sequences represented by only a single read across all MIGs in the dataset (so-called \textit{singleton} sequences, with a \snippet{CONSCOUNT} of no more than 1) could not be corrected by UMI clustering or consensus building, they are not considered reliable for downstream processing and analysis; as a result, they were here identified and separated from the other collapsed sequences:

\begin{lstlisting}
SplitSeq group -f CONSCOUNT --num 2 -s <collapsed_consensus_seqs>
\end{lstlisting}

Finally, the non-singleton sequences so identified were converted into \format{FASTA} format with \program{seqtk} \parencite{li2016seqtk} for downstream processing:

\begin{lstlisting}
seqtk seq -a <non_singleton_consensus_seqs> > <presto_fasta_output>
\end{lstlisting}

\subsubsection{Assigning VDJ identities with IgBLAST}
\label{sec:methods_comp_igpreproc_igblast}

%The \program{pRESTO} pipeline described in preceding sections converts raw IgSeq reads into corrected, collapsed consensus sequences, with each sequence representing a unique sequence present in the original RNA sample. Before analysing the diversity, structure, or other properties of the sequenced repertoires, further processing is required to assign V/D/J identities, clonotype membership, and germline sequences to each sequence in the dataset. These processing steps were here performed using \program{pRESTO}'s sister program \program{Change-O} and the ... program \program{IgBLAST}. % TODO: Describe and cite IgBLAST; cite Change-O % TODO: Move to results

To assign \vh, \dh and \jh identities to the corrected, collapsed consensus sequences produced by the \program{pRESTO} pipeline, gene segment databases were aligned to the \format{FASTA} output from \Cref{sec:methods_comp_igpreproc_collapse} with \program{IgBLAST} \parencite{ye2013igblast}. To do this, each reference file was converted into a \program{BLAST} database with \snippet{makeblastdb}, and the output \format{FASTA} file was aligned to these databases with \snippet{igblastn}:

\begin{lstlisting}
makeblastdb -parse_seqids -dbtype nucl -in <vh_reference_fasta> -out <vh_db_prefix>;
makeblastdb -parse_seqids -dbtype nucl -in <dh_reference_fasta> -out <dh_db_prefix>;
makeblastdb -parse_seqids -dbtype nucl -in <jh_reference_fasta> -out <jh_db_prefix>;
igblastn -ig_seqtype Ig -domain_system imgt -query <presto_fasta_output> -out <igblast_output> -germline_db_V <vh_db_prefix> -germline_db_D <dh_db_prefix> -germline_db_J <jh_db_prefix> -auxiliary_data <jh_aux_file> -outfmt '7 std qseq sseq btop'
\end{lstlisting}

A \jh auxiliary file was used to indicate the reading frame and CDR3 boundary co-ordinate of each \jh sequence in the reference database (\snippet{-auxiliary_data <jh_aux_file>}); for more information on this file and how it was generated, see \Cref{sec:methods_comp_locus_segments}.

\subsubsection{Clonotype inference with Change-O}
\label{sec:methods_comp_igpreproc_clones}

Following V/D/J identity assignment, the output \format{FASTA} file from \Cref{sec:methods_comp_igpreproc_collapse}, raw reference segment databases from \Cref{sec:methods_comp_locus_segments} and segment assignments from \Cref{sec:methods_comp_igpreproc_igblast} were used to construct a tab-delimited \program{Change-O} sequence database:

\begin{lstlisting}
MakeDb igblast --regions --scores --failed --partial --asis-calls --cdr3 -i <igblast_output> -s <presto_fasta_output> -r <vh_reference_fasta> <dh_reference_fasta> <jh_reference_fasta>
\end{lstlisting}

\noindent where \snippet{--failed} indicates that invalid sequences should be included in a separate database rather than discarded outright, \snippet{--regions} and \snippet{--cdr3} indicate that the database should include FR and CDR annotations, \snippet{--scores} indicates that the database should include alignment score metrics, \snippet{--partial} indicates that sequences with incomplete V/D/J alignments (e.g. those without an unambiguous V- or J-assignment) should not automaticall qualify as failed, and \snippet{--asis-calls} instructs the program to accept assignments to V/D/J databases with non-standard name formatting. Following database construction, each entry was given a unique name on the basis of its replicate identity and ordering. 

In order to compute the appropriate distance threshold for clonotype assignment, each sequence was assigned a nearest-neighbour Hamming distance within the repertoire, using the related \program{R} packages \program[R]{SHazaM} and \program[R]{Alakazam}:

\begin{lstlisting}[language=R]
tab <- readChangeoDb(<named_db_path>) %>% mutate(ROW = seq(n()))
dist_pass <- distToNearest(tab, model = "ham", normalize = "len", fields = <group_field>, first = FALSE)
dist_fail <- tab %>% filter(! ROW %in% dist_pass$ROW) %>%·mutate(DIST_NEAREST = NA) 
writeChangeoDb(bind_rows(dist_pass, dist_fail), <db_output_path>)
\end{lstlisting}

\noindent where \snippet{model = "ham"} indicates that a single-nucleotide Hamming distance metric is to be used, \snippet{normalize = "len"} that distances should be normalised by total sequence length, \snippet{first = FALSE} determines how to handle ambiguous V/J calls, and \snippet{fields = <group_field>} that only distances within an entry group (see below) should be considered.

Following assignment of nearest-neighbour distances, a distance threshold for clonotyping was computed by fitting a pair of unimodal distributions to the nearest-neighbour distribution over all sequences and selecting the threshold that minimises the maximises the average of sensitivity and specificity when assigning a point to one of these distributions (for more information, see \Cref{sec:igseq_pilot_clones} and \parencite{nouri2018threshold}). As with nearest-neighbour distance assignment, this was done in \program{R} using \program[R]{SHazaM} and \program[R]{Alakazam}; all four possible models (fitting either a normal or gamma distribution to each of the two peaks) were tried, and the one with the highest maximum likelihood was used to compute the threshold value:

\begin{lstlisting}[language=R]
tab <- readChangeoDb(<changeo_db_path_with_distances>)
models <- c("gamma-gamma", "gamma-norm", "norm-gamma", "norm-norm")
thresholds <- numeric(length(models))
likelihoods <- numeric(length(models))
for(n in 1:length(models)){
  obj <- tryCatch(findThreshold(as.numeric(tab$DIST_NEAREST), method = "gmm", model = "hmm", cutoff = "opt"), error = function(e) return(e$message), warning = function(w) return(w$message))
  thresholds[n] <- ifelse(isS4(obj), obj@threshold, NA)
  likelihoods[n] <- ifelse(isS4(obj), obj@loglk, NA)
}
if (!all(is.na(thresholds))) write(thresholds[last(which(likelihoods == max(likelihoods, na.rm = TRUE))], <threshold_output_path>)
\end{lstlisting} 

Following inference of the correct distance threshold, clonotype inference was performed on the sequence database by grouping sequences by V- and J-assignment and CDR3 length, computing pairwise Hamming distances between each pair of sequences in each group, and ... : %TODO: Finish describing clonotyping algorithm, briefly here and in more detail elsewhere

%Once the database was constructed, the sequences it contained underwent clonotype assignment, with sequences predicted to have arisen from a common na\"{i}ve B-cell ancestor annotated as belonging to the same B-cell clone. To do this, sequences from the same individual with compatible V- and J-assignments and the same CDR3 length were grouped together, and each of these groups underwent single-linkage clustering based on length-normalised pairwise Hamming distances between sequences: % TODO: Citations for various things in here % TODO: To results

\begin{lstlisting}
DefineClones --act set --model ham --sym min --norm len --failed -d <changeo_db> --dist <cluster_distance_threshold> --gf INDIVIDUAL
\end{lstlisting}

\noindent where \snippet{--act set} tells the program how to handle ambiguous V/D/J assignments, \snippet{--model ham} specifies the clustering metric as the pairwise Hamming distance; \snippet{--norm len} indicates that the Hamming distances should be normalised by sequence length; \snippet{--sym min} specifies that, in the event of asymmetric \texttt{A -> B} and \texttt{B -> A} distances (e.g. arising from length normalisation) the minimum distance should be used; and \snippet{--dist} specifies the distance threshold at which to cut the clustering dendrogram.

Finally, the clonotype numbers assigned to each group % individual? replicate?
were combined with the group ID of each clonotype to give a unique ID for each clonotype in the dataset.

\subsubsection{Germline inference with Change-O}
\label{sec:methods_comp_igpreproc_clones}

After threshold determination and clonotyping, a so-called ``full-length germline sequence" is constructed for each sequence. To do this for a given sequence, germline V/J sequences are trimmed of deleted positions and concatenated together, separated by a masked region of length corresponding to the inserted nucleotides and remaining \dh sequence:

\begin{lstlisting}
CreateGermlines -g dmask --cloned -d <clonotyped_changeo_db> -r <vh_reference_fasta> <dh_reference_fasta> <jh_reference_fasta> --failed
\end{lstlisting}

where \snippet{-g dmask} indicates that \dh sequences should be masked as well as insert sequences and \snippet{--failed} indicates that sequences that fail germline assignment should be retained in a separate database. Importantly, \snippet{--cloned} indicates that sequences from the same clone should recieve the same germline assignment, based on a simple majority rule among sequences in the clone; this process also enables assignment of unambiguous segment identities to ambiguously-assigned sequences within larger clones, improving segment calls in the dataset.

%Finally, the clonotyped and germline-inferred sequence database was split on the basis of functionality, with nonfunctional (frame-shifted, nonsense-mutated, or lacking V/J assignments) sequences moved into a separate database file before downstream analysis:
%
%\begin{lstlisting}
%ParseDb split -f FUNCTIONAL -d <germlined_changeo_db>
%\end{lstlisting} % TODO: Restore once pipeline finalised

% \subsection{Downstream analysis of IgSeq data}
% [or split into multiple sections?]

% ... 

\subsection{Downstream analysis of antibody repertoires}
\label{sec:methods_comp_igdownstream}

% ...

\subsection{Zipf approximation of rank:frequency distributions}
\label{sec:methods_comp_igdownstream_zipf}

\subsection{Repertoire Dissimilarity Index (RDI) measurements}
\label{sec:methods_comp_igdownstream_rdi}

\newabbreviation{RDI}{RDI}{Repertoire Dissimilarity Index}

The Repertoire Dissimilarity Index (RDI) \parencite{bolen2017rdi} is a method for computing a distance between any two repertoires on the basis of their V(D)J segment composition, based on the Euclidean distance between their respective V(D)J-abundance vectors. As such, it provides a pairwise distance metric which can be used to compare, cluster and visualise the relationship between different antibody repertoires.

\subsection{Computing diversity spectra}
\label{sec:methods_comp_igdownstream_spectra}



