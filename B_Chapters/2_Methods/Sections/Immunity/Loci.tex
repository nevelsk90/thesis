\section{The structure of immunoglobulin gene loci}

The sequences of the immunoglobulin gene loci in a species specify the baseline features of antibody diversification and function in that species, including the different antibody classes available, the populations of V, D and J segments available for use in V(D)J recombination, and the mechanisms available for selecting among those antibody classes present. 

\subsection{Immunoglobulin heavy chain (IgH) loci in teleosts}

\section{Heavy-chain constant regions in teleosts} % And mammals?

To date, three heavy-chain isotypes have been discovered in the \textit{IgH} loci of teleost fish: IgM, IgD and IgT (a.k.a. IgZ) \citep{fillatreau2013astonishing}. % Also ZF and trout papers
IgM and IgD are primitive isotypes found in most groups of jawed vertebrates, while IgT/Z is teleost-specific. Several important mammalian isotypes, including IgA, IgG and IgE, are absent in teleosts, as is the mechanism of AID-mediated class-switching used by mammalian B-cells to transition between the numerous classes and subclasses often available in mammalian IgH loci. As a result, while the overall structure of teleost loci can be extremely complex, the teleost isotype repertoire is typically much smaller and simpler than is often observed in mammalian loci. % Citation needed

\subsection{Immunoglobulin M (IgM)}

Immunoglobulin M (IgM) was the first IgH isotype to be identified in teleosts, and is homologous to the isotype of the same name found in mammals and other jawed vertebrates %citation needed
. As in mammals, IgM can be either expressed on the surface of B-cells in transmembrane form or secreted from the cell, depending on whether the spliced IgM mRNA retains the two transmembrane exons; however, unlike in mammals, secreted IgM is primarily produced in tetrameric, rather than pentameric, form \citep{fillatreau2013astonishing} % plus original source if possible; figure?
. In those fish species which have been tested, secreted tetrameric IgM is the main form of antibody found in serum.

\begin{table}
\begin{tabular}{lll} % Add secretory/transmembrane, other details
Isotype & Exon designator & \dots \\
IgM & C$\mu$ & \dots \\
IgD & C$\delta$ & \dots \\
IgT/Z & C$\tau$ / C$\zeta$ & \dots \\ % Add tetrapod isotypes?
\end{tabular}
\end{table}

On the chromosome, the vast majority of characterised teleost loci exhibit the standard six-exon configuration of IgM that is observed in mammalian IgH loci, with four C$\mu$ exons followed by two transmembrane domains \citep{fillatreau2013astonishing}. %Figure
As in mammals, the secretory form of IgM in teleosts comprises the four C$\mu$ exons without the transmembrane domains. However, the exon usage of transmembrane IgM in teleosts differs from mammals, and indeed varies among members of the clade. In most species, the C$\mu$4 exon is excluded from the transmembrane isoform, resulting in a C$\mu$1-C$\mu$2-C$\mu$3-TM1-TM2 configuration. However, in medaka and some other species, the third C$\mu$ exon is also excluded, giving rise to a more truncated C$\mu$1-C$\mu$2-TM1-TM2 configuration. % Citation needed, plus figure
Some species have been observed to express multiple forms of secreted IgM: for example, zebrafish express the standard five-exon isoform as well as a more unusual $\mu$1-TM1-TM2 configuration \citep{fillatreau2013astonishing}. Little is known about the functional implications of these differences in splicing pattern between teleost species; however, they illustrate once again the great variety in antibody locus structure and expression seen among teleost fish.

\subsection{Immunoglobulin D (IgD)}

Like IgM, immunoglobulin D (IgD) is a primitive antibody class, with homologues in most groups of jawed vertebrates except birds and some groups of mammals \citep{fillatreau2013astonishing}. % Citation needed
In almost all cases, it is found coexpressed with IgM; however, its role in the adaptive immune system remains unclear. % Citation needed

Secretory IgD has been observed in a minority of teleost species %citation needed
, albeit with divergent mechanisms: % detail of secretory IgD here
In other teleosts, only the transmembrane form of IgD has been observed.

While teleost IgD is homologous with the isotype of the same name observed in most mammals, teleost and mammalian IgD differ importantly in their exon makeup and protein structure. Most strikingly, all teleost IgDs observed to date contain a chimeric C$\mu$1 exon from the IgM constant region, arising from splicing from the donor site after the C$\mu$1 exon to an acceptor site before exon C$\delta$1; this configuration is universal in teleosts, but otherwise only seen in a small number of artiodactyl mammals \citep{fillatreau2013astonishing}. Teleost IgD also lacks the flexible hinge region present in mammalian versions \citep{fillatreau2013astonishing}. % better citation needed - find out about hinge regions

The number of exons present in teleost IgD is highly variable, and typically much larger than in mammalian loci. % 7 to 17
This size and variablity arises from frequent tandem duplications of C$\delta$ loci in the IgD constant region, especially of the C$\delta$2--C$\delta$3--C$\delta$4 exon block: for example, this block occurs three times in succession in channel catfish IgD, three to four times in Atlantic salmon, and four times in zebrafish \citep{fillatreau2013astonishing}. Other C$\delta$ tandem duplications have also been observed, sometimes accompanied by deletions of other C$\delta$ exons. As a result, teleost IgD can vary from seven to over 17 exons in length \citep{fillatreau2013astonishing}, with proportional variation in the size and weight of the resulting protein chain. % TODO: Add table of IgD sizes in different teleosts?

\subsection{Immunoglobulin T/Z (IgT/Z)}

Unlike IgM and IgD, the immunoglobulin T isotype (IgT, also known as IgZ) is unique to teleost fish. Also unlike IgM and IgD, IgZ is not found universally among teleost loci -- of those IgH loci characterised to date, IgZ is missing in those of medaka %citation needed
and channel catfish %citation needed
\citep{fillatreau2013astonishing} , having apparently been lost independently in these species. In those species in which it is present, IgZ appears to act as a specialised mucosal antibody class, with elevated levels observed in mucosal secretions compared to the level in serum \citep{fillatreau2013astonishing} % better citation needed
. Unlike teleost IgM, secretory IgZ in serum is predominantly monomeric % Clarify that this means one full tetramer, not only one chain
, while in mucosal secretions it is found primarily as a tetramer \citep{fillatreau2013astonishing}. % better citation needed

Among those species of teleosts possessing IgZ, the number of exons present in the IgZ constant region can vary, though not to the dramatic extend observed in teleost IgD. Most teleost species with characterised loci possess a standard configuration similar to IgM, with four C$\zeta$ exons and two transmembrane exons. However, other species have fewer C$\zeta$ exons -- three in stickleback % citation needed
, two in fugu % citation needed
%-- indicating a somewhat greater degree of variability than is typically observed for IgM.

The typical gene structure of teleost IgH loci (see below) imposes a binary choice between IgZ and IgM/D 

\subsection{Immunoglobulin heavy chain (IgH) loci in teleosts}

The structure and size of teleost heavy-chain loci is highly variable, ranging from some of the smallest to among the largest such loci observed in any vertebrate species.

The canonical teleost IgH locus adopts a V-D-J-C-D-J-C structure, with the IgZ constant region and its dedicated D/J segments located upstream of IgM, IgD and their D/J segments \citep{fillatreau2013astonishing}. % Figure
While the V segments in such a locus are generally shared among IgZ and IgM/D, the D and J segments upstream of IgZ are only rarely used by IgM/D-expressing cells.
This structure therefore imposes a binary choice between expressing IgZ and IgM/D in such loci, with recombination between a V-segment and a D-segment downstream of IgZ excising the IgZ constant region from the locus sequence. This relatively simple locus structure is found in zebrafish, grasscarp, and fugu, among other teleost species. %Citation, figure

However, in many teleost species the IgH locus structure is much more complex, with multiple tandem duplications of the entire structure to produce multiple contiguous subloci on a single chromosome. In such complex loci, it is common for several of the resulting constant-region duplicates to become pseudogenised or deleted, in addition to the pseudogenisation of VDJ-segments that is common in antigen-receptor loci % citation needed - learn more about pseudo-Vs
. This combination of sequence duplication and deletion gives rise to extremely complex locus structures, which can be very difficult to assemble accurately; nevertheless, with the exception of the Atlantic salmon and other salmonids, most teleost genomes so far analysed contain only one functional heavy-chain locus on a single chromosome, large and complex as it may be, despite the repeated whole-genome duplications that have occurred in the teleost lineage.

The two % check if this is up-to-date
teleost heavy-chain loci identified to date that lack IgZ have both %/all
been of this large, highly-duplicated type \citep{fillatreau2013astonishing}

%%%

\subsubsection{Zebrafish}

The zebrafish IgH locus occupies roughly 175 kb on chromosome 3 \citep{danilova2005zebrafishIgH} and represents the archetypal heavy-chain locus structure in teleosts, with a c. 100 kb V-region (containing 47 V-segments, of which at least 36 appear to be functional) followed by two D-J-C segment blocks. The first of these blocks contains two D and two J segments, followed by the six exons of the IgZ constant region; the second contains five D and five J, followed by IgM and IgD. The sequence identity between zebrafish IgM and IgZ (roughly 20\%) is considerably less than between zebrafish IgM and that of other teleosts (40-50\%). The identified V-genes separated into 14 families, where the members of a family were defined as showing at least 70\% sequence identity with at least one other family member. The IgD constant region contained four tandem repeats of the C$\delta$2-C$\delta$4 exon block, resulting in 17 IgD exons total. Expression data indicated that D/J segment usage was exclusively confined to transcripts containing the closest constant-region, indicating an exclusive choice between IgZ and IgM/Z based on the VDJ-recombination state of the locus in a given B-cell % Figure for this
. % Info on cysteines, splice sites, RSSs as needed for results                  

\subsubsection{Medaka}

