%\newabbreviation{bellemans}{GRZ-Bellemans}{Substrain of GRZ}

\section{Gut microbiota transfer and the turquoise killifish mucosal repertoire}
\label{sec:igseq_gut}

The results from \Cref{sec:igseq_ageing} demonstrate that the whole-body antibody repertoire of adult male turquoise killifish declines in % (clonal)
alpha-diversity with age, while increasing in VJ beta-diversity % TODO: ... other summarised results go here
These results suggest that ... . % TODO: Finish summary

However, while these results demonstrate significant age-related changes in repertoire composition at the level of the entire body, they give no specific information about the ageing process exhibited by the specialised repertoires of particular immune organs, which could differ substantially as a result of their distict antigenic environments. In particular, the gut mucosal repertoire, which polices the interface between the host organism and the gut microbiota, represents a highly important B-cell subpopulation with an intense and distinctive exprience of antigenic exposure. It would therefore be interesting to learn whether the pattern of repertoire ageing observed in the gut accords with that seen in the wider body, or exhibits its own distinct ageing phenotypes. % TODO: Citations needed for this

Smith \textit{et al.} \parencite{smith2017microbiota} demonstrated that gut microbiota transfer from young to middle-aged turquoise killifish significantly extends lifespan in this species, as well as significantly altering microbiota composition and gene expression in the gut. As part of these experiments, gut total RNA was collected from a number of male turquoise killifish of the GRZ-Bellemans substrain (a closely-related substrain to the GRZ-AD substrain used in \Cref{sec:igseq_ageing}) at different ages and following different experimental interventions (\Cref{tab:gut-cohorts-summary,tab:gut-cohorts-fish}). By using these RNA samples to perform a further round of \Igseq in the turquoise killifish, I hoped to investigate the effects of both age and microbiota transfer on the diversity of the gut mucosal repertoire; in paticular, given the observed lifespan effect of gut-microbiota transfer and the intimate relationship between the gut microbiota and the mucosal adaptive immune system, % TODO: Citation needed
I hypothesised that gut microbiotal transfer might significantly ameliorate the ageing of the adaptive-immune repertoire seen in \Cref{sec:igseq_ageing}, either through delaying the decline in diversity observed in older fish or by stimulating a renewal of repertoire diversity in mucosal B-cells.

\begin{table}[b]
% latex table generated in R 3.5.2 by xtable 1.8-3 package
% Mon Mar 11 16:51:04 2019
\begin{tabular}{lrlll}
  \toprule Group & Age (weeks) & \# Fish (Sequenced/Total) & Antibiotics? & Microbiota Transfer? \\ 
  \midrule YI\_6 & 6 & 4 / 4 & No & No \\ 
  WT\_16 & 16 & 3 / 4 & No & No \\ 
  ABX\_16 & 16 & 4 / 4 & Yes & No \\ 
  SMT\_16 & 16 & 3 / 4 & Yes & Yes (9.5-week-old donor) \\ 
  YMT\_16 & 16 & 4 / 4 & Yes & Yes (6-week-old donor) \\ 
   \bottomrule \end{tabular}

\caption{Summary of killifish used in \igseq validation and ageing experiment. All fish are GRZ-Bellemans strain and male.}
\label{tab:gut-cohorts-summary}
\end{table}

% TODO: Summary figure of gut microbiota study


