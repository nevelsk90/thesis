\section{Introduction}

%To understand and counter the complex changes that occur in the adaptive immune system with age, it is necessary to analyse the whole population of lymphocyte antigen-receptor sequences present in an individual. Such a top-down approach was impossible until relatively recently, when the advent of modern high-throughput sequencing technologies enabled the development of specialised protocols for immune-repertoire sequencing and analysis. Since then, the field of immune-repertoire studies has developed rapidly, providing a new and more powerful method for interrogating the changes ocurring in adaptive immune repertoires in a wide variety of contexts, including ageing. However, while initial human studies have indicated a decline in the diversity of these repertoires in older people, there remains a need for further research in this area. % This is weak, fill in once you've done your lit review.

%\section{Quantifying antibody repertoire composition with immunoglobulin sequencing}
%\label{sec:intro_igseq}
%
%% TODO: Move these paragraphs to IgSeq chapter intro?
%
%As a result of the primary and secondary diversification mechanisms described in \Cref{sec:intro_immunity_primary,sec:intro_affinity_maturation}, the population of B-cells present in the body of a mature jawed vertebrate together express an enormous diversity of unique antibody sequences, with a correspondingly vast array of antigen specificities.  This huge population of sequences contains a considerable amount of structure: \naive sequences in the primary repertoire are produced by the same underlying generative and selective process and so share common statistical properties, while many sequences in the secondary repertoire are related to one another as members of clones descended from a common \naive B-cell ancestor. By sampling the repertoire of antibody sequences present in a sample and reconstructing its underlying recombinatorial and clonal structure, much can be learned about the history and current state of humoral adaptive immununity in that organism.
%
%The advent of modern high-throughput sequencing mechanisms enabled the development of specialised sequencing techniques designed to interrogate the antibody repertoire, via targeted sequencing of B-cell DNA or RNA. From the beginning, these \Igseq (or \igseq) studies included teleost model organisms, with early \igseq studies investigating the antibody repertoire of developing and adult zebrafish. These studies revealed ... % Describe results from early zebrafish studies

%Caruso \textit{et al.} \parencite{caruso2009immunosenescence}  commented that, though the antibody repertoire diversity of mucosal systems had not (and, to my knowledge, has still not) been investigated specifically, the fact that these regions undergo especially frequent antigenic challenge and consequent clonal expansion of antigen-experienced cells suggests that they might be expected to show a particularly strong decline in repertoire diversity with age, impairing the ability of the organism to regulate its mucosal microbiotal environments...

% TODO: Move to discussion chapter?
% Conversely, the most well-developed teleost model organism, the zebrafish, is relatively distantly-related, with an estimated divergence time of over 200 Mya \parencite{hughes2018teleostphylo}, making the use in a killifish context of the many genetic and experimental resources developed for the zebrafish more challenging. As a result, techniques such as cell-type-specific cell sorting that depend on cell-surface markers or specific antibodies are currently unavailable for killifish studies, though many are under active development in one or more killifish laboratories.

%Turquoise killifish are strongly sexually dimorphic, with an XY sex-determination system \parencite{reichwald2015genome,valenzano2009map} and large, colourful males. Most killifish studies to date have been performed using fish of only one sex, typically males; this thesis is no exception.


%% PILOT IGOR ANALYSIS


% \subsection{Generative models and potential repertoire entropy}

% The previous two sections analysed clonal and V(D)J repertoires, respectively, as actually observed for individuals in the pilot dataset. For several reasons, such observed distributions are distinct from the original generative process giving rise to ...


% The observed V(D)J repertoire is strongly affected by selection, clonal expansion and other population-level processes taking place after the initial generation of \igh{} sequences via VDJ-recombination and its attendant junctional diversity...

% In order to access the original generative process giving rise to recombined \igh{} sequences in an organism, the program \program{IGoR} (...) infers a generative model of sequence recombination ... % TODO: Describe IGoR process, sequence selection


% TODO: Generative models